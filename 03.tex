\chapter{}

\section*{\S3. Noetherian rings}

We now study noetherian rings. We will prove Hilbert's basis 
theorem. After that we will start our study of factorization
in commutative rings. 

\begin{definition}
	A ring $R$ is said to be \textbf{noetherian} if every (increasing)
	sequence $I_1\subseteq I_2\subseteq\cdots$ of ideals of $R$
	stabilizes, that is $I_n=I_m$ for some $m\in\Z_{>0}$ and all $n\geq m$. 
\end{definition}

The ring $\Z$ of integers is noetherian.

\begin{exercise}
Let $R=\{f\colon [0,1]\to\R\}$ with 
\[
(f+g)(x)=f(x)+g(x),
\quad
(fg)(x)=f(x)g(x),
\quad
f,g\in R,\,x\in [0,1].
\]
For $n\in\Z_{>0}$ let
$I_n=\{f\in R:f|_{[0,1/n]}=0\}$. Then each $I_n$ is an ideal of $R$ and 
the sequence 
$I_1\subsetneq I_2\subsetneq\cdots$ 
does not stabilizes. Thus $R$ is not noetherian. 
\end{exercise}

\begin{definition}
	Let $R$ be a ring. An ideal $I$ of $R$ is said to be \textbf{finitely generated} if $I=(X)$ for some
	finite subset $X$ of $R$. 
\end{definition}

If $R$ is a ring, $\{0\}$ and $R$ are finitely generated. 

\begin{proposition}
Let $R$ be a ring. Then $R$ is noetherian if and only 
if every ideal of $R$ is finitely generated. 	
\end{proposition}

\begin{proof}
	Assume first that $R$ is noetherian. Let $I$ be an ideal of $R$ that is not finitely generated. 
	Thus $I\ne\{0\}$. Let $x_1\in I\setminus\{0\}$ and let $I_1=(x_1)$. Since $I$ is not finitely
	generated, $I\ne I_1$ and hence   
	$\{0\}\subsetneq I_1\subsetneq I$. Once I have the ideals $I_1,\dots,I_{k-1}$, let 
	$x_k\in I\setminus I_{k-1}$ (such an element exists because $I_{k-1}$ is finitely generated
	and $I$ is not) and $I_k=(I_{k-1},x_k)$. The sequence
	$\{0\}\subsetneq I_1\subsetneq I_2\subsetneq\cdots$ does not stabilize.  
	
	Assume now that every ideal of $R$ is finitely generated and 
	let $I_1\subseteq I_2\subseteq\cdots$ be a sequence of ideals of $R$. Then
	$I=\cup_{i\geq1}I_i$ is an ideal of $R$, so it is finitely generated, say
	$I=(x_1,\dots,x_n)$. We may assume that $x_j\in I_{i_j}$ for all $j$. Let 
	$N=\max\{j_1,\dots,j_n\}$ and $n\geq N$. Then 
	$I_N\subseteq I\subseteq I_N$ and therefore the sequence stabilizes.  
\end{proof}

\begin{exercise}
	Let $R=\C[X_1,X_2,\cdots]$ be the ring of polynomial in an infinite number of 
	commuting variables. Prove that the ideal $I=(X_1,X_2,\dots)$ of polynomials 
	with zero constant term is not finitely generated. 
\end{exercise}


The correspondence theorem and the previous proposition 
allow us to prove easily the following result. 

\begin{proposition}
	Let $I$ be an ideal of $R$. If $R$ is noetherian, then $R/I$ is noetherian.
\end{proposition}

\begin{proof}
	Let $\pi\colon R\to R/I$ be the canonical surjection and let $J$ be an ideal of $R/I$. 
	Then $\pi^{-1}(J)$ is an ideal of $R$ containing $I$. Since 
	$R$ is noetherian, $\pi^{-1}(J)$ is finitely generated, say 
	$\pi^{-1}(J)=(x_1,\dots,x_k)$ for $x_1,\dots,x_k\in R$. Thus 
	\[
	J=\pi(\pi^{-1}(J))=(\pi(x_1),\dots,\pi(x_k))
	\]
	and hence $J$ is finitely generated. 
\end{proof}

Since $\Z$ is noetherian, $\Z/n$ is noetherian for all $n\geq2$. 

\begin{exercise}
	Prove that $\R[X]$ is noetherian. 	
\end{exercise}

% usar qu ees principal
% todo: agregar despues de la prueba de la principalidad para Z que tambien R[X] es principal

\begin{theorem}[Hilbert]
	Let $R$ be a commutative ring. If $R$ is noetherian ring, then $R[X]$ is noetherian.	
\end{theorem}

\begin{proof}
	We need to show that every ideal of $R[X]$ is finitely generated. Assume that
	there is an ideal $I$ of $R[X]$ that is not finitely generated. In particular, $I\ne\{0\}$. 
	Let $f_1(X)\in I\setminus\{0\}$ be of minimal degree. For $i>1$ let 
	$f_i(X)\in I$ be of minimal degree such that $f_i(X)\not\in(f_1(X),\dots,f_{i-1}(X))$ (note
	that such an $f_i(X)$ exists because $I$ is not finitely generated). For each $i$ 
	let $a_i$ be the leading coefficient of $f_i(X)$, that is
	\[
	f_i(X)=a_iX^{n_i}+\cdots,
	\]
	where the dots denote lowest degree terms. Note that 
	$a_i\ne 0$.
	Let $J=(a_1,a_2,\dots)$. Since $R$ is noetherian, the sequence
	\[
	(a_1)\subseteq (a_1,a_2)\subseteq\cdots(a_1,a_2,\dots,a_k)\subseteq\cdots
	\]
	stabilizes, so $J$ is finitely generated, say
	$J=(a_1,\dots,a_m)$ for some $m\in\Z_{>0}$. 
	There exist $u_1,\dots,u_m\in R$ such that 
	\[
	a_{m+1}=\sum_{i=1}^m u_ia_i.
	\]
	Let 
	\[
	g(X)=\sum_{i=1}^mu_if_i(X)X^{n_{m+1}-n_i}\in (f_1(X),\dots,f_m(X)).
	\]
	The leading coefficient of $g(X)$ is $\sum_{i=1}^mu_ia_i=a_{m+1}$ and, moreover, 
	the degree of $g(X)$ is $n_{m+1}$. Thus $\deg(g(X)-f_{m+1}(X))<n_{m+1}$. 
	
	Since $f_{m+1}(X)\not\in (f_1(X)\dots,f_n(X))$, 
	\[
	g(X)-f_{m+1}(X)\not\in (f_1(X),\dots,f_n(X)),
	\]
	a contradiction to the minimality of the degree of $f_{m+1}$.  
\end{proof}

Since $R[X_1,\dots,X_n]=(R[X_1,\dots,X_{n-1}])[X_n]$, by induction 
one proves that if $R$ is a commutative noetherian ring, 
then $R[X_1,\dots,X_n]$ is noetherian. 
 
\begin{example}
	Since $\Z$ is noetherian, so is $\Z[X]$ by Hilbert's theorem. Now 
	$\Z[\sqrt{N}]$ is noetherian, as $\Z[\sqrt{N}]\simeq\Z[X]/(X^2-N)$ and quotients
	of noetherian rings are noetherian.  	
\end{example}

\begin{example}
	The ring $\Z[X,X^{-1}]$ is noetherian, as $\Z[X,X^{-1}]\simeq\Z[X,Y]/(XY-1)$. 
\end{example}
 
 
\begin{exercise}
    Let $R$ be a ring and $R[[X]]$ be the ring of formal power series with the usual operations.  
	Prove that $R[[X]]$ is noetherian if $R$ is noetherian. 	
\end{exercise}

\begin{exercise}
	Let $f\colon R\to R$ be surjective ring homomorphism. Prove that $f$ is an isomorphism
	if $R$ is noetherian. 	
\end{exercise}

\section*{\S4. Factorization}

\begin{definition}
\index{Integral domain}
\index{Domain}
    A \textbf{domain} $R$ is a ring such that $xy=0$ implies $x=0$ or $y=0$. 
    An \textbf{integral} domain is a commutative ring that is also a domain. 
% 	A commutative ring $R$ is said to be an \textbf{integral domain}
% 	if $xy=0$ implies $x=0$ or $y=0$.  	
\end{definition}

The rings $\Z$ and $\Z[i]$ are both integral domains. 
More generally, if $N$ is a square-free integer, 
then the ring $\Z[\sqrt{N}]$ is an integral domain.  
The ring $\Z/4$ of 
integers modulo 4 is not an integral domain. 

\begin{definition}
	Let $R$ be an integral domain and $x,y\in R$. Then $x$ \textbf{divides} $y$ 
	if $y=xz$ for some $x\in R$. 
	Notation: $x\mid y$ if and only if $x$ divides $y$. If $x$ does not
	divide $y$ one writes $x\nmid y$.  
\end{definition}

Note that $x\mid y$ if and only if $(y)\subseteq (x)$.
	
\begin{definition}
	Let $R$ be an integral domain and $x,y\in R$. Then $x$ and $y$ are
	\textbf{associate} in $R$ if $x=yu$ for some $u\in\mathcal{U}(R)$. 
\end{definition}

Note that $x$ and $y$ are associate if and only if $(x)=(y)$.  

\begin{example}
	The integers $2$ and $-2$ are associate in $\Z$.	
\end{example}

\begin{example}
	Let $R=\Z[i]$. 
	\begin{enumerate}
		\item Let $d\in\Z$ and $a+ib\in R$. Then $d\mid a+ib$ in $R$ if and only if 
			$d\mid a$ and $d\mid b$ in $\Z$. 
		\item $2$ and $-2i$ are associate in $R$.
	\end{enumerate} 	
\end{example}

\begin{example}
	Let $R=\R[X]$ and $f(X)\in R$. Then $f(X)$ and $\lambda f(X)$ are 
	associate in $R$ for all $\lambda\in\R^{\times}$. 	
\end{example}

\begin{definition}
	Let $R$ be an integral domain and $x\in R\setminus\{0\}$. Then $x$ is \textbf{irreducible} 
	if and only if $x\not\in\mathcal{U}(R)$ 
	and $x=ab$ with $a,b\in R$ implies that $a\in\mathcal{U}(R)$ or $b\in\mathcal{U}(R)$. 
\end{definition}

Note that $x$ is irreducible if and only if $(x)\ne R$ 
and there is no principal ideal $(y)$ such that 
$(x)\subsetneq (y)\subsetneq R$.

\begin{example}
	Let $R=\R[X]$ and $f(X)\in R\setminus\{0\}$. Then the polynomial $f(X)$ is irreducible if 
	$\lambda\in\R^{\times}$ or $\lambda f(X)$ for $\lambda\in\R^{\times}$ 
	are the only divisors
	of $f(X)$.  
\end{example}

The irreducibles of $\Z$ are the prime numbers. Note that $p\in \Z$
is prime if and only if $p\mid xy$ then $p\mid x$ or $p\mid y$. 

\begin{definition}
	Let $R$ be an integral domain and $p\in R\setminus\{0\}$. Then  
	$p$ is \textbf{prime} if $p\not\in\mathcal{U}(R)$ and 
	$yz\in (p)$ implies that $y\in (p)$ or $z\in (p)$. 
\end{definition}

In $\Z$ primes and irreducible coincide. This does not happen in full generality. However,
the following result holds. 

\begin{proposition}
	Let $R$ be an integral domain and $p\in R$. 
	If $p$ is prime, then $p$ is irreducible. 
\end{proposition}

\begin{proof}
	Let $p$ be a prime. Then $p\ne 0$ and $p\not\in\mathcal{U}(R)$. Let $x$ be such that
	$x\mid p$. Then $p=xy$ for some $y\in R$. This means $xy\in (p)$, 
	so $x\in (p)$ or $y\in (p)$ because
	$p$ is prime. If $x\in (p)$, then $x=pz$ for some $z\in R$ and hence
	\[
	p=xy=(pz)y.
	\]
	Since $p-pzy=p(1-zy)$ and $R$ is an integral domain, it follows that 
	$1-zy=0$. Thus $y\in\mathcal{U}(R)$. Similarly, if $y\in (p)$, then 
	$x\in\mathcal{U}(R)$. 
\end{proof}

To show that there are rings where some irreducibles are not prime, 
we need the following lemma. A square-free
integer is an integer which is divisible by no perfect square other than 1.

\begin{lemma}
Let $N\in\Z$ be a non-zero square-free integer and $R=\Z[\sqrt{N}]$. Then 
the map 
\[
	N\colon R\to\Z_{\geq0},
\quad a+b\sqrt{N}\mapsto 
|a^2-Nb^2|,
\]
satisfies the following properties:
\begin{enumerate}
	\item $N(\alpha)=0$ if and only if $\alpha=0$. 
	\item $N(\alpha\beta)=N(\alpha)N(\beta)$ for all $\alpha,\beta\in R$. 
	\item $\alpha\in\mathcal{U}(\Z[\sqrt{N}])$ if and only if $N(\alpha)=1$. 
	\item If $N(\alpha)$ is prime in $\Z$, then $\alpha$ is irreducible in $R$. 
\end{enumerate}	
\end{lemma}

\begin{proof}
	The first three items are left as exercises. Let us prove 4). 
	If $\alpha=\beta\gamma$ for some $\beta,\gamma\in R$, then
	$N(\alpha)=N(\beta)N(\gamma)$. Since $N(\alpha)$ is a prime number, it follows that
	$N(\alpha)=1$ or $N(\beta)=1$. Thus $\beta\in\mathcal{U}(R)$ or $\gamma\in\mathcal{U}(R)$. 	
\end{proof}

\begin{example}
	Let $R=\Z[i]$. 
	\begin{enumerate}
		\item $\mathcal{U}(R)=\{-1,1,i,-i\}$.
		\item $3$ is irreducible in $R$. In fact, if $3=\alpha\beta$, then
			$9=N(\alpha)N(\beta)$. This implies that $N(\alpha)\in\{1,3,9\}$. Write
			$\alpha=a+bi$ for $a,b\in\Z$. If $N(\alpha)=1$, then $\alpha\in\mathcal{U}(R)$ by the lemma. 
			If $N(\alpha)=9$, then $N(\beta)=1$ and hence $\beta\in\mathcal{U}(R)$ by the lemma. Finally, 
			if $N(\alpha)=3$, then $a^2+b^2=3$, which is a contradiction since $a,b\in\Z$. 
		\item $2$ is not irreducible in $R$. In fact, $2=(1+i)(1-i)$ and
			since \[
			N(1+i)=N(1-i)=2,
			\]
			it follows that $1+i\not\in\mathcal{U}(R)$ 
			and $1-i\not\in\mathcal{U}(R)$. 
	\end{enumerate}	
\end{example}

