\chapter{Some examples}

Let $G$ be a finite group and $\chi_1,\dots,\chi_r$ be the irreducible characters of $G$. Without loss of generality
we may assume that $\chi_1$ is the trivial character, i.e. $\chi_1(g)=1$ for all $g\in G$. 
Recall that $r$ is the number of conjugacy classes of $G$. Each $\chi_j$ is constant on conjugacy classes. 
The \textbf{character table} of 
$G$ is given by 
\begin{center}
\begin{tabular}{|c|cccc|}
\hline 
 & $1$ & $k_{2}$ & $\cdots$ & $k_{r}$\tabularnewline
 & $1$ & $g_{2}$ & $\cdots$ & $g_{r}$\tabularnewline
\hline 
$\chi_{1}$ & $1$ & $1$ & $\cdots$ & $1$\tabularnewline
$\chi_{2}$ & $n_{2}$ & $\chi_{2}(g_{2})$ & $\cdots$ & $\chi_{2}(g_{r})$\tabularnewline
$\vdots$ & $\vdots$ & $\vdots$ & $\ddots$ & $\vdots$\tabularnewline
$\chi_{r}$ & $n_{r}$ & $\chi_{r}(g_{2})$ & $\cdots$ & $\chi_{r}(g_{r})$\tabularnewline
\hline
\end{tabular}
\end{center}
where the $n_j$ are the degrees of the irreducible representations of $G$ and each $k_j$ is 
the size of the conjugacy class of the element $g_j$. By convention, the character table
contains not only the values of the irreducible characters of the group. 

\begin{example}
	Sea $G=\langle g:g^4=1\rangle$ 
	be the cyclic group of order four. The character table of $G$ is given by
	\begin{center}
		\begin{tabular}{|c|cccc|}
			\hline 
			& 1 & 1 & 1 & 1\tabularnewline
			& $1$ & $g$ & $g^2$ & $g^{3}$\tabularnewline
			\hline 
			$\chi_{1}$ & $1$ & $1$ & $1$ & $1$\tabularnewline
			$\chi_{2}$ & $1$ & $\lambda$ & $\lambda^2$ & $\lambda^{3}$\tabularnewline
			$\chi_{3}$ & $1$ & $\lambda^2$ & $\lambda^4$ & $\lambda^{2}$\tabularnewline
			$\chi_{4}$ & $1$ & $\lambda^{3}$ & $\lambda^{2}$ & $\lambda$\tabularnewline
			\hline
		\end{tabular}
	\end{center}
Let us see how to see this calculation in the computer:
\begin{lstlisting}
gap> C4 := CyclicGroup(4);;                       
gap> T := CharacterTable(C4);;
gap> Display(T);
CT1

     2  2  2  2  2

       1a 4a 2a 4b

X.1     1  1  1  1
X.2     1 -1  1 -1
X.3     1  A -1 -A
X.4     1 -A -1  A

A = E(4)
  = Sqrt(-1) = i
\end{lstlisting}
We need some remarks
\begin{enumerate}
    \item The symbol \lstinline{E(4)} denotes a primitive fourth root of 1.
    \item The function \lstinline{CharacterTable} computes some more information, not only the character table of the group. 
    esta función calcula algunas otras cosas. Por ejemplo:
\end{enumerate}
\begin{lstlisting}
gap> OrdersClassRepresentatives(T);
[ 1, 4, 2, 4 ]
gap> SizesCentralizers(T);
[ 4, 4, 4, 4 ]
gap> SizesConjugacyClasses(T);
[ 1, 1, 1, 1 ]
\end{lstlisting}
\end{example}

As an exercise, if $\lambda$ is a primitive root of 1 of order $n$, 
the character table of the cyclic group of order $n$ is given by 
	\begin{center}
		\begin{tabular}{|c|ccccc|}
			\hline 
			& 1 & 1 & 1 & $\cdots$ & 1\tabularnewline
			& $1$ & $g$ & $g^2$ & $\cdots$ & $g^{n-1}$\tabularnewline
			\hline 
			$\chi_{1}$ & $1$ & $1$ & $1$ & $\cdots$ & $1$\tabularnewline
			$\chi_{2}$ & $1$ & $\lambda$ & $\lambda^2$ & $\cdots$ & $\lambda^{n-1}$\tabularnewline
			$\chi_{3}$ & $1$ & $\lambda^2$ & $\lambda^4$ & $\cdots$ & $\lambda^{n-2}$\tabularnewline
			$\vdots$ & $\vdots$ & $\vdots$ & $\vdots$ & $\ddots$ & $\vdots$\tabularnewline
			$\chi_{n}$ & $1$ & $\lambda^{n-1}$ & $\lambda^{n-2}$ & $\cdots$ & $\lambda$\tabularnewline
			\hline
		\end{tabular}
	\end{center}

\begin{example}
	The character table of the group $C_2\times C_2=\{1,a,b,ab\}$ is 
	\begin{center}
		\begin{tabular}{|c|rrrr|}
			\hline 
			& 1 & 1 & 1 & 1\tabularnewline
			& $1$ & $a$ & $b$ & $ab$\tabularnewline
			\hline 
			$\chi_{1}$ & $1$ & $1$ & $1$ & $1$\tabularnewline
			$\chi_{2}$ & $1$ & $1$ & $-1$ & $-1$\tabularnewline
			$\chi_{3}$ & $1$ & $-1$ & $1$ & $-1$\tabularnewline
			$\chi_{4}$ & $1$ & $-1$ & $-1$ & $1$\tabularnewline
			\hline
		\end{tabular}
	\end{center}
	Let us do this by computer:
\begin{lstlisting}
gap> Display(CharacterTable(AbelianGroup([2,2])));
CT2

     2  2  2  2  2

       1a 2a 2b 2c

X.1     1  1  1  1
X.2     1 -1  1 -1
X.3     1  1 -1 -1
X.4     1 -1 -1  1
\end{lstlisting}
\end{example}

Clearly, the order in which the computer returns the irreducible characters is not necessarily the same we used! 

\begin{example}
	The symmetric group $\Sym_3$ has three conjugacy classes. The representatives are 
	$\id$, $(12)$ and $(123)$. There are three irreducible representations. We already found all the irreducible characters! 
	The character table of $\Sym_3$ is given by 
	\begin{center}
		\begin{tabular}{|c|rrr|}
			\hline
			& $1$ & $3$ & $2$\tabularnewline
			& $1$ & $(12)$ & $(123)$ \tabularnewline
			\hline 
			$\chi_{1}$ & $1$ & $1$ & $1$\tabularnewline
			$\chi_{2}$ & $1$ & $-1$ & $1$ \tabularnewline
			$\chi_{3}$ & $2$ & $0$ & $-1$ \tabularnewline
			\hline
		\end{tabular}
	\end{center}
	Let us recall how this table was computed. Degree-one irreducibles were easy to compute. 
	To compute the third row of the table one possible approach is to use
	the irreducible representation  
	\[
	(12)\mapsto \begin{pmatrix}-1&1\\0&1\end{pmatrix},
	\quad
	(123)\mapsto \begin{pmatrix}0&-1\\1&-1\end{pmatrix}.
	\]
    Then	
    \begin{align*}
		&\chi_3\left( (12) \right)=\trace\begin{pmatrix}-1&1\\0&1\end{pmatrix}=0,\\
		&\chi_3\left( (123) \right)=\chi_3\left( (12)(23)\right)=\trace\begin{pmatrix}0&-1\\1&-1\end{pmatrix}=-1.
	\end{align*}

	We should remark that the irreducible representation mentioned is not really needed to
	compute the third row of the character table. We can, for example, use the regular
	representation $L$. The character of $L$ is given by 
	\[
		\chi_L(g)=\begin{cases}
			6 & \text{si $g=\id$},\\
			0 & \text{si $g\ne\id$}.
		\end{cases}
	\]
	The equality $0=\chi_L\left( (12) \right)=1-1+2\chi_3( (12))$ implies that 
	$\chi_3( (12))=0$ and the equality $0=\chi_L( (123))=1+1+2\chi_3( (123))$
	implies that $\chi_3\left( (123) \right)=-1$. 

	Another approach uses the orthogonality relations. We need to compute $\chi_3( (12) )$ and $\chi_3( (123))$. 
	Let $a=\chi_3( (12) )$ and $b=\chi_3( (123))$. Then 
    we get that $a=0$ and $b=-1$. We just need to solve  
	\begin{align*}
		0&=\langle \chi_3,\chi_1\rangle=\frac16(2+3a+2b),\\
		0&=\langle \chi_3,\chi_2\rangle=\frac16(2-3a+2b).
	\end{align*}
	
	Let us use the computer:
	\begin{lstlisting}
gap> S3 := SymmetricGroup(3);;
gap> T := CharacterTable(S3);;
gap> Display(T);
CT3

     2  1  1  .
     3  1  .  1

       1a 2a 3a
    2P 1a 1a 3a
    3P 1a 2a 1a

X.1     1 -1  1
X.2     2  . -1
X.3     1  1  1
\end{lstlisting}
As we did before, some extra information was computed:
\begin{lstlisting}
gap> SizesConjugacyClasses(T);
[ 1, 3, 2 ]
gap> SizesCentralizers(T);
[ 6, 2, 3 ]
gap> SizesConjugacyClasses(T);
[ 1, 3, 2 ]
gap> OrdersClassRepresentatives(T);
[ 1, 2, 3 ]
\end{lstlisting}
\end{example}

%A challenging exercise: 
\begin{exercise}
Compute the character table of $\Sym_4$. 
\end{exercise}

\begin{example}
	We now compute the character table of the alternating group $\Alt_4$. This group has $12$ 
	elements and four conjugacy classes.
	\begin{center}
		\begin{tabular}{c|cccc}
			representante & $\id$ & $(123)$ & $(132)$ & $(123)$\tabularnewline
			\hline
			tamaño & $1$ & $4$ & $4$ & $3$ 
		\end{tabular}
	\end{center}
	Since $[\Alt_4,\Alt_4]=\{\id,(12)(34),(13)(24),(14)(23)\}$,
	$\Alt_4/[\Alt_4,\Alt_4]$ has three elements. Thus $\Alt_4$ has three degree-one irreducibles and
	an irreducible character of degree three. Let 
	$\omega=\exp(2\pi i/3)$ be a primitive cubic root of 1. If $\chi$
	is a non-trivial degree-one character, then 
	$\chi\left( (123) \right)=\omega^j$
	for some $j\in\{1,2\}$ and $\chi\left( (132) \right)=\omega^{2j}$. Since 
	$(132)(134)=(12)(34)$ and 
	the permutations $(134)$ and $(123)$ are conjugate,  
	\[
	\chi_i((12)(34))=\chi_i((132)(134))=\chi_i((132))\chi_i((134))=\omega^3=1
	\]
	for all $i\in\{1,2\}$. 
	
	To compute $\chi_4$ we use the regular representation. 
	\begin{align*}
		0&=\chi_L\left( (12)(34) \right)=1+1+1+3\chi_4\left( (12)(34) \right),\\
		0&=\chi_L\left( (123) \right)=1+\omega+\omega^2+3\chi_4\left( (123) \right),\\
		0&=\chi_L\left( (132) \right)=1+\omega+\omega^2+3\chi_4\left( (132) \right).
	\end{align*}
	Then we obtain that $\chi_4\left( (123) \right)=\chi_4\left( (132)
	\right)=0$ and $\chi_4\left( (12)(34) \right)=-1$. Therefore, the character table of $\Alt_4$
	is given by
	\begin{center}
		\begin{tabular}{|c|rrrr|}
			\hline
			& $\id$ & $(123)$ & $(132)$ & $(12)(34)$\tabularnewline
			\hline
			$\chi_1$ & $1$ & $1$ & $1$ & $1$\tabularnewline
			$\chi_2$ & $1$ & $\omega$ & $\omega^2$ & $1$\tabularnewline
			$\chi_3$ & $1$ & $\omega^2$ & $\omega$ & $1$\tabularnewline
			$\chi_4$ & $3$ & $0$ & $0$ & $-1$\tabularnewline
			\hline
		\end{tabular}
	\end{center}
	Let us use the computer:
\begin{lstlisting}
gap> A4 := AlternatingGroup(4);;
gap> T := CharacterTable(A4);;
gap> Display(T);
CT5

     2  2  2  .  .
     3  1  .  1  1

       1a 2a 3a 3b
    2P 1a 1a 3b 3a
    3P 1a 2a 1a 1a

X.1     1  1  1  1
X.2     1  1  A /A
X.3     1  1 /A  A
X.4     3 -1  .  .

A = E(3)^2
  = (-1-Sqrt(-3))/2 = -1-b3
\end{lstlisting}
The symbol \lstinline{E(3)} denotes a primitive cubic root of 1, say our $\omega$. 
To save some space, the compute uses the symbol \lstinline{A} to denote the complex number $\omega^2$ (it is the same as \lstinline{E(3)^2}) and
the symbol \lstinline{/A} to denote the complex number $\omega$, the multiplicative inverse of $\omega^2$. 
\end{example}

\begin{example}
    Let $Q_8=\{-1,1,i,-i,j,-j\}$ be the quaternion group. Let us compute the character table of $Q_8$.
    The group $Q_8$ is generated by $\{i,j\}$ and the map $\rho\colon Q_8\to\GL_2(\C)$, 
    \[
    i\mapsto\begin{pmatrix}
    i&0\\0&i
    \end{pmatrix},
    \quad
    j\mapsto\begin{pmatrix}
    0&1\\-1&0
    \end{pmatrix},
    \]
    is a representation.
    The conjugacy classes of $Q_8$ are $\{1\}$, $\{-1\}$, $\{-i,i\}$, $\{-j,j\}$ and $\{-k,k\}$. 
    So there are five irreducible representations. 
    We can compute the character of $\rho$:
    	\begin{center}
		\begin{tabular}{|c|c|c|c|c|c|}
		    \hline
			& $1$ & $-1$ & $i$ & $j$ & $k$\tabularnewline
			\hline
			$\chi_\rho$ & 2 & 2 & 0 & 0 & 0\tabularnewline
			\hline
		\end{tabular}
	\end{center}
	Then $\rho$ is irreducible, es $\langle\chi_\rho,\chi_\rho\rangle=1$. 
	
	Since $[Q_8,Q_8]=\{-1,1\}=Z(Q_8)$, the quotient group $Q_8/[Q_8,Q_8]$ has four elements and
	hence there are four irreducible degree-one representations. Since 
	$Q_8$ is non-abelian, $Q_8/Z(Q_8)$ cannot be cyclic. 
	This implies that 
	$Q_8/[Q_8,Q_8]\simeq C_2\times C_2$. This allows us
	to compute almost all the character table of $Q_8$. 
		\begin{center}
		\begin{tabular}{|c|rrrrr|}
			\hline
			& $1$ & $-1$ & $i$ & $j$ & $k$\tabularnewline
			\hline
			$\chi_1$ & $1$ & $1$ & $1$ & $1$ & $1$\tabularnewline
			$\chi_2$ & $1$ & \cellcolor{gray!30}{$1$} & $-1$ & $1$ & $-1$\tabularnewline
			$\chi_3$ & $1$ & \cellcolor{gray!30}{$1$} & $1$ & $-1$ & $-1$\tabularnewline
			$\chi_4$ & $1$ & \cellcolor{gray!30}{$1$} & $-1$ & $-1$ & $1$\tabularnewline
			$\chi_5$ & $2$ & $-2$ & $0$ & $0$ & $0$\tabularnewline
			\hline
		\end{tabular}
	\end{center}
	It remains to compute $\chi_j(-1)$ for $j\in\{2,3,4\}$, these missing values are presented in shaded
    cells. To compute these values that $\langle\chi_i,\chi_j\rangle=0$ whenever $i\ne j$. The calculations
    are left as an exercise. 
	
	To check our character table we can use the computer. 
\begin{lstlisting}
gap> Q8 := QuaternionGroup(8);;
gap> Display(CharacterTable(Q8));
CT6

     2  3  2  2  3  2

       1a 4a 4b 2a 4c
    2P 1a 2a 2a 1a 2a
    3P 1a 4a 4b 2a 4c

X.1     1  1  1  1  1
X.2     1 -1 -1  1  1
X.3     1 -1  1  1 -1
X.4     1  1 -1  1 -1
X.5     2  .  . -2  .
\end{lstlisting}
\end{example}

\begin{exercise}
    Compute the character table of the dihedral group of eight elements. 
\end{exercise}
