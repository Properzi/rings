\section*{Checkpoints}
\subsection*{Lecture 1}

\begin{enumerate}
    \item What is a group algebra $\C[G]$? When is it commutative?
    \item Define a complex representation of a group $G$. 
    \item What does it mean for two representations to be equivalent?
    \item Define an invariant subspace and an irreducible representation.
    \item Every representation of a finite group is unitary.
    \item If a representation is unitary, then it is either irreducible or
    \item Explain why Weyl’s trick implies that every representation of a
            finite group is completely reducible.
    \item Give an example of two equivalent representations and describe
            the change of basis matrix.
\end{enumerate}

\subsection*{lecture 2}
\begin{enumerate}
        \item State Maschke’s theorem and explain its significance for representations of finite groups.
        \item What does Schur’s lemma say about invariant maps between irreducible representations?
        \item Define the character of a representation and list two of its key properties.
        \item What is the statement of the Artin–Wedderburn theorem for $\mathbb{C}[G]$?
        \item Explain why Schur’s orthogonality relations are fundamental in character theory.
        \item How does the Artin–Wedderburn theorem lead to the formula $\sum_{i=1}^r n_i^2 = |G|$?
        \item Why does $\langle \chi_\rho, \chi_\rho \rangle = 1$ characterize irreducibility?
\end{enumerate}
