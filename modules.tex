\chapter{Modules, submodules, homomorphisms}

\begin{definition}
    Let $R$ be a ring. A \textbf{module} (over $R$) is an abelian group
    $M$ with a map $R\times M\to M$, $(x,m)\mapsto x\cdot m$, such that
    the following conditions hold:
    \begin{enumerate}
        \item $(r_1+r_2)\cdot m=r_1\cdot m+r_2\cdot m$ for all $r_1,r_2\in R$ y $m\in M$.
		\item $r\cdot (m_1+m_2)=r\cdot m_1+r\cdot m_2$ for all $r\in R$ y $m_1,m_2\in M$.
		\item $r_1\cdot (r_2\cdot m)=(r_1r_2)\cdot m$ for all $r_1,r_2\in R$ y $m\in M$.
		\item $1\cdot m=m$ for all $m\in M$.	
    \end{enumerate}
\end{definition}

Our definition is that of left module. Similarly one defines right modules. We will always
consider left modules, so they will be referred simply as modules.

\begin{example}
A module over a field is a vector space. 
\end{example}

\begin{example}
Every abelian group is a modulo over $\Z$.	
\end{example}

\begin{example}
Let $R$ be a ring. Then $R$ is a module (over $R$) with $x\cdot m=xm$. 
This is the \textbf{(left) regular representation} of $R$ and it usually 
be denoted by $\prescript{}{R}R$. 
\end{example}

\begin{example}
If $R$ is a ring, then $R^n=\{(x_1,\dots,x_n):x_1,\dots,x_n\in R\}$ 
is a module (over $R$) with  
$r\cdot (x_1,\dots,x_n)=(rx_1,\dots,rx_n)$. 
\end{example}

\begin{example}
If $R$ is a ring, then $M_{m,n}(R)$ is a module (over $R$) with usual matrix operations. 
\end{example}

%%El siguiente ejemplo explica por qué es útil 
%%pedir que un morfismo $f$ 
%%de anillos cumpla con la condición $f(1)=1$. 
%%
%%\begin{example}
%%\label{exa:f(1)=1}
%%Si $f\colon R\to S$ es un morfismo de anillos y $M$ es un $S$-módulo con la acción $(s,m)\mapsto sm$, entonces
%%$M$ es un $R$-módulo con $r\cdot m=f(r)m$ para $r\in R$ y $m\in M$. En efecto,
%%\begin{align*}
%%&1\cdot m=f(1)m=1m=m,\\
%%&r_1\cdot (r_2\cdot m)=f(r_1)(r_2\cdot m)=f(r_1)(f(r_2)m)=(f(r_1)f(r_2))m=f(r_1r_2)m
%%\end{align*}
%%para todo $r_1,r_2\in R$ y $m\in M$.	  	
%%\end{example}
%
\begin{example}
Let $R=\R[X]$, $T\colon\R^n\to\R^n$ be a linear map and $M=\R^n$. Then
$M$ is a module (over $R$) with  
\[
\left(\sum_{i=0}^na_iX^i\right)\cdot v=\sum_{i=0}^na_iT^i(v).
\]	
\end{example}

\begin{example}
If $\{M_i|i\in I\}$ is a family of modules, then  	
\[
\prod_{i\in I}M_i=\{(m_i)_{i\in I}:m_i\in M_i\text{ for all $i\in I$}\}
\]
is a module with 
$x\cdot (m_i)_{i\in I}=(x\cdot m_i)_{i\in I}$, 
where $(m_i)_{i\in I}$ denotes the map $I\to M_i$, $i\mapsto m_i$.
This module is the \textbf{direct product} of the family $\{M_i:i\in I\}$.
\end{example}
%
\begin{example}
If $\{M_i|i\in I\}$ is family of modules, then   	
\[
\bigoplus_{i\in I}M_i=\{(m_i)_{i\in I}:m_i\in M_i\text{ for all $i\in I$ and $m_i=0$ except finitely many $i\in I$}\}
\]
is a module with 
$x\cdot (m_i)_{i\in I}=(x\cdot m_i)_{i\in I}$. 
This module is the \textbf{direct sum} of the family $\{M_i:i\in I\}$. 
\end{example}
%
%\begin{exercise}
%Si $M$ es un módulo, entonces
%\begin{enumerate}
%\item $0\cdot m=0$ para todo $m\in M$,
%\item $x\cdot 0=0$ para todo $x\in R$ y además
%\item $-m=(-1)\cdot m$ para todo $m\in M$. 	
%\end{enumerate}
%\end{exercise}
%
%\begin{example}
%$M=\Z/6$ es un $\Z$-módulo tal que $3\cdot 2=0$ pero $3\ne 0$ (en $\Z$) y $2\ne 0$ (en $\Z/6$).  
%\end{example}
%
\begin{definition}
	Let $M$ be a module. A subset $N$ of $M$ is a \textbf{submodule} of $M$ if 
	$(N,+)$ is a subgroup of $(M,+)$ and 
	$x\cdot n\in N$ for all $x\in R$ and $n\in N$. 
\end{definition}

Clearly, if $M$ is a module, then $\{0\}$ and $M$ are submodules of $M$. 

\begin{example}
Let $R$ be a field and $M$ be a module over $R$. Then
$N$ is a submodule of $M$ if and only if $N$ is a subspace of $M$. 
\end{example}

\begin{example}
Let $R=\Z$ and $M$ be a module (over $R$). Then
$N$ is a submodule of $M$ if and only if $N$ is a subgroup of $M$
\end{example}

\begin{example}
If $M=\prescript{}{R}R$, then a subset $N\subseteq M$ is a submodule
of $M$ if and only if $N$ is a left ideal of $R$. 
\end{example}

\begin{example}
If $V$ is a vector space and $T\colon V\to V$ is a linear map, then
$V$ is a module (over $K[X]$) with  
\[
\left(\sum_{i=0}^na_iX^i\right)\cdot v=\sum_{i=0}^na_iT^i(v).
\]
A submodule $W$ is a subspace $W$ 
of $V$ such that $T(W)\subseteq W$. 
\end{example}
