\chapter{Modules, submodules, homomorphisms}

\begin{definition}
    Let $R$ be a ring. A \textbf{module} (over $R$) is an abelian group
    $M$ with a map $R\times M\to M$, $(x,m)\mapsto x\cdot m$, such that
    the following conditions hold:
    \begin{enumerate}
        \item $(r_1+r_2)\cdot m=r_1\cdot m+r_2\cdot m$ for all $r_1,r_2\in R$ y $m\in M$.
		\item $r\cdot (m_1+m_2)=r\cdot m_1+r\cdot m_2$ for all $r\in R$ y $m_1,m_2\in M$.
		\item $r_1\cdot (r_2\cdot m)=(r_1r_2)\cdot m$ for all $r_1,r_2\in R$ y $m\in M$.
		\item $1\cdot m=m$ for all $m\in M$.	
    \end{enumerate}
\end{definition}

Our definition is that of left module. Similarly one defines right modules. We will always
consider left modules, so they will be referred simply as modules.

\begin{example}
A module over a field is a vector space. 
\end{example}

\begin{example}
Every abelian group is a modulo over $\Z$.	
\end{example}

\begin{example}
Let $R$ be a ring. Then $R$ is a module (over $R$) with $x\cdot m=xm$. 
This is the \textbf{(left) regular representation} of $R$ and it usually 
be denoted by $\prescript{}{R}R$. 
\end{example}

\begin{example}
If $R$ is a ring, then $R^n=\{(x_1,\dots,x_n):x_1,\dots,x_n\in R\}$ 
is a module (over $R$) with  
$r\cdot (x_1,\dots,x_n)=(rx_1,\dots,rx_n)$. 
\end{example}

\begin{example}
If $R$ is a ring, then $M_{m,n}(R)$ is a module (over $R$) with usual matrix operations. 
\end{example}

Students usually ask why in the definition of a ring homomorphism one needs
the condition $1\mapsto 1$. The following example provides a good explanation. 

\begin{example}
%%\label{exa:f(1)=1}
If $f\colon R\to S$ is a ring homomorphism and $M$ is a module (over $S$) wih 
$(s,m)\mapsto sm$, then 
$M$ is also a module (over $R$) with $r\cdot m=f(r)m$ for all $r\in R$ and $m\in M$. In fact, 
\begin{align*}
&1\cdot m=f(1)m=1m=m,\\
&r_1\cdot (r_2\cdot m)=f(r_1)(r_2\cdot m)=f(r_1)(f(r_2)m)=(f(r_1)f(r_2))m=f(r_1r_2)m
\end{align*}
for all $r_1,r_2\in R$ and $m\in M$.	  	
\end{example}
%
\begin{example}
Let $R=\R[X]$ and $T\colon\R^n\to\R^n$ be a linear map. Then $M=\R^n$ with 
\[
\left(\sum_{i=0}^na_iX^i\right)\cdot v=\sum_{i=0}^na_iT^i(v)
\]	
is a module (over $R$).   
\end{example}

\begin{example}
If $\{M_i|i\in I\}$ is a family of modules, then  	
\[
\prod_{i\in I}M_i=\{(m_i)_{i\in I}:m_i\in M_i\text{ for all $i\in I$}\}
\]
is a module with 
$x\cdot (m_i)_{i\in I}=(x\cdot m_i)_{i\in I}$, 
where $(m_i)_{i\in I}$ denotes the map $I\to M_i$, $i\mapsto m_i$.
This module is the \textbf{direct product} of the family $\{M_i:i\in I\}$.
\end{example}
%
\begin{example}
If $\{M_i|i\in I\}$ is family of modules, then   	
\[
\bigoplus_{i\in I}M_i=\{(m_i)_{i\in I}:m_i\in M_i\text{ for all $i\in I$ and $m_i=0$ except finitely many $i\in I$}\}
\]
is a module with 
$x\cdot (m_i)_{i\in I}=(x\cdot m_i)_{i\in I}$. 
This module is the \textbf{direct sum} of the family $\{M_i:i\in I\}$. 
\end{example}
%
%\begin{exercise}
If $M$ is a module, then $0\cdot m=0$ and $-m=(-1)\cdot m$ for all $m\in M$ and 
$x\cdot 0=0$ for all $x\in R$. 
%
\begin{example}
Let $M=\Z/6$ as a module (over $\Z$). Note that 
$3\cdot 2=0$ but $3\ne 0$ (in $\Z$) and $2\ne 0$ (in $\Z/6$).  
\end{example}
%
\begin{definition}
	Let $M$ be a module. A subset $N$ of $M$ is a \textbf{submodule} of $M$ if 
	$(N,+)$ is a subgroup of $(M,+)$ and 
	$x\cdot n\in N$ for all $x\in R$ and $n\in N$. 
\end{definition}

Clearly, if $M$ is a module, then $\{0\}$ and $M$ are submodules of $M$. 

\begin{example}
Let $R$ be a field and $M$ be a module over $R$. Then
$N$ is a submodule of $M$ if and only if $N$ is a subspace of $M$. 
\end{example}

\begin{example}
Let $R=\Z$ and $M$ be a module (over $R$). Then
$N$ is a submodule of $M$ if and only if $N$ is a subgroup of $M$
\end{example}

\begin{example}
If $M=\prescript{}{R}R$, then a subset $N\subseteq M$ is a submodule
of $M$ if and only if $N$ is a left ideal of $R$. 
\end{example}

\begin{example}
If $V$ is a vector space and $T\colon V\to V$ is a linear map, then
$V$ is a module (over $\R[X]$) with  
\[
\left(\sum_{i=0}^na_iX^i\right)\cdot v=\sum_{i=0}^na_iT^i(v).
\]
A submodule is a subspace $W$ 
of $V$ such that $T(W)\subseteq W$. 
\end{example}

Clearly, a subset $N$ of $M$ is a submodule if and only 
if $r_1n_1+r_2n_2\in N$ for all
$r_1,r_2\in R$ and $n_1,n_2\in N$. 	

\begin{exercise}
If $N$ and $N_1$ are submodules of $M$, then 
\[
N+N_1=\{n+n_1:n\in N,\,n_1\in N_1\}
\]
is a submodule of $M$.
\end{exercise}

\begin{definition}
Let $M$ and $N$ be modules over $R$. 
A map $f\colon M\to N$ is a \textbf{module homomorphism} if $f(x+y)=f(x)+f(y)$ and 
$f(r\cdot x)=r\cdot f(x)$ for all $x,y\in M$ and $r\in R$. 
\end{definition}

We denote by $\Hom_R(M,N)$ the set of module homomorphisms $M\to N$. 

\begin{exercise}
Let $f\in\Hom_R(M,N)$.  
\begin{enumerate}
\item If $V$ is a submodule of $M$, then $f(V)$ is a submodule of $N$.
\item If $W$ is a submodule of $N$, then $f^{-1}(W)$ is a submodule of $M$.
\end{enumerate}
\end{exercise}

If $f\in\Hom_R(M,N)$, the \textbf{kernel} of $f$ is the submodule  
\[
\ker f=f^{-1}(\{0\})=\{m\in M:f(m)=0\}
\]
of $M$. We say that $f$ is a \textbf{monomorphism} (resp. \textbf{epimorphism}) 
if $f$ is injective (resp. surjective). Moreover, $f$ is an \textbf{isomorphism} 
if $f$ is
bijective. 

\begin{exercise}
Let $f\in\Hom_R(M,N)$. Prove that the following statements are equivalent:
\begin{enumerate}
\item $f$ is a monomorphism.
\item $\ker f=\{0\}$.
\item For every module $V$ and every $g,h\in\Hom_R(V,M)$, $f\circ g=f\circ h\implies g=h$.
\item For every module $V$ and every $g\in\Hom(V,M)$, $f\circ g=0\implies g=0$.
\end{enumerate}
\end{exercise}


\begin{example}
	Let $R=
		\begin{pmatrix}
			\R & 0\\
			0 & \R
		\end{pmatrix}$. 
	We claim that 
	$\begin{pmatrix}
			\R\\
			0
		\end{pmatrix}
		\not\simeq\begin{pmatrix}
			0\\
			\R
		\end{pmatrix}$
	as modules over $R$, where the module structure is given by the usual matrix multiplication. 
	Assume that they are isomorphic. 
	Let $f\colon\begin{pmatrix}
			0\\
			\R
		\end{pmatrix}
		\to\begin{pmatrix}
			\R\\
			0
		\end{pmatrix}$  
	be an isomorphism of modules and let  
	$x_0\in\R\setminus\{0\}$ be such that 
	$f\begin{pmatrix}0\\1\end{pmatrix}=\begin{pmatrix}x_0\\0\end{pmatrix}$. Thus 
	\[
	f\begin{pmatrix}
	0\\
	1\end{pmatrix}
	=f\left(\begin{pmatrix}
	0&0\\
	0&1\end{pmatrix}
	\cdot 
	\begin{pmatrix}
	0\\
	1
	\end{pmatrix}\right)
	=\begin{pmatrix}
	0&0\\
	0&1\end{pmatrix}\cdot f\begin{pmatrix}0\\1\end{pmatrix}
	=\begin{pmatrix}
	0&0\\
	0&1
	\end{pmatrix}
	\cdot 
	\begin{pmatrix}		
	x_0\\
	0
	\end{pmatrix}
	=\begin{pmatrix}
	0\\
	0
	\end{pmatrix},
	\]	
	a contradiction, as $f$ is injective.    
\end{example}

