\lecture{}

\begin{definition}
Let $M$ be a module and $X$ be a subset of $M$. The submodule
of $M$ generated by $X$ is defined as
\[
(X)=\bigcap\{N:N\text{ is a submodule of $M$ that contains $X$}\},
\]
the smallest submodule of $M$ containing $X$. 
\end{definition}

One can prove that  
\[
(X)=\left\{\sum_{i=1}^mr_i\cdot x_i:m\in\Z_{\geq0},\,r_1,\dots,r_m\in R,\,x_1,\dots,x_m\in X\right\}
\]

\begin{definition}
A module $M$ is \textbf{finitely-generated} if $M=(X)$ for some finite subset $X$ of $M$.
\end{definition}

If $X=\{x_1,\dots,x_m\}$ one writes $(X)=(x_1,\dots,x_n)$.
For example, $\Z=(1)=(2,3)$ and $\Z\ne (2)$.

\begin{exercise}
    Let $R$ be the ring of continuous maps $[0,1]\to\R$ with point-wise operations and 
    $M=\prescript{}{R}{R}$. Prove that
    $N=\{f\in R:f(x)\ne0\text{ for finitely many $x$}\}$ is not finitely-generated. 
\end{exercise}

\begin{exercise}
    Let $G=\{g_1,\dots,g_n\}$ be a finite group. Prove that if $M=\C[G]$ is finitely-generated, then
    $M$ is a finite-dimensional complex vector space. 
\end{exercise}

If $f\in\Hom_R(M,N)$ and $M$ is finitely-generated, then 
$f(M)$ is finitely-generated. 

\begin{proposition}
    Let $R$ be a ring and $M$ be a module over $R$. 
    Then $M$ is finitely-generated if and only if $M$ is isomorphic to a quotient of $R^k$ for some $k$.
\end{proposition}

\begin{proof}
    Assume first that $M=(m_1,\dots,m_k)$ is finitely-generated. A routine calculation shows that
    the map
    \[
    \varphi\colon R^k\to M,\quad
    (r_1,\dots,r_k)\mapsto \sum_{j=1}^kr_j\cdot m_j,
    \]
    is a surjective module homomorphism. The first isomorphism theorem implies that 
    $R^k/\ker\varphi\simeq\varphi(R^k)=M$. 
    
    Now assume that there exists a 
    surjective module homomorphism $\varphi\colon R^k\to M$. Since 
    $R^k=(e_1,\dots,e_k)$, where 
    \[
    (e_i)_j=\begin{cases}
    1 & \text{if $i=j$},\\
    0 & \text{if $i\ne j$},
    \end{cases}
    \]
    it follows that $\{\varphi(e_1),\dots,\varphi(e_k)\}$ generates $\varphi(R^k)=M$. Indeed, 
    if $m\in M$, we write $m=\varphi(r_1,\dots,r_k)$ for some $(r_1,\dots,r_k)\in R^k$ 
    and hence 
    \[
    m=\varphi(r_1,\dots,r_k)=\varphi\left(\sum_{i=1}^k r_i\cdot e_i\right)
    =\sum_{i=1}^kr_i\cdot\varphi(e_i).\qedhere
    \]
\end{proof}

\begin{definition}
    Let $R$ be a ring, $M$ be a module over $R$ and $X$ be a subset of $M$. We say that
    $X$ is \textbf{linearly independent} if for each $k\in\Z_{>0}$, $r_1,\dots,r_k\in R$
    and $m_1,\dots,m_k\in X$ such that $\sum_{i=1}^kr_i\cdot m_i=0$, then 
    $r_1=\cdots=r_k=0$. 
\end{definition}

In any ring, the set $\{1\}$ is linearly independent. 

\begin{examples}\
\begin{enumerate}
    \item $\{2,3\}$ is a linear dependent subset of $\Z$.
    \item $\{2\}$ is a linearly dependent subset of $\Z/4$.
    \item Let $R=\Z$, $M=\Q$ and $x\in M\setminus\{0\}$. 
        Then $\{x\}$ is linearly independent subset of $M$. Is $y\in M\setminus\{x\}$, then
        $\{x,y\}$ is linearly dependent.  
\end{enumerate}    
\end{examples}

\begin{examples}
Let $R=M_2(\R)$ and $M=\begin{pmatrix}
        0&\R\\
        0&\R 
    \end{pmatrix}$. Then $\left\{\begin{pmatrix}0&1\\0&0\end{pmatrix}\right\}$ is a minimal generating 
    set and it is linearly independent. 
\end{examples}

\begin{exercise}
    Let $f\in\Hom_R(M,N)$ and $X$ be a subset of $M$. 
    \begin{enumerate}
        \item If $X$ is linearly dependent, then $f(X)$ is linearly dependent.
        \item If $X$ is linearly independent and $f$ is injective, then $f(X)$ is linearly independent. 
        \item If $M=(X)$ and $f$ is surjective, then $N=(f(X))$. 
    \end{enumerate}
\end{exercise}

\begin{definition}
    Let $M$ be a module and $B$ be a subset of $M$. Then $B$ is a \textbf{basis} of $M$ if
    $B$ is linearly independent and $M=(B)$. A module $M$ is said to be \textbf{free} if it admits a basis.   
\end{definition}

As a consequence of Zorn's lemma, 
vector spaces are free. 

\begin{examples}\
\begin{enumerate}
    \item If $R$ is a ring, then $\{1\}$ is a basis of $\prescript{}{R}{R}$, so $\prescript{}{R}{R}$ is free.
    \item If $R$ is a ring, then $R^n$ is free as a module over $R$. 
\end{enumerate}
\end{examples}

\begin{exercise}
    Prove that $\Q$ is not free as a module over $\Z$. 
\end{exercise}

\begin{exercise}
    Let $R$ be a division ring and $M$ be a non-zero and finitely generated module over $R$.
    Prove the following facts:
    \begin{enumerate}
        \item Every finite set of generators contains a basis.
        \item Every linearly independent set can be extended into a basis.
        \item Any two bases contain the same number of elements.
    \end{enumerate}
\end{exercise}

\begin{example}
    $\R[X]$ is a free module (over $\R$) with basis $\{1,X,X^2,\dots\}$. 
\end{example}

\begin{exercise}
    Prove that $\{(a,b),(c,d)\}$ is a basis of $\Z\times\Z$ (as a module over $\Z$) if and only if
    $ad-bc\in\{-1,1\}$. 
\end{exercise}