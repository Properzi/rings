\lecture{}

\topic{Noetherian modules}

\begin{definition}
Let $M$ be a module and $X$ be a subset of $M$. The submodule
of $M$ generated by $X$ is defined as
\[
(X)=\bigcap\{N:N\text{ is a submodule of $M$ that contains $X$}\},
\]
the smallest submodule of $M$ containing $X$. 
\end{definition}

One can prove that  
\[
(X)=\left\{\sum_{i=1}^mr_i\cdot x_i:m\in\Z_{\geq0},\,r_1,\dots,r_m\in R,\,x_1,\dots,x_m\in X\right\}
\]

\begin{definition}
A module $M$ is \textbf{finitely-generated} if $M=(X)$ for some finite subset $X$ of $M$.
\end{definition}

If $X=\{x_1,\dots,x_m\}$ one writes $(X)=(x_1,\dots,x_n)$.
For example, $\Z=(1)=(2,3)$ and $\Z\ne (2)$.

\begin{exercise}
    Let $R$ be the ring of continuous maps $[0,1]\to\R$ with point-wise operations and 
    $M=\prescript{}{R}{R}$. Prove that
    $N=\{f\in R:f(x)\ne0\text{ for finitely many $x$}\}$ is not finitely-generated. 
\end{exercise}

\begin{exercise}
    Let $G=\{g_1,\dots,g_n\}$ be a finite group. Prove that if $M=\C[G]$ is finitely-generated, then
    $M$ is a finite-dimensional complex vector space. 
\end{exercise}

If $f\in\Hom_R(M,N)$ and $M$ is finitely-generated, then 
$f(M)$ is finitely-generated. We can prove something strong. For that purpose
we need exact sequences. 
A sequence 
	\begin{equation}
	\label{eq:exacta1}	
		\xymatrix{
        0\ar[r]
        & M
        \ar[r]^f
        & N
        \ar[r]^g
        & T\ar[r]
        & 0,
        }
  	\end{equation}
of modules and homomorphism is said to be \textbf{exact} 
if $f$ is injective, $g$ is surjective and $f(M)=\ker g$. For example,
the sequence
\[
		\xymatrix{
        0\ar[r]
        & M
        \ar[r]^f
        & M\oplus N
        \ar[r]^g
        & N\ar[r]
        & 0,
        }
\]
where $f(m)=(m,0)$ and $g(m,n)=n$ is exact.

\begin{exercise}
	Si  
	\[  
		\xymatrix{
        0\ar[r]
        & M
        \ar[r]^f
        & N
        \ar[r]^g
        & T\ar[r]
        & 0	
        }
     \]
     be an exact sequence. Prove the following statements: 
     \begin{enumerate}
     \item If $N$ if finitely generated, then $T$ is finitely generated. 
     \item If $M$ and $T$ are finitely generated, then $N$ is finitely generated. 	
     \end{enumerate}
\end{exercise}

\begin{proposition}
    Let $R$ be a ring and $M$ be a module over $R$. 
    Then $M$ is finitely-generated if and only 
    if $M$ is isomorphic to a quotient of $R^k$ for some $k$.
\end{proposition}

\begin{proof}
    Assume first that $M=(m_1,\dots,m_k)$ is finitely-generated. A routine calculation shows that
    the map
    \[
    \varphi\colon R^k\to M,\quad
    (r_1,\dots,r_k)\mapsto \sum_{j=1}^kr_j\cdot m_j,
    \]
    is a surjective module homomorphism. The first isomorphism theorem implies that 
    $R^k/\ker\varphi\simeq\varphi(R^k)=M$. 
    
    Now assume that there exists a 
    surjective module homomorphism $\varphi\colon R^k\to M$. Since 
    $R^k=(e_1,\dots,e_k)$, where 
    \[
    (e_i)_j=\begin{cases}
    1 & \text{if $i=j$},\\
    0 & \text{if $i\ne j$},
    \end{cases}
    \]
    it follows that $\{\varphi(e_1),\dots,\varphi(e_k)\}$ generates $\varphi(R^k)=M$. Indeed, 
    if $m\in M$, we write $m=\varphi(r_1,\dots,r_k)$ for some $(r_1,\dots,r_k)\in R^k$ 
    and hence 
    \[
    m=\varphi(r_1,\dots,r_k)=\varphi\left(\sum_{i=1}^k r_i\cdot e_i\right)
    =\sum_{i=1}^kr_i\cdot\varphi(e_i).\qedhere
    \]
\end{proof}

\begin{definition}
A module is \textbf{noetherian} if every sequence $M_1\subseteq M_2\subseteq\cdots$ of submodules of $M$ 
stabilizes, that is there exists $n$ such that $M_k=M_{n+k}$ for all $k$. 	
\end{definition}

\begin{proposition}
Let $M$ be a module. The following statements are equivalent:
\begin{enumerate}
\item $M$ is noetherian.
\item Submodules of $M$ are finitely generated.
\item Each non-empty family of submodules of $M$ has a maximal element (with respect to the inclusion).	
\end{enumerate}
\end{proposition}

\begin{proof}
	We first prove $2)\implies1)$. If $S_1\subseteq S_2\subseteq\cdots$ is a sequence of submodules of $M$, 
	it follows that $S=\cup_{i\geq 1}S_i$ is a submodule of $M$. Since $S$ is finitely generated, 
	$S=(x_1,\dots,x_n)$ for some $x_1,\dots,x_n\in M$. It follows that 
	$x_1,\dots,x_n\in S_N$ for some positive integer $N$. Thus 
	$S\subseteq S_N$ and hence $S_N=S_{N+k}$ for all $k$. 
	
	We now prove $1)\implies3)$. Let $F$ be a non-empty family of submodules of $M$ with no maximal elements. 
	Let $S_1\in F$. Since $S_1$ is not maximal, there exists $S_2\in F$ such that $S_1\subsetneq S_2$. 
	Having constructed with this method the submodules $S_1\subsetneq\dots\subsetneq S_k$, since $S_k$ is not
	maximal, there exists $S_{k+1}\in F$ such that $S_k\subsetneq S_{k+1}$. 
	This means that the sequence 
	$S_1\subsetneq S_2\subsetneq\cdots$ does not stabilize. 
	
	We finally prove $3)\implies2)$. Let $S$ be a submodule of $M$ and let 
	\[
	F=\{T\subseteq S:T\text{ finitely generated submodule of $M$}\}.
	\]
	By assumption, $F$ has a maximal element $N$. Thus
	$N$ is a finitely generated submodule of $M$ such that $N\subseteq S$. We may assume that 
	$N=(n_1,\dots,n_k)$. If $N=S$, then, in particular, $S$ is finitely generated. Suppose that
	$N\ne S$ and let $x\in S\setminus N$. It follows that  
	$N\subseteq (n_1,\dots,n_k,x)\subseteq S$. Since 
	$(n_1,\dots,n_k,x)\in F$ and $N$ is maximal, it follows that 
	$N=(n_1,\dots,n_k,x)$, a contradiction
	to $x\not\in N$. 
\end{proof}

\begin{exercise}
\label{xca:exacta_noetheriano}
	Let   
	\[  
		\xymatrix{
        0\ar[r]
        & M
        \ar[r]^f
        & N
        \ar[r]^g
        & T\ar[r]
        & 0	
        }
     \]
     be an exact sequence. Prove the following statements:
     \begin{enumerate}
     	\item If $N$ is noetherian, then $M$ and $T$ are noetherian.
     	\item If $M$ and $T$ are noetherian, then $N$ is noetherian.
     \end{enumerate}	
\end{exercise}

\begin{exercise}
\label{xca:regular_noetheriano}
A ring $R$ is noetherian if and only if $\prescript{}{R}R$ is noetherian.	
\end{exercise}

\begin{exercise}
\label{xca:directa_noetheriano}
If $M_1,\dots,M_n$ are noetherian, then $M_1\oplus\cdots\oplus M_n$ is noetherian. 	
\end{exercise}

The previous exercise cannot be extended to infinitely many modules. Why?
%For example, $\Z^{\Z}$ is not noetherian, as it is not finitely generated. 

\begin{proposition}
If $R$ is noetherian and $M$ is a finitely generated module, then $M$ is noetherian. 
\end{proposition}

\begin{proof}
    Assume that $M=(m_1,\dots,m_k)$. There exists a surjective homomorphism  
$R^k\to N$, $(r_1,\dots,r_k)\mapsto \sum_{i=1}^k r_i\cdot m_i$, where
$R^k=\oplus_{i=1}^k R$. Since $R$ is noetherian, $R^k$ is noetherian. Thus $M$ is noetherian.	
\end{proof}

\topic{Free modules}

\begin{definition}
    Let $R$ be a ring, $M$ be a module over $R$ and $X$ be a subset of $M$. We say that
    $X$ is \textbf{linearly independent} if for each $k\in\Z_{>0}$, $r_1,\dots,r_k\in R$
    and $m_1,\dots,m_k\in X$ such that $\sum_{i=1}^kr_i\cdot m_i=0$, then 
    $r_1=\cdots=r_k=0$. 
\end{definition}

In any ring, the set $\{1\}$ is linearly independent. 

\begin{examples}\
\begin{enumerate}
    \item $\{2,3\}$ is a linear dependent subset of $\Z$.
    \item $\{2\}$ is a linearly dependent subset of $\Z/4$.
    \item Let $R=\Z$, $M=\Q$ and $x\in M\setminus\{0\}$. 
        Then $\{x\}$ is linearly independent subset of $M$. Is $y\in M\setminus\{x\}$, then
        $\{x,y\}$ is linearly dependent.  
\end{enumerate}    
\end{examples}

\begin{examples}
Let $R=M_2(\R)$ and $M=\begin{pmatrix}
        0&\R\\
        0&\R 
    \end{pmatrix}$. Then $\left\{\begin{pmatrix}0&1\\0&0\end{pmatrix}\right\}$ is a minimal generating 
    set and it is linearly independent. 
\end{examples}

\begin{exercise}
\label{xca:LI}
    Let $f\in\Hom_R(M,N)$ and $X$ be a subset of $M$. 
    \begin{enumerate}
        \item If $X$ is linearly dependent, then $f(X)$ is linearly dependent.
        \item If $X$ is linearly independent and $f$ is injective, then $f(X)$ is linearly independent. 
        \item If $M=(X)$ and $f$ is surjective, then $N=(f(X))$. 
    \end{enumerate}
\end{exercise}

\begin{definition}
    Let $M$ be a module and $B$ be a subset of $M$. Then $B$ is a \textbf{basis} of $M$ if
    $B$ is linearly independent and $M=(B)$. A module $M$ is said to be \textbf{free} if it admits a basis.   
\end{definition}

As a consequence of Zorn's lemma, 
vector spaces are free. 

\begin{examples}\
\begin{enumerate}
    \item If $R$ is a ring, then $\{1\}$ is a basis of $\prescript{}{R}{R}$, so $\prescript{}{R}{R}$ is free.
    \item If $R$ is a ring, then $R^n$ is free as a module over $R$. 
\end{enumerate}
\end{examples}

\begin{exercise}
    Prove that $\Q$ is not free as a module over $\Z$. 
\end{exercise}

\begin{exercise}
\label{xca:linear_algebra}
    Let $R$ be a division ring and $M$ be a non-zero and finitely generated module over $R$.
    Prove the following facts:
    \begin{enumerate}
        \item Every finite set of generators contains a basis.
        \item Every linearly independent set can be extended into a basis.
        \item Any two bases contain the same number of elements.
    \end{enumerate}
\end{exercise}

This means that modules over division rings have dimension.

\begin{example}
    $\R[X]$ is a free module (over $\R$) with basis $\{1,X,X^2,\dots\}$. 
\end{example}

\begin{exercise}
    Prove that $\{(a,b),(c,d)\}$ is a basis of $\Z\times\Z$ 
    (as a module over $\Z$) if and only if
    $ad-bc\in\{-1,1\}$. 
\end{exercise}

\begin{example}
If $u\in\mathcal{U}(R)$, then $\{u\}$ is a basis of $\prescript{}{R}R$. Converely, if $R$ is an integral domain and 
$\{z\}$ is a basis of $\prescript{}{R}R$, then $z\in\mathcal{U}(R)$. Since 
$1=yz$ for some $y\in R$, it follows that $zy=1$, as  
\[
(zy-1)z=z(yz)-z=z1-z=z-z=0.
\]	
\end{example}

\begin{example}
If $I$ is a non-empty set, the module $R^{(I)}$ is free with basis 
$\{e_i:i\in I\}$, where 
\[
(e_i)_j=\begin{cases}
	1 & \text{if $i=j$},\\
	0 & \text{if $i\ne j$.}
	\end{cases}	
\]
\end{example}

\begin{example}
If $R=M_2(\Z)$, then $M=\prescript{}{R}R$ is free with bassi $\left\{\begin{pmatrix}
    1&0\\0&1\end{pmatrix}\right\}$. The submodule 
$N=\begin{pmatrix}\Z&0\\\Z&0\end{pmatrix}$ does not admits a basis as a module over $R$.
\end{example}

In modules, the size of a basis is not an invariant of the module.  

\begin{example}
Let $V$ be the complex vector space with infinite basis $e_0,e_1,e_2,\dots$ and let 
$R=\End(V)$ with the ring structucture given by 
\[
(f+g)(v)=f(v)+g(v),\quad
(fg)(v)=f(g(v))
\]
for $f,g\in R$ and $v\in V$. 

Let $M=\prescript{}{R}R$. 
The set $\{\id\}$ is a basis for $R$.  
We claim that $M$ admits a basis with two elements. If
$r,s\in R$ are such that 
\begin{align*}
&r(e_{2n})=e_n, && r(e_{2n+1})=0,\\
&s(e_{2n})=0,&& s(e_{2n+1})=e_{2n},
\end{align*}
then $\{r,s\}$ is a basis of $M$. 
If $f\in R$, then
$f=\alpha r+\beta s$, where $\alpha\colon V\to V$, $e_n\mapsto f(e_{2n})$ for all $n$, and
$\beta\in V\to V$, $e_n\mapsto f(e_{2n+1})$ for all $n$. In fact,
\begin{align*}
&(\alpha r+\beta s)(e_{2n})=\alpha(r(e_{2n}))+\beta(s(e_{2n}))=f(e_{2n}),\\
&(\alpha r+\beta s)(e_{2n+1})=\alpha(r(e_{2n+1}))+\beta(s(e_{2n+1}))=f(e_{2n+1}).
\end{align*}
Moreover, $\{r,s\}$ is linearly independent. Indeed, if $\alpha r+\beta s=0$ for some $\alpha,\beta\in R$,  
Evaluation on $e_{2n}$ yields $\alpha=0$ and evaluation on $e_{2n+1}$ yields $\beta=0$.   
\end{example} 

\begin{example}
Si $M$ es un módulo libre con base $X$ y $N$ es un módulo libre con base $Y$, entonces
$M\oplus N$ es un módulo libre con base 
\[
\{(x,0):x\in X\}\cup \{(0,y):y\in Y\}.
\]	
\end{example}

\begin{exercise}
Sea $R$ es un anillo conmutativo. Si $M$ y $N$ son libres y finitamente generados, entonces
$\Hom_R(M,N)$ es libre y finitamente generado.	
\end{exercise}

% Como $R$ es conmutativo, $\Hom_R(M,N)$ es un $R$-módulo. Si $\{m_1,\dots,m_k\}$ es base de $M$ 
% y $\{n_1,\dots,n_l\}$ es base de $N$, entonces definimos para cada $i\in\{1,\dots,k\}$ y $j\in\{1,\dots,l\}$ 
% definimos $f_{ij}$ 
% \[
% f_{ij}(m_k)=\begin{cases}
% n_j & \text{si $k=i$},\\
% 0 & \text{si $k\ne i$}.
% \end{cases}
% \]
% Entonces $\{i_ij}$ es base de $\Hom_R(M,N)$. 
Some properties of free modules:

\begin{proposition}
Si $M$ es libre, entonces existe un subconjunto $\{m_i:i\in I\}$ de $M$ tal que 
para cada $m\in M$ existen únicos $r_i\in R$, $i\in I$, 
donde $r_i=0$ salvo finitos $i\in I$ 
tales que $m=\sum r_i\cdot m_i$. 
\end{proposition}

\begin{proof}
Como $M$ es libre, existe una base $\{m_i:i\in I\}$ de $M$. Si $m\in M$, entonces
podemos escribir $m=\sum r_i\cdot m_i$ (suma finita) para ciertos $r_i\in I$. Veamos que los $r_i$ son únicos. Si $m=\sum s_i\cdot m_i$, entonces
$\sum (r_i-s_i)\cdot m_i=0$. La independencia lineal del conjunto $\{m_i:i\in I\}$ implica entonces que $r_i=s_i$ para todo $i\in I$.  	
\end{proof}

\begin{proposition}
\label{pro:libre}
Sea $M$ libre con base $\{m_i:i\in I\}$ y sea $N$ un submódulo de $M$. Si $\{n_i:i\in I\}\subseteq N$ es un subcobjunto, existe
un único $f\in\Hom_R(M,N)$ tal que $f(m_i)=n_i$ para todo $i\in I$.  
\end{proposition}

\begin{proof}[Bosquejo de la demostración]
Basta con observar que el único morfismo $f\colon M\to N$ debe definirse como $f(\sum r_i\cdot m_i)=\sum r_i\cdot n_i$.  	
\end{proof}

Una aplicación sencilla de la proposición anterior:

\begin{example}
Veamos que no existe un epimorfismo $\Z\to\Z\times\Z$ (de $\Z$-módulos). En efecto, si $f\colon\Z\to\Z\times\Z$ es un epimorfismo, 
sea $\{u,v\}$ una base de $\Z\times\Z$. Entonces $f(k)=u$ y $f(l)=v$ para ciertos $k,l\in\Z$. La proposición~\ref{pro:libre} implica
que existe un morfismo $g\colon\Z\times\Z\to\Z$ tal que $g(u)=k$ y $g(v)=l$. En particular, $f\circ g=\id_{\Z\times\Z}$ y luego 
$g$ es un monomorfismo. 	Como 
\[
g(lu-kv)=lg(u)-kg(v)=lk-kl=0,
\]
se tiene entonces que $lu-kv=0$ y luego $k=l=0$, pues $\{u,v\}$ es linealmente independiente, una contradicción.  
\end{example}

Otra propiedad importante de los módulos libres:

\begin{proposition}
Si $M$ es libre, entonces $M\simeq R^{(I)}$ para algún conjunto $I$.
\end{proposition}

\begin{proof}
Supongamos que $M$ es libre con base $\{m_i:i\in I\}$. Vimos una proposición que nos dice que 
existe un único morfismo $f\in\Hom_R(M,R^{(I)})$ tal que 
$f(m_i)=e_i$ para todo $i\in I$, donde 
\[
(e_i)_j=\begin{cases}
	1 & \text{si $i=j$},\\
	0 & \text{si $i\ne j$.}
	\end{cases}	
\]	
Veamos que $f$ es un isomorfismo. 
Primero veamos que $f$ es epimorfismo: dado $(r_i)_{i\in I}\in R^{(I)}$, entonces
$f(\sum r_i\cdot m_i)=(r_i)_{i\in I}$. Para ver que es monomorfismo:
\[
0=f(\sum r_i\cdot m_i)=\sum r_i\cdot f(m_i)=\sum r_i\cdot e_i\implies r_i=0\text{ para todo $i\in I$}.\qedhere
\]
\end{proof}

\begin{corollary}
Todo módulo es (isomorfo a un) cociente de un módulo libre.
\end{corollary}

\begin{proof}
Si $M$ es un módulo, vamos a demostrar que existe un módulo libre $L$ y un epimorfismo $f\in\Hom_R(L,M)$, ya que
entonces $L/\ker f\simeq M$ por el primer teorema de isomorfismos. Sea 
$\{m_i:i\in I\}$ un conjunto de generadores de $M$, que siempre existe, ya que podríamos tomar por ejemplo el conjunto
$\{m:m\in M\}$ y sea $L=R^{(I)}$. Entonces $L$ es libre y $f\colon R^{(I)}\to M$, $e_i\mapsto m_i$, es un epimorfismo. 
\end{proof}

Otra propiedad importante de los módulos libres:

\begin{proposition}
	Si $M$ es un módulo libre, $f\in\Hom_R(N,T)$ es un epimorfismo y $h\in\Hom_R(M,T)$, entonces existe
	$\varphi\in\Hom_R(M,N)$ tal que $f\circ \varphi=h$. 
	% todo: diagrama
\end{proposition}

\begin{proof}
Si $\{m_i:i\in I\}$ es base de $M$, como $f$ es un epimorfismo, para cada $i\in I$ existe $n_i\in N$ tal que $f(n_i)=h(m_i)$. Como 
$M$ es libre, existe un único morfismo $\varphi\colon M\to N$ tal que $\varphi(m_i)=n_i$ para todo $i\in I$. Este morfismo
cumple $f\circ \varphi=h$. 
\end{proof}

%\index{Sumando directo!de un submódulo}
%Recordemos que un submódulo $N$ de $M$ es un \textbf{sumando directo} de $M$ si existe un submódulo $T$ de $M$ tal que
%$M=N\oplus T$. 

\index{Cuerpo!de fracciones}
Sea $R$ un dominio íntegro y sea $S=R\setminus\{0\}$. 

En el conjunto $R\times S$ definimos
la siguiente relación:
\[
(r,s)\sim (r_1,s_1)\Longleftrightarrow rs_1-r_1s=0.
\]

Dejamos como ejercicio verificar que $\sim$ es una relación de equivalencia. 

La clase
de equivalencia del par $(r,s)$ será denotada por $r/s$ o bien $\frac{r}{s}$. 

Puede demostrarse que 
el conjunto
$K(R)=(R\times S)/{\sim}$ de clases de equivalencia 
es un cuerpo con las operaciones
\begin{equation}
\label{eq:K(R)}
\frac{r}{s}+\frac{r_1}{s_1}=\frac{rs_1+r_1s}{ss_1},
\quad
\frac{r}{s}\frac{r_1}{s_1}=\frac{rr_1}{ss_1}.
\end{equation}
$K(R)$ se llama el \textbf{cuerpo de fracciones} de $R$. 
Por ejemplo, 
$K(\Z)=\Q$.

Para demostrar que $K(R)$ es un cuerpo primero
debemos ver que las operaciones~\ref{eq:K(R)} están bien definidas. Por ejemplo, si $r/s\sim r'/s'$ y $r_1/s_1\sim r_1'/s_1'$, entonces
$r/s+r_1/s_1\sim r'/s'+r_1'/s_1'$. En efecto,
como $r/s\sim r'/s'$, entonces $rs'-r's=0$. Similarmente, $r_1s_1'-r_1's_1=0$, pues $r_1/s_1\sim r_1'/s_1'$. Luego 
\begin{gather*}
\frac{rs_1+r_1s}{ss_1}=\frac{r's_1'+r_1's'}{s's_1'}
\shortintertext{pues}
(rs_1+r_1s)s's_1'=rs_1s's_1'+r_1ss's_1'
=r'ss_1s_1'+r_1's_1ss'
=(r's_1'+r_1's)ss_1.
\end{gather*}
De la misma forma se demuestra que el producto también está bien definido. Dejamos como ejercicio
verificar que con estas operaciones $K(R)$ es un cuerpo. 

\begin{theorem}
Sea $R$ un dominio íntegro. 
Si $M$ es un módulo libre con base finita, entonces dos bases cualesquiera de $M$ 
tienen la misma cantidad de elementos.
\end{theorem}

\begin{proof}
Sea $K=K(R)$ el cuerpo de fracciones de $R$. Es sencillo verificar que el grupo abeliano  
$V=\Hom_R(M,K)$ es un $K$-espacio vectorial con la acción 
\[
(\lambda f)(m)=\lambda f(m),
\]
donde $\lambda\in K$, $f\in V$ y $m\in M$.

El espacio vectorial $V$ tiene bien definida su dimensión. 
Calculemos entonces $\dim V$. Para eso, sea $\{e_1,\dots,e_n\}$ una base de $M$. 
Para cada $i\in\{1,\dots,n\}$ sea
\[
f_i\colon M\to K,\quad
e_j\mapsto\begin{cases}
1 & \text{si $i=j$},\\
0 & \text{si $i\ne j$}.
\end{cases}
\]
Veamos que $\{f_1,\dots,f_n\}$ es base de $V$. Es un conjunto de generadores de $V$ pues 
si $f\in V$, entonces
\[
f=\sum_{i=1}^n f(e_i)f_i,
\]
pues estos morfismos coinciden en los elementos de una base de $M$, es decir 
$f(e_j)=(\sum_{i=1}^n f(e_i)f_i)(e_j)$ para todo $j\in\{1,\dots,n\}$.    
Además $\{f_1,\dots,f_n\}$ es linealmente independiente, pues si $0=\sum_{i=1}^n\lambda_if_i$, 
entonces, al evaluar en cada $e_j$, se tiene que 
\begin{align*}
0=\left(\sum_{i=1}^n \lambda_if_i\right)(e_j)=\lambda_j
\end{align*}
para todo $j\in\{1,\dots,n\}$. Luego $n=\dim V$. 
\end{proof}

\begin{definition}
\index{Rango!de un módulo}
Sea $R$ un dominio íntegro. Si $M$ es un módulo finitamente generado, se 
define el \textbf{rango} de $M$ como el cardinal de una base de $M$. Si $M$ no es finitamente
generado diremos que el rango de $M$ es infinito. El rango del módulo $M$ será denotado por $\rank(M)$.   
\end{definition}