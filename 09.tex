\lecture{}

\begin{definition}
Let $M$ be a module and $X$ be a subset of $M$. The submodule
of $M$ generated by $X$ is defined as
\[
(X)=\bigcap\{N:N\text{ is a submodule of $M$ that contains $X$}\},
\]
the smallest submodule of $M$ containing $X$. 
\end{definition}

One can prove that  
\[
(X)=\left\{\sum_{i=1}^mr_i\cdot x_i:m\in\Z_{\geq0},\,r_1,\dots,r_m\in R,\,x_1,\dots,x_m\in X\right\}
\]

\begin{definition}
A module $M$ is \textbf{finitely-generated} if $M=(X)$ for some finite subset $X$ of $M$.
\end{definition}

If $X=\{x_1,\dots,x_m\}$ one writes $(X)=(x_1,\dots,x_n)$.
For example, $\Z=(1)=(2,3)$ and $\Z\ne (2)$.

\begin{exercise}
    Let $R$ be the ring of continuous maps $[0,1]\to\R$ with point-wise operations and 
    $M=\prescript{}{R}{R}$. Prove that
    $N=\{f\in R:f(x)\ne0\text{ for finitely many $x$}\}$ is not finitely-generated. 
\end{exercise}

\begin{exercise}
    Let $G=\{g_1,\dots,g_n\}$ be a finite group. Prove that if $M=\C[G]$ is finitely-generated, then
    $M$ is a finite-dimensional complex vector space. 
\end{exercise}

If $f\in\Hom_R(M,N)$ and $M$ is finitely-generated, then 
$f(M)$ is finitely-generated. We can prove something strong. For that purpose
we need exact sequences. 
A sequence 
	\begin{equation}
	\label{eq:exacta1}	
		\xymatrix{
        0\ar[r]
        & M
        \ar[r]^f
        & N
        \ar[r]^g
        & T\ar[r]
        & 0,
        }
  	\end{equation}
of modules and homomorphism is said to be \textbf{exact} 
if $f$ is injective, $g$ is surjective and $f(M)=\ker g$. For example,
the sequence
\[
		\xymatrix{
        0\ar[r]
        & M
        \ar[r]^f
        & M\oplus N
        \ar[r]^g
        & N\ar[r]
        & 0,
        }
\]
where $f(m)=(m,0)$ and $g(m,n)=n$ is exact.

\begin{exercise}
	Si  
	\[  
		\xymatrix{
        0\ar[r]
        & M
        \ar[r]^f
        & N
        \ar[r]^g
        & T\ar[r]
        & 0	
        }
     \]
     be an exact sequence. Prove the following statements: 
     \begin{enumerate}
     \item If $N$ if finitely generated, then $T$ is finitely generated. 
     \item If $M$ and $T$ are finitely generated, then $N$ is finitely generated. 	
     \end{enumerate}
\end{exercise}

\begin{proposition}
    Let $R$ be a ring and $M$ be a module over $R$. 
    Then $M$ is finitely-generated if and only 
    if $M$ is isomorphic to a quotient of $R^k$ for some $k$.
\end{proposition}

\begin{proof}
    Assume first that $M=(m_1,\dots,m_k)$ is finitely-generated. A routine calculation shows that
    the map
    \[
    \varphi\colon R^k\to M,\quad
    (r_1,\dots,r_k)\mapsto \sum_{j=1}^kr_j\cdot m_j,
    \]
    is a surjective module homomorphism. The first isomorphism theorem implies that 
    $R^k/\ker\varphi\simeq\varphi(R^k)=M$. 
    
    Now assume that there exists a 
    surjective module homomorphism $\varphi\colon R^k\to M$. Since 
    $R^k=(e_1,\dots,e_k)$, where 
    \[
    (e_i)_j=\begin{cases}
    1 & \text{if $i=j$},\\
    0 & \text{if $i\ne j$},
    \end{cases}
    \]
    it follows that $\{\varphi(e_1),\dots,\varphi(e_k)\}$ generates $\varphi(R^k)=M$. Indeed, 
    if $m\in M$, we write $m=\varphi(r_1,\dots,r_k)$ for some $(r_1,\dots,r_k)\in R^k$ 
    and hence 
    \[
    m=\varphi(r_1,\dots,r_k)=\varphi\left(\sum_{i=1}^k r_i\cdot e_i\right)
    =\sum_{i=1}^kr_i\cdot\varphi(e_i).\qedhere
    \]
\end{proof}

\topic{Noetherian modules}

\begin{definition}
A module is \textbf{noetherian} if every sequence $M_1\subseteq M_2\subseteq\cdots$ of submodules of $M$ 
stabilizes, that is there exists $n$ such that $M_k=M_{n+k}$ for all $k$. 	
\end{definition}

\begin{proposition}
Let $M$ be a module. The following statements are equivalent:
\begin{enumerate}
\item $M$ is noetherian.
\item Submodules of $M$ are finitely generated.
\item Each non-empty family of submodules of $M$ has a maximal element (with respect to the inclusion).	
\end{enumerate}
\end{proposition}

\begin{proof}
	We first prove $2)\implies1)$. If $S_1\subseteq S_2\subseteq\cdots$ is a sequence of submodules of $M$, 
	it follows that $S=\cup_{i\geq 1}S_i$ is a submodule of $M$. Since $S$ is finitely generated, 
	$S=(x_1,\dots,x_n)$ for some $x_1,\dots,x_n\in M$. It follows that 
	$x_1,\dots,x_n\in S_N$ for some positive integer $N$. Thus 
	$S\subseteq S_N$ and hence $S_N=S_{N+k}$ for all $k$. 
	
	We now prove $1)\implies3)$. Let $F$ be a non-empty family of submodules of $M$ with no maximal elements. 
	Let $S_1\in F$. Since $S_1$ is not maximal, there exists $S_2\in F$ such that $S_1\subsetneq S_2$. 
	Having constructed with this method the submodules $S_1\subsetneq\dots\subsetneq S_k$, since $S_k$ is not
	maximal, there exists $S_{k+1}\in F$ such that $S_k\subsetneq S_{k+1}$. 
	This means that the sequence 
	$S_1\subsetneq S_2\subsetneq\cdots$ does not stabilize. 
	
	We finally prove $3)\implies2)$. Let $S$ be a submodule of $M$ and let 
	\[
	F=\{T\subseteq S:T\text{ finitely generated submodule of $M$}\}.
	\]
	By assumption, $F$ has a maximal element $N$. Thus
	$N$ is a finitely generated submodule of $M$ such that $N\subseteq S$. We may assume that 
	$N=(n_1,\dots,n_k)$. If $N=S$, then, in particular, $S$ is finitely generated. Suppose that
	$N\ne S$ and let $x\in S\setminus N$. It follows that  
	$N\subseteq (n_1,\dots,n_k,x)\subseteq S$. Since 
	$(n_1,\dots,n_k,x)\in F$ and $N$ is maximal, it follows that 
	$N=(n_1,\dots,n_k,x)$, a contradiction
	to $x\not\in N$. 
\end{proof}

\begin{exercise}
\label{xca:exacta_noetheriano}
	Let   
	\[  
		\xymatrix{
        0\ar[r]
        & M
        \ar[r]^f
        & N
        \ar[r]^g
        & T\ar[r]
        & 0	
        }
     \]
     be an exact sequence. Prove the following statements:
     \begin{enumerate}
     	\item If $N$ is noetherian, then $M$ and $T$ are noetherian.
     	\item If $M$ and $T$ are noetherian, then $N$ is noetherian.
     \end{enumerate}	
\end{exercise}

\begin{exercise}
\label{xca:regular_noetheriano}
A ring $R$ is noetherian if and only if $\prescript{}{R}R$ is noetherian.	
\end{exercise}

\begin{exercise}
\label{xca:directa_noetheriano}
If $M_1,\dots,M_n$ are noetherian, then $M_1\oplus\cdots\oplus M_n$ is noetherian. 	
\end{exercise}

The previous exercise cannot be extended to infinitely many modules. Why?
%For example, $\Z^{\Z}$ is not noetherian, as it is not finitely generated. 

\begin{proposition}
If $R$ is noetherian and $M$ is a finitely generated module, then $M$ is noetherian. 
\end{proposition}

\begin{proof}
    Assume that $M=(m_1,\dots,m_k)$. There exists a surjective homomorphism  
$R^k\to N$, $(r_1,\dots,r_k)\mapsto \sum_{i=1}^k r_i\cdot m_i$, where
$R^k=\oplus_{i=1}^k R$. Since $R$ is noetherian, $R^k$ is noetherian. Thus $M$ is noetherian.	
\end{proof}

\topic{Free modules}

\begin{definition}
    Let $R$ be a ring, $M$ be a module over $R$ and $X$ be a subset of $M$. We say that
    $X$ is \textbf{linearly independent} if for each $k\in\Z_{>0}$, $r_1,\dots,r_k\in R$
    and $m_1,\dots,m_k\in X$ such that $\sum_{i=1}^kr_i\cdot m_i=0$, then 
    $r_1=\cdots=r_k=0$. 
\end{definition}

In any ring, the set $\{1\}$ is linearly independent. 

\begin{examples}\
\begin{enumerate}
    \item $\{2,3\}$ is a linear dependent subset of $\Z$.
    \item $\{2\}$ is a linearly dependent subset of $\Z/4$.
    \item Let $R=\Z$, $M=\Q$ and $x\in M\setminus\{0\}$. 
        Then $\{x\}$ is linearly independent subset of $M$. Is $y\in M\setminus\{x\}$, then
        $\{x,y\}$ is linearly dependent.  
\end{enumerate}    
\end{examples}

\begin{examples}
Let $R=M_2(\R)$ and $M=\begin{pmatrix}
        0&\R\\
        0&\R 
    \end{pmatrix}$. Then $\left\{\begin{pmatrix}0&1\\0&0\end{pmatrix}\right\}$ is a minimal generating 
    set and it is linearly independent. 
\end{examples}

\begin{exercise}
\label{xca:LI}
    Let $f\in\Hom_R(M,N)$ and $X$ be a subset of $M$. 
    \begin{enumerate}
        \item If $X$ is linearly dependent, then $f(X)$ is linearly dependent.
        \item If $X$ is linearly independent and $f$ is injective, then $f(X)$ is linearly independent. 
        \item If $M=(X)$ and $f$ is surjective, then $N=(f(X))$. 
    \end{enumerate}
\end{exercise}

\begin{definition}
    Let $M$ be a module and $B$ be a subset of $M$. Then $B$ is a \textbf{basis} of $M$ if
    $B$ is linearly independent and $M=(B)$. A module $M$ is said to be \textbf{free} if it admits a basis.   
\end{definition}

As a consequence of Zorn's lemma, 
vector spaces are free. 

\begin{examples}\
\begin{enumerate}
    \item If $R$ is a ring, then $\{1\}$ is a basis of $\prescript{}{R}{R}$, so $\prescript{}{R}{R}$ is free.
    \item If $R$ is a ring, then $R^n$ is free as a module over $R$. 
\end{enumerate}
\end{examples}

\begin{exercise}
    Prove that $\Q$ is not free as a module over $\Z$. 
\end{exercise}

\begin{exercise}
\label{xca:linear_algebra}
    Let $R$ be a division ring and $M$ be a non-zero and finitely generated module over $R$.
    Prove the following facts:
    \begin{enumerate}
        \item Every finite set of generators contains a basis.
        \item Every linearly independent set can be extended into a basis.
        \item Any two bases contain the same number of elements.
    \end{enumerate}
\end{exercise}

\begin{example}
    $\R[X]$ is a free module (over $\R$) with basis $\{1,X,X^2,\dots\}$. 
\end{example}

\begin{exercise}
    Prove that $\{(a,b),(c,d)\}$ is a basis of $\Z\times\Z$ 
    (as a module over $\Z$) if and only if
    $ad-bc\in\{-1,1\}$. 
\end{exercise}