

\begin{proposition}
    Let $G$ be a finite group and $\rho\colon G\to\GL(V)$ be a representation. Then
    each $\rho_g$ is diagonalizable. 
\end{proposition}

\begin{proof}
    Let $n=\dim V$. Fix a basis of the finite-dimensional vector space $V$ and consider
    a matrix representation $\rho\colon G\to\GL_n(V)$. 
    Since $g$ is finite, $g^m=1$ for some $m\in\N$. This means that $\rho_g$ is a root of $X^m-1\in\C[X]$. Since 
    the roots of the polynomial $X^m-1$ are all different and $X^m-1$ factorizes linearly on $\C[X]$, it follows
    that the minimal polynomial of $\rho_g$ also factorizes linearly in $\C[X]$. Hence $\rho_g$ is diagonalizable. 
\end{proof}

\begin{definition}
    Let $\rho\colon G\to\GL(V)$ and $\psi\colon G\to\GL(W)$ be
    representations of a finite group $G$. A linear map $T\colon V\to W$ is said to be $G$-invariant
    if the diagram
    \[\begin{tikzcd}
	V & V \\
	W & W
	\arrow["{\rho_g}", from=1-1, to=1-2]
	\arrow["T"', from=1-1, to=2-1]
	\arrow["{\psi_g}"', from=2-1, to=2-2]
	\arrow["T", from=1-2, to=2-2]
\end{tikzcd}\]
    commutes, i.e. $\psi_gT=T\rho_g$ for all $g\in G$. 
\end{definition}

\begin{definition}
    The representations $\rho\colon G\to\GL(V)$ and $\psi\colon G\to\GL(W)$ are \textbf{equivalent}
    if there exists a bijective $G$-invariant map $T\colon V\to W$. 
\end{definition}

\begin{example}
    Let $G=\Z/n$. The representations 
    \begin{align*}
    &\rho\colon G\to\GL_2(\C),\quad
    m\mapsto
    \begin{pmatrix}
        \cos(2\pi m/n) & -\sin(2\pi m/n)\\
        \sin(2\pi m/n) & \cos(2\pi m/n)
    \end{pmatrix}
    \shortintertext{and}    
    &\psi\colon G\to\GL_2(\C),\quad
    m\mapsto
    \begin{pmatrix}
        e^{2\pi im/n} & 0\\
        0 & e^{-2\pi im/n}
    \end{pmatrix}
    \end{align*}
    are equivalent, as $\rho_mT=T\psi_m$ for all $m\in G$ if $T=\begin{pmatrix}
        i&-i\\
        1&1
    \end{pmatrix}$.
\end{example}

\begin{definition}
    Let $\rho\colon G\to\GL(V)$
    a representation. A subspace $W$ of $V$ is said to be $G$-invariant (with respect to $\rho$)
    if $\rho_g(W)\subseteq W$ for all $g\in G$. 
\end{definition}

If $\rho\colon G\to\GL(V)$ is a representation and $W\subseteq V$ is $G$-invariant, then
the map $\rho|_W\colon G\to\GL(W)$, $g\mapsto (\rho_g)|_W$, is a representation.   

\begin{definition}
    Let $\rho\colon G\to\GL(V)$
    a representation. A \textbf{subrepresentation} of $\rho$ is a representation
    of the form $\rho|_W\colon G\to\GL(W)$ for some $G$-invariant subspace $W$ of $V$. 
\end{definition}

\begin{example}
    Let $G=\langle g:g^3=1\rangle$ be the 
    cyclic group of order three  
    and 
    \[
    \rho\colon G\to\GL_3(\R),
    \quad
    g\mapsto\begin{pmatrix}
        0&1&0\\
        0&0&1\\
        1&0&0
    \end{pmatrix}.
    \]
    The subspace
    \[
    W=\left\{
    \begin{pmatrix}
    x\\
    y\\
    z
    \end{pmatrix}\in\R^3:x+y+z=0\right\}
    \]
    is a $G$-invariant subspace of $\R^3$. 
\end{example}

\begin{definition}
    \index{Representation|irreducible}
    A representation $\rho\colon G\GL(V)$ is \textbf{irreducible} if
    $\{0\}$ and $V$ are the only $G$-invariant subspaces of $V$. 
\end{definition}

Clearly, degree-one representations are irreducible.

\begin{example}
        Let $G=\langle g:g^3=1\rangle$ be the 
    cyclic group of order three  
    and 
    \[
    \rho\colon G\to\GL_3(\R),
    \quad
    g\mapsto\begin{pmatrix}
        0&1&0\\
        0&0&1\\
        1&0&0
    \end{pmatrix}.
    \]
    We claim that the $G$-invariant subspace
    \[
    W=\left\{
    \begin{pmatrix}
    x\\
    y\\
    z
    \end{pmatrix}\in\R^3:x+y+z=0\right\}\subseteq\R^3
    \]
    is irreducible. Let $S$ be a non-zero $G$-invariant subspace of $W$ and let $s=\begin{pmatrix}x_0\\y_0\\z_0\end{pmatrix}\in S$ be a non-zero element. Then
    \[
    t=\begin{pmatrix}y_0\\z_0\\x_0\end{pmatrix}
    =\begin{pmatrix}
        0&1&0\\
        0&0&1\\
        1&0&0
    \end{pmatrix}
    \begin{pmatrix}x_0\\y_0\\z_0\end{pmatrix}\in S.
    \]
    We claim that $\{s,t\}$ are linearly independent. If not, there exists $\lambda\in\R$ such that
    $\lambda s=t$. Thus $\lambda x_0=y_0$, $\lambda y_0=z_0$ and $\lambda z_0=x_0$. This implies that
    $\lambda^3x_0=x_0$. Since $x_0\ne 0$ (because if $x_0=0$, then $y_0=z_0=0$, a contradiction), it follows that
    $\lambda=1$ and hence $x_0=y_0=z_0$, a contradiction because $x_0+y)0+z_0=0$. 
    Therefore $\dim S=2$ and hence $S=W$. 
\end{example}

\begin{exercise}
    Let $\rho\colon G\to\GL(V)$ be a degree-two representation. Prove that
    $\rho$ is irreducible if and only if there is no common eigenvector for the $\rho_g$, $g\in G$. 
\end{exercise}

The previous exercise can be used to show that the representation
$\Sym_3\to\GL_2(\C)$
of the symmetric group $\Sym_3$ 
given by
\[
(12)\mapsto\begin{pmatrix}
-1&1\\0&1
\end{pmatrix},
\quad
(123)\mapsto\begin{pmatrix}
1&0\\
1&-1
\end{pmatrix}
\]
is irreducible. 

\begin{definition}
    \index{Representation|completely irreducible}
    A representation $\rho\colon G\to\GL(V)$ is \textbf{completely irreducible}
    if $V$ can be decomposed as 
    $V=V_1\oplus\cdots\oplus V_n$, where each $V_i$ is a $G$-invariant subspace of $V$ and
    each $\rho|_{V_i}$ is irreducible. 
\end{definition}

Since we are considering finite-dimensional vector spaces, our vector spaces are
Hilbert spaces, so they have
an inner product $V\times V\to\C$, $(v,w)\mapsto\langle v,w\rangle$. 

\begin{definition}
    \index{Representation|unitary}
    A representation $\rho\colon G\to\GL(V)$ is \textbf{unitary} if 
    $\langle \rho_gv,\rho_gw\rangle=\langle v,w\rangle$ for all $g\in G$ and $v,w\in V$. 
\end{definition}

\begin{definition}}
A representation 
$\rho\colon G\to\GL(V)$ is \textbf{decomposable} if $V$ can be decomposed as $V=S\otimes T$
where $S$ and $T$ are non-zero $G$-invariant subspaces of $V$. 
\end{definition}

\begin{exercise}
Let $\rho\colon G\to\GL(V)$ be a unitary representation. Prove that $\rho$ is either
irreducible or decomposable. 
\end{exercise}

\begin{example}
Let $G$ be a finite group and $V=\C[G]$. The \textbf{left regular representation}
of $G$ is the representation
\[
L\colon G\to\GL(V),
\quad
g\mapsto L_g,
\]
where $L_g(h)=gh$. With the inner product
\[
\left\langle\sum_{g\in G}\lambda_gg,\sum_{g\in G}\mu_gg\right\rangle=\sum_{g\in G}\lambda_g\overline{\mu_g}
\]
the representation $L$ is unitary. 
\end{example}

\begin{proposition}[Weyl's trick]
    Every representation of a finite group is equivalent to a unitary representation.
\end{proposition}

\begin{proof}
    Let $\rho\colon G\to\GL(V)$ and $V\times V\to\C$, $(v,w)\mapsto\langle v,w\rangle_0$ be an inner
    product on $V$. A straighforward calculation shows that 
    \[
    \langle v,w\rangle=\sum_{g\in G}\langle\rho_gv,\rho_gw\rangle_0
    \]
    is an inner product of $V$. Since
    \begin{align*}
    \langle\rho_gv,\rho_gw\rangle&=\sum_{h\in G}\langle\rho_h\rho_gv,\rho_h\rho_gw\rangle_0\\
    &=\sum_{h\in G}\langle\rho_{hg}v,\rho_{hg}w\rangle_0=\sum_{x\in G}\langle\rho_xv,\rho_xw\rangle_0=\langle v,w\rangle,
    \end{align*}
    the representation $\rho$ is unitary.
\end{proof}

Weyl's trick has several interesting corollaries. Let $\rho\colon G\to\GL(V)$ be a representation
of a finite group $G$. Then 1) every non-zero representation is either 
irreducible or decomposable, and 2) every $\rho_g$ is diagonalizable 
(as unitary operators are diagonalizable). 

\begin{exercise}
    If $G$ is an infinite group it is not longer true that every non-zero representation
    is either irreducible or decomposable. Find an example. 
\end{exercise}

Recall that we only consider finite-dimensional representations of finite groups. 

\begin{theorem}[Maschke]
    Every representation of a finite group is completely reducible. 
\end{theorem}

\begin{proof}
    Let $G$ be a finite group and $\rho\colon G\to\GL(V)$ be a representation of $G$. We proceed
    by induction on $\dim V$. 
    If $\dim V=1$, the result is trivial, as degree-one representations are irreducible. Assume that
    the result holds for representations of degree $\leq n$. Let $\rho\colon G\to\GL(V)$ be a representation
    of degree $n+1$. If $\rho$ is irreducible, we are done. If not, write $V=S\otimes T$, where $S$ and $T$
    are non-zero $G$-invariant subspaces. Since $\dim S<\dim V$ and $\dim T<\dim V$, it follows from 
    the inductive hypothesis that 
    both $S$ and $T$ are completely irreducible. Thus $V$ is completely irreducible. 
\end{proof}

\begin{example}
    Let $G=\Sym_3$ and $\rho\colon G\to\GL_3(\C)$ be the representation given by
    \[
    (12)\mapsto\begin{pmatrix}
    0&1&0\\
    1&0&0\\
    0&0&1
    \end{pmatrix},\quad
    (123)\mapsto\begin{pmatrix}
    0&0&1\\
    1&0&0\\
    0&1&0
    \end{pmatrix}
    \]
    Then $\rho_g$ is unitary for all $g\in G$ (because $\rho_{(12)}$ and $\rho_{(123)}$ are both
    unitary). Moreover, 
    \[
    S=\left\langle \begin{pmatrix}
    1\\1\\1
    \end{pmatrix}
    \right\rangle,
    \quad
    T=S^{\perp}=\left\langle
    \begin{pmatrix}
    -1\\1\\0
    \end{pmatrix},
    \begin{pmatrix}
    0\\-1\\1
    \end{pmatrix}
    \right\rangle,
    \]
    are irreducible $G$-invariant subspaces of $V=\C^3$. A direct calculation shows that 
    in the orthogonal basis $\left\{\begin{pmatrix}
    1\\1\\1
    \end{pmatrix},
    \begin{pmatrix}
    -1\\1\\0
    \end{pmatrix},
    \begin{pmatrix}
    0\\-1\\1
    \end{pmatrix}
    \right\}$ 
    the matrices $\rho_{(12)}$ and $\rho_{(123)}$ can be written as 
    \[
    \rho_{(12)}=\begin{pmatrix}
        1&0&0\\
        0&-1&1\\
        0&0&1
    \end{pmatrix},
    \quad
    \rho_{(123)}=
    \begin{pmatrix}
        1&0&0\\
        0&0&-1\\
        0&1&-1
    \end{pmatrix}.
    \]
\end{example}

\begin{exercise}
Let $G$ be a finite group. 
Prove that there is a bijection between degree-one representations of $G$ and
degree-one representations of $G/[G,G]$.
\end{exercise}
