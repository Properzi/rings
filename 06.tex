\section{Lecture: 30/10/2024}
\label{6}

We will dedicate three lectures to the development of the character theory for complex representations 
of finite groups. Our approach will remain entirely elementary, 
relying solely on fundamental linear algebra principles over complex numbers.

\subsection{Group algebras}

Fix a field $K$. 
For a finite group $G$, let $K[G]$ be the vector space (over $K$)
with basis $\{g:g\in G\}$. 
Thus $K[G]$ is a vector space of dimension $\dim K[G]=|G|$.
Moreover, $K[G]$ is a ring
with
\[
\left(\sum_{g\in G}\lambda_gg\right)\left(\sum_{h\in G}\mu_hh\right)
=\sum_{g,h\in G}\lambda_g\mu_h(gh).
\] 
Note that $K[G]$ is commutative if and only if $G$ is abelian. 

\begin{example}
    Let 
    \[
    \Sym_3=\{\id,(12),(13),(23),(123),(132)\}
    \]
    the symmetric group in three letters. Every element of $\C[\Sym_3]$
    is of the form 
    \[
    a\id+b(12)+c(13)+d(23)+e(123)+f(132)
    \]
    for some $a,b,c,d,e,f\in\C$. For example, 
    \[
    \alpha=5\id +3(123)\quad\text{ and }\quad 
    \beta=-4\id+(132)
    \]
    are elements of
    $\C[\Sym_3]$. We compute
    \begin{align*}
    \alpha+\beta &=1\id+3(123)+(132)
    \shortintertext{and}
    \alpha\beta &=(5\id +3(123))(-4\id+(132))\\
    &=-20\id +5(132)-12(123)+3(123)(132)\\
    &=-20\id +5(132)-12(123)+3\id\\
    &=-17\id+ 5(132)-12(123).
    \end{align*}
    Another example:
    \begin{align*}
    (\id+5(13))(2\id-(12))
    &=2\id-(12)+10(13)-5(13)(12)\\
    &=2\id-(12)+10(13)-5(123),
    \end{align*}
    as $(13)(12)=(123)$. 
\end{example}

Note that $K[G]$ is a ring and also a vector space (over $K$) and these structures
are somewhat compatible, as the formulas 
\[
(\lambda a+\mu b)c=\lambda (ac)+\mu (bc),\quad
a(\lambda b+\mu c)=\lambda (ab)+\mu (ac)
\]
hold for all $\lambda,\mu\in K$ and $a,b,c\in K[G]$. 

\begin{definition}
\index{Algebra}
Let $A$ be a ring and $K$ be a field. Then $A$ is a \emph{$K$-algebra} (or an \emph{algebra} 
over the field $K$) if $A$ is a vector space (over $K$)
such that $\lambda(ab)=(\lambda a)b=a(\lambda b)$ for all $\lambda\in K$ and $a,b\in A$. 
\end{definition}

The fields $\R$ and $\C$ are examples of $\R$-algebras. 
Other examples of $\R$-algebras are the 
polynomial rings $\R[X]$ and $\R[X,Y]$ and the matrix rings $M_n(\R)$.  

\begin{example}
	If $A$ is an algebra, then $M_n(A)$ is an algebra.	
\end{example}

\begin{example}
    Let $G$ be a finite group and $K$ be a field. Then $K[G]$ is a $K$-algebra. 
\end{example}

\begin{example}
	Let $G=\langle g:g^3=1\rangle=\{1,g,g^2\}\simeq C_3$ be the cyclic group of order three. 
	If $\alpha=a_11+_2g+a_3g^2$ and $\beta=b_11+b_2g+b_3g^2$, then
	\[
		\alpha\beta=(a_1b_1+a_2b_3+a_3b_2)1+(a_1b_2+a_2b_1+a_3b_3)g+a_1b_3+a_2b_2+a_3b_1)g^2.
	\]
	One can check that $\C[G]\simeq\C[X]/(X^3-1)$. 
\end{example}

In general, one proves that the group algebra of $C_n$, the cyclic group of order $n\geq2$, 
is isomorphic to $\C[X]/(X^n-1)$.

\begin{exercise}
\label{xca:RC3}
	Prove that $\R[C_3]\simeq\R\times\C$. 	
\end{exercise}

\begin{exercise}
	Let $G=\{1,g\}\simeq C_2$ be the cyclic group of order two. The product
	of $\C[G]$ is 
	\[
	(a1+bg)(c1+gd)=(ac+bd)1+(ad+bc)g.
	\]
	Prove that the map 
 \[
 \C[G]\to \C\times \C,\quad a1+bg\mapsto (a+b,a-b),
 \]
	is a linear isomorphism of rings. 
\end{exercise}

\begin{exercise}
	Let $K=\Z/2$ and $G=\{1,g\}\simeq C_2$ be the cyclic group of order two. 
	Prove that the map 
 \[
 K[G]\to\begin{pmatrix}
		K&K\\
		0&K
	\end{pmatrix},\quad 
 a1+bg\mapsto\begin{pmatrix}
		a+b&b\\
		0&a+b		
	\end{pmatrix},
    \]
    is a linear isomorphism of rings.
\end{exercise}

The group ring has the following property, which is left as an exercise. 
Let $A$ be an algebra and
$G$ be a finite group. If $f\colon G\to\mathcal{U}(A)$ is a group homomorphism, 
then there exists a unique algebra homomorphism $\varphi\colon K[G]\to A$ such that
$\varphi|_G=f$. 

\begin{example}
	Let $\D_3=\langle r,s:r^3=s^2=1,\,srs^{-1}=r^{-1}\rangle$ be the dihedral
	group of six elements. We claim that 
	\[
	\C[\D_3]\simeq\C\times\C\times M_2(\C).
	\]
	Let $\omega$ be a primitive cubic root of one. Let 
	\[
	R=\begin{pmatrix}
		\omega&0\\
		0&\omega^2	
	\end{pmatrix},
	\quad
	S=\begin{pmatrix}
		0&1\\
		1&0
	\end{pmatrix}.
 	\]
 	One easily checks that $SRS^{-1}=R^{-1}$ and $R^3=S^2=\begin{pmatrix}
		1&0\\
		0&1	
	\end{pmatrix}$. It follows that there exists a group homomorphism
	$G\to\C\times\C\times M_2(\C)$ such that
	$r\mapsto (1,1,R)$ and $s\mapsto (1,-1,S)$. This group homomorphism
	is a ring isomorphism.  
\end{example}


\subsection{Group representations}

\begin{definition}
	\index{Representation}
	A \emph{representation} (over the field $K$) of a group $G$ is a group homomorphism
	$\rho\colon G\to\GL(V)$, $g\mapsto\rho_g$, for some non-zero finite-dimensional 
        vector space $V$ (over $K$).
\end{definition}

\index{Representation!degree}
The \emph{degree} of the representation $\rho\colon G\to\GL(V)$ will be the dimension of $V$. Note that
since $V$ is finite-dimensional, say $\dim V=n$, after fixing a basis of $V$ we get a \emph{matrix representation} 
$\rho\colon G\to\GL(V)\simeq\GL_n(K)$ of $G$. Depending on the context, we will use either representations on vector spaces or matrix representations.  

\begin{example}
	We use group representations to show that 
	$G=\langle x,y:x^2=y^2=1\rangle$ is infinite. Note that
	\[
	\rho\colon G\to\GL_2(\C),
	\quad
	x\mapsto\begin{pmatrix}
		1&0\\
		0&-1	
	\end{pmatrix},\quad
 	y\mapsto\begin{pmatrix}
		1&1\\
		0&-1	
	\end{pmatrix},
 	\]
 	is a group homomorphism, as 
 	$\rho_x^2=\rho_y^2=\begin{pmatrix}
		1&0\\
		0&1	
	\end{pmatrix}$. We claim that the elements of the form $(xy)^n$ are
	different for all $n$. It is enough to show that   
	$(xy)^n=(xy)^m$, then $n=m$. Note that
	\[
	\rho_{xy}=\rho_x\rho_y=\begin{pmatrix}
		1&0\\
		0&-1	
	\end{pmatrix}
	\begin{pmatrix}
		1&1\\
		0&-1	
	\end{pmatrix}
	=\begin{pmatrix}
		1&1\\
		0&1	
	\end{pmatrix}.
	\]
	Thus $\rho_{xy}^n=\begin{pmatrix}
		1&n\\
		0&1	
	\end{pmatrix}$ and 
	$\rho_{xy}^m=\begin{pmatrix}
		1&m\\
		0&1	
	\end{pmatrix}$. Then 
	\[
	(xy)^n=(xy)^m\implies\rho_{xy}^n=\rho_{xy}^m\implies n=m.
	\] 
\end{example}

The previous example demonstrates the power of group representations, even for infinite groups. 
However, in this course, we will focus on complex finite-dimensional representations of finite groups.

\begin{convention}
    In Lectures \ref{6}, \ref{7}, and \ref{8}, we will focus on complex representations of finite groups, unless stated otherwise.
\end{convention}

\begin{example}
\index{Representation!trivial}
	If $G$ is a group, then $\rho\colon G\to\GL_1(\C)=\C^\times$, $g\mapsto 1$, 
	is a representation. This representation is known as the \emph{trivial (complex) representation} of $G$. 	
\end{example}

\begin{example}
	The sign of permutations yields a representation of $\Sym_n$. It is the group homomorphism
	$\Sym_n\to\C^\times$, $\sigma\mapsto\sgn(\sigma)$.  	
\end{example}

\begin{example}
	Let $G=\langle g:g^6=1\rangle$ be the cyclic group of order six. Then
	\[
	\rho\colon G\to\GL_2(\C),
	\quad 
	g\mapsto\begin{pmatrix}1&1\\-1&0\end{pmatrix}
	\] 
	is a group representation of degree two. 
\end{example}

%To prove the following result we need
%the \emph{minimal polynomial characterization of diagonazability",  
%see for example, \cite[Corollary 2.5.12]{zbMATH06049469}.
%
%\begin{proposition}
%    Let $G$ be a finite group and $\rho\colon G\to\GL_n(\C)$ 
%    be a representation. Then
%    each $\rho_g$ is diagonalizable.
%\end{proposition}
%
%\begin{proof}
%    Since $g$ is finite, $g^m=1$ for some $m\in\Z_{>0}$. This means that $\rho_g$ is a root of $X^m-1\in\C[X]$. Since
%    the roots of the polynomial $X^m-1$ are all different and $X^m-1$ factorizes linearly on $\C[X]$, it follows
%    that the minimal polynomial of $\rho_g$ also factorizes linearly in $\C[X]$. Hence $\rho_g$ is diagonalizable.
%\end{proof}
%
%In page \pageref{rho_diagonalizable} we will see an alternative  
%proof of the previous proposition. 
%
\begin{definition}
	\index{Invariant!map}
    Let $\rho\colon G\to\GL(V)$ and $\psi\colon G\to\GL(W)$ be
    representations. A linear map $T\colon V\to W$ is said to be \emph{invariant} 
    if the diagram
    \[\begin{tikzcd}
	V & V \\
	W & W
	\arrow["{\rho_g}", from=1-1, to=1-2]
	\arrow["T"', from=1-1, to=2-1]
	\arrow["{\psi_g}"', from=2-1, to=2-2]
	\arrow["T", from=1-2, to=2-2]
\end{tikzcd}
\]
    commutes for all $g\in G$, this means that 
    $\psi_gT=T\rho_g$ for all $g\in G$.
\end{definition}

It is convenient to introduce an alternative
notation for group representations. 
Let 
\[ 
\rho\colon G\to\GL(V),\quad g\mapsto \rho_g,
\]
be a representation. For
$g\in G$ and $v\in V$ write $g\cdot v=\rho_g(v)$. Then the following properties hold: 
\begin{enumerate}
	\item $1\cdot v=v$ for all $v\in V$, 
	\item $g\cdot (h\cdot v)=(gh)\cdot v$ for all $g,h\in G$ and $v\in V$, 
	\item $g\cdot (v+w)=g\cdot v+g\cdot w$ for all $g\in G$ and $v,w\in V$, and 
	\item $g\cdot (\lambda v)=\lambda (g\cdot v)$ for all $g\in G$, $\lambda\in\C$ and $v\in V$.  	
\end{enumerate}
Using this notation for the representations $\rho\colon G\to\GL(V)$ 
and $\psi\colon G\to\GL(W)$, a linear
map $T\colon V\to W$ is invariant if and only if
\[
T(g\cdot v)=g\cdot T(v)
\]
for all $v\in V$ and $g\in G$. 
Although we use the same notation for different representations, there is no risk of confusion. 

\begin{definition}
	\index{Representations!equivalence}
    The representations $\rho\colon G\to\GL(V)$ and $\psi\colon G\to\GL(W)$ are \emph{equivalent}
    if there exists a bijective linear 
    map $T\colon V\to W$ invariant with respect to $\rho$ and $\psi$.
\end{definition}

If the representations $\rho$ and 
$\psi$ are equivalent, we write 
$\rho\simeq\psi$. 

Now we can understand why we can alternatively use
representations or matrix representations. 

\begin{example}
\label{xca:change_of_basis}
Let $\rho\colon G\to\GL(V)$ be a representation, say, of degree 
$n=\dim V$. Choose a basis 
$B=\{v_1,\dots,v_n\}$ of $V$ and let 
\[
T\colon V\to\C^n,
\quad
\sum_{i=1}^n\lambda_iv_i\mapsto (\lambda_1,\dots,\lambda_n),
\]
be the isomorphism that takes coordinates 
for the basis $B$. For each $g\in G$, 
the composition of linear maps 
% https://q.uiver.app/#q=WzAsNCxbMCwwLCJcXENebiJdLFsyLDAsIlYiXSxbMSwwLCJWIl0sWzMsMCwiXFxDXm4iXSxbMCwyLCJUXnstMX0iXSxbMiwxLCJcXHJob19nIl0sWzEsMywiVCJdLFswLDMsIiIsMix7ImN1cnZlIjozfV1d
\[\begin{tikzcd}
	{\C^n} & V & V & {\C^n}
	\arrow["{T^{-1}}", from=1-1, to=1-2]
	\arrow["{\rho_g}", from=1-2, to=1-3]
	\arrow["T", from=1-3, to=1-4]
\end{tikzcd}
\]
produces a representation 
$\psi\colon G\to\GL(C^n)$
%\quad 
%g\mapsto T\rho T^{-1},
%\]
is a representation 
of $G$ equivalent to $\rho$.   
\end{example}

\begin{example}
    Let $G=\Z/n$. The representations
    \begin{align*}
    &\rho\colon G\to\GL_2(\C),\quad
    m\mapsto
    \begin{pmatrix}
        \cos(2\pi m/n) & -\sin(2\pi m/n)\\
        \sin(2\pi m/n) & \cos(2\pi m/n)
    \end{pmatrix},
    \shortintertext{and}
    &\psi\colon G\to\GL_2(\C),\quad
    m\mapsto
    \begin{pmatrix}
        e^{2\pi im/n} & 0\\
        0 & e^{-2\pi im/n}
    \end{pmatrix},
    \end{align*}
    are equivalent, as $\rho_mT=T\psi_m$ for all $m\in G$ if $T=\begin{pmatrix}
        i&-i\\
        1&1
    \end{pmatrix}$.
\end{example}

\begin{definition}
	\index{Invariant!subspace}
    Let $\rho\colon G\to\GL(V)$ be a representation. 
    A subspace $W$ of $V$ is said to be \emph{invariant} (with respect to $\rho$)
    if $\rho_g(W)\subseteq W$ for all $g\in G$.
\end{definition}

Let $\rho\colon G\to\GL(V)$ be a representation. Then $\{0\}$ and $V$ are always   
invariant subspaces of~$V$. Moreover, if $W\subseteq V$ is invariant, then
the map $\rho|_W\colon G\to\GL(W)$, $g\mapsto (\rho_g)|_W$, is a representation. The
map $\rho|_W$ is the \emph{restriction} of $\rho$ to $W$. 

\begin{example}
	If $\rho\colon G\to\GL(V)$ and $\psi\colon G\to\GL(W)$ are representations of a given group $G$ 
	and 
 $T\colon V\to W$
 is an invariant map, then the \emph{kernel} 
	\[
	\ker T=\{v\in V:T(v)=0\}
	\]
	is an invariant subspace of $V$ and the \emph{image} 
	\[
		T(V)=\{T(v):v\in V\}
	\]
	is an invariant subspace of $W$. 
\end{example}


\begin{definition}
\index{Subrepresentation}
    Let $\rho\colon G\to\GL(V)$
    be a representation. A \emph{subrepresentation} of $\rho$ is a restricted representation 
    of the form $\rho|_W\colon G\to\GL(W)$ for some invariant subspace $W$ of~$V$.
\end{definition}

%\begin{example}
%    Let $G=\langle g:g^3=1\rangle$ be the
%    cyclic group of order three
%    and
%    \[
%    \rho\colon G\to\GL_3(\R),
%    \quad
%    g\mapsto\begin{pmatrix}
%        0&1&0\\
%        0&0&1\\
%        1&0&0
%    \end{pmatrix}.
%    \]
%    The subspace
%    \[
%    W=\left\{
%    \begin{pmatrix}
%    x\\
%    y\\
%    z
%    \end{pmatrix}\in\R^3:x+y+z=0\right\}
%    \]
%    is a invariant subspace of $\R^3$.
%\end{example}

\begin{definition}
    \index{Representation!irreducible}
    A non-zero representation $\rho\colon G\to\GL(V)$ is \emph{irreducible} if
    $\{0\}$ and $V$ are the only invariant subspaces of $V$.
\end{definition}

Degree-one representations are irreducible. 

The irreducibility 
of a representation, of course, depends on the base field. 
In the following example, it is crucial using real representations.

\begin{example}
    Let $G=\langle g:g^3=1\rangle$ be the
    cyclic group of order three
    and
    \[
    \rho\colon G\to\GL_3(\R),
    \quad
    g\mapsto\begin{pmatrix}
        0&1&0\\
        0&0&1\\
        1&0&0
    \end{pmatrix}.
    \]
    Note that in this example we work with a real representation! 
    It is a routine calculation to prove that 
    \[
    W=\left\{
    \begin{pmatrix}
    x\\
    y\\
    z
    \end{pmatrix}\in\R^3:x+y+z=0\right\}
    \]
    is an invariant subspace of $\R^3$. We claim that $W$ 
    is irreducible. Let $S$ be a non-zero invariant subspace of 
    $W$ and let $s=\begin{pmatrix}x_0\\y_0\\z_0\end{pmatrix}\in S$ be a non-zero element. Then
    \[
    t=\begin{pmatrix}y_0\\z_0\\x_0\end{pmatrix}
    =\begin{pmatrix}
        0&1&0\\
        0&0&1\\
        1&0&0
    \end{pmatrix}
    \begin{pmatrix}x_0\\y_0\\z_0\end{pmatrix}\in S.
    \]
    We claim that $\{s,t\}$ is linearly independent. If not, there exists $\lambda\in\R$ such that
    $\lambda s=t$. Thus $\lambda x_0=y_0$, $\lambda y_0=z_0$ and $\lambda z_0=x_0$. This implies that
    $\lambda^3x_0=x_0$. Since $x_0\ne 0$ (because if $x_0=0$, then $y_0=z_0=0$, a contradiction), it follows that
    $\lambda=1$ and hence $x_0=y_0=z_0$, a contradiction because $x_0+y_0+z_0=0$.
    Therefore $\dim S=2$ and hence $S=W$.
\end{example}

What happens in the previous example if we work over complex numbers?

\begin{exercise}
    Let $\rho\colon G\to\GL(V)$ be a degree-two representation. Prove that
    $\rho$ is irreducible if and only if there is no common eigenvector for the $\rho_g$, $g\in G$.
\end{exercise}

The previous exercise can be used to show that the representation
$\Sym_3\to\GL_2(\C)$
of the symmetric group $\Sym_3$
given by
\[
(12)\mapsto\begin{pmatrix}
-1&-1\\0&1
\end{pmatrix},
\quad
(123)\mapsto\begin{pmatrix}
-1&-1\\
1&0
\end{pmatrix},
\]
is irreducible.

\begin{example}
\label{exa:direct_sum_reps}
Let $\rho\colon G\to\GL_n(\C)$ and $\psi\colon G\to\GL_m(\C)$ be representations. One defines
the \emph{direct sum} $\rho\oplus\psi$ of $\rho$ and $\psi$ as 
\[
\rho\oplus\psi\colon G\to\GL_{n+m}(\C),\quad
g\mapsto 
\begin{pmatrix}
\rho_g & 0\\
0 & \psi_g	
\end{pmatrix}.
\]
\end{example}

Let us describe the previous example without using matrix representations. If $V$ and 
$W$ are complex vector spaces, the (external) direct sum of $V$ and $W$ is defined
as the set $V\times W$ with the complex vector space structure given by
\[
(v,w)+(v_1,w_1)=(v+v_1,w+w_1),\quad
\lambda (v,w)=(\lambda v,\lambda w)
\]
for all $v,v_1\in V$, $w,w_1\in W$ and $\lambda\in\C$. 
If $\rho\colon G\to\GL(V)$ and $\psi\colon G\to\GL(W)$ are representations, 
the \emph{direct sum} $\rho\oplus\psi$ of $\rho$ and $\psi$ is then 
\[
\rho\oplus\psi\colon G\to\GL(V\oplus W),\quad
g\mapsto (\rho\oplus\psi)_g,
\]
where $(\rho\oplus\psi)_g\colon V\oplus W\to V\oplus W$, 
$(v,w)\mapsto (\rho_g(v),\psi_g(w))$. Note that both 
$V\simeq V\oplus\{0\}$ and $W\simeq \{0\}\oplus W$ are
invariant subspaces of $V\oplus W$. 

\begin{definition}
    \index{Representation!completely reducible}
    A representation $\rho\colon G\to\GL(V)$ is said to be 
    \emph{completely reducible}
    if $\rho$ can be decomposed as
    $\rho=\rho_1\oplus\cdots\oplus \rho_n$ for some irreducible
    representations $\rho_1,\dots,\rho_n$ of $G$. 
\end{definition}

Note that if $\rho\colon G\to\GL(V)$ is completely reducible and 
$\rho=\rho_1\oplus\cdots\oplus \rho_n$ for some irreducible representations 
$\rho_i\colon G\to\GL(V_i)$, $i\in\{1,\dots,n\}$, then 
each $V_i$ is an invariant subspace of $V$ and $V=V_1\oplus \cdots V_n$. 
Moreover, on some basis of $V$, the matrix  
$\rho_g$ can be written as 
\[
\rho_g=\begin{pmatrix}
(\rho_1)_g &  \\
& (\rho_2)_g  \\
&&\ddots\\
&&&(\rho_n)_g	
\end{pmatrix}.
\]

\begin{definition}
\index{Representation!decomposable}
\index{Representation!indecomposable}
A representation
$\rho\colon G\to\GL(V)$ is \emph{decomposable} if $V$ can be decomposed as $V=S\oplus T$
where $S$ and $T$ are non-zero invariant subspaces of $V$. 
\end{definition}

A representation is 
\emph{indecomposable} if it is not decomposable. 

\begin{exercise}
\label{xca:equivalence}
	Let $\rho\colon G\to\GL(V)$ and $\psi\colon G\to\GL(W)$ be equivalent representations.
	Prove the following facts:
	\begin{enumerate}
		\item If $\rho$ is irreducible, then $\psi$ is irreducible.
		\item If $\rho$ is decomposable, then $\psi$ is decomposable.
		\item If $\rho$ is completely reducible, then $\psi$ is completely reducible. 
	\end{enumerate}	
\end{exercise}

Since we are considering finite-dimensional vector spaces, our vector spaces admit 
an \emph{inner product}, that is a map $V\times V\to\C$, $(v,w)\mapsto\langle v,w\rangle$, 
that satisfies the following properties:
\begin{enumerate}
    \item $\langle y,x\rangle=\overline{\langle x,y\rangle}$ for all $x,y\in V$, where $\overline{a+bi}=a-bi$ denotes the complex conjugate.
    \item $\langle ax+by,z\rangle=a\langle x,z\rangle+b\langle y,z\rangle$ for all $x,y,z\in V$ and $a,b\in\C$.
    \item $\langle x,x\rangle>0$ for all $x\in V$ and $\langle x,x\rangle=0$ if $x=0$. 
\end{enumerate}

For example, if $V=\C^n$, $v=\colvec{3}{v_1}{\vdots}{v_n}\in V$ and $w=\colvec{3}{w_1}{\vdots}{w_n}\in V$, 
then the \emph{usual inner product} between $v$ and $w$ 
is given by
\begin{equation}
    \label{eq:inner_product}
V\times V\to\C,\quad 
(v,w)\mapsto v_1\overline{w_1}+\cdots+v_n\overline{w_n}.
\end{equation}

\begin{definition}
    \index{Representation!unitary}
    Let $V$ be a finite-dimensional complex vector space with inner product $(v,w)\mapsto\langle v,w\rangle$. 
    A representation $\rho\colon G\to\GL(V)$ is \emph{unitary} if
    $\langle \rho_gv,\rho_gw\rangle=\langle v,w\rangle$ for all $g\in G$ and $v,w\in V$.
\end{definition}

\begin{example}
Let $G$ be a finite group and $V=\C[G]$. The \emph{left regular representation}
of $G$ is the representation
\[
L\colon G\to\GL(V),
\quad
g\mapsto L_g,
\]
where $L_g(h)=gh$. With respect to the inner product
\[
\left\langle\sum_{g\in G}\lambda_gg,\sum_{g\in G}\mu_gg\right\rangle=\sum_{g\in G}\lambda_g\overline{\mu_g},
\]
the representation $L$ is unitary. Prove it! 
\end{example}



If $V$ is a vector space and $W$ is a subspace of $V$, 
\[
W^\perp = \{v\in V:\langle v,w\rangle=0\text{ for all $w\in W$}\}, 
\]
is called the 
\emph{orthogonal complement} of $W$. The following result is extremely important: 

\begin{proposition}
\label{pro:irr_or_dec}
Let $\rho\colon G\to\GL(V)$ be a unitary representation. Then $\rho$ is either
irreducible or decomposable.
\end{proposition}

\begin{proof}
	If $\rho$ is not irreducible, there exists an invariant subspace $W$ of $V$ such that
	$\{0\}\subsetneq W\subsetneq V$. Can we find an invariant complement? Yes, and this is because 
        our representation is unitary.  
	We know that $V=W\oplus W^{\perp}$ as complex vector spaces, where 
	$W^\perp\ne\{0\}$. So $\rho$ will be decomposable if we can prove that
	$W^\perp$ is an invariant subspace of $V$. 
	Let $v\in W^\perp$, $w\in W$ and $g\in G$. Then  
	\[
	\langle \rho_g(v),w\rangle=\langle v,\rho_g^{-1}(w)\rangle=0   
	\]
	since $\rho_g^{-1}(w)\in W$, as $W$ is an invariant subspace of $V$. This means that 
	$\rho_g(v)\in W^\perp$ and hence $V=W\oplus W^{\perp}$ is decomposable.  
\end{proof}

\begin{theorem}[Weyl's trick]
\index{Weyl's trick}
    Every representation $\rho\colon G\to\GL_n(\C)$ 
    of a finite group is unitary.
\end{theorem}

\begin{proof}
    Let $V=\C^n$ and 
    $V\times V\to\C$, $(v,w)\mapsto v\cdot w$, be an\footnote{It does not matter which inner product we have chosen. 
    We use this particular notation for our inner product of $V$ because  
    with this inner product, we will 
    produce a new inner product (denoted by $(v,w)\mapsto \langle v, w\rangle$) that will turn $\rho$ 
    into a unitary representation.} inner
    product on $V$. 
   
    A straightforward calculation shows that
    \[
    \langle v,w\rangle=\sum_{g\in G}(\rho_gv)\cdot (\rho_gw)
    \]
    is an inner product of $V$. For example, 
    \begin{align*}
    \langle v,w+w_1\rangle&=\sum_{g\in G}v\cdot(w+w_1)\\
    &=\sum_{g\in G}\left(v\cdot w+v\cdot w_1\right)\\
    &=\sum_{g\in G}v\cdot w+\sum_{g\in G}v\cdot w_1\\
    &=\langle v,w\rangle+\langle v,w_1\rangle 
    \end{align*}
    holds for all $v,w,w_1\in V$. 
    
    Since
    \begin{align*}
    \langle\rho_gv,\rho_gw\rangle&=\sum_{h\in G}(\rho_h\rho_gv)\cdot(\rho_h\rho_gw)\\
    &=\sum_{h\in G}(\rho_{hg}v)\cdot(\rho_{hg}w)=\sum_{x\in G}(\rho_xv)\cdot(\rho_xw)=\langle v,w\rangle,
    \end{align*}
    the representation $\rho$ is unitary.
\end{proof}

\begin{example}
    Let $A=\begin{pmatrix}
        -1 & -1\\
        -1 & 0
    \end{pmatrix}$. Since $A^3=I$, we can use this matrix to 
    represent the (multiplicative) cyclic group 
    \[
    C_3=\langle g:g^3=1\rangle
    \]
    of order three. Note that the map $1\mapsto g$ yields an isomorphism 
    $C_3\simeq \Z/3$. 
        
    The map $\rho\colon \Z/3\to\GL_2(\C)$, $1\mapsto A$, is a group homomorphism. In particular, we have a representation $C_3\to\GL(V)$, where $V=\C^2$ with the usual inner product given by \eqref{eq:inner_product}, that is 
    \[
    v\cdot w=v^T\overline{w}
    \]
    for $v,w\in V$. 
    For example, the inner product between the vectors $\colvec{2}{1}{i}$ and $\colvec{2}{1}{2}$
    is 
    \[
    (1,-i)\colvec{2}{1}{2}=1+2i.
    \]
    Since $A\in M_2(\R)$, we compute 
    \[
    \langle v,w\rangle=v^T\overline{w}+(Av)^T\overline{(Aw)}+(A^2v)^T\overline{(A^2w)}=v^T\left (I+A^TA+(A^2)^TA^2\right)\overline{w}
    \]
    and this is an invariant inner product on $V$. Note that 
    \[
        B=I+A^TA+(A^2)^TA^2=\begin{pmatrix}
            4&2\\
            2&4
        \end{pmatrix}.
    \]
\end{example}

\label{rho_diagonalizable}
Weyl's trick has several interesting consequences:  

\begin{corollary}
\label{cor:consequences}
	Let $\rho\colon G\to\GL(V)$ be a representation of a finite group $G$. 
 The following properties
	hold:
	\begin{enumerate}
		\item $\rho$ is equivalent to a unitary representation.
		\item $\rho$ is either irreducible or decomposable.
		\item Each $\rho_g$ is diagonalizable. 
	\end{enumerate}
\end{corollary}

\begin{proof}
	The first claim follows from Example \ref{xca:change_of_basis} and 
	Weyl's trick. The second claim, from 1) and
	Proposition \ref{pro:irr_or_dec}. Finally, 3) follows 
	from Weyl's trick immediately, as unitary matrices are diagonalizable.  
\end{proof}

\begin{exercise}
\label{xca:not_decomposable}
    If $G$ is an infinite group, it is no longer true that every non-zero representation
    is either irreducible or decomposable. Find an example.
\end{exercise}

For the previous exercise. consider the representation $\Z\to\GL_2(\C)$, 
$1\mapsto\begin{pmatrix}1&1\\0&1\end{pmatrix}$. 