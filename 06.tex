\chapter{}

\section*{\S6. Group representations}

We will spend 
four lectures studying the basics on complex representations of finite groups. 
The topics to cover are: 1) Basic definitions and examples, 2) characters, 3)
Schur's orthogonality relations and applications. 

\begin{definition}
	\index{Representation}
	A \textbf{representation} (over the field $K$) of a group $G$ is a group homomorphism
	$\rho\colon G\to\GL(V)$, $g\mapsto\rho_g$, for some vector space $V$ (over $K$).
\end{definition}

The \textbf{degree} of the representation $\rho\colon G\to\GL(V)$ will be the dimension of $V$. Note that
if $V$ is finite-dimensional, say $\dim V=n$, fixing a basis of $V$ we get a \textbf{matrix representation} 
$\rho\colon G\to\GL(V)\simeq\GL_n(K)$ of $G$.  

\begin{example}
	We use group representations to show that 
	$G=\langle x,y:x^2=y^2=1\rangle$ is infinite. Note that
	\[
	\rho\colon G\to\GL_2(\C),
	\quad
	x\mapsto\begin{pmatrix}
		1&0\\
		0&-1	
	\end{pmatrix},\quad
 	y\mapsto\begin{pmatrix}
		1&1\\
		0&-1	
	\end{pmatrix}
 	\]
 	is a group homomorphism, as 
 	$\rho_x^2=\rho_y^2=\begin{pmatrix}
		1&0\\
		0&1	
	\end{pmatrix}$. We claim that the elements of the form $(xy)^n$ are
	different for all $n$. It is enough to show that   
	$(xy)^n=(xy)^m$, then $n=m$. Note that
	\[
	\rho_{xy}=\rho_x\rho_y=\begin{pmatrix}
		1&0\\
		0&-1	
	\end{pmatrix}
	\begin{pmatrix}
		1&0\\
		0&-1	
	\end{pmatrix}
	=\begin{pmatrix}
		1&1\\
		0&1	
	\end{pmatrix}.
	\]
	Thus $\rho_{xy}^n=\begin{pmatrix}
		1&n\\
		0&1	
	\end{pmatrix}$ and 
	$\rho_{xy}^m=\begin{pmatrix}
		1&m\\
		0&1	
	\end{pmatrix}$. From this the claim follows.
\end{example}

The previous example shows the power of group representations, even for infinite groups.   
However, in this course we will work with complex finite-dimensional 
representations of finite
groups.  

\begin{example}
	If $G$ is a group, then $\rho\colon G\to\C^\times$, $g\mapsto 1$, is a representation. This representation is known as the \textbf{trivial representation} of $G$. 	
\end{example}

\begin{example}
	The sign yields a representation of $\Sym_n$. It is the group homomorphism
	$\Sym_n\to\C^\times$, $\sigma\mapsto\sgn(\sigma)$.  	
\end{example}


\begin{example}
	Let $G=\langle g:g^6=1\rangle$ be the cyclic group of order six. Then
	\[
	\rho\colon G\to\GL_2(\C),
	\quad 
	g\mapsto\begin{pmatrix}1&1\\-1&0\end{pmatrix}
	\] 
	is a group representation of degree two. 
\end{example}

\begin{proposition}
    Let $G$ be a finite group and $\rho\colon G\to\GL(V)$ be a representation of finite degree. Then
    each $\rho_g$ is diagonalizable.
\end{proposition}

\begin{proof}
    Let $n=\dim V$. Fix a basis of the finite-dimensional vector space $V$ and consider
    a matrix representation $\rho\colon G\to\GL_n(V)$.
	Since $g$ is finite, $g^m=1$ for some $m\in\Z_{>0}$. This means that $\rho_g$ is a root of $X^m-1\in\C[X]$. Since
    the roots of the polynomial $X^m-1$ are all different and $X^m-1$ factorizes linearly on $\C[X]$, it follows
    that the minimal polynomial of $\rho_g$ also factorizes linearly in $\C[X]$. Hence $\rho_g$ is diagonalizable.
\end{proof}

In page \pageref{rho_diagonalizable} we will see an alternative  
proof of the previous proposition. 

\begin{definition}
	\index{Invariant map}
    Let $\rho\colon G\to\GL(V)$ and $\psi\colon G\to\GL(W)$ be
    representations of a finite group $G$. A linear map $T\colon V\to W$ is said to be invariant
    if the diagram
    \[\begin{tikzcd}
	V & V \\
	W & W
	\arrow["{\rho_g}", from=1-1, to=1-2]
	\arrow["T"', from=1-1, to=2-1]
	\arrow["{\psi_g}"', from=2-1, to=2-2]
	\arrow["T", from=1-2, to=2-2]
\end{tikzcd}\]
    commutes, i.e. $\psi_gT=T\rho_g$ for all $g\in G$.
\end{definition}

\begin{definition}
	\index{Representations!equivalence}
    The representations $\rho\colon G\to\GL(V)$ and $\psi\colon G\to\GL(W)$ are \textbf{equivalent}
    if there exists a bijective map $T\colon V\to W$ invariant with respect to $\rho$ and $\psi$.
\end{definition}

\begin{example}
    Let $G=\Z/n$. The representations
    \begin{align*}
    &\rho\colon G\to\GL_2(\C),\quad
    m\mapsto
    \begin{pmatrix}
        \cos(2\pi m/n) & -\sin(2\pi m/n)\\
        \sin(2\pi m/n) & \cos(2\pi m/n)
    \end{pmatrix}
    \shortintertext{and}
    &\psi\colon G\to\GL_2(\C),\quad
    m\mapsto
    \begin{pmatrix}
        e^{2\pi im/n} & 0\\
        0 & e^{-2\pi im/n}
    \end{pmatrix}
    \end{align*}
    are equivalent, as $\rho_mT=T\psi_m$ for all $m\in G$ if $T=\begin{pmatrix}
        i&-i\\
        1&1
    \end{pmatrix}$.
\end{example}

\begin{definition}
	\index{Invariant subspace}
    Let $\rho\colon G\to\GL(V)$
    a representation. A subspace $W$ of $V$ is said to be \textbf{invariant} (with respect to $\rho$)
    if $\rho_g(W)\subseteq W$ for all $g\in G$.
\end{definition}

If $\rho\colon G\to\GL(V)$ is a representation and $W\subseteq V$ is invariant, then
the map $\rho|_W\colon G\to\GL(W)$, $g\mapsto (\rho_g)|_W$, is a representation. The
map $\rho|_W$ is the \textbf{restriction} of $\rho$ to $W$. 

\begin{example}
	If $\rho\colon G\to\GL(V)$ and $\psi\colon G\to\GL(W)$ are representations
	and $T\colon V\to W$ is an invariant map, then the \textbf{kernel} 
	\[
	\ker T=\{v\in V:T(v)=0\}
	\]
	is an invariant of $V$ and the \textbf{image} 
	\[
		T(V)=\{T(v):v\in V\}
	\]
	is an invariant subspace of $W$. 
\end{example}


\begin{definition}
\index{Subrepresentation}
    Let $\rho\colon G\to\GL(V)$
    a representation. A \textbf{subrepresentation} of $\rho$ is a restricted representation 
    of the form $\rho|_W\colon G\to\GL(W)$ for some invariant subspace $W$ of $V$.
\end{definition}

\begin{example}
    Let $G=\langle g:g^3=1\rangle$ be the
    cyclic group of order three
    and
    \[
    \rho\colon G\to\GL_3(\R),
    \quad
    g\mapsto\begin{pmatrix}
        0&1&0\\
        0&0&1\\
        1&0&0
    \end{pmatrix}.
    \]
    The subspace
    \[
    W=\left\{
    \begin{pmatrix}
    x\\
    y\\
    z
    \end{pmatrix}\in\R^3:x+y+z=0\right\}
    \]
    is a invariant subspace of $\R^3$.
\end{example}

\begin{definition}
    \index{Representation!irreducible}
    A representation $\rho\colon G\to\GL(V)$ is \textbf{irreducible} if
    $\{0\}$ and $V$ are the only invariant subspaces of $V$.
\end{definition}

Degree-one representations are irreducible.

\begin{example}
        Let $G=\langle g:g^3=1\rangle$ be the
    cyclic group of order three
    and
    \[
    \rho\colon G\to\GL_3(\R),
    \quad
    g\mapsto\begin{pmatrix}
        0&1&0\\
        0&0&1\\
        1&0&0
    \end{pmatrix}.
    \]
    We claim that the invariant subspace
    \[
    W=\left\{
    \begin{pmatrix}
    x\\
    y\\
    z
    \end{pmatrix}\in\R^3:x+y+z=0\right\}\subseteq\R^3
    \]
    is irreducible. Let $S$ be a non-zero invariant subspace of $W$ and let $s=\begin{pmatrix}x_0\\y_0\\z_0\end{pmatrix}\in S$ be a non-zero element. Then
    \[
    t=\begin{pmatrix}y_0\\z_0\\x_0\end{pmatrix}
    =\begin{pmatrix}
        0&1&0\\
        0&0&1\\
        1&0&0
    \end{pmatrix}
    \begin{pmatrix}x_0\\y_0\\z_0\end{pmatrix}\in S.
    \]
    We claim that $\{s,t\}$ are linearly independent. If not, there exists $\lambda\in\R$ such that
    $\lambda s=t$. Thus $\lambda x_0=y_0$, $\lambda y_0=z_0$ and $\lambda z_0=x_0$. This implies that
    $\lambda^3x_0=x_0$. Since $x_0\ne 0$ (because if $x_0=0$, then $y_0=z_0=0$, a contradiction), it follows that
    $\lambda=1$ and hence $x_0=y_0=z_0$, a contradiction because $x_0+y_0+z_0=0$.
    Therefore $\dim S=2$ and hence $S=W$.
\end{example}

\begin{exercise}
    Let $\rho\colon G\to\GL(V)$ be a degree-two representation. Prove that
    $\rho$ is irreducible if and only if there is no common eigenvector for the $\rho_g$, $g\in G$.
\end{exercise}

The previous exercise can be used to show that the representation
$\Sym_3\to\GL_2(\C)$
of the symmetric group $\Sym_3$
given by
\[
(12)\mapsto\begin{pmatrix}
-1&-1\\0&1
\end{pmatrix},
\quad
(123)\mapsto\begin{pmatrix}
-1&-1\\
1&0
\end{pmatrix}
\]
is irreducible.

\begin{example}
Let $\rho\colon G\to\GL(V)$ and $\psi\colon G\to\GL(W)$ be representations. One defines
the \textbf{direct sum} $\rho\oplus\psi$ of $\rho$ and $\psi$ as 
\[
\rho\oplus\psi\colon G\to\GL(V\oplus W),\quad
g\mapsto (\rho\oplus\psi)_g,
\]
where $(\rho\oplus\psi)_g\colon V\oplus W\to V\oplus W$ is given by 
$(v,w)\mapsto (\rho_g(v),\psi_g(w))$.
\end{example}

\begin{definition}
    \index{Representation!completely irreducible}
    A representation $\rho\colon G\to\GL(V)$ is \textbf{completely irreducible}
    if $V$ can be decomposed as
    $V=V_1\oplus\cdots\oplus V_n$, where each $V_i$ is a invariant subspace of $V$ and
    each $\rho|_{V_i}$ is irreducible.
\end{definition}

Since we are considering finite-dimensional vector spaces, our vector spaces are
Hilbert spaces, so they have
an inner product $V\times V\to\C$, $(v,w)\mapsto\langle v,w\rangle$.

\begin{definition}
    \index{Representation!unitary}
    A representation $\rho\colon G\to\GL(V)$ is \textbf{unitary} if
    $\langle \rho_gv,\rho_gw\rangle=\langle v,w\rangle$ for all $g\in G$ and $v,w\in V$.
\end{definition}

\begin{definition}
\index{Representation!decomposable}
\index{Representation!indecomposable}
A representation
$\rho\colon G\to\GL(V)$ is \textbf{decomposable} if $V$ can be decomposed as $V=S\otimes T$
where $S$ and $T$ are non-zero invariant subspaces of $V$. 
\end{definition}

A representation is 
\textbf{indecomposable} if it is not decomposable. 

\begin{exercise}
Let $\rho\colon G\to\GL(V)$ be a unitary representation. Prove that $\rho$ is either
irreducible or decomposable.
\end{exercise}

\begin{example}
Let $G$ be a finite group and $V=\C[G]$. The \textbf{left regular representation}
of $G$ is the representation
\[
L\colon G\to\GL(V),
\quad
g\mapsto L_g,
\]
where $L_g(h)=gh$. With the inner product
\[
\left\langle\sum_{g\in G}\lambda_gg,\sum_{g\in G}\mu_gg\right\rangle=\sum_{g\in G}\lambda_g\overline{\mu_g}
\]
the representation $L$ is unitary.
\end{example}

\begin{proposition}[Weyl's trick]
\index{Weyl's trick}
    Every representation of a finite group is equivalent to a unitary representation.
\end{proposition}

\begin{proof}
    Let $\rho\colon G\to\GL(V)$ and $V\times V\to\C$, $(v,w)\mapsto\langle v,w\rangle_0$ be an inner
    product on $V$. A straighforward calculation shows that
    \[
    \langle v,w\rangle=\sum_{g\in G}\langle\rho_gv,\rho_gw\rangle_0
    \]
    is an inner product of $V$. Since
    \begin{align*}
    \langle\rho_gv,\rho_gw\rangle&=\sum_{h\in G}\langle\rho_h\rho_gv,\rho_h\rho_gw\rangle_0\\
    &=\sum_{h\in G}\langle\rho_{hg}v,\rho_{hg}w\rangle_0=\sum_{x\in G}\langle\rho_xv,\rho_xw\rangle_0=\langle v,w\rangle,
    \end{align*}
    the representation $\rho$ is unitary.
\end{proof}

\label{rho_diagonalizable}
Weyl's trick has several interesting corollaries. Let $\rho\colon G\to\GL(V)$ be a representation
of a finite group $G$. Then 1) every non-zero representation is either
irreducible or decomposable, and 2) every $\rho_g$ is diagonalizable
(as unitary operators are diagonalizable).

\begin{exercise}
    If $G$ is an infinite group it is not longer true that every non-zero representation
    is either irreducible or decomposable. Find an example.
\end{exercise}

Recall that we only consider finite-dimensional representations of finite groups.

\begin{theorem}[Maschke]
\index{Maschke theorem}
    Every representation of a finite group is completely reducible.
\end{theorem}

\begin{proof}
    Let $G$ be a finite group and $\rho\colon G\to\GL(V)$ be a representation of $G$. We proceed
    by induction on $\dim V$.
    If $\dim V=1$, the result is trivial, as degree-one representations are irreducible. Assume that
    the result holds for representations of degree $\leq n$. Let $\rho\colon G\to\GL(V)$ be a representation
    of degree $n+1$. If $\rho$ is irreducible, we are done. If not, write $V=S\oplus T$, where $S$ and $T$
    are non-zero invariant subspaces. Since $\dim S<\dim V$ and $\dim T<\dim V$, it follows from
    the inductive hypothesis that
    both $S$ and $T$ are completely irreducible. Thus $V$ is completely irreducible.
\end{proof}

\begin{example}
    Let $G=\Sym_3$ and $\rho\colon G\to\GL_3(\C)$ be the representation given by
    \[
    (12)\mapsto\begin{pmatrix}
    0&1&0\\
    1&0&0\\
    0&0&1
    \end{pmatrix},\quad
    (123)\mapsto\begin{pmatrix}
    0&0&1\\
    1&0&0\\
    0&1&0
    \end{pmatrix}
    \]
    Then $\rho_g$ is unitary for all $g\in G$ (because $\rho_{(12)}$ and $\rho_{(123)}$ are both
    unitary). Moreover,
    \[
    S=\left\langle \begin{pmatrix}
    1\\1\\1
    \end{pmatrix}
    \right\rangle,
    \quad
    T=S^{\perp}=\left\langle
    \begin{pmatrix}
    -1\\1\\0
    \end{pmatrix},
    \begin{pmatrix}
    0\\-1\\1
    \end{pmatrix}
    \right\rangle,
    \]
    are irreducible invariant subspaces of $V=\C^3$. A direct calculation shows that
    in the orthogonal basis $\left\{\begin{pmatrix}
    1\\1\\1
    \end{pmatrix},
    \begin{pmatrix}
    -1\\1\\0
    \end{pmatrix},
    \begin{pmatrix}
    0\\-1\\1
    \end{pmatrix}
    \right\}$
    the matrices $\rho_{(12)}$ and $\rho_{(123)}$ can be written as
    \[
    \rho_{(12)}=\begin{pmatrix}
        1&0&0\\
        0&-1&1\\
        0&0&1
    \end{pmatrix},
    \quad
    \rho_{(123)}=
    \begin{pmatrix}
        1&0&0\\
        0&0&-1\\
        0&1&-1
    \end{pmatrix}.
    \]
\end{example}

\begin{exercise}
Let $G$ be a finite group.
Prove that there is a bijection between degree-one representations of $G$ and
degree-one representations of $G/[G,G]$.
\end{exercise}

\begin{lemma}[Schur]
    Let $\rho\colon G\to\GL(V)$ and $\psi\colon G\to\GL(W)$ be irreducible representations. If 
    $T\colon V\to W$ is a non-zero invariant map, then $T$ is bijective.  
\end{lemma}

\begin{proof}
    Since $T$ is non-zero and $\ker T$ is an invariant subspace of $V$, it follows that $\ker T=\{0\}$, as $\rho$ is irreducible. Thus 
    $T$ is injective. Since $T(V)$ is a non-zero invariant subspace of $W$, it follows from the fact that $\psi$ is irreducible 
    that $T$ is surjective. Therefore $T$ 
    is bijective.  
\end{proof}

Two applications:

\begin{proposition}
    If $\rho\colon G\to\GL(V)$ is an irreducible representation and $T\colon V\to V$ is invariant, then 
    $T=\lambda\id$ for some $\lambda\in\C$. 
\end{proposition}

\begin{proof}
    Let $\lambda$ be an eigenvector of $T$. Then $T-\lambda\id$ is invariant and it is 
    not bijective. Thus $T-\lambda\id=0$ by Schur's lemma.
\end{proof}

\begin{proposition}
    Let $G$ be a finite abelian group. If $\rho\colon G\to\GL(V)$ is an irreducible representation, then
    $\dim V=1$. 
\end{proposition}

\begin{proof}
    Let $h\in G$. Note that since $G$ is abelian, $T=\rho_h$ is invariant:
    \[
    T\rho_g=\rho_h\rho_g=\rho_{hg}=\rho_{gh}=\rho_g\rho_h=\rho_gT.
    \]
    By Schur's lemma, there exists $\lambda_h\in\C$ such that $\rho_h=\lambda_h\id$. If $v\in V\setminus\{0\}$, 
    then $V=\langle v\rangle$, as $\rho$ is irreducible. 
\end{proof}
