\chapter{Some solutions}
\addcontentsline{toc}{chapter}{Some solutions}

% \section*{Lecture 1}
\section*{Lecture 2}

\begin{sol}{xca:x_unit}
If $x$ is a unit, then $yx=xy=1$ for some $y\in R$ and hence $ryx=r$ for all $r\in R$. Conversely, 
if $R=(x)$, then, in particular, $1\in R=(x)$ and hence $1=xy$ and $1=zx$ for some $y,z\in R$. Now
$z=z1=z(xy)=(zx)y=1y=1$ and hence $x$ is a unit. 
\end{sol}

% \section*{Lecture 3}
\section*{Lecture 4}

\begin{sol}{xca:principal=>noetherian}
	Let $I_1\subsetneq I_2\subsetneq$ be a sequence of ideals of $R$.  
	Since $R$ is principal, each $I_j$ is principal, 
	say $I_j=(a_j)$ for some $a_j\in R$, so the sequence is of the form
	\[
	(a_1)\subsetneq (a_2)\subsetneq\cdots.
	\]
	Since $I=\cup_{i\geq1}(a_i)$ is an ideal of $R$, 
	there exists $x\in R$ such that $I=(x)$. Since $x\in (a_n)$ for some $n\in\Z_{>0}$, 
	it follows that $(a_k)\subseteq I=(x)\subseteq (a_n)$ for all $k\in\Z_{>0}$. 
\end{sol}
	
\section*{Lecture 5}


\begin{sol}{xca:content}
	Let $d$ be 
	the greatest common divisor
	of the coefficients of $f$. 
	Since $a$ divides all the coefficients of $f$, it follows that $a$ divides $d$. If $p$ is a prime that
	divides $d$, then either $p$ divides $a$ or $p$ divides all the coefficients of the primitive
	polynomial $f_1$. It follows that $d$ divides $a$.    
\end{sol}

\begin{sol}{xca:Gauss}
\begin{enumerate}
	\item Let $f=\sum_{i=0}^na_iX^i$, $g=\sum_{i=0}^mb_iX^i$. Suppose that 
	$fg$ is not primitive and let $p$ be a prime number dividing all the coefficients  
	of $fg$. Since both $f$ and $g$ are primitive, there exist $i\in\{0,\dots,n\}$ 
 	and $j\in\{0,\dots,m\}$ minimal such that $p\nmid a_i$ and $p\nmid b_j$. 
 	If $c_{i+j}$ is the coefficient of $X^{i+j}$ in $fg$, then  
	\[
 	c_{i+j}=\sum_{k>i}a_kb_{i+j-k}+\sum_{k<i}a_kb_{i+j-k}+a_ib_j.
 	\]
 	Thus $p$ divides $\sum_{k>i}a_kb_{i+j-k}+\sum_{k<i}a_kb_{i+j-k}$ and does not divide
 	the integer $a_ib_j$, that is $p$ does not divide $c_{i+j}$, a contradiction.
	\item Suppose that $f$ is irreducible in $\Z[X]$. If $f$ is not primitive, then
	$f=af_1$ for some $a\in\Z\setminus\{-1,1\}$ and some primitive polynomial $f_1$, a contradiction 
	to the irreducibility of $f$. Let us prove that $f$ is irreducible in $\Q[X]$. If not, say 
	$f=gh$ for some $g,h\in\Q[X]$ of positive degree. After multiplying by a 
	suitable rational number, we may assume that  
	\[
 	f=\frac{a}{b}g_1h_1,
 	\]
 	where $\gcd(a,b)=1$ and $g_1,h_1\in\Z[X]$ are primitive polynomials, that is 
 	\[
 	bf=ag_1h_1.
 	\]
 	The greatest common divisor of the coefficients of $bf$ is $b$. 
	Since $g_1h_1$ is primitive by the previous item, the greatest common divisor 
	of the coefficients of $ag_1h_1$ is $a$. Thus $a=b$ or $a=-b$, 
  	that is $f=g_1h_1$ or $f=-g_1h_1$ in $\Z[X]$.  
\end{enumerate}
\end{sol}

\begin{sol}{xca:Z[X]_UFD}
	Since $\Z$ is noetherian, it follows that $\Z[X]$ is noetherian by Hilbert's theorem. 
	Then $\Z[X]$ admits factorizations. We need to show that the factorization into irreducibles is unique. 
	Let $f\in\Z[X]$ be non-zero and
	assume that 
	\[
	f=f_1\cdots f_k=g_1\cdots g_l
	\]
	be factorizations of $f$ into non-constant irreducibles integer polynomials. Since
	$f_1,\dots,f_k$ and $g_1,\dots,g_k$ 
	are irreducible in $\Q[X]$ and $\Q[X]$ is a unique factorization domain, it follows
	that $k=s$ and there exits $\sigma\in\Sym_k$ such that $g_i$ and $h_{\sigma(i)}$ are
	associate for all $i\in\{1,\dots,k\}$. After reordering we may assume that 
	for each $i\in\{1,\dots,k\}$ there
	exists $a_i/b_i\in\Q$ such that $b_ig_i=a_ih_i$. Since both $g_i$ and $h_i$ are
	irreducible integer polynomials, it follows from Exercise \ref{xca:Gauss} that
	both $g_i$ and $h_i$ are primitive. By Exercise \ref{xca:content}, 
	$a_i$ (resp. $b_i$) is the greatest common divisor of the coefficients 
	of $a_ih_i$ (resp. $b_ig_i$). This implies that $a_i$ and $b_i$ are associate, so 
	$a_i/b_i$ is a unit. Hence $g_i$ and $h_i$ are associate in $\Z[X]$.     
\end{sol}

\begin{sol}{xca:Z[i]irreducibles}
	Let $x\in\Z[i]$ be irreducible. Since 
	$x\not\in\mathcal{U}(\Z[i])$, it follows that $N(x)>1$. Write
	the integer $N(x)$ as a product of (not necessarily different) primes $p_1,\dots,p_k$. Since 
	$x$ divides $N(x)=x\overline{x}=p_1\cdots p_k$ and $x$ is prime, 
	$x$ divides $p_i$ for some $i\in\{1,\dots,k\}$. 
	
	If $p_i$ is irreducible in $\Z[i]$, 
	then $x$ and $p_i$ are associate, so $x\in\{p_i,-p_i,ip_i,-ip_i\}$ for some
	prime number $p_i$ that is irreducible in $\Z[i]$. 
	
	If $p_i$ is not irreducible, say $p_i=\alpha\beta$ for some $\alpha,\beta\in\Z[i]$ that
	are not units of $\Z[i]$. Then 
	$p_i^2=N(p_i)=N(\alpha)N(\beta)$ with $N(\alpha)>1$ and $N(\beta)>1$. The unique factorization
	of $\Z$ implies that $N(\alpha)=N(\beta)=p_i$ and hence $\alpha$ and $\beta$ 
	are irreducible in $\Z[i]$. Thus $x$ divides 
	$\alpha\beta$. If $x$ divides the irreducible $\alpha$, 
	then $x$ and $\alpha$ are associate in $\Z[i]$. This means that 
	$\alpha\in\{x,-x,xi,-xi\}$ and $N(x)=N(\alpha)=p_i$ for some prime number $p_i$.  
	
	Combining the previous paragraphs with Fermat's theorem it follows that 
	the irreducibles of $\Z[i]$ are $1+i$, $1-i$, $-1+i$ and $-1-i$, 
	$p$, $-p$, $pi$ and $-p$ for some prime number $p\in\Z$ such that $p\equiv3\bmod 4$, and 
	the elements $x\in\Z[i]$ such that $N(x)=p$ where $p\in\Z$ is a prime number such that $p\equiv1\bmod 4$. 
\end{sol}

\section*{Lecture 6}

\begin{sol}{xca:fx=cx}
    Consider $\R$ as a $\Q$-vector space. 
    Since $\sqrt{2}\not\in\Q$, the subset $\{1,\sqrt{2}\}$ of $\R$ 
    is linearly independent 
    over $\Q$. Extend $\{1,\sqrt{2}\}$ 
    to a basis $B$ of $\R$ as a $\Q$-vector space. The
    linear map $f\colon\R\to\R$ such that
    $f(1)=\sqrt{2}$, $f(\sqrt{2})=1$ and $f(b)=b$ for all $b\in B\setminus\{1,\sqrt{2}\}$ 
    is not of the form $f(x)=\lambda x$. 
\end{sol}

\begin{sol}{xca:Rn=R}
    Let $\{x_i:i\in I\}$ be a basis of $\R$ as a $\Q$-vector space and
    let $\{e_1,\dots,e_n\}$ be the standard basis of $\R^n$. Since
    $I$ and $I\times\{e_1,\dots,e_n\}$ have the same cardinality, 
    there exists a bijective map $f\colon I\to I\times\{e_1,\dots,e_n\}$. Since
    the set $\{x_ie_j:i\in I,1\leq j\leq n\}$ is a basis of $\R^n$, it follows
    that the bases of $\R^n$ and $\R$ have the same cardinality and hence
    $\R^n$ and $\R$ are isomorphic as vector spaces. 
\end{sol}

\begin{sol}{xca:aut}
    Let $G$ be a group such that $\Aut(G)=\{\id\}$. 
    Then $G$ is abelian, as the map $G\to G$, $x\mapsto gxg^{-1}$,   
    is an automorphism of $G$. Since $G$ is abelian, 
    the map $G\to G$, $x\mapsto x^{-1}$, is an automorphism of $G$. Since 
    the automorphism group of $G$ is trivial, 
    $x=x^2$ for all $x\in G$. It follows that $G$ is a vector space
    over the field $\Z/2$ of two elements. Let $B$ be   
    a basis of $G$ over $\Z/2$. If $|I|\geq 2$, 
    the automorphism of $G$ that exchanges to basis elements of $B$ and  
    and fixes all other elements of $B$ would be non-trivial. Hence $|I|=1$ and 
    $G$ is either trivial of cyclic of order two. Both groups have trivial
    automorphism group. 
\end{sol}

\section*{Lecture 7}

\begin{sol}{xca:freshman_dream}
    The tricky part of the exercise is to prove that 
    if $p$ is a prime and $1\leq k\leq p^n-1$, then $\binom{p^n}{k}$ is divisible by $p$. The rest is routine. 
    Write
    \[
    \binom{p^n}{k}=\frac{(p^n)!}{k!(p^n-k)!}=\frac{p^n}{k}\binom{p^n-1}{k-1}.
    \]
    Then
    \[
    k\binom{p^n}{k}=p^n\binom{p^n-1}{k-1},
    \]
    that is $p^n$ divides $k\binom{p^n}{k}$. 
    
    If $\gcd(p,k)=1$, then it follows that $p$ divides $\binom{p^n}{k}$ by unique decomposition
    of every integer as a product of primes. We may assume then that $\gcd(p,k)\ne1$, 
    say $k=p^\alpha m$ for some integer $m$ not divisible by $p$. Then
    \[
    p^n\binom{p^n-1}{k-1}=k\binom{p^n}{k}=p^{\alpha}m\binom{p^n}{k}
    \]
    and hence 
    \[
    p^{n-\alpha}\binom{p^n-1}{k-1}=m\binom{p^n}{k}.
    \]
    Since $k<p^n$, it follows that
    $m-\alpha\geq 1$. Thus $p$ divides $m\binom{p^n}{k}$ and hence
    $p$ divides $\binom{p^n}{k}$ because $p$ and $m$ are coprime. 
\end{sol}

\section*{Lecture 8}

\begin{sol}{xca:RC3}
Let $G=\langle g:g^3=1\rangle$ be the cyclic group of order three and $\omega$ be a primitive cubic root of one.  
The map $G\to \R\times\C$, $g\mapsto (1,\omega)$, extends to an algebra
homomorphism $\varphi\colon\R[G]\to\R\times\C$. Since 
$\dim_\R\R\times\C=\dim\R[G]=3$, it follows that $\varphi$ is a bijective. 
\end{sol}

% \begin{sol}{xca:projector}
% \end{sol}

% \begin{sol}{xca:submodules}
% \end{sol}

% \begin{sol}{xca:commuting}
% \end{sol}

% \begin{sol}{xca:Hom}
% \end{sol}

% \section*{Lecture 9}

% \begin{sol}{xca:linear_algebra}
% Para demostrar la primera afirmación procederemos por inducción en la cantidad de generadores de $M$. Si $M=(m)$, entonces
% $\{m\}$ es base pues $\{m\}$ es linealmente independiente: si $r\cdot m=0$ y $r\ne 0$, entonces
% \[
% m=1\cdot m=(r^{-1}r)\cdot m=r^{-1}\cdot (r\cdot m)=0.
% \]
% Si vale para $k-1$ generadores, sea $M=(m_1,\dots,m_k)$. Si $\{m_1,\dots,m_k\}$ no es linealmente
% independiente, entonces existen $r_1,\dots,r_k\in R$ no todos cero tales
% que
% \[
% r_1\cdot m_1+\cdots+r_k\cdot m_k=0.
% \]
% Sin perder generalidad podemos suponer que $r_k\ne 0$. Entonces
% \[
% v_k=\sum_{i=1}^{k-1} (r_k^{-1}r_i)\cdot m_i\in (m_1,\dots,m_{k-1}).
% \]
% Como entonces $M=(m_1,\dots,m_k)=(m_1,\dots,m_{k-1})$, la hipótesis inductiva implica que
% $M$ es libre. 

% Vamos a demostrar ahora que todo
% conjunto $X$ linealmente independiente puede extenderse a una base. 	Sea $X=\{x_1,\dots,x_k\}$ tal que $M=(X)$. 
% Como $M\ne\{0\}$, sin perder generalidad podemos suponer que $x_1\ne 0$. Como $R$ es de división,
% el conjunto $\{x_1\}$ es linealmente independiente, pues si $r\ne 0$ y $r\cdot x_1=0$, entonces 
% \[
% x_1=1\cdot r=(r^{-1}r)\cdot x_1=r^{-1}\cdot (r\cdot x)=r^{-1}\cdot 0=0.
% \]
% Sea $Y=\{y_1,\dots,y_l\}$ un subconjunto de $X$ maximal tal que $Y$ es linealmente independiente. Veamos que $X\subseteq (Y)$. Sea $x\in X$. Si $x\not\in Y$,  
% entonces, como $Y\subseteq Y\cup \{x\}$, la maximalidad de $Y$ implica que $\{x\}\cup Y$ es linealmente dependiente, es decir
% que existen $r,r_1,\dots,r_k\in R$ no todos cero tales que
% \[
% r\cdot x+\sum_{i=1}^l r_i\cdot y_i=0.
% \]
% Si $r=0$, entonces $r_1=\cdots=r_l=0$ porque los $y_j$ son linealmente independientes, una contradicción. Luego $r\ne 0$ y entonces
% \[
% x=-\sum_{i=1}^l (-r^{-1}r_i)\cdot y_i\in (Y).
% \]
% Luego $X\subseteq (Y)$. En conclusión $Y$ es una base de $M$ pues $M=(Y)$ 
% y además $Y$ es linealmente independiente. 

% Demostremos que dos bases finitas cualesquiera tienen la misma cantidad de elementos. 
% Para eso es suficiente demostrar
% que si $X$ e $Y$ son conjuntos finitos linealmente independientes
% tales que $(X)\subseteq (Y)$, entonces $|X|\leq |Y|$. Supongamos que $|X|=k$ e $|Y|=l$. 
% Procederemos por inducción en $l$. Si $l=1$ y $k>1$, entonces
% exiten $r_1,r_2\in R$ tales que $x_1=r_1\cdot y_1$ y $x_2=r_2\cdot y_1$. Luego
% \[
% x_2=r_2\cdot y_1=r_2\cdot (r_1^{-1}\cdot x_1)=(r_2r_1^{-1})\cdot x_1,
% \]
% una contradicción pues $\{x_1,x_2\}$ es linealmente independiente. 
% Supongamos ahora que el resultado 
% es verdadero para $l-1$ y sea $l=|Y|$. Para cada $j$ escribimos
% \[
% x_j = \sum_{i=1}^l r_{ji}\cdot y_i,
% \]
% donde $r_{ji}\in R$. Si $r_{j1}\ne 0$ para todo $j$, entonces 
% $x_j=\sum_{i=2}^l r_{ji}\cdot y_i$ para todo $j$ y luego $(X)\subseteq (y_2,\dots,y_l)$, que implica que
% $|X|\leq l-1<l=|Y|$. Si existe $j$ tal que $r_{j1}\ne 0$, sin perder generalidad podemos
% suponer que $r_{11}\ne 0$. Para cada $j\in\{2,\dots,k\}$ sea
% \[
% z_j = x_j-(r_{j1}r_{11}^{-1})\cdot x_1.
% \]
% Como $z_j\in (y_2,\dots,y_l)$ para todo $j$ y los $z_j$ son linealmente independientes, 
% la hipótesis inductiva impica que $k-1\leq l-1$, es decir $|X|\leq |Y|$.  
% \end{sol}

% \section*{Lecture 10}
% \section*{Lecture 11}
% \section*{Lecture 12}
%\section*{Lecture 13}

