\chapter{Characters}

\begin{definition}
	\index{Character}
	Let $\rho\colon G\to\GL(V)$ be a representation. The \textbf{character} of $\rho$ 
	is the map $\chi_\rho\colon G\to\C$, $g\mapsto\trace\rho_g$. 	
\end{definition}

If a representation $\rho$ is irreducible, its character is said to be an 
\textbf{irreducible character}. The \textbf{degree} of a character is the degree of the affording
representation. 

\begin{proposition}
	Let $\rho\colon G\to\GL(V)$ be a representation, $\chi$ be its character and $g\in G$.
	The following statements hold:
	\begin{enumerate}
		\item $\chi(1)=\dim V$. 
		\item $\chi(g)=\chi(hgh^{-1})$ for all $h\in G$.
		\item $\chi(g)$ is the sum of $\chi(1)$ roots of one of order $|g|$. 
		\item $\chi(g^{-1})=\overline{\chi(g)}$. 
		\item $|\chi(g)|\leq\chi(1)$.  
	\end{enumerate} 
\end{proposition}

\begin{proof}
	The first statement is trivial. 	To prove 2) note that
	\[
	\chi(hgh^{-1})=\trace(\rho_{hgh^{-1}})=\trace(\rho_h\rho_g\rho_h^{-1})=\trace\rho_g=\chi(g).
	\]
	Statement 3) follows from the fact that the trace of $\rho_g$ is the sum
	of the eigenvalues of $\rho_g$ and these numbers are roots of the polynomial
	$X^{|g|}-1\in\C[X]$. To prove 4) write $\chi(g)=\lambda_1+\cdots+\lambda_k$, where 
	the $\lambda_j$ are roots of one. Then
	\[
	\overline{\chi(g)}=\sum^k_{j=1}\overline{\lambda_j}
	=\sum_{j=1}^k\lambda_j^{-1}
	=\trace(\rho_g^{-1})
	=\trace(\rho_{g^{-1}})
	=\chi(g^{-1}).
	\] 
	Finally, we prove 5). Use 3) to write $\chi(g)$ as the sum of
	$\chi(1)$ roots of one, say $\chi(g)=\lambda_1+\cdots+\lambda_k$ for
	$k=\chi(1)$. Then 
	\[
	|\chi(g)|=|\lambda_1+\cdots+\lambda_k|\leq |\lambda_1|+\cdots+|\lambda_k|
	=\underbrace{1+\cdots+1}_{\text{$k$-times}}=k.
	\]
\end{proof}

If two representations are equivalent, their characters are equal.

\begin{definition}
	Let $G$ be a group and 
	$f\colon G\to\C$ be a map. Then $f$ is a \textbf{class function} if
	$f(g)=f(hgh^{-1})$ for all $g,h\in G$. 	
\end{definition}

Characters are class functions. 

\begin{proposition}
    If $\rho\colon G\to\GL(V)$ and
    $\psi\colon G\to\GL(W)$ are representations, then
    $\chi_{\rho\oplus\psi}=\chi_\rho+\chi_\psi$.
\end{proposition}

\begin{proof}
  For $g\in G$, it follows that 
  $(\rho\oplus\psi)_g=
  \begin{pmatrix}
    \rho_g & 0\\ 
    0 & \psi_g
  \end{pmatrix}$. 
  Thus  
  \[
    \chi_{\rho\oplus\psi}(g)=\trace((\rho\oplus\phi)_g)=\trace(\rho_g)+\trace(\psi_g)=\chi_\rho(g)+\chi_\psi(g).\qedhere
  \]
\end{proof}

Let $V$ be a vector space with basis $\{v_1,\dots,v_k\}$ and 
$W$ be a vector space with basis $\{w_1,\dots,w_l\}$. A 
\textbf{tensor product} of $V$ and $W$ is a vector space $X$ with 
together with a bilinear map 
\[
V\times W\to X,
\quad
(v,w)\mapsto v\otimes w,
\]
such that $\{v_i\otimes w_j:1\leq i\leq k,\,1\leq j\leq l\}$ is a basis 
basis of $X$. The tensor product of $V$ and $W$ is unique up to isomorphism 
and it is denoted by $V\otimes W$. Note that
\[
\dim(V\otimes W)=(\dim V)(\dim W).
\]

\begin{definition}
	Let $\rho\colon G\to\GL(V)$ and $\psi\colon G\to\GL(W)$ be representations. The \textbf{tensor product} of $\rho$ and $\psi$ is the representation of $G$ given by 
	\begin{gather*}
	\rho\otimes\psi\colon G\to\GL(V\otimes W),
	\quad 
	g\mapsto (\rho\otimes\psi)_g,
	\shortintertext{where}
	(\rho\otimes\psi)_g(v\otimes w)=\rho_g(v)\otimes \psi_g(w)
	\end{gather*}
	for $v\in V$ and $w\in W$.  	
\end{definition}

A direct calculation shows that the tensor product of representations is indeed a representation. 

\begin{proposition}
  	If $\rho\colon G\to\GL(V)$ and
    $\psi\colon G\to\GL(W)$ are representations, then
    \[
    \chi_{\rho\otimes\psi}=\chi_\rho\chi_\psi.
    \]
    %	\item $\chi_{V^*}=\overline{\chi_V}$.
\end{proposition}

\begin{proof}
	For each $g\in G$ the map $\rho_g$ is diagonalizable. Let $\{v_1,\dots,v_n\}$
	be a basis of eigenvectors of $\phi_g$ and let $\lambda_1,\dots,\lambda_n\in\C$ be such that
	$\rho_g(v_i)=\lambda_iv_i$ for all $i\in\{1,\dots,n\}$. Similarly, 
	let $\{w_1,\dots,w_m\}$ be a basis of eigenvectors of $\psi_g$ and $\mu_1,\dots,\mu_m\in\C$ be such that $\psi_g(w_j)=\mu_jw_j$ for all $j\in\{1,\dots,m\}$. Each 
	$v_i\otimes w_j$ is eigenvector of $\phi\otimes\psi$ with eigenvalue 
	$\lambda_i\mu_j$, as  
	\[
		(\rho\otimes\psi)_g(v_i\otimes w_j)=\rho_gv_i\otimes \psi_gw_j=\lambda_iv_i\otimes \mu_jv_j=(\lambda_i\mu_j)v_i\otimes w_j.
	\]
	Thus  
	$\{v_i\otimes w_j:1\leq i\leq n,1\leq j\leq m\}$ is a basis of eigenvectors and the 
	$\lambda_i\mu_j$ are the eigenvalues of $(\phi\otimes\psi)_g$. It follows that 
	\[
	\chi_{\rho\otimes\psi}(g)
	=\sum_{i,j}\lambda_i\mu_j
	=\left(\sum_i\lambda_i\right)\left(\sum_j\mu_j\right)
	=\chi_\rho(g)\chi_\psi(g).\qedhere 
	\]
\end{proof}

For completeness we mention without proof that
it is also possible to define the dual $\rho^*\colon G\to\GL(V^*)$  
of a representation
$\rho\colon G\to\GL(V)$ by the formula
\[
(\rho^*_gf)(v)=f(\rho^{-1}_gv),\quad
g\in G,\,f\in V^*\text{ and }v\in V.
\]  
We claim that the character of the dual representation is then 
$\overline{\chi_\rho}$. Let $\{v_1,\dots,v_n\}$ be a basis of $V$
and $\lambda_1,\dots,\lambda_n\in\C$ be such that $\rho_gv_i=\lambda_iv_i$ for all $i\in\{1,\dots,n\}$. If $\{f_1,\dots,f_n\}$ is the dual basis of $\{v_1,\dots,v_n\}$, then 
\[
(\rho^*_gf_i)(v_j)=f_i(\rho_g^{-1}v_j)
=\overline{\lambda_j}f_i(v_j)
=\overline{\lambda_j}\delta_{ij}
\]
and the claim follows. 
%\begin{proposition}
%	
%\end{proposition}
%
%\begin{proof}
%	Let $g\in G$ and $\{v_1,\dots,v_n\}$
%	be a basis of eigenvectors of $\rho_g$. Let
%	$\lambda_1,\dots,\lambda_n\in\C$ be such that $\rho_g(v_i)=\lambda_iv_i$ for all
%	$i\in\{1,\dots,n\}$. Let $\{f_1,\dots,f_n\}$ be the dual basis of $\{v_1,\dots,v_n\}$.
%	Since $\rho_g$ is invertible, each eigenvector of $\rho_g$ is non-zero. 
%	Thus $\rho_g(v_i)=\lambda_iv_i$ implies that 
%	$\rho_{g^{-1}}v_i=\lambda_i^{-1}v_i=\overline{\lambda_i}v_i$... 
%%	Now 
%%	\[
%		(\rho_g f_i)(v_j)=f_i(g^{-1}v_j)=\overline{\lambda_j}f_i(v_j)=\overline{\lambda_j}\delta_{ij}.
%	\]
%	
%	  
%	
%	 We claim
%	that $\{f_1,\dots,f_n\}$ is a basis of eigenvectors...
%	with $\overline{\lambda_1},\dots,\overline{\lambda_n}$. En efecto, si $gv_j=\lambda_jv_j$, entonces
%	$g^{-1}v_j=\lambda_j^{-1}v_j=\overline{\lambda_j}v_j$ (observemos que como $\phi_g$ es inversible, los $\lambda_j$ son no nulos). Luego
%	\[
%		(gf_i)(v_j)=f_i(g^{-1}v_j)=\overline{\lambda_j}f_i(v_j)=\overline{\lambda_j}\delta_{ij}.
%	\]
%	En conclusión
%	\[
%		\chi_{V^*}(g)=\sum_{i=1}^n\overline{\lambda_i}=\overline{\chi_V(g)}.\qedhere
%	\]	
%\end{proof}

