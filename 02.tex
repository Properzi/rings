\section{Lecture: 02/10/2024}

\subsection{Quotients}

Let $R$ be a ring and $I$ be an ideal of $R$. 
Then $R/I$ is an abelian group
with 
\[
(x+I)+(y+I)=(x+y)+I
\]
and the 
\emph{canonical map} 
$R\to R/I$, $x\mapsto x+I$,
is a surjective group homomorphism with kernel $I$. Recall that 
$R/I$ is the set of cosets $x+I$, where 
\[
x+I=y+I\Longleftrightarrow x-y\in I.
\]

Note that here we only used
that $I$ is an additive subgroup of $R$. We need an ideal to put a ring structure
on the set $R/I$ of cosets modulo $I$. As in the case of the integers, 
we use the following notation. For $x,y\in R$ 
we write 
\[
x\equiv y\bmod I\Longleftrightarrow x-y\in I.
\]

How can we put a ring structure on $R/I$? It makes sense
to define a multiplication on $R/I$ so that
the canonical map $R\to R/I$ is a surjective ring homomorphism. For that purpose, 
we define 
\[
(x+I)(y+I)=(xy)+I.
\]
Since $I$ is an ideal of $R$, this multiplication is well-defined. In fact, let 
$x+I=x_1+I$ and $y+I=y_1+I$. We want to show that
$xy+I=x_1y_1+I$. Since $x-x_1\in I$, 
\[
xy-x_1y=(x-x_1)y\in I
\]
because $I$ is a right ideal. Similarly, since $y-y_1\in I$, it follows that 
\[
x_1y-x_1y_1=x_1(y-y_1)\in I,
\]
as $I$ is a left ideal. Thus
\[
xy-x_1y_1=xy-x_1y+x_1y-x_1y_1=(x-x_1)y+x_1(y-y_1)\in I.
\]

\begin{theorem}
\label{thm:quotient_ring}
	Let $R$ be a ring and $I$ be an ideal of $R$. Then
	$R/I$ with 
	\[
	(x+I)+(y+I)=(x+y)+I,\quad
	(x+I)(y+I)=(xy)+I,
	\]
	is a ring and the canonical map $R\to R/I$, $x\mapsto x+I$, 
	is a surjective ring homomorphism with kernel~$I$. 
\end{theorem}

We have already seen that multiplication is well-defined. 
The rest of the proof is left as an exercise. As an example, we show that 
the left distributive property holds
in $R/I$ because it holds in $R$, that is 
\begin{align*}
    (x+I)\left((y+I)+(z+I)\right) &= (x+I)(y+z+I)\\
    &=x(y+z)+I\\
    &=xy+xz+I\\
    &=(xy+I)(xz+I)\\
    &=(x+I)(y+I)+(x+I)(z+I).
\end{align*}

\begin{exercise}
\label{xca:quotient_ring}
    Prove Theorem \ref{thm:quotient_ring}.
\end{exercise}

\begin{example}
	Let $R=(\Z/3)[X]$ and $I=(2X^2+X+2)$ be the ideal of $R$ 
	generated by the polynomial $2X^2+X+2$. 	If $f(X)\in R$, 
	the division algorithm allows us to write
	\[
	f(X)=(2X^2+X+2)q(X)+r(X),
	\]
	for some $q(X),r(X)\in R$, where either $r(X)=0$ or $\deg r(X)<2$. 
	This means
	that $r(X)=aX+b$ for some $a,b\in\Z/3$.
	Note that
	\[
    f(X)\equiv aX+b\bmod (2X^2+X+2)
    \]
	for some $a,b\in\Z/3$, 
	so the quotient ring $R/I$ has 
	nine elements.  Can you find an expression for the product 
	$(aX+b)(cX+d)$ in $R/I$?
\end{example}

\index{Isomorphism}
An \emph{isomorphism} between the rings $R$ and $S$ is a bijective
ring homomorphism $R\to S$. If such a homomorphism exists, then $R$ and $S$ are said to be isomorphic, and the notation is 
$R\simeq S$. 

\begin{exercise}
\label{xca:iso}
    Prove that ring isomorphism is an equivalence relation.
\end{exercise}

As it happens in the case of groups, 
to understand quotient rings, one has 
the first isomorphism theorem. 

\begin{theorem}[First isomorphism theorem]
\index{First isomorphism theorem!for rings}
\label{thm:ring_iso1}
	If $f\colon R\to S$ is a ring homomorphism, then $R/\ker f\simeq f(R)$.  	
\end{theorem}

This is somewhat similar to the result one knows from group theory. 
One needs to show that, if $I=\ker f$, then 
the map $R/I\to f(R)$, $x+I\mapsto f(x)$, is a well-defined 
bijective ring homomorphism. 

\begin{exercise}
\label{xca:first_iso}
    Prove Theorem \ref{thm:ring_iso1}.
\end{exercise}

\begin{example}
Let 
\[
R=\left\{\begin{pmatrix}
a&b\\
0&a
\end{pmatrix}:a,b\in\Q\right\}
\]
with the usual matrix operations. 
A direct calculation shows that the map $R\to\Q$, $\begin{pmatrix}a&b\\0&a\end{pmatrix}\mapsto a$, is a surjective 
ring homomorphism with  
\[
\ker f=\left\{\begin{pmatrix}
0&b\\
0&0\end{pmatrix}:b\in\Q\right\}.
\]
Thus $R/\ker f\simeq\Q$. 	
\end{example}

\begin{example}
    We will prove that 
	\[
	\R[X]/(X^2+1)\simeq\C. 
	\]
    But before going into the technical proof, let us show how this
    isomorphism works. 
    Let $f(X)\in\R[X]$. 
	The division algorithm on $\R[X]$ allows us to write
	\[
	f(X)=(X^2+1)q(X)+r(X)
	\]
	for some $q(X),r(X)\in\R[X]$, where $r(X)=0$ or $\deg r(X)<2$. Thus
	$r(X)=aX+b$ for some $a,b\in\R$. This implies that
	\[
	f(X)\equiv aX+b\bmod (X^2+1).
	\] 
	It is quite easy to describe the ring operation of 
	$\R[X]/(X^2+1)$. Clearly 
	\[
	(aX+b)+(cX+d)\equiv (a+c)X+(b+d)\bmod (X^2+1),
	\]
	Since $X^2\equiv -1\bmod (X^2+1)$,   	
	\[
	(aX+b)(cX+d)\equiv X(ad+bc)+(bd-ac), 
	\]
	which reminds us of the usual 
	multiplication rule of the field of complex numbers. This means that, in practice, 
	this quotient ring looks pretty much like the ring of 
	complex numbers.

    Now is time to prove that
	\[
		\R[X]/(X^2+1)\simeq\C. 
	\] 

    Let $\varphi\colon \R[X]\to\C$, $f(X)\mapsto f(i)$, be the evaluation map. Then
    $\varphi$ is a ring homomorphism. Moreover, $\varphi$ is surjective, as
    $\varphi(a+bX)=a+bi$ for all $a,b\in\R$. 
	
	We claim that $\ker\varphi=(X^2+1)$, the ideal
	of $\R[X]$ generated by the polynomial $X^2+1$.  
    Clearly, every multiple of $X^2+1$, that is every polynomial
    of the form $(X^2+1)g(X)$ for some $g(X)\in\R[X]$, belongs to the kernel of $\varphi$. 
    Thus $(X^2+1)\subseteq\ker\varphi$. Conversely, if 
    $f(X)\in\ker\varphi$, then $f(i)=0$. Thus 
	\[
        f(X)=(X-i)g(X)
    \]
    for some $g(X)\in\R[X]$. Since $f(X)\in\R[X]$, 
    $f(-i)=0$ and hence $g(-i)=0$. Thus $X+i$ divides $g(X)$. Therefore 
    \[
        f(X)=(X-i)(x+i)h(X)=(X^2+1)h(X)
    \]
    for some $h(X)\in\R[X]$. Therefore $f(X)\in (X^2+1)$. 
	
	By the first 
    isomorphism theorem, 
    \[
	\R[X]/\ker\varphi=\R[X]/(X^2+1)\simeq\C.
	\] 
\end{example}

\begin{exercise}
	Prove that $\Z[\sqrt{-5}]\simeq\Z[X]/(X^2+5)$. 	
\end{exercise}

Similarly, if $N$ is a square-free integer, then 
\[
\Z[\sqrt{N}]\simeq\Z[X]/(X^2-N).
\]

\begin{exercise}
\label{xca:ring_isos}
Prove the following isomorphisms:
\begin{enumerate}
	\item $\Z[X]/(7)\simeq (\Z/7)[X]$.
	\item $\Q[\sqrt{2}]\simeq\Q[X]/(X^2-2)$.
	\item $\R[X]/(X^2-1)\simeq\R\times\R$.
	\item $\Q[X]/(X-2)\simeq\Q$.
	\item $\R[X,Y]/(X)\simeq\R[Y]$. 
\end{enumerate}
\end{exercise}

\begin{exercise}
\label{xca:sqrt2and3}
Are the rings $\Q[\sqrt{2}]$ and $\Q[\sqrt{3}]$ isomorphic?	
\end{exercise}

\begin{exercise}
\label{xca:continuos}
	Let $R$ be the ring of continuous maps $[0,2]\to\R$, where the operations are given by 
	\begin{align*}	    
	(f+g)(x)&=f(x)+g(x),\\
	(fg)(x)&=f(x)g(x).
	\end{align*}
	Prove that the set 
	$I=\{f\in R:f(1)=0\}$ is an ideal of $R$ and that $R/I\simeq\R$.   	
\end{exercise}

\begin{exercise}
\label{xca:matrices}
	Let $n\geq1$. 
	Let $R$ be a ring and $I$ be an ideal of $R$. Prove that $M_n(I)$ is an ideal 
	of $M_n(R)$ and that 
    \[
    M_n(R)/M_n(I)\simeq M_n(R/I).
    \]
\end{exercise}

\begin{exercise}
    \label{xca:Z[sqrt10]/(2,sqrt10)}
    Let $R=\Z[\sqrt{10}]$ and $I=(2,\sqrt{10})$. Prove that $R/I\simeq\Z/2$. 	
\end{exercise}

Hint: Use the ring homomorphism $\Z[\sqrt{10}]\to\Z/2$, $a+b\sqrt{10}\mapsto a\bmod 2$. 	

\begin{exercise}
\label{xca:Z[i]/(1+3i)}
	Prove that $\Z[i]/(1+3i)\simeq\Z/10$. 	
\end{exercise}

Hint: Use 
the ring homomorphism 
\[
\Z\hookrightarrow\Z[i]\xrightarrow{\pi}\Z[i]/(1+3i),
\]
where
$\pi$ is the canonical map. 

\begin{exercise}
\label{xca:Z15}
	Prove that there is no ideal $I$ of $\Z[i]$ 
	such that $\Z[i]/I\simeq\Z/15$. 
\end{exercise}

\begin{exercise}
	Let $R=(\Z/2)[X]/(X^2+X+1)$. 
	\begin{enumerate}
		\item How many elements does $R$ have?
		\item Can you recognize the additive group of $R$?
		\item Prove that $R$ is a field. 	
	\end{enumerate}
\end{exercise}

Recall that if $f\colon X\to Y$ is a map and $A\subseteq X$ and $B\subseteq Y$ are
subsets, then
\begin{align*}
    f(A)=\{f(a):a\in A\},\quad 
    f^{-1}(B)=\{x\in X:f(x)\in B\}.
\end{align*}
The following statements are easily proved:
\begin{enumerate}
	\item $A\subseteq f^{-1}(f(A))$. 
	\item $A=f^{-1}(f(A))$ if $f$ is injective. 
	\item $f(f^{-1}(B))\subseteq B$. 
	\item $f(f^{-1}(B))=B$ if $f$ is surjective. 
\end{enumerate}

\begin{theorem}[Correspondence theorem]
\index{Correspondence theorem!for rings}
\label{thm:ring_correspondence}
	Let $f\colon R\to S$ be a surjective ring homomorphism. There exists a
	bijective correspondence between 
	the set of ideals of $R$ containing $\ker f$ and
	the set of ideals of $S$.  Moreover, if $f(I)=J$, then
	$R/I\simeq S/J$. 
\end{theorem}

\begin{proof}[Sketch of the proof]
Let $I$ be an ideal of $R$ containing $\ker f$ and
let $J$ be an ideal of $S$. 
We need to prove the following facts:
\begin{enumerate}
\item $f(I)$ is an ideal of $S$.
\item $f^{-1}(J)$ is an ideal of $R$ containing $\ker f$. 
\item $f(f^{-1}(J))=J$ and $f^{-1}(f(I))=I$. 
\item If $f(I)=J$, then $R/I\simeq S/J$. 
\end{enumerate}
We only prove the fourth statement, the others are left as exercises. Note that
the third claim implies that $f(I)=J$ if and only if $I=f^{-1}(J)$. 
Let 
$\pi\colon S\to S/J$ be the canonical map. The composition
$g=\pi\circ f\colon R\to S/J$ is a surjective ring homomorphism and
\[
\ker g=\{x\in R:g(x)=0\}=\{x\in R:f(x)\in J\}=\{x\in R:x\in f^{-1}(J)=I\}=I.
\]
Then the first isomorphism theorem implies that $R/I\simeq S/J$.
\end{proof}

\begin{exercise}
    Prove Theorem \ref{thm:ring_correspondence}.
\end{exercise}

As we did for groups, for the correspondence theorem, 
it helps to have in mind the following diagram: 
\[
\begin{tikzcd}
        && R \\
        & I=f^{-1}(J) && f(R)=S \\
        \ker f && f(I)=J \\
        & {\{0\}}
        \arrow["f", from=1-3, to=2-4]
        \arrow[no head, from=1-3, to=2-2]
        \arrow[no head, from=2-2, to=3-1]
        \arrow["f", from=3-1, to=4-2]
        \arrow["f", from=2-2, to=3-3]
        \arrow[no head, from=3-3, to=4-2]
        \arrow[no head, from=2-4, to=3-3]
\end{tikzcd}
\]

\subsection{Direct products and direct sums}

In Example~\ref{exa:direct_rings}, we defined the \emph{direct product} of rings. There is another similar, yet important, construction.

\begin{definition}
\label{def:direct_ideals}
\index{Direct sum of ideals}
Let $R$ be a ring and $I$ and $J$ be ideals of $R$. We say that 
$R=I\oplus J$ (that is, $R$ is a \emph{direct sum} of ideals) 
if $R=I+J$ and $I\cap J=\{0\}$.  
\end{definition}

\begin{exercise}
\label{xca:uniqueness}
    Let $R$ be a ring and $I$ and $J$ be ideals of $R$. 
    Prove that $R=I\oplus J$ as a direct sum of ideals if and only if 
    every element $r\in R$ can be written uniquely as 
    $r=u+v$ for $u\in I$ and $v\in J$.
\end{exercise}

The following exercise illustrates the relationship between decomposing a 
ring as a direct product of two rings and as a direct sum of two ideals:

\begin{exercise}
\label{xca:products}
    Let $R$ be a ring. Prove that $R$ admits a decomposition 
    $R=S\times T$ as a direct product of rings if and only if $R$ admits a decomposition 
    $R=I+J$ as a direct sum of ideals. 
\end{exercise}

The constructions in Example~\ref{exa:direct_rings} and Definition~\ref{def:direct_ideals}
can be generalized to a finite number of objects. 
If $R$ is a ring and $I_1,\dots,I_n$ are ideals of $R$, we say 
$R$ is the \emph{direct sum} of the ideals $I_1,\dots,I_n$, 
that is 
\[
R=I_1\oplus\cdots\oplus I_n, 
\]
if $R=I_1+\cdots+I_n$ and 
for every $i\in\{1,\dots,n\}$
\[
I_i\cap (I_1+\cdots+I_{i-1}+I_{i+1}+\cdots+I_n)=\{0\}.
\]

As we did in Exercise~\ref{xca:uniqueness}, one can prove that 
$R=I_1\oplus\cdots\oplus I_n$ as a direct sum of ideals
if and only if every $r\in R$ can be written uniquely 
as $r=e_1+\cdots+e_n$ for $e_1\in I_1,\dots,e_n\in I_n$. Moreover, $R$ admits a decomposition 
$R=R_1\times\cdots\times R_n$ as a direct product 
of rings if and only if $R$ admits a decomposition 
$R=I_1\oplus\cdots\oplus I_n$ as 
a direct sum of ideals. 

\subsection{The Chinese remainder theorem}

We now work with commutative rings. 
If $R$ is a commutative ring and $I$ and $J$ are ideals of $R$, then
\[
I+J=\{u+v:u\in I,\,v\in J\}
\]
is an ideal of $R$. The sum of ideals makes sense also in non-commutative rings.  

\begin{definition}
\index{Coprime ideals}
\index{Comaximal ideals}
	Let $R$ be a commutative ring. The ideals $I$ and $J$ of $R$ are said to be
	\emph{coprime} if $R=I+J$.  
\end{definition}

The terminology is motivated by the following example. If $I$ and $J$ are
ideals of $\Z$, then $I=(a)$ and $J=(b)$ for some $a,b\in\Z$. Then Bezout's theorem 
states that 
\begin{align*}
\text{$a$ and $b$ are coprime}
&\Longleftrightarrow 1=ra+sb\text{ for some $r,s\in\Z$}\\
&\Longleftrightarrow\text{ $I$ and $J$ are coprime.}
\end{align*}

In some books, coprime ideals are called \emph{comaximal} ideals. 

If $I$ and $J$ are ideals of $R$, then 
\[
	IJ=\left\{\sum_{i=1}^mu_iv_i:m\in\Z_{\geq0},\,u_1,\dots,u_m\in I,\,v_1,\dots,v_m\in J\right\}
\]
is an ideal of $R$. Note that 
$IJ$ is the set of all finite sum of elements of the form $uv$ for $u\in I$ and $v\in J$. 
For example, if $I=(u)$ and $J=(v)$, then $IJ=(uv)$. 
Why we need to consider finite sums of elements of the form $uv$
for $u\in I$ and $v\in J$?

\begin{exercise}
    \label{xca:IJ}
    Let $R=\R[X,Y]$ and $I=J=(X,Y)$. Prove that the
    set \[
    \{uv:u\in I,\,v\in J\}
    \]
    is not an ideal of $R$.
\end{exercise}

Note that $IJ\subseteq I\cap J$. Equality does not hold in general. Take
for example $R=\Z$ and $I=J=(2)$. Then $IJ=(4)\subsetneq (2)=I\cap J$. 

\begin{proposition}
Let $R$ be a commutative ring. If $I$ and $J$ are coprime ideals, then $IJ=I\cap J$. 	
\end{proposition}

\begin{proof}
Let $x\in I\cap J$. Since $I$ and $J$ are coprime, 
$1=u+v$ for some $u\in I$ and $v\in J$, 
\[ 
x=x1=x(u+v)=xu+xv\in IJ\qedhere.
\]
\end{proof}

\begin{bonus}
\label{xca:strongly_regular}
\index{Ring!strongly regular}
    Let $R$ be a commutative ring. 
    Prove that $I\cap J=IJ$ 
    for all ideals $I$ and $J$ of $R$
    if and only if $R$ is \emph{strongly regular}, 
    that is  
    for each $a\in R$ there 
    exists $x\in R$ such that $a=xa^2$. 
\end{bonus}

One can also prove that if $R$ is a ring,
$I\cap J=IJ$ holds 
for all left ideals $I$ and $J$ of $R$ 
if and only if $R$ is strongly regular. 
% steps
% Let $I$ be a left ideal of $R$. Since $RI=I=I\cap R=IR$, $I$ 
% is an ideal of $R$. Let $a\in R$. Then $Ra=RaR$ and
% $Ra=(Ra)(Ra)=(RaR)a=(Ra)a$ and hence $a=xa^2$ for some $x$. 
% Conversely, let $a\in I\cap J$. There exists $x$ such that $a=xa^2$. Then
% a=(xa)a\in IJ$. Claim: $I$ is an ideal of $R$. Let $a\in I$ and $r\in R$.
% Then there exists $x$ such that $a=xa^2$. Since $xa$ is central, 
% $ar=xa^2r$...
%. https://ysharifi.wordpress.com/category/noncommutative-ring-theory-notes/von-neumann-regular-rings/


\begin{theorem}[Chinese remainder theorem]
\index{Chinese remainder theorem}
Let $R$ be a commutative ring and $I$ and $J$ be coprime ideals of $R$. 
If $u,v\in R$, then 
there exists $x\in R$ such that 
\[
\begin{cases}	
x\equiv u\bmod I,\\
x\equiv v\bmod J.
\end{cases}
\]
\end{theorem}

\begin{proof}
Since the ideals $I$ and $J$ are coprime, $1=a+b$ for some $a\in I$ and $b\in J$. 
Let $x=av+bu$. Then
\[
x-u=av+(b-1)u=av-au=a(v-u)\in I,
\]
that is $x\equiv u\bmod I$. Similarly, $x-v\in J$ and $x\equiv v\bmod J$.  	
\end{proof}

For the following result, we need 
to use the direct product of rings. See Example~\ref{exa:direct_rings}. 

\begin{corollary}
\label{cor:chinese}
	Let $R$ be a commutative ring. If $I$ and $J$ are coprime ideals of $R$, 
	then $R/(I\cap J)\simeq R/I\times R/J$.
\end{corollary}

\begin{proof}
	Let $\pi_I\colon R\to R/I$ and $\pi_J\colon R\to R/J$ be the canonical maps. A direct 
	calculation shows that the
	map $\varphi\colon R\to R/I\times R/J$, $x\mapsto (\pi_I(x),\pi_J(x))$, 
	is a ring homomorphism with $\ker\varphi=I\cap J$. For example, to prove
 that 
 \[
 \varphi(xy)=\varphi(x)\varphi(y)
 \]
 we proceed as follows: if $x,y\in R$, then 
        \begin{align*}
            \varphi(xy) &= (\pi_I(xy),\pi_J(xy))\\
            &= (\pi_I(x)\pi_I(y),\pi_J(x)\pi_J(y))\\
            &= (\pi_I(x),\pi_J(x))(\pi_I(y),\pi_J(y))\\
            &= \varphi(x)\varphi(y). 
        \end{align*}

 We now claim that the map $\varphi$ is surjective. To prove this, 
 we will use the Chinese remainder theorem. If $(u+I,v+J)\in R/I\times R/J$, 
	then there exists $x\in R$ such that 
	$x-u\in I$ and $x-v\in J$. This translates into the surjectivity of $\varphi$,
 as \[
 \varphi(x)=(\pi_I(x),\pi_J(x))=(x+I,x+J)=(u+I,v+J).
 \]
 Now
	$R/(I\cap J)\simeq R/I\times R/J$ by the first isomorphism theorem. 
\end{proof}

\begin{example}
    If $n$ and $m$ are coprime integers, then
    \[
    \Z/(nm)\simeq \Z/n\times \Z/m
    \]
    as rings. This follows from Corollary \ref{cor:chinese} with 
    $R=\Z$, $I=(n)$ and $J=(m)$. 
\end{example}

Let $R$ be a commutative ring and $I_1,\dots,I_n$ be ideals of $R$. Then
$I_1\cdots I_n$, defined as the set of all finite sums of the form 
$x_1\cdots x_n$ such that $x_j\in I_j$ for all $j\in\{1,\dots,n\}$, 
is an ideal of $R$. For example, if $I_j=(x_j)$ for all $j\in\{1,\dots,n\}$, 
then $I_1\cdots I_j=(x_1\dots x_j)$. 

If $I_1$ and $I_j$ are coprime for all $j\in\{2,\dots,n\}$, 
then $I_1$ and $I_2\cdots I_n$ are coprime. If $I_i$ and $I_j$ are coprime
whenever $i\ne j$, then 
$I_1\cap\cdots\cap I_n=I_1\cdots I_n$ and 
\[
R/(I_1\cap\cdots\cap I_n)\simeq R/I_1\times\cdots\times R/I_n.
\]

Applying the previous observations in the case of the ring of integers, we get the following result: If $n=p_1^{\alpha_1}\cdots p_k^{\alpha_k}$ is the factorization 
of $n$ into powers of distinct primes, then 
\[
\Z/n\simeq \Z/(p_1^{\alpha_1})\times\cdots \times \Z/(p_k^{\alpha_k})
\]
as rings. 

\begin{bonus}
\index{Euler $\varphi$-function}
    Let $n=p_1^{\alpha_1}\cdots p_k^{\alpha_k}$ be the factorization 
    of $n$ into powers of distinct primes. Prove that 
    $\varphi(n)=\varphi(p_1^{\alpha_1})\cdots \varphi(p_k^{\alpha_k})$, where $\varphi$ is the Euler $\varphi$-function. 
\end{bonus}


% exercise: every map $\Z/p\to\Z/p$ is a polynomial map
% todo: ejercicios de cuentas, hay que usar minipages o algo simliar
%
%\begin{exercise}
%	Solve 
%	\begin{enumerate}
%	\item $\begin{cases}
%		x\equiv 3\bmod 5,\\
%		x\equiv 1\bmod 6.
%		\end{cases}$
%	\item $\begin{cases}
%		x\equiv 5\bmod 7,\\
%		x\equiv 7\bmod 11,\\
%		x\equiv 3\bmod 13.
%		\end{cases}$			
%	\end{enumerate}
%\end{exercise}

The story says that, 
instead of counting troops, 
the Chinese generals would order them to 
assemble in rows of different depths 
(hence obtaining the remainder of the number 
of soldiers modulo different integers) and then they 
would use the theorem to 
compute the size of the army.

\begin{bonus}
\label{xca:gather_people}
	Let us gather people in the following way. When I 
	count by three, there are two people left. 	When I count by four, 
	there is one person left over and when I count by five there is
	one missing. How many people are there?
\end{bonus}

\begin{bonus}
\label{xca:no_solution}
	Prove that 
	\[
	\begin{cases}
		x\equiv 29\bmod 52,\\
		x\equiv 19\bmod 72,
		\end{cases}
	\]
	does not have a solution.
\end{bonus}

\begin{bonus}
\label{xca:consecutive}
	Find three consecutive integers such that the first one is divisible by a square, 
	the second one is divisible by a cube and the third one is divisible by a fourth power. 	
\end{bonus}

\begin{bonus}
\label{xca:perfect_square}
	Prove that 
	for each $n>0$ there are $n$ consecutive integers such that 
	each integer is divisible by a perfect square $\ne 1$. 	
\end{bonus}

A special application of the Chinese remainder theorem leads to the construction of the Lagrange 
interpolation polynomial.

\begin{bonus}[Lagrange's interpolation theorem]
\index{Lagrange's interpolation theorem}
	The Chinese remainder theorem proves the following well-known result. 
	Let $x_1,\dots,x_k\in\R$ be such that $x_i\ne x_j$ whenever $i\ne j$ 
	and $y_1,\dots,y_k\in\R$. Then there exists $f(X)\in\R[X]$ such that
	\[
	\begin{cases}
		f(X)\equiv y_1\bmod (X-x_1),\\
		f(X)\equiv y_2\bmod (X-x_2),\\
		\phantom{f(X)}\vdots\\
		f(X)\equiv y_k\bmod (X-x_k).  	
	\end{cases}
 	\]
 	The solution $f(X)$ is unique modulo $(X-x_1)(X-x_2)\cdots (X-x_n)$. 
\end{bonus}

% A similar application of the Chinese remainder theorem 
% leads to the Hermite interpolation polynomial. Hermite interpolation computes a polynomial of degree less than $n$ such that the polynomial and its first few derivatives have the same values at $m<n$ given points as the given function and its first few derivatives at those points. The number of pieces of information, function values and derivative values, must add up to 
% $n$.

% Dedekind's theorem
% https://en.wikipedia.org/wiki/Chinese_remainder_theorem

With small changes, the Chinese remainder theorem can be proved 
in arbitrary non-commutative rings, see for example 
\cite[Chapter III, Theorem 2.25]{MR600654}. 
