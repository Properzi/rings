\section*{Introduction}

The notes correspond to the bachelor 
course \emph{Ring and Modules} of the 
Vrije Universiteit Brussel, 
Faculty of Sciences, 
Department of Mathematics and Data Sciences. The course
is divided into twelve or thirteen two-hours lectures. 

The material is somewhat standard. Basic texts on abstract algebra
are for example \cite{MR1129886}, \cite{MR2286236} and \cite{MR600654}. 
Lang's book \cite{MR783636} is also a standard reference, but 
maybe a little bit more advanced. 
We based the lectures on the representation theory of finite
groups on \cite{MR0450380} and 
\cite{MR2867444}. 

We also mention a set of great expository papers by 
Keith Conrad available at 
\url{https://kconrad.math.uconn.edu/blurbs/}. 
The notes are extremely well-written and are useful at  
every stage of a mathematical career. 

% Bibtex information:
% {\footnotesize\begin{verbatim}
% @misc{rings,
%     author={Vendramin, L.},
%     title={Rings and modules},
%     year={2022},
%     note={Available at www.github.com/vendramin/rings},
%     pages={108}
% }
% \end{verbatim}}

 Thanks go to Wouter Appelmans, Arne van Antwerpen, Ilaria Colazzo, Luk De Block, 
 Luca Descheemaeker, Carsten Dietzel, {\L}ukas Kubat, Lucas Simons, Senne Trappeniers, 
and Geoffrey Jassens. 

This version 
was compiled on \today~at~\currenttime. 
Please send comments and corrections to me at \url{Leandro.Vendramin@vub.be}. 


\bigskip
\begin{flushright}
Leandro Vendramin\\Brussels, Belgium\par
\end{flushright}
