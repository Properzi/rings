\usepackage[foot]{amsaddr} 
\usepackage{hyperref}
\usepackage{listings}
\usepackage{tikz-cd}
\usepackage{datetime}
\usepackage{amssymb}
\usepackage{mathtools}
\usepackage{stmaryrd}
\usepackage{mdframed}
\usepackage{textcomp}
\usepackage{colortbl}
\usepackage[most]{tcolorbox}

\hypersetup{
  final, 
  colorlinks, 
  linkcolor=-red!55!green!50, 
  citecolor=blue!50!red,
  urlcolor=green!30!black
}

\swapnumbers

% For Dyslexic replace the next three lines by 
%\usepackage{fontspec}
%\setmainfont{OpenDyslexic}
%\usepackage{mathptmx}
%\usepackage{newtxtext}

\usepackage[margin=1in,footskip=.25in]{geometry}

\overfullrule=1mm

\renewcommand\emph[1]{\textcolor{blue!50!red}
{\bfseries #1}}

\renewcommand\thesection{\arabic{section}}
\renewcommand\thesubsection{\arabic{section}.\arabic{subsection}}


\lstdefinelanguage{Julia}%
  {morekeywords={abstract,break,case,catch,const,continue,do,else,elseif,%
      end,export,false,for,function,immutable,import,importall,if,in,%
      macro,module,otherwise,quote,return,switch,true,try,type,typealias,%
      using,while},%
   sensitive=true,%
   alsoother={$},%
   morecomment=[l]\#,%
   morecomment=[n]{\#=}{=\#},%
   morestring=[s]{"}{"},%
   morestring=[m]{'}{'},%
}[keywords,comments,strings]%

\definecolor{background}{HTML}{F5F5F5}
\definecolor{jlstring}{HTML}{880000}%          % julia's strings
\definecolor{jlbase}{HTML}{444444}%            % julia's base color
\definecolor{jlkeyword}{HTML}{444444}%         % julia's keywords
\definecolor{jlliteral}{HTML}{78A960}%         % julia's literals
\definecolor{jlbuiltin}{HTML}{397300}%         % julia's built-ins
\definecolor{jlmacros}{HTML}{1F7199}%          % julia's macros
\definecolor{jlfunctions}{HTML}{444444}%       % julia's functions
\definecolor{jlcomment}{HTML}{888888}%         % julia's comments
\definecolor{jlstring}{HTML}{880000}%          % julia's strings


\lstset{%
    language         = Julia,
    basicstyle       = \color{jlstring}\ttfamily\scriptsize,
    backgroundcolor  = \color{background},
    keywordstyle     = \color{jlkeyword},
    stringstyle      = \color{jlstring},
    commentstyle     = \color{jlcomment},
    showstringspaces = false,
    columns=fixed,
}

%\renewcommand\sectionname{Lecture}
\renewcommand\subsectionname{\S}

% para enumerar
\renewcommand{\labelenumi}{\textbf{\arabic{enumi})}}

\usepackage[most]{tcolorbox}

%\newtcolorbox{mybox}%[colback=red!5!white,colframe=red!75!black]
% enhanced,
% boxrule=0pt,frame hidden,
% %borderline west={4pt}{0pt}{black},
% %colback={gray!20},
% sharp corners,
% left=.5cm
% %left=18.0pt
% }

\makeindex             

\newcommand{\Irr}{\operatorname{Irr}}
\newcommand{\Ann}{\operatorname{Ann}}
\newcommand{\op}{\operatorname{op}}
\newcommand{\Gal}{\operatorname{Gal}}
\newcommand{\supp}{\operatorname{supp}}
\newcommand{\Q}{\mathbb{Q}}
\newcommand{\Z}{\mathbb{Z}}
\newcommand{\F}{\mathbb{F}}
\newcommand{\R}{\mathbb{R}}
\newcommand{\B}{\mathbb{B}}
\newcommand{\C}{\mathbb{C}}
\newcommand{\D}{\mathbb{D}}
\newcommand{\rank}{\operatorname{rank}}
\newcommand{\norm}{\operatorname{norm}}
\newcommand{\Hom}{\operatorname{Hom}}
\newcommand{\Syl}{\mathrm{Syl}}
\newcommand{\id}{\operatorname{id}}
\newcommand{\Aut}{\operatorname{Aut}}
\newcommand{\Inn}{\operatorname{Inn}}
\newcommand{\End}{\operatorname{End}}
\newcommand{\Alt}{\mathbb{A}}
\newcommand{\Sym}{\mathbb{S}}
\newcommand{\lcm}{\operatorname{lcm}}
\newcommand{\trace}{\operatorname{trace}}
\newcommand{\sgn}{\operatorname{sign}}
\newcommand{\ch}{\operatorname{char}}
\newcommand{\im}{\operatorname{im}}
\newcommand{\Ret}{\operatorname{Ret}}
\newcommand{\GL}{\mathbf{GL}}
\newcommand{\SL}{\mathbf{SL}}
\newcommand{\PSL}{\mathbf{PSL}}
\newcommand{\PGL}{\mathbf{PGL}}
\newcommand{\Fix}{\operatorname{Fix}}
\newcommand{\Aff}{\operatorname{Aff}}
\newcommand{\Soc}{\operatorname{Soc}}
\newcommand{\Core}{\operatorname{Core}}
\newcommand{\legendre}[2]{\left(\frac{#1}{#2}\right)}
\newcommand{\Fun}{\operatorname{Fun}}
\newcommand{\Res}{\operatorname{Res}}
\newcommand{\Ind}{\operatorname{Ind}}
\newcommand{\cp}{\operatorname{cp}}
\newcommand{\cf}{\operatorname{cf}}
\newcommand{\Char}{\operatorname{Char}}
\newcommand{\gl}{\mathfrak{gl}}
\renewcommand{\sl}{\mathfrak{sl}}
\newcommand{\ad}[1]{\operatorname{ad}\,#1}
\newcommand{\A}{\mathbb{A}}


% column vector
\newcount\colveccount
\newcommand*\colvec[1]{
\global\colveccount#1
\begin{pmatrix}
        \colvecnext
        }
        \def\colvecnext#1{
        #1
        \global\advance\colveccount-1
        \ifnum\colveccount>0
        \\
        \expandafter\colvecnext
        \else
\end{pmatrix}
\fi
}

\newtheorem{theorem}{Theorem}[section]
\newtheorem{lemma}[theorem]{Lemma}
\newtheorem{proposition}[theorem]{Proposition}
\newtheorem{corollary}[theorem]{Corollary}

\theoremstyle{definition}
\newtheorem{definition}[theorem]{Definition}
\newtheorem{example}[theorem]{Example}
%\newtheorem{examples}[theorem]{Examples}
\newtheorem{xca}[theorem]{Exercise}
\newtheorem{bxca}[theorem]{Bonus exercise}
%\newtheorem{exa}[theorem]{Example}
\newtheorem{remark}[theorem]{Remark}
\newtheorem{que}[theorem]{Question}
\newtheorem{conj}[theorem]{Conjecture}
\newtheorem{open}[theorem]{Open problem}
%\newtheorem{exercise}[theorem]{Exercise}

\theoremstyle{remark}
\newtheorem*{claim}{Claim}

\newenvironment{sol}[1]
{\renewcommand{\qedsymbol}{}\begin{proof}[\ref{#1}]}
  {\end{proof}}

% \newtcolorbox{mybox}{
% enhanced,
% boxrule=0pt,frame hidden,
% %borderline west={4pt}{0pt}{black},
% %colback={gray!20},
% sharp corners,
% left=.5cm
% %left=18.0pt
% }
% \newenvironment{exercise}
%   {\begin{mybox}\begin{xca}}
%   {\end{xca}\end{mybox}}

\newenvironment{problem}
{\begin{tcolorbox}[boxrule=0pt,frame hidden,colback=blue!5!white,
  left=.3em, right=.3em, top=-.2em, bottom=.3em,
  beforeafter skip balanced=.4\baselineskip plus 2pt,
  before upper={\parindent4mm\noindent},
colframe=blue!55!white]\begin{open}}{\end{open}\end{tcolorbox}}

\newenvironment{question}
{\begin{tcolorbox}[boxrule=0pt,frame hidden,colback=blue!5!white,
  left=.3em, right=.3em, top=-.2em, bottom=.3em,
  beforeafter skip balanced=.4\baselineskip plus 2pt,
  before upper={\parindent4mm\noindent},
colframe=blue!55!white]\begin{que}}{\end{que}\end{tcolorbox}}

\newenvironment{conjecture}
{\begin{tcolorbox}[boxrule=0pt,frame hidden,colback=blue!5!white,
  left=.3em, right=.3em, top=-.2em, bottom=.3em,
  beforeafter skip balanced=.4\baselineskip plus 2pt,
  before upper={\parindent4mm\noindent},
colframe=blue!55!white]\begin{conj}}{\end{conj}\end{tcolorbox}}


\newenvironment{exercise}
{\begin{tcolorbox}[boxrule=0pt,frame hidden,colback=green!5!white,
  left=.3em, right=.3em, top=-.2em, bottom=.3em,
  beforeafter skip balanced=.4\baselineskip plus 2pt,
  before upper={\parindent4mm\noindent},
colframe=green!55!black]\begin{xca}}{\end{xca}\end{tcolorbox}}

\newenvironment{bonus}
{\begin{tcolorbox}[boxrule=0pt,frame hidden,colback=yellow!15!white,
  left=.3em, right=.3em, top=-.2em, bottom=.3em,
  beforeafter skip balanced=.4\baselineskip plus 2pt,
  before upper={\parindent4mm\noindent},
colframe=yellow!45!black]\begin{bxca}}{\end{bxca}\end{tcolorbox}}

% \newenvironment{example}
% {\begin{tcolorbox}[boxrule=0pt,frame hidden,colback=red!5!white,
%   left=.3em, right=.3em, top=-.2em, bottom=.3em,
%   beforeafter skip balanced=.4\baselineskip plus 2pt,
%   before upper={\parindent4mm\noindent},
% colframe=red!55!black]\begin{exa}}{\end{exa}\end{tcolorbox}}

\numberwithin{figure}{section}
\numberwithin{equation}{section}

\makeindex

\title{\course}
\author{Leandro Vendramin}
\address{Department of Mathematics and Data
Science, Vrije Universiteit Brussel, Pleinlaan 2, 1050 Brussel}
\email{Leandro.Vendramin@vub.be}
\thanks{}
\date{}

\makeatletter
\renewcommand\section{\@startsection{section}{1}%
  \z@{.7\linespacing\@plus\linespacing}{.5\linespacing}%
  {\color{green!30!black}\normalfont\bfseries\centering}}
\renewcommand\subsection{\@startsection{subsection}{2}%
  \normalparindent{.5\linespacing\@plus.7\linespacing}{-.5em}%
  {\color{-red!55!green!50}\normalfont\bfseries}}

\usepackage{fancyhdr}
\pagestyle{fancy}
\fancyhf{}
\fancyfoot[R]{\thepage}
\fancyhead[L]{\course}
\fancyhead[R]{Lecture \thesection}
\setlength{\headheight}{14pt}

\usepackage[english.nosectiondot]{babel}
