\chapter{Chinese remainder theorem}

Note that if $R$ is a commutative ring and $I$ and $J$ are ideals of $R$, then
\[
I+J=\{u+v:u\in I,\,v\in J\}
\]
is an ideal of $R$. 

\begin{definition}
	Let $R$ be a commutative ring. The ideals $I$ and $J$ of $R$ are said to be
	\textbf{coprime} if $R=I+J$.  
\end{definition}

The terminology is motivated by the following example. If $I$ and $J$ are
ideals of $\Z$, then $I=(a)$ and $J=(b)$ for some $a,b\in\Z$. Then
\begin{align*}
\text{$a$ and $b$ are coprime}
\Longleftrightarrow 1=ra+sb\text{ for some $r,s\in\Z$}
\Longleftrightarrow\text{ $I$ and $J$ are coprime.}
\end{align*}

If $I$ and $J$ are ideals of $R$, then 
\[
IJ=\left\{\sum_{i=1}^mu_iv_i:m\in\N_0,\,u_1,\dots,u_m\in I,\,v_1,\dots,v_m\in J\right\}
\]
is an ideal of $R$. Note that $IJ\subseteq I\cap J$. The equality does not hold in general. Take
for example $R=\Z$ and $I=J=(2)$. Then $IJ=(4)\subsetneq (2)=I\cap J$. 

\begin{proposition}
Let $R$ be a commutative ring. If $I$ and $J$ are coprime ideals, then $IJ=I\cap J$. 	
\end{proposition}

\begin{proof}
Let $x\in I\cap J$. Since $I$ and $J$ are coprime, 
$1=u+v$ for some $u\in I$ and $v\in J$, 
$x=x1=x(u+v)=xu+xv\in IJ$. 	
\end{proof}

\begin{theorem}[chinese remainder theorem]
\index{Chinese remainder theorem}
Let $R$ be a commutative ring and $I$ and $J$ be coprime ideals. If $u,v\in R$, then 
there exists $x\in R$ such that 
\[
\begin{cases}	
x\equiv u\bmod I,\\
x\equiv v\bmod J.
\end{cases}
\]
\end{theorem}

\begin{proof}
Since the ideals $I$ and $J$ are coprime, $1=a+b$ for some $a\in I$ and $b\in J$. 
Let $x=av+bu$. Then
\[
x-u=av+(b-1)u=av-au=a(v-u)\in I,
\]
that is $x\equiv u\bmod I$. Similarly, $x-v\in J$ and $x\equiv v\bmod J$.  	
\end{proof}

\begin{corollary}
	Let $R$ be a commutative ring. If $I$ and $J$ are coprime ideals of $R$, 
	then $R/(I\cap J)\simeq R/I\times R/J$.
\end{corollary}

\begin{proof}
	Let $\pi_I\colon R\to R/$ and $\pi_J\colon R\to R/J$ be the canonical maps. A straightforward
	calculation shows that the
	map $\varphi\colon R\to R/I\times R/J$, $x\mapsto (\pi_I(x),\pi_J(x))$, 
	is an injective ring homomorphism with $\ker\varphi=I\cap J$.
	The chinese remainder theorem implies that $\varphi$ is surjective. If $(u+I,v+J)\in R/I\times R/J$, 
	then there exists $x\in R$ such that 
	$x-u\in I$ and $x-v\in J$. This translates into the surjectivity of $\varphi$. Now
	$R/(I\cap J)\simeq R/I\times R/J$ by the first isomorphism theorem. 
\end{proof}

Let $R$ be a commutative ring and $I_1,\dots,I_n$ be ideals of $R$. Then
\[
I_1\cdots I_n=\left\{
\sum_{i=1}^mu_{i_1}\cdots u_{i_n}:m\in\N_0,\,u_{i_1},\dots,u_{i_n}\in I_{i_j}
\right\}
\]
is an ideal of $R$. If $I_1$ and $I_j$ are coprime for all $j\in\{2,\dots,n\}$, 
then $I_1$ and $I_2\cdots I_n$ are coprime. If $I_i$ and $I_j$ are coprime
whenever $i\ne j$, then 
\[
R/(I_1\cap\cdots\cap I_n)\simeq R/I_1\times\cdots\times R/I_n.
\]

\begin{exercise}[Lagrange's interpolation theorem]
	The chinese remainder theorem roves the following well-known result. 
	Let $x_1,\dots,x_k\in\R$ be such that $x_i\ne x_j$ whenever $i\ne j$ 
	and $y_1,\dots,y_k\in\R$. Then there exists $f(X)\in\R[X]$ such that
	\[
	\begin{cases}
		f(X)\equiv y_1\bmod (X-x_1),\\
		f(X)\equiv y_2\bmod (X-x_2),\\
		\phantom{f(X)}\vdots\\
		f(X)\equiv y_k\bmod (X-x_k).  	
	\end{cases}
 	\]
 	The solution $f(X)$ is unique modulo $(X-x_1)(X-x_2)\cdots (X-x_n)$. 
\end{exercise}

% todo: ejercicios de cuentas, hay que usar minipages o algo simliar
%
%\begin{exercise}
%	Solve 
%	\begin{enumerate}
%	\item $\begin{cases}
%		x\equiv 3\bmod 5,\\
%		x\equiv 1\bmod 6.
%		\end{cases}$
%	\item $\begin{cases}
%		x\equiv 5\bmod 7,\\
%		x\equiv 7\bmod 11,\\
%		x\equiv 3\bmod 13.
%		\end{cases}$			
%	\end{enumerate}
%\end{exercise}


\begin{exercise}
\label{xca:gather_people}
	Let us gather people in the following way. When I 
	count by three, there are two persons left. 	When I count by four, 
	there is one person left over and when I count by five there is
	one missing. How many persons are there?
\end{exercise}

\begin{exercise}
\label{xca:no_solution}
	Prove that 
	\[
	\begin{cases}
		x\equiv 29\bmod 52,\\
		x\equiv 19\bmod 72.
		\end{cases}
	\]
	does not have solution.
\end{exercise}

\begin{exercise}
\label{xca:consecutive}
	Find three consecutive integers such that the first one is divisible by a square, 
	the second one is divisible by a cube and the third one is divisible by a fourth power. 	
\end{exercise}

\begin{exercise}
\label{xca:perfect_square}
	Prove that 
	for each $n\in\N$ there are $n$ consecutive integers such that 
	each integer is divisible by a perfect square $\ne 1$. 	
\end{exercise}


