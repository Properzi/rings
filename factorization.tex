\chapter{Factorization}

\begin{definition}
\index{Integral domain}
	A commutative ring $R$ is said to be an \textbf{integral domain}
	if $xy=0$ implies $x=0$ or $y=0$.  	
\end{definition}

The rings $\Z$ and $\Z[i]$ are both integral domains. 
More generally, if $N$ is a square-free integer, 
then the ring $\Z[\sqrt{N}]$ is an integral domain.  
The ring $\Z/4$ of 
integers modulo 4 is not an integral domain. 

\begin{definition}
	Let $R$ be an integral domain and $x,y\in R$. Then $x$ \textbf{divides} $y$ 
	if $y=xz$ for some $x\in R$. 
	Notation: $x\mid y$ if and only if $x$ divides $y$. If $x$ does not
	divide $y$ one writes $x\nmid y$.  
\end{definition}

Note that $x\mid y$ if and only if $(y)\subseteq (x)$.
	
\begin{definition}
	Let $R$ be an integral domain and $x,y\in R$. Then $x$ and $y$ are
	\textbf{associate} in $R$ if $x=yu$ for some $u\in\mathcal{U}(R)$. 
\end{definition}

Note that $x$ and $y$ are associate if and only if $(x)=(y)$.  

\begin{example}
	The integers $2$ and $-2$ are associate in $\Z$.	
\end{example}

\begin{example}
	Let $R=\Z[i]$. 
	\begin{enumerate}
		\item Let $d\in\Z$ and $a+ib\in R$. Then $d\mid a+ib$ in $R$ if and only if 
			$d\mid a$ and $d\mid b$ in $\Z$. 
		\item $2$ and $-2i$ are associate in $R$.
	\end{enumerate} 	
\end{example}

\begin{example}
	Let $R=\R[X]$ and $f(X)\in R$. Then $f(X)$ and $\lambda f(X)$ are 
	associate in $R$ for all $\lambda\in\R^{\times}$. 	
\end{example}

\begin{definition}
	Let $R$ be an integral domain and $x\in R\setminus\{0\}$. Then $x$ is \textbf{irreducible} 
	if and only if $x\not\in\mathcal{U}(R)$ 
	and $x=ab$ with $a,b\in R$ implies that $a\in\mathcal{U}(R)$ or $b\in\mathcal{U}(R)$. 
\end{definition}

Note that $x$ is irreducible if and only if $(x)\ne R$ 
and there is no principal ideal $(y)$ such that 
$(x)\subsetneq (y)\subsetneq R$.

\begin{example}
	Let $R=\R[X]$ and $f(X)\in R\setminus\{0\}$. Then $f(X)$ is irreducible if 
	$\lambda\in\R^{\times}$ or $\lambda f(X)$ for $\lambda\in\R^{\times}$ 
	are the only divisors
	of $f(X)$.  
\end{example}

The irreducibles of $\Z$ are the prime numbers. 

\begin{definition}
	Let $R$ be an integral domain and $p\in R\setminus\{0\}$. Then  
	$p$ is \textbf{prime} if $p\not\in\mathcal{U}(R)$ and 
	$yz\in (p)$ implies that $y\in (p)$ or $z\in (p)$. 
\end{definition}

In $\Z$ primes and irreducible coincide. This does not happend in full generality. However,
the following result holds. 

\begin{proposition}
	Let $R$ be an integral domain and $x\in R$. 
	If $x$ is prime, then $x$ is irreducible. 
\end{proposition}

\begin{proof}
	Let $p$ be a prime. Then $p\ne 0$ and $p\not\in\mathcal{U}(R)$. Let $x$ be such that
	$x\mid p$. Then $p=xy$ for some $y\in R$. This means $xy\in (p)$, 
	so $x\in (p)$ or $y\in (p)$ because
	$p$ is prime. If $x\in (p)$, then $x=pz$ for some $z\in R$ and hence
	\[
	p=xy=(pz)y.
	\]
	Since $p-pzy=p(1-zy)$ and $R$ is an integral domain, it follows that 
	$1-zy=0$. Thus $y\in\mathcal{U}(R)$. Similarly, if $y\in (p)$, then 
	$x\in\mathcal{U}(R)$. 
\end{proof}

To show that there rings where some irreducibles are not prime, 
we need the following lemma. 

\begin{lemma}
Let $N\in\Z$ be a square-free integer and $R=\Z[\sqrt{N}]$. The map 
\[
N\colon R\to\N,
\quad a+b\sqrt{N}\mapsto 
|a^2-Nb^2|,
\]
satisfies the following properties:
\begin{enumerate}
	\item $N(\alpha)=0$ if and only if $\alpha=0$. 
	\item $N(\alpha\beta)=N(\alpha)N(\beta)$ for all $\alpha,\beta\in R$. 
	\item $\alpha\in\mathcal{U}(\Z[\sqrt{N}])$ if and only if $N(\alpha)=1$. 
	\item If $N(\alpha)$ is prime in $\Z$, then $\alpha$ is irreducible in $R$. 
\end{enumerate}	
\end{lemma}

\begin{proof}
	The first three items are left as an exercises. Let us prove 4). 
	If $\alpha=\beta\gamma$ for some $\beta,\gamma\in R$, then
	$N(\alpha)=N(\beta)N(\gamma)$. Since $N(\alpha)$ is a prime number, it follows that
	$N(\alpha)=1$ or $N()\beta)=1$. Thus $\beta\in\mathcal{U}(R)$ or $\gamma\in\mathcal{U}(R)$. 	
\end{proof}

\begin{example}
	Let $R=\Z[i]$. 
	\begin{enumerate}
		\item $\mathcal{U}(R)=\{-1,1,i,-i\}$.
		\item $3$ is irreducible in $R$. In fact, if $3=\alpha\beta$, then
			$9=N(\alpha)N(\beta)$. This implies that $N(\alpha)\in\{1,3,9\}$. Write
			$\alpha=a+bi$ for $a,b\in\Z$. If $N(\alpha)=1$, then $\alpha\in\mathcal{U}(R)$ by the lemma. 
			If $N(\alpha)=9$, then $N(\beta)=1$ and hence $\beta\in\mathcal{U}(R)$ by the lemma. Finally, 
			if $N(\alpha)=3$, then $a^2+b^2=3$, which is a contradiction since $a,b\in\Z$. 
		\item $2$ is not irreducible in $R$. In fact, $2=(1+i)(1-i)$ and
			since \[
			N(1+i)=N(1-i)=2,
			\]
			it follows that $1+i\not\in\mathcal{U}(R)$ 
			and $1-i\not\in\mathcal{U}(R)$. 
	\end{enumerate}	
\end{example}

\begin{exercise}
	Let $R=\Z[\sqrt{-5}]$. 
	\begin{enumerate}
		\item Prove that $1+\sqrt{-5}$ and $1-\sqrt{-5}$ are irreducible in $R$. 
		\item Prove that $2,3,1+\sqrt{-5}$ and $1-\sqrt{-5}$ are not associate in $R$.
	\end{enumerate}
\end{exercise}

\begin{example}
	Let $R=\Z[\sqrt{-3}]$ and $x=1+\sqrt{-3}$. 
	\begin{enumerate}
		\item $x$ is irreducible. 
	If $x=\alpha\beta$ for some $\alpha,\beta\in R$, then 
	$4=N(x)=N(\alpha)N(\beta)$. Write $\alpha=a+b\sqrt{-3}$ for some $a,b\in\Z$. Then
	$N(\alpha)=a^2+3b^2\ne 2$. If $N(\alpha)=2$, then $a^2+3b^2=2$ and then $a$ and $b$
	both have the same parity. 
	
	If both $a$ and $b$ are even, say $a=2k$ and $b=2l$ for
	some $k,l\in\Z$, then 
	\[
	2=a^2+3b^2=4k^2+12l^2
	\]
	is divisible by 4, a contradiction.  
	
	If both $a$ and $b$ are odd, say $a=2k+1$ and $b=2l+1$ for some $k,l\in\Z$, then
	\[
	2=a^2+3b^2=4k^2+4k+12l^2+12l+4
	\]
	is divisible by 4, a contradiction. 
		\item $x$ is not prime. Note that $(1+\sqrt{-3})(1-\sqrt{-3})=4$, then 
		$x$ divides $4=2\cdot 2$. But $1+\sqrt{-3}\nmid 2$, as 
		\[
		(a-3b)+(a+b)\sqrt{3}=(1+\sqrt{-3})(a+b\sqrt{-3})=2
		\]
		implies that $a-3b=2$ and $a+b=0$, which yields 
		$a=1/2\not\in\Z$, a contradiction.
	\end{enumerate}
\end{example}

\begin{exercise}
	Let $R=\Z[\sqrt{5}]$. Prove that $1+\sqrt{5}$ is irreducible and not prime in $R$. 	
\end{exercise}



