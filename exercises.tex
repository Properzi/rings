\chapter*{Exercises}

\begin{exercise}
	\label{xca:Gauss_finite}
	Let $I$ be a non-zero ideal of $\Z[i]$. Prove that 
	$\Z[i]/I$ is finite.
\end{exercise}

\begin{exercise}
	\label{xca:Zsqrt10}
	Let $R=\Z[\sqrt{10}]$. Prove that $I=(2,\sqrt{10})$ is a maximal ideal. 
\end{exercise}

\begin{exercise}
	\label{xca:RXY_prime}
	Let $R=\R[X,Y]$.  Prove that $(X^3-Y^2)$ is a prime ideal. Is it a maximal
	ideal?
\end{exercise}

\begin{exercise}
	\label{xca:augmentation}
	Let $G$ be a finite group. 
	\begin{enumerate}
		\item Prove that $I=\{\sum_{g\in G}\lambda_gg:\sum_{g\in G}\lambda_g=0\}$ is an ideal of $\C[G]$.
		\item Prove that $I$ is the ideal generated by $\{g-1_G:g\in G\}$.
		\item If $G=\langle g\rangle\simeq\Z$, then $I$ is generated by $g-1_G$. 
	\end{enumerate}
\end{exercise}

\begin{sol}{xca:Gauss_finite}
	Let $R=\Z[i]$. For $u+vi\in R$ let $N(u+iv)=u^2+v^2$. 
	Each element of $R/I$ is of the form $(a+bi)+I$ for $a,b\in \Z$. Since $R$
	is principal, $I=(\alpha)$ for some $\alpha\in R$. Apply the division
	algorithm to write $a+bi=\alpha q+r$ for some $q,r\in R$, where $r=0$ or
	$N(r)<N(\alpha)$. Now $(a+bi)+I=r+I$, as $a+bi-r\in I$. Hence $R/I$ is
	finite as the are finitely many $r$ with $N(r)<N(\alpha)$. 
\end{sol}

\begin{sol}{xca:Zsqrt10}
	Let $f\colon R\to\Z/2$, $a+b\sqrt{10}\mapsto a\bmod 2$. Then $f$ is a
	surjective ring homomorphism such that $\ker
	f=I=\{2x+y\sqrt{10}:x,y\in\Z\}$. By the first isomorphism theorem,
	$R/I\simeq\Z/2$ is a field. Hence $I$ is maximal. 
\end{sol}

\begin{sol}{xca:RXY_prime}
	Let $\varphi\colon\R[X,Y]\to\R[t]$, $f(X,Y)\mapsto f(t^2,t^3)$. 
	Then $\varphi$ is a surjective ring homomorphism. We claim that
	$\ker\varphi=(X^3-Y^2)$. Since 
	\[
		X^mY^n=-(X^3-y^2)X^mY^{n-2}+X^{m+3}Y^{n-2}, 
	\]
	it follows
	that $f(X,Y)=g(X)+h(X)Y+(X^3-Y^2)f_1(X,Y)$, that is 
	\[
	X^mY^n\equiv X^{m+3}Y^{n-2}\bmod I
	\implies
	f(X,Y)\equiv g(X)+h(X)Y\bmod I.
	\]
	By the first isomorphism theorem, $\R[X,Y]/I\simeq\R[t]$. Since $\R[t]$ is
	a domain, it follows that $I$ is a prime ideal. Since $\R[t]$ is not a
	field, $I$ is not a maximal ideal.
\end{sol}

\begin{sol}{xca:augmentation}
	First note that $\epsilon\colon\C[G]\to\C$, $\sum_{g\in G}\lambda_gg\mapsto \sum_{g\in G}\lambda_g$, 
	is a ring homomorphism with kernel equal to $I$. To prove
	that $I$ is generated by the set $\{g-1_G:g\in G\}$ note that, 
	if $\alpha=\sum_{g\in G}\lambda_gg\in \ker\epsilon$, then 
	\[
		\sum_{g\in G}\lambda_gg=\sum_{g\in G}\lambda_g(g-1_G)+\left(\sum_{g\in G}\lambda_g\right)1_G
		=\sum_{g\in G}\lambda_g(g-1_G).
	\]
	
	Now assume that $G=\langle g\rangle\simeq\Z$. Then $G=\{g^k:k\in\Z\}$. 
	Since 
	\[
		g^k-1_G=(g-1_G)(g^{k-1}+\cdots+g+1_G),
	\]
	for all $k\in\Z$, it follows that $g^k-1\in (g-1_G)$. 
\end{sol}
