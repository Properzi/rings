\chapter{Rings}

\begin{definition}
\index{Ring}
A \textbf{ring} is a set $R$ with two binary operations, the addition
$R\times R\to R$, $(x,y)\mapsto x+y$, and the multiplication
$R\times R\to R$, $(x,y)\mapsto xy$, such that
the following properties hold:
\begin{enumerate}
    \item $(R,+)$ is an abelian group.
    \item $(xy)z=x(yz)$ for all $x,y,z\in R$.
    \item $x(y+z)=xy+xz$ for all $x,y,z\in R$.
    \item $(x+y)z=xz+yz$ for all $x,y,z\in R$.
    \item There exists $e\in R$ such that $xe=ex=x$ for all $x\in R$.
\end{enumerate}
\end{definition}

\begin{definition}
\index{Ring!commutative}
A ring $R$ is said to be \textbf{commutative} if $xy=yx$ for all $x,y\in R$. 
\end{definition}

\begin{example}
$\Z$, $\Q$, $\R$ and $\C$ are commutative rings.
\end{example}

\begin{example}
    If $R$ is a commutative ring, then the set 
    \[
    R[X]=\left\{\sum_{i=0}^na_iX^i:n\in\N_0,\,a_1,\dots,a_n\in R\right\}
    \]
    of polynomials is a commutative ring with the usual operations. 
    \item If $A$ is an abelian group, then $\End(A)$ is a ring with
    \[
    (f+g)(x)=f(x)+g(x),\quad
    (fg)(x)=f(g(x)),\quad f,g\in\End(A)\text{ and }x\in A.
    \]
\end{example}

Some facts 