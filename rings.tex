\chapter{Rings}

\begin{definition}
\index{Ring}
A \textbf{ring} is a set $R$ with two binary operations, the addition
$R\times R\to R$, $(x,y)\mapsto x+y$, and the multiplication
$R\times R\to R$, $(x,y)\mapsto xy$, such that
the following properties hold:
\begin{enumerate}
    \item $(R,+)$ is an abelian group.
    \item $(xy)z=x(yz)$ for all $x,y,z\in R$.
    \item $x(y+z)=xy+xz$ for all $x,y,z\in R$.
    \item $(x+y)z=xz+yz$ for all $x,y,z\in R$.
    \item There exists $e\in R$ such that $xe=ex=x$ for all $x\in R$.
\end{enumerate}
\end{definition}

\begin{definition}
\index{Ring!commutative}
A ring $R$ is said to be \textbf{commutative} if $xy=yx$ for all $x,y\in R$. 
\end{definition}

\begin{example}
$\Z$, $\Q$, $\R$ and $\C$ are commutative rings.
\end{example}

\begin{example}
    If $R$ is a commutative ring, then the set 
    \[
    R[X]=\left\{\sum_{i=0}^na_iX^i:n\in\N_0,\,a_1,\dots,a_n\in R\right\}
    \]
    of polynomials is a commutative ring with the usual operations. 
\end{example}

\begin{example}
    If $A$ is an abelian group, then $\End(A)$ is a ring with
    \[
    (f+g)(x)=f(x)+g(x),\quad
    (fg)(x)=f(g(x)),\quad f,g\in\End(A)\text{ and }x\in A.
    \]
\end{example}

Let $R$ be a ring. 
Some facts:
\begin{enumerate}
    \item $x0=0x=x$ for all $x\in R$.
    \item $x(-y)=-xy$ for all $x,y\in R$.
    \item If $1=0$, then $|R|=1$. 
\end{enumerate}

\begin{example}
    The real vector space $H(\R)=\{a1+bi+cj+dk:a,b,c,d\in\R\}$ with basis $\{1,i,j,k\}$ 
    is a ring with the multiplication induced by
    the formulas 
    \[
    i^2=j^2=k^2=-1,
    \quad ij=k,
    \quad jk=i,
    \quad ki=j.
    \]
    As an example, let us perform a calculation in $H(\R)$: 
    \[
    (1+i+j)(i+k)=i+k-1+ik+ji+jk=i+k-1-j-k+i=-1+2i-j,
    \]
    as $ij=i(ij)=-j$. 
\end{example}

\begin{example}
    Let $n\geq2$. 
    The abelian group $\Z/n=\{0,1,\dots,n\}$ of integers modulo $n$ is a ring 
    with the usual multiplication modulo $n$. 
\end{example}

\begin{example}
    Let $n\geq1$. 
    The set $M_n(\R)$ of real $n\times n$ matrices is a ring with the usual matrix operations. Recall
    that if $a=(a_{ij})$ and $b=(b_{ij})$, the multiplication $ab$ is given by
    \[
    (ab)_{ij}=\sum_{k=1}^n a_{ik}b_{kj}.
    \]
\end{example}

\begin{example}
    Real polynomials in two commuting variables form a ring. This ring will be denoted by $\R[X,Y]$. 
\end{example}

\begin{definition}
\index{Subring}
    Let $R$ be a ring. A \textbf{subring} $S$ of $R$ is a subset $S$ such that
    $(S,+)$ is a subgroup of $(R,+)$ such that $1\in S$ and 
    if $x,y\in S$, then $xy\in S$. 
\end{definition}

\begin{example}
    Clearly $\Z$ is a subring of $\Z$. 
\end{example}

\begin{example}
    $\Z\subseteq\Q\subseteq\R\subseteq\C$ is a chain of subrings. 
\end{example}

\begin{example}
    \index{Gauss integers}
    $\Z[i]=\{a+bi:a,b\in\Z\}$ is a subring of $\C$. This is known as the ring of \textbf{Gauss integers}.  
\end{example}

\begin{example}
    $\Q[\sqrt{2}]=\{a+b\sqrt{2}:a,b\in\Q\}$ is a subring of $\R$. 
\end{example}

\begin{example}
    \index{Center!of ring}
    If $R$ is a ring, then the \textbf{center} $Z(R)=\{x\in R:xy=yx\text{for all $y\in R$}\}$ is a subring of $R$. 
\end{example}
