\chapter{Rings}

\begin{definition}
\index{Ring}
A \textbf{ring} is a set $R$ with two binary operations, the addition
$R\times R\to R$, $(x,y)\mapsto x+y$, and the multiplication
$R\times R\to R$, $(x,y)\mapsto xy$, such that
the following properties hold:
\begin{enumerate}
    \item $(R,+)$ is an abelian group.
    \item $(xy)z=x(yz)$ for all $x,y,z\in R$.
    \item $x(y+z)=xy+xz$ for all $x,y,z\in R$.
    \item $(x+y)z=xz+yz$ for all $x,y,z\in R$.
    \item There exists $e\in R$ such that $xe=ex=x$ for all $x\in R$.
\end{enumerate}
\end{definition}

\begin{definition}
\index{Ring!commutative}
A ring $R$ is said to be \textbf{commutative} if $xy=yx$ for all $x,y\in R$. 
\end{definition}

\begin{example}
$\Z$, $\Q$, $\R$ and $\C$ are commutative rings.
\end{example}

\begin{example}
    The set  
    \[
    \R[X]=\left\{\sum_{i=0}^na_iX^i:n\in\N_0,\,a_1,\dots,a_n\in \R\right\}
    \]
    of real polynomials in one variable 
    is a commutative ring with the usual operations. 
\end{example}

More generally, if $R$ is a commutative ring, then $R[X]$ is a commutative ring. This construction
allows us to define 
the polynomial ring $R[X,Y]$ in two commuting variables $X$ and $Y$ and coefficients in $R$ as 
$R[X,Y]=(R[X])[Y]$. One can also define the ring  
$R[X_1,\dots,X_n]$ of real polynomials 
in $n$ commuting variables $X_1,\dots,X_n$ with coefficients in $R$ as 
$R[X_1,\dots,X_n]
=(R[X_1,\dots,X_{n-1}])[X_n]$.

\begin{example}
    If $A$ is an abelian group, then $\End(A)$ is a ring with
    \[
    (f+g)(x)=f(x)+g(x),\quad
    (fg)(x)=f(g(x)),\quad f,g\in\End(A)\text{ and }x\in A.
    \]
\end{example}

Let $R$ be a ring. 
Some facts:
\begin{enumerate}
    \item $x0=0x=x$ for all $x\in R$.
    \item $x(-y)=-xy$ for all $x,y\in R$.
    \item If $1=0$, then $|R|=1$. 
\end{enumerate}

\begin{example}
    The real vector space $H(\R)=\{a1+bi+cj+dk:a,b,c,d\in\R\}$ with basis $\{1,i,j,k\}$ 
    is a ring with the multiplication induced by
    the formulas 
    \[
    i^2=j^2=k^2=-1,
    \quad ij=k,
    \quad jk=i,
    \quad ki=j.
    \]
    As an example, let us perform a calculation in $H(\R)$: 
    \[
    (1+i+j)(i+k)=i+k-1+ik+ji+jk=i+k-1-j-k+i=-1+2i-j,
    \]
    as $ij=i(ij)=-j$. 
\end{example}

\begin{example}
    Let $n\geq2$. 
    The abelian group $\Z/n=\{0,1,\dots,n\}$ of integers modulo $n$ is a ring 
    with the usual multiplication modulo $n$. 
\end{example}

\begin{example}
    Let $n\geq1$. 
    The set $M_n(\R)$ of real $n\times n$ matrices is a ring with the usual matrix operations. Recall
    that if $a=(a_{ij})$ and $b=(b_{ij})$, the multiplication $ab$ is given by
    \[
    (ab)_{ij}=\sum_{k=1}^n a_{ik}b_{kj}.
    \]
\end{example}

\begin{definition}
\index{Subring}
    Let $R$ be a ring. A \textbf{subring} $S$ of $R$ is a subset $S$ such that
    $(S,+)$ is a subgroup of $(R,+)$ such that $1\in S$ and 
    if $x,y\in S$, then $xy\in S$. 
\end{definition}

Clearly, $\Z\subseteq\Q\subseteq\R\subseteq\C$ is a chain of subrings. 

\begin{example}
    \index{Gauss integers}
    $\Z[i]=\{a+bi:a,b\in\Z\}$ is a subring of $\C$. 
    This is known as the ring of \textbf{Gauss integers}.  
\end{example}

\begin{example}
    $\Q[\sqrt{2}]=\{a+b\sqrt{2}:a,b\in\Q\}$ is a subring of $\R$. 
\end{example}

\begin{example}
    \index{Center!of ring}
    If $R$ is a ring, then the \textbf{center} 
    $Z(R)=\{x\in R:xy=yx\text{ for all $y\in R$}\}$ 
    is a subring of $R$. 
\end{example}

If $S$ is a subring of a ring $R$, then the zero element 
of $S$ is the zero element of $R$, i.e. $0_R=0_S$. Moreover, 
the additive inverse of an element $s\in S$ 
is the additive inverse of $s$ as an element of $R$. 	

\begin{exercise}\
\begin{enumerate}
	\item If $S$ and $T$ are subrings of $R$, then $S\cap T$ is a subring of $R$.
	\item If $R_1\subseteq R_2\subseteq\cdots$ is a sequence of subrings of $R$, then 
	$\cup_{i\geq1}R_i$ is a subring of $R$. 
\end{enumerate}
\end{exercise}

\begin{definition}
\index{Units}
	Let $R$ be a ring. An element $x\in R$ is a \textbf{unit} if there exists $y\in R$ such that $xy=yx=1$. 
\end{definition}

The set $\mathcal{U}(R)$ of units of a ring $R$ form a group with the multiplication. For example,
$\mathcal{U}(\Z/8)=\{1,3,5,7\}$. 

\begin{definition}
	\index{Ring!division}
	A \textbf{division ring} is a ring $R$ 
	such that $\mathcal{U}(R)=R\setminus\{0\}$.  	
\end{definition}

One shows that $H(\R)$ is a non-commutative division ring. 

\begin{definition}
\index{Field}
	A \textbf{field} is a commutative division ring with $1\ne 0$. 
\end{definition}

Clearly, $\Q$, $\R$ and $\C$ are fields. 
If $p$ is a prime number, then $\Z/p$ is a field. 	

\begin{exercise}
	$\Q[\sqrt{2}]$ is a field. 
	Find the multiplicative inverse of $x+y\sqrt{2}\in\Q[\sqrt{2}]$.  
\end{exercise}

More challenging: Prove that 
\[
\Q[\sqrt[3]{2}]=\{x+y\sqrt[3]{2}+z\sqrt[3]{4}:x,y,z\in\Q\}
\]
is a field. What is the inverse of $x+y\sqrt[3]{2}+z\sqrt[3]{4}$?

\begin{definition}
\index{Ideal!left}
	Let $R$ be a ring. A \textbf{left ideal} of $R$ is a subset $I$ such that 
	$(I,+)$ is a subgroup of $(R,+)$ and such that $RI\subseteq I$, 
	i.e. $ry\in I$ for all $r\in R$ and $y\in I$. 
\end{definition}

Similarly one defines right ideals. 

\begin{example}
	
\end{example}

Can you find an 
example of a left ideal that is not a right ideal?

\begin{definition}
\index{Ideal}	
Let $R$ be a ring. An ideal of $R$ is a subset that is both a left and a right ideal of $R$. 
\end{definition}
 
If $R$ is a ring, then $\{0\}$ and $R$ are both ideals of $R$. 

\begin{exercise}
Let $R$ be a ring. 
\begin{enumerate}
\item If $\{I_\alpha:\alpha\}$ is a collection of ideals of $R$, then $\cap_{\alpha}I_\alpha$ is an ideal of $R$.  	
\item If $I_1\subseteq I_2\subseteq\cdots$ is a sequence of ideals of $R$, then $\cup_{i\geq1}I_i$ is an ideal of $R$. 
\end{enumerate}
\end{exercise}

\begin{example}
Let $R=\R[X]$. If $f(X)\in R$, then the set 
\[
(f(X))=\{f(X)g(X):g(X)\in R\}
\]
of multiples of $f(X)$ is an ideal of $R$. One can prove that this is the smallest 
ideal of $R$ containing $f(X)$.  	
\end{example}

If $R$ is a ring and $X$ is a subset of $R$, one defines
the ideal generated by $X$ as the smallest ideal of $R$ containing $X$, that is 
\[
(X)=\bigcap\{I:\text{$I$ ideal of $R$ such that $X\subseteq I$}\}.
\]
It is possible to prove that 
\[
(X)=\left\{\sum_{i=1}^mr_ix_is_i:m\in\N_0,\,r_1,\dots,r_m,s_1,\dots,s_m\in R\right\}, 
\]
where by convention the empty sum is equal to zero. If $X=\{x_1,\dots,x_n\}$ is a finite
set, then by write $(X)=(x_1,\dots,x_n)$. 

\begin{exercise}
Prove that every ideal of $\Z$ is of the form $n\Z$ for some $n\geq0$. 	
\end{exercise}

\begin{exercise}
Let $n\geq2$. Find the ideals of $\Z/n$. 	
\end{exercise}

\begin{exercise}
Find the ideals of $\R$.	
\end{exercise}

One proves that a field $K$ has only two ideals. 

\begin{definition}
\index{Ring!homomorphism}
Let $R$ and $S$ be rings. A map $f\colon R\to S$ is a \textbf{ring homomorphism}  
if $f(1)=1$, $f(x+y)=f(x)+f(y)$ and $f(xy)=f(x)f(y)$ for all $x,y\in R$. 	
\end{definition}

Our definition of a ring is that of a ring with identity. This means
that the identity element $1$ of a ring $R$ 
is part of the structure. For that reason, in 
the definition
of a ring homomorphism $f$ one needs $f(1)=1$.  

\begin{example}
The map $f\colon\Z/6\to\Z/6$, $x\mapsto 3x$, is not a ring homomorphism because
$f(1)=3$. 	
\end{example}
 
If $R$ is a ring, then  
the identity map $\id\colon R\to R$, $x\mapsto x$, is a ring homomorphism. 	

\begin{example}
The inclusions $\Z\hookrightarrow\Q\hookrightarrow\R\hookrightarrow\C$ are ring homomorphisms. 	
\end{example}

More generally, if $S$ is a subring of a ring $R$, then the inclusion map 
$S\hookrightarrow R$ is a ring homomorphism. 

\begin{example}
Let $R$ be a ring. 
The map $\Z\to R$, $k\mapsto k1$, is a ring homomorphism. 	
\end{example}

\begin{example}
Let $x_0\in\R$. The evaluation map $\R[X]\to\R$, $f\mapsto f(x_0)$, 
is a ring homomorphism. 	
\end{example}

The \textbf{kernel} of a ring homomorphism
$f\colon R\to S$ is the subset
\[
\ker f=\{x\in R:f(x)=0\}.
\]
One proves that the kernel of $f$ is an ideal of $R$. 
Moreover, $\ker f=\{0\}$ if and only if $f$ is injective. The image 
\[
f(R)=\{f(x):x\in R\}
\]
is a subring of $S$. In general, $f(R)$ is not an ideal of $S$. 

\begin{example}
The map $\C\to M_2(\R)$, $a+bi\mapsto\begin{pmatrix}a&b\\-b&a\end{pmatrix}$, is an injective
ring homomorphism. 	
\end{example}

\begin{example}
The map $\Z[i]\to\Z/5$, $a+bi\mapsto a+2b\bmod 5$, is a ring homomorphism 
with $\ker f=\{a+bi:a+2b\equiv 0\bmod 5\}$. 	
\end{example}

\begin{exercise}
There is no ring homomorphism $\Z/6\to\Z/15$. Why?	
\end{exercise}

\begin{exercise}
If $f\colon\R[X]\to\R$ is a ring homomorphism such that the restriction $f|_{\R}$ of 
$f$ onto $\R$ is the identity, then there exists $x_0\in\R$ such that 
$f$ is the evaluation map at $x_0$. 
\end{exercise}

We now define ring quotients. 


