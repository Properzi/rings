\chapter{Zorn's lemma}

\begin{definition}
A non-empty set $R$ is said to be a \textbf{partially ordered set} (or poset, for short) 
if there is a subset $X\subseteq R\times R$ such that
\begin{enumerate}
    \item $(r,r)\in X$ for all $r\in R$, 
    \item if $(r,s)\in X$ and $(s,t)\in X$, then $(r,t)\in X$, and 
    \item if $(r,s)\in X$ and $(s,r)\in X$, then $r+s$. 
\end{enumerate}
\end{definition}

The set $X$ is a partial order relation on $R$.  
We will use the following notation: $(r,s)\in X$ if and only if $r\leq s$. Moreover, 
$r<s$ if and only if $r\leq s$ and $r\ne s$. 

\begin{definition}
Let $R$ be a poset and $r,s\in R$. Then $r$ and $s$ are \textbf{comparable}
if either $r<s$ or $s<r$.
\end{definition}

\begin{example}
Let $U=\{1,2,3,4,5\}$ and $T$ be the set of subsets of $U$. Then $T$ is a poset
with the usual inclusion, that is $C\leq D$ if and only if $C\subseteq D$. The subsets
$\{1,2\}$ and $\{3,4\}$ of $U$ are elements of $T$ that are not comparable. 
\end{example}

\begin{definition}
    Let $R$ be a poset and $r\in R$. Then $r$ is \textbf{maximal} in $R$ if 
    such that $r\leq t$ implies $r=t$. 
\end{definition}

\begin{example}
$\Z$ has no maximal elements. 
\end{example}

\begin{example}
Let $R=\{(x,y)\in\R^2:y\leq0\}$ with $(x_1,y_1)\leq(x_2,y_2)$ if and only if $x_1=x_2$ and $y_1\leq y_2$. Then
$R$ is a poset and each $(x,0)$ is maximal. Thus $R$ has infinitely many maximal elements.
\end{example}

\begin{definition}
    Let $R$ be a poset and $S$ be a non-empty subset of $R$. An \textbf{upper bound}
    for $S$ is an element $u\in R$ such that $s\leq u$ for all $s\in S$. 
\end{definition}

\begin{example}
    Let $S=\{6\Z,12\Z,24\Z\}$ be a subset of the set of subgroups of $\Z$. Then 
    $6\Z=6\Z\cap 12\Z\cap 24\Z$ is an upper bound of $S$. 
\end{example}

\begin{definition}
    Let $R$ be a poset. A \textbf{chain} is a non-empty subset $S$ of $R$ such that
    any two elements of $S$ are comparable. 
\end{definition}

We now state Zorn's lemma:

\begin{quote}
Let $R$ be a poset such that 
every chain in $R$ admits an upper bound in $R$. 
Then $R$ contains a maximal element. 
\end{quote}

It is not intuitive, but it is logically equivalent to a more 
intuitively statement in set theory, the Axiom of Choice, 
which says every cartesian product of non-empty sets is non-empty. 
It is more an axiom that a lemma. 
The reason for calling Zorn’s lemma a lemma rather 
than an axiom is purely historical.

\begin{definition}
	\index{Ideal!maximal}
	Let $R$ be a ring. An ideal $I$ of $R$ is said to be \textbf{maximal}
	if $I\ne R$ and if $J$ is an ideal of $R$ such that $I\subseteq J$, then 
	either $I=J$ or $J=R$. 
\end{definition}

If $p$ is a prime number, then $p\Z$ is a maximal ideal of $\Z$.

\begin{exercise}
Let $R$ be a commutative ring. Prove that $R$ is a 
field if and only if $\{0\}$ is a maximal ideal of $R$. 	
\end{exercise}

\begin{exercise}
Let $R$ be a commutative ring and $I$ be an ideal of $R$. Prove that $I$ is maximal
if and only if $R/I$ is a field.  	
\end{exercise}

The following application of Zorn's lemma uses 
the identity of a ring. 

\begin{theorem}
	Let $R$ be a ring. Each proper ideal $I$ of $R$ 
	is contained in a maximal ideal. 
	In particular, all rings have maximal ideals. 	
\end{theorem}

\begin{proof}
	Let $X=\{J\subseteq R:J\text{ is an ideal of $R$ such that }I\subseteq J\subsetneq R\}$.
	Since $I\in X$, it follows that $X$ is non-empty. Moreover, $X$ is a poset
	with respect to the inclusion. If $C$ is a chain in $X$, then 
	$\cup_{J\in C}J$ is an upper bound for $C$, as $\cup_{J\in C}J$ is an ideal and
	$\cup_{J\in C}J\ne R$ because $1\not\in\cup_{J\in C}J$. 	Zorn's lemma implies that
	there exists $M\in X$ maximal. We claim that $M$ is a maximal ideal of $R$. The definition
	of $X$ implies that $M$ is a proper ideal of $R$. If $M_1$ is an ideal of $R$
	such that $M\subseteq M_1$, it follows that $I\subseteq M_1$ and hence $M_1\in X$. The maximality
	of $M$ implies that $M=M_1$.  
\end{proof}

\begin{exercise}
	Compute the maximal ideals of $\R[X]$ and $\C[X]$. 	
\end{exercise}

One can also compute the maximal ideals of $K[X]$ for any field $K$. 

\begin{exercise}
	Let $R$ be a principal domain and $p\in R$. Then $p$ is irreducible 
	if and only if $(p)$ is maximal.	
\end{exercise}

\begin{example}
	The ideal $(X^2+2X+2)$ is maximal in $\Q[X]$ because
	\[
	X^2+2X+2=(X+1)^2+1
	\]
	has degree two and no rational roots. 
	Hence $X^2+2X+2$ is irreducible in $\Q[X]$ and it generates 
	a maximal ideal. 	
\end{example}

\begin{exercise}
	Let $R$ be a commutative ring and $I$ be an ideal of $R$. Then 
	$I$ is maximal if and only if $R/I$ is a field. 
\end{exercise}

\begin{example}
	Let $R=(\Z/2)[X]$ and $f(X)=X^2+X+1$. Since $f(X)$ is irreducible (because $\deg f(X)=2$ and
	$f(X)$ has no roots in $\Z/2$, it follow that $(f(X))$ is a maximal ideal. 
	Thus $R/I$ is a field. 
\end{example}

\begin{exercise}
	Compute the maximal ideals of $\Z/n$. 	
\end{exercise}

\begin{exercise}
	Let $R$ be a commutative ring and $J(R)$ be the intersection of all maximal ideals 
	of $R$. Prove that $x\in J(R)$ 
	if and only if $1-xy\in\mathcal{U}(R)$ for all $y\in R$. The ideal $J(R)$ is proper
	and it is known
	as the \textbf{Jacobson radical} of $R$.  	
\end{exercise}

We conclude the chaper with a different application of Zorn's lemma.  

\begin{exercise}
	Prove that every non-zero vector space has a basis.
\end{exercise}

The previous exercise can be used to solve the following exercises:

\begin{exercise}
    Prove that there exists a group homomorphism $f\colon\R\to\R$ that 
    is not of the form $f(x)=\lambda x$ for some $\lambda\in\R$. 
\end{exercise}

\begin{exercise}
    Prove that the abelian groups $\R^n$ and $\R$ are isomorphic.
\end{exercise}

\begin{exercise}
    Prove that if $G$ is a group such that $|G|>2$, then $|\Aut(G)|>1$.
\end{exercise}
