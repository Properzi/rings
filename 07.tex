\chapter{}

Recall that, by convention, we only consider complex 
finite-dimensional representations of finite groups.

\begin{theorem}[Maschke]
\index{Maschke's theorem}
    Every representation of a finite group is completely reducible.
\end{theorem}

\begin{proof}
    Let $G$ be a finite group and $\rho\colon G\to\GL(V)$ be a representation of $G$. We proceed
    by induction on $\dim V$.
    If $\dim V=1$, the result is trivial, as degree-one representations are irreducible. Assume that
    the result holds for representations of degree $\leq n$. Suppose that $\rho$ has degree $n+1$. 
    If $\rho$ is irreducible, we are done. If not, write $V=S\oplus T$, where $S$ and $T$
    are non-zero invariant subspaces of $V$. Since $\dim S<\dim V$ and $\dim T<\dim V$, it follows from
    the inductive hypothesis that
    both $S$ and $T$ are spaces of completely reducible representations. 
    Thus $\rho$ is completely reducible.
\end{proof}

\begin{example}
    Let $G=\Sym_3$ and $\rho\colon G\to\GL_3(\C)$ be the representation given by
    \[
    (12)\mapsto\begin{pmatrix}
    0&1&0\\
    1&0&0\\
    0&0&1
    \end{pmatrix},\quad
    (123)\mapsto\begin{pmatrix}
    0&0&1\\
    1&0&0\\
    0&1&0
    \end{pmatrix}
    \]
    Then $\rho_g$ is unitary for all $g\in G$ (because $\rho_{(12)}$ and $\rho_{(123)}$ are both
    unitary). Moreover,
    \[
    S=\left\langle \begin{pmatrix}
    1\\1\\1
    \end{pmatrix}
    \right\rangle,
    \quad
    T=S^{\perp}=\left\langle
    \begin{pmatrix}
    -1\\1\\0
    \end{pmatrix},
    \begin{pmatrix}
    0\\-1\\1
    \end{pmatrix}
    \right\rangle,
    \]
    are irreducible invariant subspaces of $V=\C^3$. A direct calculation shows that
    in the orthogonal basis $\left\{\begin{pmatrix}
    1\\1\\1
    \end{pmatrix},
    \begin{pmatrix}
    -1\\1\\0
    \end{pmatrix},
    \begin{pmatrix}
    0\\-1\\1
    \end{pmatrix}
    \right\}$
    the matrices $\rho_{(12)}$ and $\rho_{(123)}$ can be written as
    \[
    \rho_{(12)}=\begin{pmatrix}
        1&0&0\\
        0&-1&1\\
        0&0&1
    \end{pmatrix},
    \quad
    \rho_{(123)}=
    \begin{pmatrix}
        1&0&0\\
        0&0&-1\\
        0&1&-1
    \end{pmatrix}.
    \]
\end{example}

\begin{exercise}
Let $G$ be a finite group.
Prove that there is a bijection between degree-one representations of $G$ and
degree-one representations of $G/[G,G]$.
\end{exercise}

The following result is simple and crucial. 

\begin{lemma}[Schur]
\index{Schur's!lemma}
    Let $\rho\colon G\to\GL(V)$ and $\psi\colon G\to\GL(W)$ be irreducible representations. If 
    $T\colon V\to W$ is a non-zero invariant map, then $T$ is bijective.  
\end{lemma}

\begin{proof}
    Since $T$ is non-zero and $\ker T$ is an invariant subspace of $V$, it follows that $\ker T=\{0\}$, as $\rho$ is irreducible. Thus 
    $T$ is injective. Since $T(V)$ is a non-zero invariant subspace of $W$, it follows from the fact that $\psi$ is irreducible 
    that $T$ is surjective. Therefore $T$ 
    is bijective.  
\end{proof}

Two applications:

\begin{proposition}
    If $\rho\colon G\to\GL(V)$ is an irreducible representation and $T\colon V\to V$ is invariant, then 
    $T=\lambda\id$ for some $\lambda\in\C$. 
\end{proposition}

\begin{proof}
    Let $\lambda$ be an eigenvalue of $T$. Then $T-\lambda\id$ is invariant, as 
    \[
    (T-\lambda\id)\rho_g=T\rho_g-\lambda\rho_g=\rho_g(T-\lambda\id)
    \]
    for all $g\in G$ since $T$ is invariant. By definition, 
    $T-\lambda\id$ is not bijective. Thus $T-\lambda\id=0$ by Schur's lemma.
\end{proof}

\begin{proposition}
    Let $G$ be a finite abelian group. 
    If $\rho\colon G\to\GL(V)$ is an irreducible representation, then
    $\dim V=1$. 
\end{proposition}

\begin{proof}
    Let $h\in G$. Note that since $G$ is abelian, $T=\rho_h$ is invariant:
    \[
    T\rho_g=\rho_h\rho_g=\rho_{hg}=\rho_{gh}=\rho_g\rho_h=\rho_gT.
    \]
    By the previous proposition, 
    there exists $\lambda_h\in\C$ such that $\rho_h=\lambda_h\id$. If $v\in V\setminus\{0\}$, 
    then $V=\langle v\rangle$. In fact, since 
    $\langle v\rangle$ is a non-zero invariant subspace of $V$ and $\rho$ is irreducible, 
    it follows that $V=\langle v\rangle$. 
\end{proof}

\topic{Characters}

This lecture is devoted to study character theory. Fix a group and consider
(matrix) representations of groups. How can we study those matrices? Since 
equivalence of representations translates into equivalence of matrices, 
it makes sense to use linear algebra. We can use 
the characteristic polynomial of the matrix, 
\[
A\in\C^{n\times n}\rightsquigarrow \chi_A(X)=a_0+a_1X+\cdots+a_nX^n\in \C[X], 
\]
or any of the numbers $a_0,\dots,a_n\in\C$, as all of them are indeed invariants of the matrix $A$.
The determinant and the trace of $A$ are examples of such numbers. In the context of group representations,  
the trace is particularly interesting. 

\begin{definition}
	\index{Character}
	Let $\rho\colon G\to\GL(V)$ be a representation. The \textbf{character} of $\rho$ 
	is the map $\chi_\rho\colon G\to\C$, $g\mapsto\trace\rho_g$. 	
\end{definition}

If a representation $\rho$ is irreducible, its character is said to be an 
\textbf{irreducible character}. The \textbf{degree} of a character is the degree of the affording
representation. 

\begin{proposition}
	Let $\rho\colon G\to\GL(V)$ be a representation, $\chi$ be its character and $g\in G$.
	The following statements hold:
	\begin{enumerate}
		\item $\chi(1)=\dim V$. 
		\item $\chi(g)=\chi(hgh^{-1})$ for all $h\in G$.
		\item $\chi(g)$ is the sum of $\chi(1)$ roots of one of order $|g|$. 
		\item $\chi(g^{-1})=\overline{\chi(g)}$. 
		\item $|\chi(g)|\leq\chi(1)$.  
	\end{enumerate} 
\end{proposition}

\begin{proof}
	The first statement is trivial. 	To prove 2) note that
	\[
	\chi(hgh^{-1})=\trace(\rho_{hgh^{-1}})=\trace(\rho_h\rho_g\rho_h^{-1})=\trace\rho_g=\chi(g).
	\]
	Statement 3) follows from the fact that the trace of $\rho_g$ is the sum
	of the eigenvalues of $\rho_g$ and these numbers are roots of the polynomial
	$X^{|g|}-1\in\C[X]$. To prove 4) write $\chi(g)=\lambda_1+\cdots+\lambda_k$, where 
	the $\lambda_j$ are roots of one. Then
	\[
	\overline{\chi(g)}=\sum^k_{j=1}\overline{\lambda_j}
	=\sum_{j=1}^k\lambda_j^{-1}
	=\trace(\rho_g^{-1})
	=\trace(\rho_{g^{-1}})
	=\chi(g^{-1}).
	\] 
	Finally, we prove 5). Use 3) to write $\chi(g)$ as the sum of
	$\chi(1)$ roots of one, say $\chi(g)=\lambda_1+\cdots+\lambda_k$ for
	$k=\chi(1)$. Then 
	\[
	|\chi(g)|=|\lambda_1+\cdots+\lambda_k|\leq |\lambda_1|+\cdots+|\lambda_k|
	=\underbrace{1+\cdots+1}_{\text{$k$-times}}=k.\qedhere
	\]
\end{proof}

If two representations are equivalent, their characters are equal.

\begin{definition}
	Let $G$ be a group and 
	$f\colon G\to\C$ be a map. Then $f$ is a \textbf{class function} if
	$f(g)=f(hgh^{-1})$ for all $g,h\in G$. 	
\end{definition}

Characters are class functions. 

\begin{proposition}
    If $\rho\colon G\to\GL(V)$ and
    $\psi\colon G\to\GL(W)$ are representations, then
    $\chi_{\rho\oplus\psi}=\chi_\rho+\chi_\psi$.
\end{proposition}

\begin{proof}
  For $g\in G$, it follows that 
  $(\rho\oplus\psi)_g=
  \begin{pmatrix}
    \rho_g & 0\\ 
    0 & \psi_g
  \end{pmatrix}$. 
  Thus  
  \[
    \chi_{\rho\oplus\psi}(g)=\trace((\rho\oplus\phi)_g)=\trace(\rho_g)+\trace(\psi_g)=\chi_\rho(g)+\chi_\psi(g).\qedhere
  \]
\end{proof}

Let $V$ be a vector space with basis $\{v_1,\dots,v_k\}$ and 
$W$ be a vector space with basis $\{w_1,\dots,w_l\}$. A 
\textbf{tensor product} of $V$ and $W$ is a vector space $X$ with 
together with a bilinear map 
\[
V\times W\to X,
\quad
(v,w)\mapsto v\otimes w,
\]
such that $\{v_i\otimes w_j:1\leq i\leq k,\,1\leq j\leq l\}$ is a  
basis of $X$. The tensor product of $V$ and $W$ is unique up to isomorphism 
and is denoted by $V\otimes W$. Note that
\[
\dim(V\otimes W)=(\dim V)(\dim W).
\]
Note that every element of $V\otimes W$ is of the form
\[
\sum_{i,j}\lambda_{ij}v_i\otimes v_j
\]
and not of the form $v\otimes w$ for $v\in V$ and $w\in W$. 

The bilinearity of tensor products is crucial. 
For example,
\begin{align*}
    (v_1+v_3)\otimes (3w_1+w_2) 
    &=v_1\otimes (3w_1+w_2)+v_3\otimes (3w_1+w_2)\\
    &=v_1\otimes (3w_1)+v_1\otimes w_2+v_3\otimes (3w_1)+v_3\otimes w_2\\
    &=3v_1\otimes w_1+v_1\otimes w_2+3v_3\otimes w_1+v_3\otimes w_2.
\end{align*}

\begin{definition}
	Let $\rho\colon G\to\GL(V)$ and $\psi\colon G\to\GL(W)$ be representations. The \textbf{tensor product} of $\rho$ and $\psi$ is the representation of $G$ given by 
	\begin{gather*}
	\rho\otimes\psi\colon G\to\GL(V\otimes W),
	\quad 
	g\mapsto (\rho\otimes\psi)_g,
	\shortintertext{where}
	(\rho\otimes\psi)_g(v\otimes w)=\rho_g(v)\otimes \psi_g(w)
	\end{gather*}
	for $v\in V$ and $w\in W$.  	
\end{definition}

A direct calculation shows that the tensor product of representations is indeed a representation. 

\begin{proposition}
  	If $\rho\colon G\to\GL(V)$ and
    $\psi\colon G\to\GL(W)$ are representations, then
    \[
    \chi_{\rho\otimes\psi}=\chi_\rho\chi_\psi.
    \]
\end{proposition}

\begin{proof}
	For each $g\in G$, the map $\rho_g$ is diagonalizable. Let $\{v_1,\dots,v_n\}$
	be a basis of eigenvectors of $\rho_g$ and let $\lambda_1,\dots,\lambda_n\in\C$ be such that
	$\rho_g(v_i)=\lambda_iv_i$ for all $i\in\{1,\dots,n\}$. Similarly, 
	let $\{w_1,\dots,w_m\}$ be a basis of 
	eigenvectors of $\psi_g$ and $\mu_1,\dots,\mu_m\in\C$ be such that $\psi_g(w_j)=\mu_jw_j$ for all $j\in\{1,\dots,m\}$. Each 
	$v_i\otimes w_j$ is eigenvector of $\rho\otimes\psi$ with eigenvalue 
	$\lambda_i\mu_j$, as  
	\[
		(\rho\otimes\psi)_g(v_i\otimes w_j)=\rho_gv_i\otimes \psi_gw_j=\lambda_iv_i\otimes \mu_jv_j=(\lambda_i\mu_j)v_i\otimes w_j.
	\]
	Thus  
	$\{v_i\otimes w_j:1\leq i\leq n,1\leq j\leq m\}$ is a basis of eigenvectors and the 
	$\lambda_i\mu_j$ are the eigenvalues of $(\rho\otimes\psi)_g$. It follows that 
	\[
	\chi_{\rho\otimes\psi}(g)
	=\sum_{i,j}\lambda_i\mu_j
	=\left(\sum_i\lambda_i\right)\left(\sum_j\mu_j\right)
	=\chi_\rho(g)\chi_\psi(g).\qedhere 
	\]
\end{proof}

For completeness, we mention without proof that
it is also possible to define the dual $\rho^*\colon G\to\GL(V^*)$  
of a representation
$\rho\colon G\to\GL(V)$ by the formula
\[
(\rho^*_gf)(v)=f(\rho^{-1}_gv),\quad
g\in G,\,f\in V^*\text{ and }v\in V.
\]  
We claim that the character of the dual representation is then 
$\overline{\chi_\rho}$. Let $\{v_1,\dots,v_n\}$ be a basis of $V$
and $\lambda_1,\dots,\lambda_n\in\C$ be such that $\rho_gv_i=\lambda_iv_i$ for all $i\in\{1,\dots,n\}$. If $\{f_1,\dots,f_n\}$ is the dual basis of $\{v_1,\dots,v_n\}$, then 
\[
(\rho^*_gf_i)(v_j)=f_i(\rho_g^{-1}v_j)
=\overline{\lambda_j}f_i(v_j)
=\overline{\lambda_j}\delta_{ij}
\]
and the claim follows. 
%\begin{proposition}
%	
%\end{proposition}
%
%\begin{proof}
%	Let $g\in G$ and $\{v_1,\dots,v_n\}$
%	be a basis of eigenvectors of $\rho_g$. Let
%	$\lambda_1,\dots,\lambda_n\in\C$ be such that $\rho_g(v_i)=\lambda_iv_i$ for all
%	$i\in\{1,\dots,n\}$. Let $\{f_1,\dots,f_n\}$ be the dual basis of $\{v_1,\dots,v_n\}$.
%	Since $\rho_g$ is invertible, each eigenvector of $\rho_g$ is non-zero. 
%	Thus $\rho_g(v_i)=\lambda_iv_i$ implies that 
%	$\rho_{g^{-1}}v_i=\lambda_i^{-1}v_i=\overline{\lambda_i}v_i$... 
%%	Now 
%%	\[
%		(\rho_g f_i)(v_j)=f_i(g^{-1}v_j)=\overline{\lambda_j}f_i(v_j)=\overline{\lambda_j}\delta_{ij}.
%	\]
%	
%	  
%	
%	 We claim
%	that $\{f_1,\dots,f_n\}$ is a basis of eigenvectors...
%	with $\overline{\lambda_1},\dots,\overline{\lambda_n}$. En efecto, si $gv_j=\lambda_jv_j$, entonces
%	$g^{-1}v_j=\lambda_j^{-1}v_j=\overline{\lambda_j}v_j$ (observemos que como $\phi_g$ es inversible, los $\lambda_j$ son no nulos). Luego
%	\[
%		(gf_i)(v_j)=f_i(g^{-1}v_j)=\overline{\lambda_j}f_i(v_j)=\overline{\lambda_j}\delta_{ij}.
%	\]
%	En conclusión
%	\[
%		\chi_{V^*}(g)=\sum_{i=1}^n\overline{\lambda_i}=\overline{\chi_V(g)}.\qedhere
%	\]	
%\end{proof}

\topic{Schur's orthogonality relations}

Let $\rho\colon G\to\GL(V)$ and $\psi\colon G\to\GL(W)$ be representations of a finite group
$G$. Since $V$ and $W$ are vector spaces, the set 
\[
\Hom(V,W)=\{T\colon V\to W:\text{$T$ is linear}\}
\]
is a vector space with 
\begin{align*}
&(\lambda T)(v)=\lambda T(v) && \text{for all $\lambda\in\C$ and all $v\in V$,}\\ 
&(T+T_1)(v)=T(v)+T_1(v) &&\text{for all $v\in V$.}
\end{align*}
We claim that the set $\Hom_G(V,W)$ of invariant maps
is a subspace of $\Hom(V,W)$. Indeed, the zero map is clearly invariant. If $T,T_1\in\Hom_G(V,W)$ 
and $\lambda\in\C$, then
\[
(T+\lambda T_1)(\rho_g v)
=T(\rho_gv)+\lambda T_1(\rho_gv)
=\psi_gT(v)+\lambda \psi_gT_1(v)
=\psi_g((T+\lambda T_1)(v))
\]
for all $v\in V$. 
 
\begin{proposition}
	Let $\rho\colon G\to\GL(V)$ and $\psi\colon G\to\GL(W)$ be representations
	and $T\colon V\to W$ be a linear map. Then
	\[
	T^{\#}=\frac{1}{|G|}\sum_{g\in G}\psi_{g^{-1}}T\rho_g\in\Hom_G(V,W).
	\]
	Moreover, the map $\Hom(V,W)\to\Hom_G(V,W)$, $T\mapsto T^{\#}$, is linear and surjective.  
\end{proposition}

\begin{proof}
  Let $h\in G$ and $v\in V$. Then 
  \begin{align*}
	T^{\#}\rho_h(v)
	&=\frac{1}{|G|}\sum_{g\in G}\psi_{g^{-1}}T\rho_g\rho_h(v)
	=\frac{1}{|G|}\sum_{g\in G}\psi_{g^{-1}}T\rho_{gh}(v)\\
	&=\frac{1}{|G|}\sum_{x\in G}\psi_{hx^{-1}}T\rho_x(v)
	=\frac{1}{|G|}\sum_{x\in G}\psi_h\psi_{x^{-1}}T\rho_x(v)
	=\psi_hT^{\#}(v).
      \end{align*}

    It is a routine calculation to show that $T\mapsto T^{\#}$ is a linear map. The map
    $T\mapsto T^{\#}$ is surjective since 
    if $T\in\Hom_G(V,W)$, then  
	\[
	T^{\#}(v)=\frac{1}{|G|}\sum_{g\in G}\psi_{g^{-1}}T\rho_g(v)
	=\frac{1}{|G|}\sum_{g\in G}\psi_{g^{-1}}\psi_gT(v)
	=T(v)
      \]
    holds for all $v\in V$, so $T^{\#}=T$.  
\end{proof}

If $\rho\colon G\to\GL(V)$ and $\psi\colon
	G\to\GL(W)$ are non-equivalent irreducible representations and $T\colon
	V\to W$ is a linear map, then $T^{\#}=0$, as 
	$T^{\#}\in\Hom_{G}(V,W)=\{0\}$ by the previous proposition and Schur's lemma.

\begin{theorem}[Ergodic theorem]
\index{Ergodic theorem}
  Let $\rho\colon G\to\GL(V)$ be an irreducible representation. 
  If $T\colon V\to V$ is linear, then 
  $T^{\#}=(\dim V)^{-1}\trace(T)\id$.
\end{theorem}

\begin{proof}
  The previous proposition and Schur's lemma imply that
  $T^{\#}=\lambda\id$ for some $\lambda\in\C$.
  We now compute the trace of $T^{\#}$. On the one hand, 
  \[
	\trace(T^{\#})=\trace(\lambda\id)=(\dim V)\lambda.
  \]
  On the other hand.  
  \[
	\trace(T^{\#})
	=\frac{1}{|G|}\sum_{g\in G}\trace(\rho_{g^{-1}}T\rho_g)
	=\frac{1}{|G|}\sum_{g\in G}\trace(T)
	=\trace(T),
  \]
  as $\trace(ABA^{-1})=\trace(B)$ for all $A$ and $B$. 
  Hence 
  \[
  \trace(T^{\#})=(\dim V)^{-1}\trace(T)\id.\qedhere 
  \]
\end{proof}

