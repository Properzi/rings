\chapter*{Some hints}


%\section*{Lecture 1}
\section*{Lecture 2}

\begin{sol}{xca:Z[sqrt10]/(2,sqrt10)}
	Use the ring homomorphism $\Z[\sqrt{10}]\to\Z/2$, $a+b\sqrt{10}\mapsto a\bmod 2$. 	
\end{sol}

\begin{sol}{xca:Z[i]/(1+3i)}
	Use the ring homomorphism $\Z\hookrightarrow\Z[i]\xrightarrow{\pi}\Z[i]/(1+3i)$, where
	$\pi$ is the canonical map. 	
\end{sol}

% \section*{Lecture 3}
% \section*{Lecture 4}
% \section*{Lecture 5}
\section*{Lecture 6}

\begin{sol}{xca:extend}
    If $X$ is a linearly independent set, a basis 
    of $V$ that contains $X$ will be found as a maximal linearly independent set 
    containing $X$. 
\end{sol}

\section*{Lecture 7}

\begin{sol}{xca:freshman_dream}
    The tricky part of the exercise is to prove that 
    if $p$ is a prime and $1\leq k\leq p^n-1$, then $\binom{p^n}{k}$ is divisible by $p$.
    We proceed as follows. Write
    \[
    \binom{p^n}{k}=\frac{(p^n)!}{k!(p^n-k)!}=\frac{p^n}{k}\binom{p^n-1}{k-1}.
    \]
    Then
    \[
    k\binom{p^n}{k}=p^n\binom{p^n-1}{k-1},
    \]
    that is $p^n$ divides $k\binom{p^n}{k}$. 
    
    If $\gcd(p,k)=1$, then it follows that $p$ divides $\binom{p^n}{k}$ by unique decomposition
    of every integer as a product of primes. We may assume then that $\gcd(p,k)\ne1$, 
    say $k=p^\alpha m$ for some integer $m$ not divisible by $p$. Then
    \[
    p^n\binom{p^n-1}{k-1}=k\binom{p^n}{k}=p^{\alpha}m\binom{p^n}{k}
    \]
    and hence 
    \[
    p^{n-\alpha}\binom{p^n-1}{k-1}=m\binom{p^n}{k}.
    \]
    Since $k<p^n$, it follows that
    $m-\alpha\geq 1$. Thus $p$ divides $m\binom{p^n}{k}$ and hence
    $p$ divides $\binom{p^n}{k}$ because $p$ and $m$ are coprime. 
\end{sol}

%\section*{Lecture 8}


% \section*{Lecture 9}
% \section*{Lecture 10}
% \section*{Lecture 11}
% \section*{Lecture 12}
% \section*{Lecture 13}
