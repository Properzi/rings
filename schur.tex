\chapter{Schur's orthogonality relations}

Let $\rho\colon G\to\GL(V)$ and $\psi\colon G\to\GL(W)$ be representations of a finite group
$G$. Since $V$ and $W$ are vector spaces, the set 
\[
\Hom(V,W)=\{T\colon V\to W:\text{$T$ is linear}\}
\]
is a vector space with 
\begin{align*}
&(\lambda T)(v)=\lambda T(v) && \text{for all $\lambda\in\C$ and all $v\in V$,}\\ 
&(T+T_1)(v)=T(v)+T_1(v) &&\text{for all $v\in V$.}
\end{align*}
We claim that the set $\Hom_G(V,W)$ of invariant maps
is a subspace of $\Hom(V,W)$. Indeed, the zero map is clearly invariant. If $T,T_1\in\Hom_G(V,W)$ 
and $\lambda\in\C$, then
\[
(T+\lambda T_1)(\rho_g v)
=T(\rho_gv)+\lambda T_1(\rho_gv)
=\psi_gT(v)+\lambda \psi_gT_1(v)
=\psi_g((T+\lambda T_1)(v))
\]
for all $v\in V$. 
 
\begin{proposition}
	Let $\rho\colon G\to\GL(V)$ and $\psi\colon G\to\GL(W)$ be representations
	and $T\colon V\to W$ be a linear map. Then
	\[
	T^{\#}=\frac{1}{|G|}\sum_{g\in G}\psi_{g^{-1}}T\rho_g\in\Hom_G(V,W).
	\]
	Moreover, the map $\Hom(V,W)\to\Hom_G(V,W)$, $T\mapsto T^{\#}$, is linear and surjective.  
\end{proposition}

\begin{proof}
  Let $h\in G$ and $v\in V$. Then 
  \begin{align*}
	T^{\#}\phi_h(v)
	&=\frac{1}{|G|}\sum_{g\in G}\psi_{g^{-1}}T\phi_g\phi_h(v)
	=\frac{1}{|G|}\sum_{g\in G}\psi_{g^{-1}}T\phi_{gh}(v)\\
	&=\frac{1}{|G|}\sum_{x\in G}\psi_{hx^{-1}}T\phi_x(v)
	=\frac{1}{|G|}\sum_{x\in G}\psi_h\psi_{x^{-1}}T\phi_x(v)
	=\psi_hT^{\#}(v).
      \end{align*}

	  If $T\in\Hom_{\C[G]}(V,V)$, then 
      \[
	T^{\#}(v)=\frac{1}{|G|}\sum_{g\in G}\psi_{g^{-1}}T\phi_g(v)
	=\frac{1}{|G|}\sum_{g\in G}\psi_{g^{-1}}\psi_gT(v)
	=T(v)
      \]
      for all $v\in V$.
\end{proof}

If $\phi\colon G\to\GL(V)$ and $\psi\colon
	G\to\GL(W)$ are non-equivalent irreducible representations and $T\colon
	V\to W$ is a linear map, then $T^{\#}=0$, as 
	$T^{\#}\in\Hom_{\C[G]}(V,W)=\{0\}$ by the previous proposition and Schur's lemma.

\begin{theorem}[ergodic theorem]
  Let $\rho\colon G\to\GL(V)$ be an irreducible representation. 
  If $T\colon V\to V$ is linear, then 
  $T^{\#}=(\deg\phi)^{-1}\trace(T)\id$.
\end{theorem}

\begin{proof}
  The previous proposition and Schur's lemma imply that
  $T^{\#}=\lambda\id$ for some $\lambda\in\C$.
  We now compute the trace of $T^{\#}$. On the one hand, 
  \[
	\trace(T^{\#})=\trace(\lambda\id)=(\dim V)\lambda.
  \]
  On the other hand.  
  \[
	\trace(T^{\#})
	=\frac{1}{|G|}\sum_{g\in G}\trace(\rho_{g^{-1}}T\rho_g)
	=\frac{1}{|G|}\sum_{g\in G}\trace(T)
	=\trace(T),
  \]
  as $\trace(ABA^{-1})=\trace(B)$ for all $A$ and $B$. 
  Hence 
  \[
  \trace(T^{\#})=(\dim V)^{-1}\trace(T)\id.\qedhere 
  \]
\end{proof}
