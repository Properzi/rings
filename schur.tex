\chapter{Schur's orthogonality relations}

Let $\rho\colon G\to\GL(V)$ and $\psi\colon G\to\GL(W)$ be representations of a finite group
$G$. Since $V$ and $W$ are vector spaces, the set 
\[
\Hom(V,W)=\{T\colon V\to W:\text{$T$ is linear}\}
\]
is a vector space with 
\begin{align*}
&(\lambda T)(v)=\lambda T(v) && \text{for all $\lambda\in\C$ and all $v\in V$,}\\ 
&(T+T_1)(v)=T(v)+T_1(v) &&\text{for all $v\in V$.}
\end{align*}
We claim that the set $\Hom_G(V,W)$ of invariant maps
is a subspace of $\Hom(V,W)$. Indeed, the zero map is clearly invariant. If $T,T_1\in\Hom_G(V,W)$ 
and $\lambda\in\C$, then
\[
(T+\lambda T_1)(\rho_g v)
=T(\rho_gv)+\lambda T_1(\rho_gv)
=\psi_gT(v)+\lambda \psi_gT_1(v)
=\psi_g((T+\lambda T_1)(v))
\]
for all $v\in V$. 
 
\begin{proposition}
	Let $\rho\colon G\to\GL(V)$ and $\psi\colon G\to\GL(W)$ be representations
	and $T\colon V\to W$ be a linear map. Then
	\[
	T^{\#}=\frac{1}{|G|}\sum_{g\in G}\psi_{g^{-1}}T\rho_g\in\Hom_G(V,W).
	\]
	Moreover, the map $\Hom(V,W)\to\Hom_G(V,W)$, $T\mapsto T^{\#}$, is linear and surjective.  
\end{proposition}

\begin{proof}
	
\end{proof}
