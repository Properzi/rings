\chapter{}

This chapter contains examples of character tables. Then we
study modules over rings, submodules, module homomorphisms
and finitely-generated modules. 

Let $G$ be a finite group and $\chi_1,\dots,\chi_r$ be the irreducible characters of $G$. Without loss of generality
we may assume that $\chi_1$ is the trivial character, i.e. $\chi_1(g)=1$ for all $g\in G$. 
Recall that $r$ is the number of conjugacy classes of $G$. Each $\chi_j$ is constant on conjugacy classes. 
The \textbf{character table} of 
$G$ is given by 
\begin{center}
\begin{tabular}{|c|cccc|}
\hline 
 & $1$ & $k_{2}$ & $\cdots$ & $k_{r}$\tabularnewline
 & $1$ & $g_{2}$ & $\cdots$ & $g_{r}$\tabularnewline
\hline 
$\chi_{1}$ & $1$ & $1$ & $\cdots$ & $1$\tabularnewline
$\chi_{2}$ & $n_{2}$ & $\chi_{2}(g_{2})$ & $\cdots$ & $\chi_{2}(g_{r})$\tabularnewline
$\vdots$ & $\vdots$ & $\vdots$ & $\ddots$ & $\vdots$\tabularnewline
$\chi_{r}$ & $n_{r}$ & $\chi_{r}(g_{2})$ & $\cdots$ & $\chi_{r}(g_{r})$\tabularnewline
\hline
\end{tabular}
\end{center}
where the $n_j$ are the degrees of the irreducible representations of $G$ and each $k_j$ is 
the size of the conjugacy class of the element $g_j$. By convention, the character table
contains not only the values of the irreducible characters of the group. 

\begin{example}
	Sea $G=\langle g:g^4=1\rangle$ 
	be the cyclic group of order four. The character table of $G$ is given by
	\begin{center}
		\begin{tabular}{|c|cccc|}
			\hline 
			& 1 & 1 & 1 & 1\tabularnewline
			& $1$ & $g$ & $g^2$ & $g^{3}$\tabularnewline
			\hline 
			$\chi_{1}$ & $1$ & $1$ & $1$ & $1$\tabularnewline
			$\chi_{2}$ & $1$ & $\lambda$ & $\lambda^2$ & $\lambda^{3}$\tabularnewline
			$\chi_{3}$ & $1$ & $\lambda^2$ & $\lambda^4$ & $\lambda^{2}$\tabularnewline
			$\chi_{4}$ & $1$ & $\lambda^{3}$ & $\lambda^{2}$ & $\lambda$\tabularnewline
			\hline
		\end{tabular}
	\end{center}
% Let us see how to see this calculation in the computer:
% \begin{lstlisting}
% gap> C4 := CyclicGroup(4);;                       
% gap> T := CharacterTable(C4);;
% gap> Display(T);
% CT1

%      2  2  2  2  2

%       1a 4a 2a 4b

% X.1     1  1  1  1
% X.2     1 -1  1 -1
% X.3     1  A -1 -A
% X.4     1 -A -1  A

% A = E(4)
%   = Sqrt(-1) = i
% \end{lstlisting}
% We need some remarks
% \begin{enumerate}
%     \item The symbol \lstinline{E(4)} denotes a primitive fourth root of 1.
%     \item The function \lstinline{CharacterTable} computes some more information, not only the character table of the group. 
%     esta función calcula algunas otras cosas. Por ejemplo:
% \end{enumerate}
% \begin{lstlisting}
% gap> OrdersClassRepresentatives(T);
% [ 1, 4, 2, 4 ]
% gap> SizesCentralizers(T);
% [ 4, 4, 4, 4 ]
% gap> SizesConjugacyClasses(T);
% [ 1, 1, 1, 1 ]
% \end{lstlisting}
\end{example}

\begin{exercise}
	Let $n\in\Z_{>0}$ be such that $n\geq2$. Let 
$C_n=\langle g:g^n=1\rangle$ be the cyclic group of order $n$.
\begin{enumerate}
    \item Prove that the maps  
        $\chi_i\colon C_n\to\C^\times$, $g^k\mapsto e^{2\pi ik/n}$, where $i\in\{0,1,\dots,n-1\}$, 
        are the irreducible representations of $C_n$. 
    \item Let $\lambda$ be a primitive root of 1 of order $n$. Prove that 
        the character table of $C_n$ of order $n$ is given by 
	\begin{center}
		\begin{tabular}{|c|ccccc|}
			\hline 
			& 1 & 1 & 1 & $\cdots$ & 1\tabularnewline
			& $1$ & $g$ & $g^2$ & $\cdots$ & $g^{n-1}$\tabularnewline
			\hline 
			$\chi_{1}$ & $1$ & $1$ & $1$ & $\cdots$ & $1$\tabularnewline
			$\chi_{2}$ & $1$ & $\lambda$ & $\lambda^2$ & $\cdots$ & $\lambda^{n-1}$\tabularnewline
			$\chi_{3}$ & $1$ & $\lambda^2$ & $\lambda^4$ & $\cdots$ & $\lambda^{n-2}$\tabularnewline
			$\vdots$ & $\vdots$ & $\vdots$ & $\vdots$ & $\ddots$ & $\vdots$\tabularnewline
			$\chi_{n}$ & $1$ & $\lambda^{n-1}$ & $\lambda^{n-2}$ & $\cdots$ & $\lambda$\tabularnewline
			\hline
		\end{tabular}
	\end{center}
\end{enumerate}
\end{exercise}

\begin{exercise}
    Let $A$ and $B$ be abelian groups. We write $\Irr(A)=\{\rho_1,\dots,\rho_r\}$ and 
    $\Irr(B)=\{\phi_1,\dots,\phi_s\}$. Prove
    that the maps 
    \[
    \varphi_{ij}\colon A\times B\to\C^\times,\quad
    (a,b)\mapsto\rho_i(a)\phi_j(b),
    \]
    where $i\in\{1,\dots,r\}$ and $j\in\{1,\dots,s\}$, are the irreducible representations of $A\times B$. 
\end{exercise}

Let us show a particular example of the previous exercise. 

\begin{example}
	The character table of the group $C_2\times C_2=\{1,a,b,ab\}$ is 
	\begin{center}
		\begin{tabular}{|c|rrrr|}
			\hline 
			& 1 & 1 & 1 & 1\tabularnewline
			& $1$ & $a$ & $b$ & $ab$\tabularnewline
			\hline 
			$\chi_{1}$ & $1$ & $1$ & $1$ & $1$\tabularnewline
			$\chi_{2}$ & $1$ & $1$ & $-1$ & $-1$\tabularnewline
			$\chi_{3}$ & $1$ & $-1$ & $1$ & $-1$\tabularnewline
			$\chi_{4}$ & $1$ & $-1$ & $-1$ & $1$\tabularnewline
			\hline
		\end{tabular}
	\end{center}
% 	Let us do this by computer:
% \begin{lstlisting}
% gap> Display(CharacterTable(AbelianGroup([2,2])));
% CT2

%      2  2  2  2  2

%       1a 2a 2b 2c

% X.1     1  1  1  1
% X.2     1 -1  1 -1
% X.3     1  1 -1 -1
% X.4     1 -1 -1  1
% \end{lstlisting}
\end{example}

Clearly, the order in which the computer returns the irreducible characters is not necessarily the same we used! 

\begin{example}
	The symmetric group $\Sym_3$ has three conjugacy classes. The representatives are 
	$\id$, $(12)$ and $(123)$. There are three irreducible representations. We already found all the irreducible characters! 
	The character table of $\Sym_3$ is given by 
	\begin{center}
		\begin{tabular}{|c|rrr|}
			\hline
			& $1$ & $3$ & $2$\tabularnewline
			& $1$ & $(12)$ & $(123)$ \tabularnewline
			\hline 
			$\chi_{1}$ & $1$ & $1$ & $1$\tabularnewline
			$\chi_{2}$ & $1$ & $-1$ & $1$ \tabularnewline
			$\chi_{3}$ & $2$ & $0$ & $-1$ \tabularnewline
			\hline
		\end{tabular}
	\end{center}
	Let us recall how this table was computed. Degree-one irreducibles were easy to compute. 
	To compute the third row of the table one possible approach is to use
	the irreducible representation  
	\[
	(12)\mapsto \begin{pmatrix}-1&1\\0&1\end{pmatrix},
	\quad
	(123)\mapsto \begin{pmatrix}0&-1\\1&-1\end{pmatrix}.
	\]
    Then	
    \begin{align*}
		&\chi_3\left( (12) \right)=\trace\begin{pmatrix}-1&1\\0&1\end{pmatrix}=0,\\
		&\chi_3\left( (123) \right)=\chi_3\left( (12)(23)\right)=\trace\begin{pmatrix}0&-1\\1&-1\end{pmatrix}=-1.
	\end{align*}

	We should remark that the irreducible representation mentioned is not really needed to
	compute the third row of the character table. We can, for example, use the regular
	representation $L$. The character of $L$ is given by 
	\[
		\chi_L(g)=\begin{cases}
			6 & \text{si $g=\id$},\\
			0 & \text{si $g\ne\id$}.
		\end{cases}
	\]
	The equality $0=\chi_L\left( (12) \right)=1-1+2\chi_3( (12))$ implies that 
	$\chi_3( (12))=0$ and the equality $0=\chi_L( (123))=1+1+2\chi_3( (123))$
	implies that $\chi_3\left( (123) \right)=-1$. 

	Another approach uses the orthogonality relations. We need to compute $\chi_3( (12) )$ and $\chi_3( (123))$. 
	Let $a=\chi_3( (12) )$ and $b=\chi_3( (123))$. Then 
    we get that $a=0$ and $b=-1$. We just need to solve  
	\begin{align*}
		0&=\langle \chi_3,\chi_1\rangle=\frac16(2+3a+2b),\\
		0&=\langle \chi_3,\chi_2\rangle=\frac16(2-3a+2b).
	\end{align*}
	
% 	Let us use the computer:
% 	\begin{lstlisting}
% gap> S3 := SymmetricGroup(3);;
% gap> T := CharacterTable(S3);;
% gap> Display(T);
% CT3

%      2  1  1  .
%      3  1  .  1

%       1a 2a 3a
%     2P 1a 1a 3a
%     3P 1a 2a 1a

% X.1     1 -1  1
% X.2     2  . -1
% X.3     1  1  1
% \end{lstlisting}
% As we did before, some extra information was computed:
% \begin{lstlisting}
% gap> SizesConjugacyClasses(T);
% [ 1, 3, 2 ]
% gap> SizesCentralizers(T);
% [ 6, 2, 3 ]
% gap> SizesConjugacyClasses(T);
% [ 1, 3, 2 ]
% gap> OrdersClassRepresentatives(T);
% [ 1, 2, 3 ]
% \end{lstlisting}
\end{example}

%A challenging exercise: 
\begin{exercise}
Compute the character table of $\Sym_4$. 
\end{exercise}

\begin{example}
	We now compute the character table of the alternating group $\Alt_4$. This group has $12$ 
	elements and four conjugacy classes.
	\begin{center}
		\begin{tabular}{c|cccc}
			representative & $\id$ & $(123)$ & $(132)$ & $(123)$\tabularnewline
			\hline
			size & $1$ & $4$ & $4$ & $3$ 
		\end{tabular}
	\end{center}
	Since $[\Alt_4,\Alt_4]=\{\id,(12)(34),(13)(24),(14)(23)\}$,
	$\Alt_4/[\Alt_4,\Alt_4]$ has three elements. Thus $\Alt_4$ has three degree-one irreducibles and
	an irreducible character of degree three. Let 
	$\omega=\exp(2\pi i/3)$ be a primitive cubic root of 1. If $\chi$
	is a non-trivial degree-one character, then 
	$\chi\left( (123) \right)=\omega^j$
	for some $j\in\{1,2\}$ and $\chi\left( (132) \right)=\omega^{2j}$. Since 
	$(132)(134)=(12)(34)$ and 
	the permutations $(134)$ and $(123)$ are conjugate,  
	\[
	\chi_i((12)(34))=\chi_i((132)(134))=\chi_i((132))\chi_i((134))=\omega^3=1
	\]
	for all $i\in\{1,2\}$. 
	
	To compute $\chi_4$ we use the regular representation. 
	\begin{align*}
		0&=\chi_L\left( (12)(34) \right)=1+1+1+3\chi_4\left( (12)(34) \right),\\
		0&=\chi_L\left( (123) \right)=1+\omega+\omega^2+3\chi_4\left( (123) \right),\\
		0&=\chi_L\left( (132) \right)=1+\omega+\omega^2+3\chi_4\left( (132) \right).
	\end{align*}
	Then we obtain that $\chi_4\left( (123) \right)=\chi_4\left( (132)
	\right)=0$ and $\chi_4\left( (12)(34) \right)=-1$. Therefore, the character table of $\Alt_4$
	is given by
	\begin{center}
		\begin{tabular}{|c|rrrr|}
			\hline
			& $\id$ & $(123)$ & $(132)$ & $(12)(34)$\tabularnewline
			\hline
			$\chi_1$ & $1$ & $1$ & $1$ & $1$\tabularnewline
			$\chi_2$ & $1$ & $\omega$ & $\omega^2$ & $1$\tabularnewline
			$\chi_3$ & $1$ & $\omega^2$ & $\omega$ & $1$\tabularnewline
			$\chi_4$ & $3$ & $0$ & $0$ & $-1$\tabularnewline
			\hline
		\end{tabular}
	\end{center}
% 	Let us use the computer:
% \begin{lstlisting}
% gap> A4 := AlternatingGroup(4);;
% gap> T := CharacterTable(A4);;
% gap> Display(T);
% CT5

%      2  2  2  .  .
%      3  1  .  1  1

%       1a 2a 3a 3b
%     2P 1a 1a 3b 3a
%     3P 1a 2a 1a 1a

% X.1     1  1  1  1
% X.2     1  1  A /A
% X.3     1  1 /A  A
% X.4     3 -1  .  .

% A = E(3)^2
%   = (-1-Sqrt(-3))/2 = -1-b3
% \end{lstlisting}
% The symbol \lstinline{E(3)} denotes a primitive cubic root of 1, say our $\omega$. 
% To save some space, the compute uses the symbol \lstinline{A} to denote the complex number $\omega^2$ (it is the same as \lstinline{E(3)^2}) and
% the symbol \lstinline{/A} to denote the complex number $\omega$, the multiplicative inverse of $\omega^2$. 
\end{example}

\begin{example}
    Let $Q_8=\{-1,1,i,-i,j,-j\}$ be the quaternion group. Let us compute the character table of $Q_8$.
    The group $Q_8$ is generated by $\{i,j\}$ and the map $\rho\colon Q_8\to\GL_2(\C)$, 
    \[
    i\mapsto\begin{pmatrix}
    i&0\\0&i
    \end{pmatrix},
    \quad
    j\mapsto\begin{pmatrix}
    0&1\\-1&0
    \end{pmatrix},
    \]
    is a representation.
    The conjugacy classes of $Q_8$ are $\{1\}$, $\{-1\}$, $\{-i,i\}$, $\{-j,j\}$ and $\{-k,k\}$. 
    So there are five irreducible representations. 
    We can compute the character of $\rho$:
    	\begin{center}
		\begin{tabular}{|c|c|c|c|c|c|}
		    \hline
			& $1$ & $-1$ & $i$ & $j$ & $k$\tabularnewline
			\hline
			$\chi_\rho$ & 2 & 2 & 0 & 0 & 0\tabularnewline
			\hline
		\end{tabular}
	\end{center}
	Then $\rho$ is irreducible, es $\langle\chi_\rho,\chi_\rho\rangle=1$. 
	
	Since $[Q_8,Q_8]=\{-1,1\}=Z(Q_8)$, the quotient group $Q_8/[Q_8,Q_8]$ has four elements and
	hence there are four irreducible degree-one representations. Since 
	$Q_8$ is non-abelian, $Q_8/Z(Q_8)$ cannot be cyclic. 
	This implies that 
	$Q_8/[Q_8,Q_8]\simeq C_2\times C_2$. This allows us
	to compute almost all the character table of $Q_8$. 
		\begin{center}
		\begin{tabular}{|c|rrrrr|}
			\hline
			& $1$ & $-1$ & $i$ & $j$ & $k$\tabularnewline
			\hline
			$\chi_1$ & $1$ & $1$ & $1$ & $1$ & $1$\tabularnewline
			$\chi_2$ & $1$ & \cellcolor{gray!30}{$1$} & $-1$ & $1$ & $-1$\tabularnewline
			$\chi_3$ & $1$ & \cellcolor{gray!30}{$1$} & $1$ & $-1$ & $-1$\tabularnewline
			$\chi_4$ & $1$ & \cellcolor{gray!30}{$1$} & $-1$ & $-1$ & $1$\tabularnewline
			$\chi_5$ & $2$ & $-2$ & $0$ & $0$ & $0$\tabularnewline
			\hline
		\end{tabular}
	\end{center}
	It remains to compute $\chi_j(-1)$ for $j\in\{2,3,4\}$, these missing values are presented in shaded
    cells. To compute these values that $\langle\chi_i,\chi_j\rangle=0$ whenever $i\ne j$. The calculations
    are left as an exercise. 
% 	To check our character table we can use the computer. 
% \begin{lstlisting}
% gap> Q8 := QuaternionGroup(8);;
% gap> Display(CharacterTable(Q8));
% CT6

%      2  3  2  2  3  2

%       1a 4a 4b 2a 4c
%     2P 1a 2a 2a 1a 2a
%     3P 1a 4a 4b 2a 4c

% X.1     1  1  1  1  1
% X.2     1 -1 -1  1  1
% X.3     1 -1  1  1 -1
% X.4     1  1 -1  1 -1
% X.5     2  .  . -2  .
% \end{lstlisting}
\end{example}

\begin{exercise}
    Compute the character table of the dihedral group of eight elements. 
\end{exercise}

\section*{Modules}

The rest of the course will be devoted to study modules over rings. 
We first start with the main definitions and basic examples.

\begin{definition}
    Let $R$ be a ring. A \textbf{module} (over $R$) is an abelian group
    $M$ with a map $R\times M\to M$, $(x,m)\mapsto x\cdot m$, such that
    the following conditions hold:
    \begin{enumerate}
        \item $(r_1+r_2)\cdot m=r_1\cdot m+r_2\cdot m$ for all $r_1,r_2\in R$ y $m\in M$.
		\item $r\cdot (m_1+m_2)=r\cdot m_1+r\cdot m_2$ for all $r\in R$ y $m_1,m_2\in M$.
		\item $r_1\cdot (r_2\cdot m)=(r_1r_2)\cdot m$ for all $r_1,r_2\in R$ y $m\in M$.
		\item $1\cdot m=m$ for all $m\in M$.	
    \end{enumerate}
\end{definition}

Our definition is that of left module. Similarly one defines right modules. We will always
consider left modules, so they will be referred simply as modules.

\begin{example}
A module over a field is a vector space. 
\end{example}

\begin{example}
Every abelian group is a module over $\Z$.	
\end{example}

\begin{example}
Let $R$ be a ring. Then $R$ is a module (over $R$) with $x\cdot m=xm$. 
This is the \textbf{(left) regular representation} of $R$ and it usually 
be denoted by $\prescript{}{R}R$. 
\end{example}

\begin{example}
If $R$ is a ring, then $R^n=\{(x_1,\dots,x_n):x_1,\dots,x_n\in R\}$ 
is a module (over $R$) with  
$r\cdot (x_1,\dots,x_n)=(rx_1,\dots,rx_n)$. 
\end{example}

\begin{example}
If $R$ is a ring, then $M_{m,n}(R)$ is a module (over $R$) with usual matrix operations. 
\end{example}

Students usually ask why in the definition of a ring homomorphism one needs
the condition $1\mapsto 1$. The following example provides a good explanation. 

\begin{example}
%%\label{exa:f(1)=1}
If $f\colon R\to S$ is a ring homomorphism and $M$ is a module (over $S$) wih 
$(s,m)\mapsto sm$, then 
$M$ is also a module (over $R$) with $r\cdot m=f(r)m$ for all $r\in R$ and $m\in M$. In fact, 
\begin{align*}
&1\cdot m=f(1)m=1m=m,\\
&r_1\cdot (r_2\cdot m)=f(r_1)(r_2\cdot m)=f(r_1)(f(r_2)m)=(f(r_1)f(r_2))m=f(r_1r_2)m
\end{align*}
for all $r_1,r_2\in R$ and $m\in M$.	  	
\end{example}
%
\begin{example}
Let $R=\R[X]$ and $T\colon\R^n\to\R^n$ be a linear map. Then $M=\R^n$ with 
\[
\left(\sum_{i=0}^na_iX^i\right)\cdot v=\sum_{i=0}^na_iT^i(v)
\]	
is a module (over $R$).   
\end{example}

\begin{example}
If $\{M_i|i\in I\}$ is a family of modules, then  	
\[
\prod_{i\in I}M_i=\{(m_i)_{i\in I}:m_i\in M_i\text{ for all $i\in I$}\}
\]
is a module with 
$x\cdot (m_i)_{i\in I}=(x\cdot m_i)_{i\in I}$, 
where $(m_i)_{i\in I}$ denotes the map $I\to M_i$, $i\mapsto m_i$.
This module is the \textbf{direct product} of the family $\{M_i:i\in I\}$.
\end{example}
%
\begin{example}
If $\{M_i|i\in I\}$ is family of modules, then   	
\[
\bigoplus_{i\in I}M_i=\{(m_i)_{i\in I}:m_i\in M_i\text{ for all $i\in I$ and $m_i=0$ except finitely many $i\in I$}\}
\]
is a module with 
$x\cdot (m_i)_{i\in I}=(x\cdot m_i)_{i\in I}$. 
This module is the \textbf{direct sum} of the family $\{M_i:i\in I\}$. 
\end{example}
%
%\begin{exercise}
If $M$ is a module, then $0\cdot m=0$ and $-m=(-1)\cdot m$ for all $m\in M$ and 
$x\cdot 0=0$ for all $x\in R$. 
%
\begin{example}
Let $M=\Z/6$ as a module (over $\Z$). Note that 
$3\cdot 2=0$ but $3\ne 0$ (in $\Z$) and $2\ne 0$ (in $\Z/6$).  
\end{example}
%
\begin{definition}
	Let $M$ be a module. A subset $N$ of $M$ is a \textbf{submodule} of $M$ if 
	$(N,+)$ is a subgroup of $(M,+)$ and 
	$x\cdot n\in N$ for all $x\in R$ and $n\in N$. 
\end{definition}

Clearly, if $M$ is a module, then $\{0\}$ and $M$ are submodules of $M$. 

\begin{example}
Let $R$ be a field and $M$ be a module over $R$. Then
$N$ is a submodule of $M$ if and only if $N$ is a subspace of $M$. 
\end{example}

\begin{example}
Let $R=\Z$ and $M$ be a module (over $R$). Then
$N$ is a submodule of $M$ if and only if $N$ is a subgroup of $M$
\end{example}

\begin{example}
If $M=\prescript{}{R}R$, then a subset $N\subseteq M$ is a submodule
of $M$ if and only if $N$ is a left ideal of $R$. 
\end{example}

\begin{example}
If $V$ is a vector space and $T\colon V\to V$ is a linear map, then
$V$ is a module (over $\R[X]$) with  
\[
\left(\sum_{i=0}^na_iX^i\right)\cdot v=\sum_{i=0}^na_iT^i(v).
\]
A submodule is a subspace $W$ 
of $V$ such that $T(W)\subseteq W$. 
\end{example}

Clearly, a subset $N$ of $M$ is a submodule if and only 
if $r_1n_1+r_2n_2\in N$ for all
$r_1,r_2\in R$ and $n_1,n_2\in N$. 	

\begin{exercise}
If $N$ and $N_1$ are submodules of $M$, then 
\[
N+N_1=\{n+n_1:n\in N,\,n_1\in N_1\}
\]
is a submodule of $M$.
\end{exercise}

\begin{definition}
Let $M$ and $N$ be modules over $R$. 
A map $f\colon M\to N$ is a \textbf{module homomorphism} if $f(x+y)=f(x)+f(y)$ and 
$f(r\cdot x)=r\cdot f(x)$ for all $x,y\in M$ and $r\in R$. 
\end{definition}

We denote by $\Hom_R(M,N)$ the set of module homomorphisms $M\to N$. 

\begin{exercise}
Let $f\in\Hom_R(M,N)$.  
\begin{enumerate}
\item If $V$ is a submodule of $M$, then $f(V)$ is a submodule of $N$.
\item If $W$ is a submodule of $N$, then $f^{-1}(W)$ is a submodule of $M$.
\end{enumerate}
\end{exercise}

If $f\in\Hom_R(M,N)$, the \textbf{kernel} of $f$ is the submodule  
\[
\ker f=f^{-1}(\{0\})=\{m\in M:f(m)=0\}
\]
of $M$. We say that $f$ is a \textbf{monomorphism} (resp. \textbf{epimorphism}) 
if $f$ is injective (resp. surjective). Moreover, $f$ is an \textbf{isomorphism} 
if $f$ is
bijective. 

\begin{exercise}
Let $f\in\Hom_R(M,N)$. Prove that the following statements are equivalent:
\begin{enumerate}
\item $f$ is a monomorphism.
\item $\ker f=\{0\}$.
\item For every module $V$ and every $g,h\in\Hom_R(V,M)$, $f\circ g=f\circ h\implies g=h$.
\item For every module $V$ and every $g\in\Hom(V,M)$, $f\circ g=0\implies g=0$.
\end{enumerate}
\end{exercise}

Later we will see a similar exercise for surjective module homomorphisms.

\begin{example}
	Let $R=
		\begin{pmatrix}
			\R & 0\\
			0 & \R
		\end{pmatrix}$. 
	We claim that 
	$\begin{pmatrix}
			\R\\
			0
		\end{pmatrix}
		\not\simeq\begin{pmatrix}
			0\\
			\R
		\end{pmatrix}$
	as modules over $R$, where the module structure is given by the usual matrix multiplication. 
	Assume that they are isomorphic. 
	Let $f\colon\begin{pmatrix}
			0\\
			\R
		\end{pmatrix}
		\to\begin{pmatrix}
			\R\\
			0
		\end{pmatrix}$  
	be an isomorphism of modules and let  
	$x_0\in\R\setminus\{0\}$ be such that 
	$f\begin{pmatrix}0\\1\end{pmatrix}=\begin{pmatrix}x_0\\0\end{pmatrix}$. Thus 
	\[
	f\begin{pmatrix}
	0\\
	1\end{pmatrix}
	=f\left(\begin{pmatrix}
	0&0\\
	0&1\end{pmatrix}
	\cdot 
	\begin{pmatrix}
	0\\
	1
	\end{pmatrix}\right)
	=\begin{pmatrix}
	0&0\\
	0&1\end{pmatrix}\cdot f\begin{pmatrix}0\\1\end{pmatrix}
	=\begin{pmatrix}
	0&0\\
	0&1
	\end{pmatrix}
	\cdot 
	\begin{pmatrix}		
	x_0\\
	0
	\end{pmatrix}
	=\begin{pmatrix}
	0\\
	0
	\end{pmatrix},
	\]	
	a contradiction, as $f$ is injective.    
\end{example}

If $N$ and $N_1$ are submodules of $M$, we say that $M$ is the \textbf{direct sum} of $N$ and $N_1$
if $M=N+N_1$ and $N\cap N_1=\{0\}$. In this case, we write $M=N\oplus N_1$. Note that if
$M=N\oplus N_1$, then each $m\in M$ can be written uniquely as $m=n+n_1$ for some
 $n\in N$ and $n_1\in N_1$. 
Such a decomposition exists because $M=S+T$. If $m\in M$ can be written as 
$m=n+n_1=n'+n_1'$ for some $n,n'\in N$ and $n_1,n_1'\in N_1$, then 
$-n'+n=n_1'-n_1\in N\cap N_1=\{0\}$ and hence $n=n'$ and $n_1=n_1'$. If $M=N\oplus N_1$, the submodule
$N$ (resp. $N_1$) is a \textbf{direct summand} of $M$ and the submodule $N_1$ (resp $N$) is a \textbf{complement} of $N$ 
in $M$.   	

\begin{example}
If $M=\R^2$ as a vector space, then every subspace of $M$ is a direct summand of $M$.
\end{example}

Clearly, the submodules $\{0\}$ and $M$ are direct summands of $M$.

\begin{example}
If $M=\Z$ as a module over $\Z$, then $m\Z$ is a direct sum of $M$ if and only if 
$m\in\{0,1\}$, as $n\Z\cap m\Z=\{0\}$ if and only if $nm=0$.
\end{example}

\begin{exercise}
\label{xca:projector}
Let $M$ be a module. 
A module $N$ is isomorphic to a direct summand of $M$ if and only if
there are module homomorphisms $i\colon N\to M$ and $p\colon M\to N$ 
such that $p\circ i=\id_N$. In this case, $M=\ker p\oplus i(N)$.  
\end{exercise}

The \textbf{direct sum} of submodules can be defined for finitely many summands. 
If $V_1,\dots,V_n$ are submodules of $M$, we say that $M=V_1\oplus\cdots\oplus V_n$ 
if every $m\in M$ can be written uniquely as $m=v_1+\cdots+v_n$ for some $v_1\in V_1,\dots,v_n\in V_n$. 

\begin{exercise}
Prove that $M=V_1\oplus\cdots\oplus V_n$ if and only if 
$M=V_1+\cdots+V_n$ and 
\[
V_i\cap\left(\sum_{j\ne i}V_j\right)=\{0\}
\]	
for all $i\in\{1,\dots,n\}$.
\end{exercise}

If $\{N_i:i\in I\}$ is a family of submodules of a module $M$, then the intersection 
$\cap_{i\in I}N_i$ is also a submodule of $M$.

\begin{exercise}
\label{xca:submodules}
Let $T\colon\R^2\to\R^2$ be a linear map and $M=\R^2$ with the module structure 
over $\R[X]$ given by
\[
\left(\sum_{i=0}^n a_iX^i\right)\cdot (x,y)=\sum_{i=0}^n a_iT^i(x,y).
\]
Find the submodules of $M$ in the following cases. 
\begin{enumerate}
    \item $T(x,y)=(0,y)$.
    \item $T(x,y)=(y,x)$.
\end{enumerate}
% Prove that 
% $\{0\}$, $M$, $\R\times\{0\}$ and $\{0\}\times\R$ 
% are the only submodules of $M$.
% 
% Vamos a demostrar que
% $\{0\}$, $M$, $\R\times\{0\}$ y $\{0\}\times\R$ son los únicos submódulos de $M$. Si $N$ es un submódulo no nulo de $M$, sea
% $(x_0,y_0)\in N\setminus\{(0,0)\}$. Si $(x,y)\in M$ es tal que $xy\ne 0$, entonces
% \[
% \left(\frac{x}{x_0}+\left(\frac{y}{y_0}-\frac{x}{x_0}\right)X\right)\cdot (x_0,y_0)=(x,y) 
% \]
% y luego $N=M$. Si $y_0=0$, entonces $N=\R\times\{0\}$, pues $\frac{x}{x_0}\cdot (x_0,0)=(x,0)$. Si $x_0=0$, entonces
% $N=\{0\}\times\R$, pues 
% $\frac{y}{y_0}\cdot (0,y_0)=(0,y)$ 
\end{exercise}


% \begin{example}
% Sea $M=\R^2$ como $\R[X]$-módulo con la acción
% \[
% \left(\sum_{i=0}^n a_iX^i\right)\cdot (x,y)=\sum_{i=0}^n a_iT^i(x,y),
% \]
% donde $T\colon M\to M$, $T(x,y)=(y,x)$. 
% Vamos a calcular todos los submódulos de $M$. 
% Si $N\subseteq M$ es un submódulo entonces $N$ es un espacio vectorial real. Supongamos que $N\ne\{(0,0)\}$ y que $N\ne\R^2$. Como entonces $\dim N=1$, 
% sea $\{(a_0,b_0)\}$ una base de $N$. Como $N$ es un submódulo,
% $(b_0,a_0)=X\cdot (a_0,b_0)\in N$. En particular, existe $\lambda\in\R$ tal que $(b_0,a_0)=\lambda (a_0,b_0)$. Como $(a_0,b_0)\ne(0,0)$, sin perder generalidad
% podemos suponer que $a_0\ne 0$. Esto implica que
% $\lambda^2 a_0=\lambda (\lambda a_0)=\lambda b_0=a_0$ y entonces $\lambda^2=1$. Si $\lambda=1$, entonces
% $a_0=b_0$. Si $\lambda=-1$, entonces $a_0=-b_0$. En conclusión, $N$ está generado por $(1,1)$ o por $(1,-1)$.  
% \end{example}

\begin{example}
\label{xca:commuting}
Let $V$ be a real vector space and $T\colon V\to V$ be a linear map. Then 
$V$ is a module over $\R[X]$ with  
\[
\left(\sum_{i=0}^n a_iX^i\right)\cdot v=\sum_{i=0}^n a_iT^i(v).
\]
Prove that a module homomorphism $g\colon V\to V$ commutes with $T$. 
% \[
% (g\circ T)(v)=g(T(v))=g(X\cdot v)=X\cdot g(v)=T(g(v))=(T\circ g)(v)
% \]
% para todo $v\in V$.x
\end{example}

\begin{exercise}
\label{xca:Hom}
Prove that $\Hom_R(M,N)$ is a module over $Z(R)$. 
\end{exercise}

Let $M$ be a module and $N$ be a submodule of $M$. In particular, $M/N$ is an abelian group
and the map $\pi\colon M\to M/N$, $m\mapsto m+N$, is a surjective group homomorphism with kernel equal to $N$. We claim
that the \textbf{quotient} $M/N$ is a module with  
\[
r\cdot (m+N)=(r\cdot m)+N,\quad
r\in R,\,m\in M. 
\]
Let us check that this operation on $M/N$ is well-defined. If $x+N=y+N$, then
$x-y\in N$ implies that  
\[
r\cdot x-r\cdot y=r\cdot (x-y)\in N,
\]
that is $r\cdot (x+N)=r\cdot (y+N)$. It is an exercise to show that
the map $\pi\colon M\to M/N$, $x\mapsto x+N$, is a 
surjective module homomorphism. 

\begin{example}
If $R=M=\Z$ and $N=2\Z$, then $M/N\simeq\Z/2$. 
\end{example}

\begin{example}
Let $R$ be a commutative ring. We claim that 
\[
M\simeq\Hom_R(\prescript{}{R}R,M).
\]
Since $R$ is commutative, it follows that $\Hom_R(\prescript{}{R}R,M)$ is a module, see Exercise~\ref{xca:Hom}.
Let $\varphi\colon M\to\Hom_R(\prescript{}{R}R,M)$, $m\mapsto f_m$, where $f_m\colon R\to M$, $r\mapsto r\cdot m$. 
To show that $\varphi$ is well-defined it is enough to see that $\varphi(m)\in\Hom_R(\prescript{}{R}R,M)$, that is 
\[
f_m(r+s)=(r+s)\cdot m=r\cdot m+s\cdot m,\quad
f_m(rs)=(rs)\cdot m=r\cdot (s\cdot m)=r\cdot f_m(s).
\]
Let us show that $\varphi$ is a module homomorphism. We first note that 
\[
\varphi(m+n)=\varphi(m)+\varphi(n)
\]
for all $m,n\in M$, as  
\begin{align*}
\varphi(m+n)(r)&=f_{m+n}(r)=r\cdot (m+n)\\
&=r\cdot m+r\cdot n=f_m(r)+f_n(r)=\varphi(m)(r)+\varphi(n)(r).
\end{align*}
Moreover, 
\[
\varphi(r\cdot m)=r\cdot\varphi(m)
\]
for all  
$r\in R$ and $m\in M$, as  
\begin{align*}
\varphi(r\cdot m)(s)&=f_{r\cdot m}(s)
=s\cdot (r\cdot m)
=(sr)\cdot m\\
&=(rs)\cdot m=f_m(rs)=\varphi(m)(rs)=(r\cdot\varphi(m))(s).
\end{align*}
It remains to show that $\varphi$ is bijective. We first prove that
$\varphi$ is injective. If $\varphi(m)=0$, then $r\cdot m=\varphi(m)(r)=0$ for all $r\in R$. In particular, 
$m=1\cdot m=0$. We now prove that $\varphi$ is surjective. If $f\in\Hom_R(\prescript{}{R}R,M)$, 
let $m=f(1)$. Then $\varphi(m)=f$, as
\[
\varphi(m)(r)=r\cdot m=r\cdot f(1)=f(r).
\]
\end{example}

As one does for groups, it is possible to show that if $M$ is a 
module and $N$ is a submodule of $M$, the pair 
$(M/N,\pi\colon M\to M/N)$ has the following properties:
\begin{enumerate}
\item $N\subseteq \ker \pi$.
\item If $f\colon M\to T$ is a homomorphism such that $N\subseteq \ker f$, then there exists a 
unique module homomorphism $\varphi\colon M/N\to T$ such that 
the diagram
\[
\begin{tikzcd}
	M & T \\
	{M/N}
	\arrow["\pi"', from=1-1, to=2-1]
	\arrow["f", from=1-1, to=1-2]
	\arrow["{\varphi }"', dashed, from=2-1, to=1-2]
\end{tikzcd}
\]
is commutative, that is 
$\varphi\circ\pi =f$.  
\end{enumerate}

Recall that if $S$ and $T$ are submodules of a module $M$, then 
both $S\cap T$ and 
\[
S+T=\{s+t:s\in S,\,t\in T\}
\]
are submodules of $M$. 
The \textbf{isomorphism theorems} hold:
\begin{enumerate}
	\item If $f\in\Hom_R(M,N)$, then $M/\ker f\simeq f(M)$.
	\item If $T\subseteq N\subseteq M$ are submodules, then  
	\[
	\frac{M/T}{N/T}\simeq M/N
	\]
	\item If $S$ and $T$ are submodules of $M$, then $(S+T)/S\simeq T/(S\cap T)$. 
\end{enumerate}

\begin{example}
If $K$ is a field and $V$ is a module over $K$, then $V$ is, by definition, a vector space over $K$. 
If $S$ and $T$ are subspaces of $V$, then they are submodules of $V$. 
By the second isomorphism theorem, $(S+T)/T\simeq S/(S\cap T)$ 
as vector spaces. By applying dimension,  
\[
\dim(S+T)-\dim T=\dim(S)-\dim(S\cap T).
\]
\end{example}

\begin{example}
If $N$ is a direct summand of $M$ and $M$ and $X$ is a complement for $N$, then $X\simeq M/N$, as
\[
M/N=(N\oplus X)/N\simeq X/(N\cap X)=X/\{0\}\simeq X
\]
by the second isomorphism theorem. So all complements of $N$ in $M$ are isomorphic. 
\end{example}

It is also possible to prove that there exists a bijective correspondence between
submodules of $M/N$ and submodules of $M$ containing $N$. The correspondence is given by 
$\pi^{-1}(Y)\mapsfrom Y$ and $X\mapsto \pi(X)$. 

\begin{exercise}
Let $f\in\Hom_R(M,N)$. Prove that the following statements are equivalent: 
\begin{enumerate}
\item $f$ is an epimorphism.
\item $N/f(M)\simeq\{0\}$. 
\item For every module $W$ and every $g,h\in\Hom_R(N,T)$, $g\circ f=h\circ f\implies g=h$.
\item For every module $T$ and every $g\in\Hom_R(N,T)$, $g\circ f=0\implies g=0$. 
\end{enumerate}
\end{exercise}

\begin{exercise}
\label{xca:mod_iso_max}
    Let $R$ be a ring and $M_1$ and $M_2$ be maximal ideals of $R$. Prove that
    $R/M_1\simeq R/M_2$ as modules over $R$ if and only if there exists
    $r\in R\setminus M_2$ such that $rM_1\subseteq M_2$. 
\end{exercise}





