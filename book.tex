\RequirePackage{amsmath} 

\documentclass[graybox,envcountsect]{svmono}

\usepackage[T1]{fontenc}
\usepackage[utf8]{inputenc}

\usepackage{amsmath}
\usepackage[notref,notcite]{showkeys}
\usepackage{float}
\usepackage{amssymb}
\usepackage{amstext}
\usepackage{mathtools}
\usepackage{xcolor} 
\usepackage{centernot}
\usepackage{listings}
\usepackage{multicol}
\usepackage{mathptmx}
%\usepackage{newtxtext,newtxmath}
%\usepackage{txfonts}
\usepackage{datetime}
\usepackage{tikz-cd}

\usepackage{helvet}
\usepackage{courier}
\usepackage{type1cm}         
\usepackage{makeidx}        
\usepackage{graphicx}        
\usepackage{multicol}        
\usepackage[all]{xy}
\usepackage{hyperref} 
%\usepackage{tikz-cd}

\usepackage[small,bf]{caption}

\usepackage{tikz}
\usetikzlibrary{braids}
	
\usepackage[bottom]{footmisc}

% for QED
\let\proof\relax\let\endproof\relax
\usepackage{amsthm}

\overfullrule=1mm

%%% for Spanish
% \def\abstractname{Resumen}%
% \def\ackname{Agradecimientos}%
% \def\andname{y}%
% \def\bibname{Referencias}%
% \def\lastandname{, y}%
% \def\appendixname{Apéndice}%
% \def\chaptername{Capítulo}%
% \def\claimname{Afirmación}%
% \def\conjecturename{Conjetura}%
% \def\contentsname{Contenidos}%
% \def\corollaryname{Corolario}%
% \def\definitionname{Definici\'on}%
% \def\emailname{e-mail}%
% \def\examplename{Ejemplo}%
% \def\examplesname{Ejemplos}%
% \def\exercisename{Ejercicio}%
\def\figurename{Figure}%
% \def\forewordname{Foreword}%
% \def\keywordname{{\bf Palabras clave:}}%
% \def\indexname{Índice}%
% \def\lemmaname{Lema}%
% \def\listfigurename{Figuras}%
% \def\listtablename{Tablas}%
% \def\notename{Nota}%
% \def\partname{Parte}%
% \def\prefacename{Prefacio}%
\def\problemname{Open problem}%
% \def\proofname{Demostración}%
% \def\propertyname{Propiedad}%
% \def\propositionname{Proposici\'on}%
% \def\questionname{Pregunta}%
% \def\refname{Referencias}%
% \def\remarkname{Observación}%
% \def\seename{see}%
% \def\solutionname{Solución}%
% \def\tablename{Tabla}%
% \def\theoremname{Teorema}
\def\notationname{Notation}
\def\stepsname{Algorithm}
% \def\conventionname{Convención}

% change numbers 
\let\remark\relax
\let\theorem\relax
\let\lemma\relax
\let\definition\relax
\let\proposition\relax
\let\corollary\relax
\let\exercise\relax
\let\example\relax
\let\conjecture\relax
\spnewtheorem{theorem}{\theoremname}[section]{\bfseries}{\itshape}
\renewcommand\thetheorem{\thesection.\arabic{theorem}}
\spnewtheorem{lemma}[theorem]{\lemmaname}{\bfseries}{\itshape}
\spnewtheorem{definition}[theorem]{\definitionname}{\bfseries}{\upshape}
\spnewtheorem{proposition}[theorem]{\propositionname}{\bfseries}{\itshape}
\spnewtheorem{corollary}[theorem]{\corollaryname}{\bfseries}{\itshape}
\spnewtheorem{exercise}[theorem]{\exercisename}{\bfseries}{\upshape}
\spnewtheorem{example}[theorem]{\examplename}{\bfseries}{\upshape}
\spnewtheorem{examples}[theorem]{\examplesname}{\bfseries}{\upshape}
\spnewtheorem{remark}[theorem]{\remarkname}{}{\upshape}
\spnewtheorem{conjecture}[theorem]{\conjecturename}{\bfseries}{\upshape}
\spnewtheorem{notation}[theorem]{\notationname}{\bfseries}{\upshape}
\spnewtheorem{steps}[theorem]{\stepsname}{\bfseries}{\upshape}
\spnewtheorem{convention}[theorem]{\conventionname}{\bfseries}{\upshape}

% para enumerar
\renewcommand{\labelenumi}{\textbf{\arabic{enumi})}}

\setcounter{secnumdepth}{1}

\makeindex             

\renewcommand{\I}{\operatorname{I}}
\newcommand{\II}{\operatorname{II}}

\newcommand{\GAP}{\textsf{GAP}}
\newcommand{\FK}{\mathcal{E}}
\newcommand{\ad}[1]{\operatorname{ad}\,#1}

\newcommand{\N}{\mathbb{N}}
\newcommand{\Q}{\mathbb{Q}}
\newcommand{\Z}{\mathbb{Z}}
\newcommand{\F}{\mathbb{F}}
\newcommand{\R}{\mathbb{R}}
\newcommand{\C}{\mathbb{C}}
\renewcommand{\H}{\mathbb{H}}
\newcommand{\A}{\mathbb{A}}
\newcommand{\K}{\mathbb{K}}
\newcommand{\T}{\mathbb{T}}
\renewcommand{\D}{\mathbb{D}}
\newcommand{\B}{\mathbb{B}}
\newcommand{\Fun}{\operatorname{Fun}}
\newcommand{\mpl}{\operatorname{mpl}}
\newcommand{\cL}{\mathcal{L}}
\newcommand{\cE}{\mathcal{E}}
\newcommand{\cH}{\mathcal{H}}

\newcommand{\GF}{\mathsf{GF}}
\newcommand{\MAX}{\operatorname{MAX}}
\newcommand{\MIN}{\operatorname{MIN}}
\newcommand{\cf}{\operatorname{cf}}
\newcommand{\cl}{\operatorname{cl}}
\newcommand{\cd}{\operatorname{cd}}
\newcommand{\bL}{\mathbf{L}}
\newcommand{\bP}{\mathbf{P}}

\newcommand{\Nil}{\operatorname{Nil}}
\newcommand{\rad}{\operatorname{rad}}
\newcommand{\rank}{\operatorname{rank}}

\newcommand{\Aff}{\mathrm{Aff}}
\newcommand{\Ann}{\operatorname{Ann}}
\newcommand{\Der}{\operatorname{Der}}
\newcommand{\Core}{\operatorname{Core}}
\newcommand{\Soc}{\operatorname{Soc}}
\newcommand{\Fix}{\operatorname{Fix}}
\newcommand{\Rad}{\mathrm{rad}}
\newcommand{\Inn}{\mathrm{Inn}}
\newcommand{\dist}{\mathrm{dist}}
\newcommand{\Out}{\mathrm{Out}}
\newcommand{\Ext}{\mathrm{Ext}}
\newcommand{\Img}{\mathrm{im}}
\newcommand{\Hol}{\operatorname{Hol}}
\newcommand{\Hom}{\operatorname{Hom}}
\newcommand{\Alg}{\operatorname{Alg}}
\newcommand{\Bil}{\operatorname{Bil}}
\newcommand{\op}{\operatorname{op}}
\newcommand{\gr}{\operatorname{gr}}
\newcommand{\Syl}{\mathrm{Syl}}
\newcommand{\id}{\operatorname{id}}
\newcommand{\Aut}{\operatorname{Aut}}
\newcommand{\End}{\operatorname{End}}
\newcommand{\Irr}{\operatorname{Irr}}
\newcommand{\Alt}{\mathbb{A}}
\newcommand{\Sym}{\mathbb{S}}
\newcommand{\lcm}{\mathrm{mcm}}
\newcommand{\diag}{\operatorname{diag}}
\newcommand{\spec}{\operatorname{Spec}}
\newcommand{\supp}{\operatorname{supp}}
\newcommand{\trace}{\operatorname{traza}}
\newcommand{\sgn}{\operatorname{signo}}

\newcommand{\inner}{\operatorname{inn}}
\newcommand{\ext}{\operatorname{ext}}
\newcommand{\im}{\operatorname{im}}
\newcommand{\Ret}{\operatorname{Ret}}

\newcommand{\GL}{\mathbf{GL}}
\newcommand{\SL}{\mathbf{SL}}
\newcommand{\PSL}{\mathbf{PSL}}
\newcommand{\PGL}{\mathbf{PGL}}

\newcommand{\legendre}[2]{\left(\frac{#1}{#2}\right)}

%\newcommand{\char}{\operatorname{char}}

% multiset
\def\multiset#1#2{\ensuremath{\left(\kern-.3em\left(\genfrac{}{}{0pt}{}{#1}{#2}\right)\kern-.3em\right)}}

% column vector
\newcount\colveccount
\newcommand*\colvec[1]{
\global\colveccount#1
\begin{pmatrix}
	\colvecnext
	}
	\def\colvecnext#1{
	#1
	\global\advance\colveccount-1
	\ifnum\colveccount>0
	\\
	\expandafter\colvecnext
	\else
\end{pmatrix}
\fi
}

% numero como secciones
\renewcommand{\thesection}{\arabic{chapter}}
\renewcommand{\thesubsection}{\Alph{section}}

% To remove Springer from the title page
\usepackage{etoolbox}
\makeatletter
\patchcmd{\@maketitle}{{\Large Springer\par}}{}{}{}
\makeatother

\begin{document}
 
\lstset{language=GAP,
  showstringspaces=false,
  xleftmargin=0.6cm,
  xrightmargin=0.6cm,
  basicstyle=\small\ttfamily,
  frame=single,
  framerule=0pt,
}


\author{Leandro Vendramin}
\title{Rings and modules}
\subtitle{Notes}
\maketitle

\frontmatter

%\include{dedic}
\preface

The notes correspond to the bachelor 
course \emph{Ring and Modules} of the 
Vrije Universiteit Brussel, 
Faculty of Sciences, 
Department of Mathematics and Data Sciences. The course
is divided into thirteen two-hours lectures. 

The material is somewhat standard. Basic texts on abstract algebra
are for example \cite{MR1129886}, \cite{MR2286236} and \cite{MR600654}. 
Lang's book \cite{MR783636} is also a standard reference, but 
maybe a little bit more advanced. 
We based the lectures on representation theory of finite
groups on \cite{MR0450380} and 
\cite{MR2867444}. 

We also mention a set of great expository papers by 
Keith Conrad available at 
\url{https://kconrad.math.uconn.edu/blurbs/}. 
The notes are extremely well-written and are useful at  
at every stage of a mathematical career. 
 
Thanks go to Arne van Antwerpen, Luca Descheemaeker, Lucas Simons
and Geoffrey Jassens. 

This version 
was compiled on \today~at~\currenttime.

\begin{flushright}
Leandro Vendramin\\Brussels, Belgium\par
\end{flushright}

%\include{foreword}

\tableofcontents

\mainmatter

\part{Rings}
\chapter{Rings}

\begin{definition}
\index{Ring}
A \textbf{ring} is a set $R$ with two binary operations, the addition
$R\times R\to R$, $(x,y)\mapsto x+y$, and the multiplication
$R\times R\to R$, $(x,y)\mapsto xy$, such that
the following properties hold:
\begin{enumerate}
    \item $(R,+)$ is an abelian group.
    \item $(xy)z=x(yz)$ for all $x,y,z\in R$.
    \item $x(y+z)=xy+xz$ for all $x,y,z\in R$.
    \item $(x+y)z=xz+yz$ for all $x,y,z\in R$.
    \item There exists $e\in R$ such that $xe=ex=x$ for all $x\in R$.
\end{enumerate}
\end{definition}

\begin{definition}
\index{Ring!commutative}
A ring $R$ is said to be \textbf{commutative} if $xy=yx$ for all $x,y\in R$. 
\end{definition}

\begin{example}
$\Z$, $\Q$, $\R$ and $\C$ are commutative rings.
\end{example}

\begin{example}
    If $R$ is a commutative ring, then the set 
    \[
    R[X]=\left\{\sum_{i=0}^na_iX^i:n\in\N_0,\,a_1,\dots,a_n\in R\right\}
    \]
    of polynomials is a commutative ring with the usual operations. 
\end{example}

\begin{example}
    If $A$ is an abelian group, then $\End(A)$ is a ring with
    \[
    (f+g)(x)=f(x)+g(x),\quad
    (fg)(x)=f(g(x)),\quad f,g\in\End(A)\text{ and }x\in A.
    \]
\end{example}

Let $R$ be a ring. 
Some facts:
\begin{enumerate}
    \item $x0=0x=x$ for all $x\in R$.
    \item $x(-y)=-xy$ for all $x,y\in R$.
    \item If $1=0$, then $|R|=1$. 
\end{enumerate}

\begin{example}
    The real vector space $H(\R)=\{a1+bi+cj+dk:a,b,c,d\in\R\}$ with basis $\{1,i,j,k\}$ 
    is a ring with the multiplication induced by
    the formulas 
    \[
    i^2=j^2=k^2=-1,
    \quad ij=k,
    \quad jk=i,
    \quad ki=j.
    \]
    As an example, let us perform a calculation in $H(\R)$: 
    \[
    (1+i+j)(i+k)=i+k-1+ik+ji+jk=i+k-1-j-k+i=-1+2i-j,
    \]
    as $ij=i(ij)=-j$. 
\end{example}

\begin{example}
    Let $n\geq2$. 
    The abelian group $\Z/n=\{0,1,\dots,n\}$ of integers modulo $n$ is a ring 
    with the usual multiplication modulo $n$. 
\end{example}

\begin{example}
    Let $n\geq1$. 
    The set $M_n(\R)$ of real $n\times n$ matrices is a ring with the usual matrix operations. Recall
    that if $a=(a_{ij})$ and $b=(b_{ij})$, the multiplication $ab$ is given by
    \[
    (ab)_{ij}=\sum_{k=1}^n a_{ik}b_{kj}.
    \]
\end{example}

\begin{example}
    Real polynomials in two commuting variables form a ring. This ring will be denoted by $\R[X,Y]$. 
\end{example}

\begin{definition}
\index{Subring}
    Let $R$ be a ring. A \textbf{subring} $S$ of $R$ is a subset $S$ such that
    $(S,+)$ is a subgroup of $(R,+)$ such that $1\in S$ and 
    if $x,y\in S$, then $xy\in S$. 
\end{definition}

\begin{example}
    Clearly $\Z$ is a subring of $\Z$. 
\end{example}

\begin{example}
    $\Z\subseteq\Q\subseteq\R\subseteq\C$ is a chain of subrings. 
\end{example}

\begin{example}
    \index{Gauss integers}
    $\Z[i]=\{a+bi:a,b\in\Z\}$ is a subring of $\C$. This is known as the ring of \textbf{Gauss integers}.  
\end{example}

\begin{example}
    $\Q[\sqrt{2}]=\{a+b\sqrt{2}:a,b\in\Q\}$ is a subring of $\R$. 
\end{example}

\begin{example}
    \index{Center!of ring}
    If $R$ is a ring, then the \textbf{center} $Z(R)=\{x\in R:xy=yx\text{for all $y\in R$}\}$ is a subring of $R$. 
\end{example}

\chapter{Chinese remainder theorem}
\chapter{Noetherian rings}

In this chapter we will work with commutative rings. 

\begin{definition}
	A ring $R$ is said to be \textbf{noetherian} if every (increasing)
	sequence $I_1\subseteq I_2\subseteq\cdots$ of ideals of $R$
	stabilizes, that is $I_n=I_m$ for some $m\in\N$ and all $n\geq m$. 
\end{definition}

The ring $\Z$ of integers is noetherian.

\begin{example}
Let $R=\{f\colon [0,1]\to\R\}$ with 
\[
(f+g)(x)=f(x)+g(x),
\quad
(fg)(x)=f(x)g(x),
\quad
f,g\in R,\,x\in [0,1].
\]
For $n\in\N$ let
$I_n=\{f\in R:f|_{[0,1/n]}=0\}$. Then each $I_n$ is an ideal of $R$ and 
the sequence 
$I_1\subsetneq I_2\subsetneq\cdots$ 
does not stabilizes. Thus $R$ is not noetherian. 
\end{example}

\begin{definition}
	Let $R$ be a ring. An ideal $I$ of $R$ is said to be \textbf{finitely generated} if $I=(X)$ for some
	finite subset $X$ of $R$. 
\end{definition}

The zero ideal is always finitely generated. 

\begin{proposition}
Let $R$ be a ring. Then $R$ is noetherian if and only 
if every ideal of $R$ is finitely generated. 	
\end{proposition}

\begin{proof}
	Assume first that $R$ is noetherian. Let $I$ be an ideal of $R$ that is not finitely generated. 
	Thus $I\ne\{0\}$. Let $x_1\in I\setminus\{0\}$ and let $I_1=(x_1)$. Since $I$ is not finitely
	generated, $I\ne I_1$ and hence   
	$\{0\}\subsetneq I_1\subsetneq I$. Once I have the ideals $I_1,\dots,I_{k-1}$, let 
	$x_k\in I\setminus I_{k-1}$ (such an element exists because $I_{k-1}$ is finitely generated
	and $I$ is not) and $I_k=(I_{k-1},x_k)$. The sequence
	$\{0\}\subsetneq I_1\subsetneq I_2\subsetneq\cdots$ does not stabilize.  
	
	Assume now that every ideal of $R$ is finitely generated and 
	let $I_1\subseteq I_2\subseteq\cdots$ be a sequence of ideals of $R$. Then
	$I=\cup_{i\geq1}I_i$ is an ideal of $R$, so it is finitely generated, sayç
	$I=(x_1,\dots,x_n)$. We may assume that $x_j\in I_{i_j}$ for all $j$. Let 
	$N=\max\{j_1,\dots,j_n\}$ and $n\geq N$. Then 
	$I_N\subseteq I\subseteq I_N$ and therefore the seuqence stabilizes.  
\end{proof}

\begin{exercise}
	Let $R=\C[X_1,X_2,\cdots]$ be the ring of polynomial in an infinite number of 
	commuting variables. Prove that the ideal $I=(X_1,X_2,\dots)$ of polynomials 
	with zero contant term is not finitely generated. 
\end{exercise}


The correspondence theorem and the previous proposition 
allow us to prove easily the following result. 

\begin{proposition}
	Let $I$ be an ideal of $R$. If $R$ is noetherian, then $R/I$ is noetherian.
\end{proposition}

\begin{proof}
	Let $\pi\colon R\to R/I$ be the canonical surjection and let $J$ be an ideal of $R/I$. 
	Then $\pi^{-1}(J)$ is an ideal of $R$ containing $I$. Since 
	$R$ is noetherian, $\pi^{-1}(J)$ is finitely generated, say 
	$\pi^{-1}(J)=(x_1,\dots,x_k)$ for $x_1,\dots,x_k\in R$. Thus 
	\[
	J=\pi(\pi^{-1}(J))=(\pi(x_1),\dots,\pi(x_k))
	\]
	and hence $J$ is finitely generated. 
\end{proof}

Since $\Z$ is noetherian, $\Z/n$ is noetherian for all $n\geq2$. 

\begin{exercise}
	Prove that $\R[X]$ is noetherian. 	
\end{exercise}

% usar qu ees principal
% todo: agregar despues de la prueba de la principalidad para Z que tambien R[X] es principal

\begin{theorem}[Hilbert]
	Let $R$ be a commutative ring. If $R$ is noetherian ring, then $R[X]$ is noetherian.	
\end{theorem}

\begin{proof}
	We need to show that every ideal of $R[X]$ is finitely generated. Assume that
	there is an ideal $I$ of $R[X]$ that is not finitely generated. In particular, $I\ne\{0\}$. 
	Let $f_1(X)\in I\setminus\{0\}$ be of minimal degree. For $i>1$ let 
	$f_i(X)\in I$ be of minimal degree such that $f_i(X)\not\in(f_1(X),\dots,f_{i-1}(X))$ (note
	that such an $f_i(X)$ exists because $I$ is not finitely generated). For each $i$ 
	let $a_i$ be the leading coefficient of $f_i(X)$, that is
	\[
	f_i(X)=a_iX^{n_i}+\cdots,
	\]
	where the dots denote lowest degree terms. Note that 
	$a_i\ne 0$.
	Let $J=(a_1,a_2,\dots)$. Since $R$ is noetherian, the sequence
	\[
	(a_1)\subseteq (a_1,a_2)\subseteq\cdots(a_1,a_2,\dots,a_k)\subseteq\cdots
	\]
	stabilizes, so $J$ is finitely generated, say
	$J=(a_1,\dots,a_m)$ for some $m\in\N$. 
	There exist $u_1,\dots,u_m\in R$ such that 
	\[
	a_{m+1}=\sum_{i=1}^m u_ia_i.
	\]
	Let 
	\[
	g(X)=\sum_{i=1}^mu_if_i(X)X^{n_{m+1}-n_i}\in (f_1(X),\dots,f_m(X)).
	\]
	The leading coefficient of $g(X)$ is $\sum_{i=1}^mu_ia_i=a_{m+1}$ and, moreover, 
	the degree of $g(X)$ is $n_{m+1}$. Thus $\deg(g(X))<n_{m+1}$. 
	Since $f_{m+1}(X)\not\in (f_1(X)\dots,f_n(X))$, 
	\[
	g(X)-f_{m+1}(X)\not\in (f_1(X),\dots,f_n(X)),
	\]
	a contradiction to the minimality of the degree of $f_{m+1}$.  
\end{proof}

Since $R[X_1,\dots,X_n]=(R[X_1,\dots,X_{n-1}])[X_n]$, by induction 
one proves that if $R$ is a commutative noetherian ring, 
then $R[X_1,\dots,X_n]$ is noetherian. 
 
\begin{example}
	Since $\Z$ is noetherian, so is $\Z[X]$ by Hilbert's theorem. Now 
	$\Z[\sqrt{N}]$ is noetherian, as $\Z[\sqrt{N}]\simeq\Z[X]/(X^2-N)$ and quotients
	of noetherian rings are noetherian.  	
\end{example}

\begin{example}
	The ring $\Z[X,X^{-1}]$ is noetherian, as $\Z[X,X^{-1}]\simeq\Z[X,Y]/(XY-1)$. 
\end{example}
 
 
\begin{exercise}
	Prove that $R[[X]]$ is noetherian if $R$ is noetherian. 	
\end{exercise}

\begin{exercise}
	Let $f\colon R\to R$ be surjective ring homomorphism. Prove that $f$ is an isomorphism
	if $R$ is noetherian. 	
\end{exercise}



\chapter{Factorization}

\begin{definition}
\index{Integral domain}
	A commutative ring $R$ is said to be an \textbf{integral domain}
	if $xy=0$ implies $x=0$ or $y=0$.  	
\end{definition}

The rings $\Z$ and $\Z[i]$ are both integral domains. 
More generally, if $N$ is a square-free integer, 
then the ring $\Z[\sqrt{N}]$ is an integral domain.  
The ring $\Z/4$ of 
integers modulo 4 is not an integral domain. 

\begin{definition}
	Let $R$ be an integral domain and $x,y\in R$. Then $x$ \textbf{divides} $y$ 
	if $y=xz$ for some $x\in R$. 
	Notation: $x\mid y$ if and only if $x$ divides $y$. If $x$ does not
	divide $y$ one writes $x\nmid y$.  
\end{definition}

Note that $x\mid y$ if and only if $(y)\subseteq (x)$.
	
\begin{definition}
	Let $R$ be an integral domain and $x,y\in R$. Then $x$ and $y$ are
	\textbf{associate} in $R$ if $x=yu$ for some $u\in\mathcal{U}(R)$. 
\end{definition}

Note that $x$ and $y$ are associate if and only if $(x)=(y)$.  

\begin{example}
	The integers $2$ and $-2$ are associate in $\Z$.	
\end{example}

\begin{example}
	Let $R=\Z[i]$. 
	\begin{enumerate}
		\item Let $d\in\Z$ and $a+ib\in R$. Then $d\mid a+ib$ in $R$ if and only if 
			$d\mid a$ and $d\mid b$ in $\Z$. 
		\item $2$ and $-2i$ are associate in $R$.
	\end{enumerate} 	
\end{example}

\begin{example}
	Let $R=\R[X]$ and $f(X)\in R$. Then $f(X)$ and $\lambda f(X)$ are 
	associate in $R$ for all $\lambda\in\R^{\times}$. 	
\end{example}

\begin{definition}
	Let $R$ be an integral domain and $x\in R\setminus\{0\}$. Then $x$ is \textbf{irreducible} 
	if and only if $x\not\in\mathcal{U}(R)$ 
	and $x=ab$ with $a,b\in R$ implies that $a\in\mathcal{U}(R)$ or $b\in\mathcal{U}(R)$. 
\end{definition}

Note that $x$ is irreducible if and only if $(x)\ne R$ 
and there is no principal ideal $(y)$ such that 
$(x)\subsetneq (y)\subsetneq R$.

\begin{example}
	Let $R=\R[X]$ and $f(X)\in R\setminus\{0\}$. Then $f(X)$ is irreducible if 
	$\lambda\in\R^{\times}$ or $\lambda f(X)$ for $\lambda\in\R^{\times}$ 
	are the only divisors
	of $f(X)$.  
\end{example}

The irreducibles of $\Z$ are the prime numbers. 

\begin{definition}
	Let $R$ be an integral domain and $p\in R\setminus\{0\}$. Then  
	$p$ is \textbf{prime} if $p\not\in\mathcal{U}(R)$ and 
	$yz\in (p)$ implies that $y\in (p)$ or $z\in (p)$. 
\end{definition}

In $\Z$ primes and irreducible coincide. This does not happend in full generality. However,
the following result holds. 

\begin{proposition}
	Let $R$ be an integral domain and $x\in R$. 
	If $x$ is prime, then $x$ is irreducible. 
\end{proposition}

\begin{proof}
	Let $p$ be a prime. Then $p\ne 0$ and $p\not\in\mathcal{U}(R)$. Let $x$ be such that
	$x\mid p$. Then $p=xy$ for some $y\in R$. This means $xy\in (p)$, 
	so $x\in (p)$ or $y\in (p)$ because
	$p$ is prime. If $x\in (p)$, then $x=pz$ for some $z\in R$ and hence
	\[
	p=xy=(pz)y.
	\]
	Since $p-pzy=p(1-zy)$ and $R$ is an integral domain, it follows that 
	$1-zy=0$. Thus $y\in\mathcal{U}(R)$. Similarly, if $y\in (p)$, then 
	$x\in\mathcal{U}(R)$. 
\end{proof}

To show that there rings where some irreducibles are not prime, 
we need the following lemma. 

\begin{lemma}
Let $N\in\Z$ be a square-free integer and $R=\Z[\sqrt{N}]$. The map 
\[
N\colon R\to\N,
\quad a+b\sqrt{N}\mapsto 
|a^2-Nb^2|,
\]
satisfies the following properties:
\begin{enumerate}
	\item $N(\alpha)=0$ if and only if $\alpha=0$. 
	\item $N(\alpha\beta)=N(\alpha)N(\beta)$ for all $\alpha,\beta\in R$. 
	\item $\alpha\in\mathcal{U}(\Z[\sqrt{N}])$ if and only if $N(\alpha)=1$. 
	\item If $N(\alpha)$ is prime in $\Z$, then $\alpha$ is irreducible in $R$. 
\end{enumerate}	
\end{lemma}

\begin{proof}
	The first three items are left as an exercises. Let us prove 4). 
	If $\alpha=\beta\gamma$ for some $\beta,\gamma\in R$, then
	$N(\alpha)=N(\beta)N(\gamma)$. Since $N(\alpha)$ is a prime number, it follows that
	$N(\alpha)=1$ or $N()\beta)=1$. Thus $\beta\in\mathcal{U}(R)$ or $\gamma\in\mathcal{U}(R)$. 	
\end{proof}

\begin{example}
	Let $R=\Z[i]$. 
	\begin{enumerate}
		\item $\mathcal{U}(R)=\{-1,1,i,-i\}$.
		\item $3$ is irreducible in $R$. In fact, if $3=\alpha\beta$, then
			$9=N(\alpha)N(\beta)$. This implies that $N(\alpha)\in\{1,3,9\}$. Write
			$\alpha=a+bi$ for $a,b\in\Z$. If $N(\alpha)=1$, then $\alpha\in\mathcal{U}(R)$ by the lemma. 
			If $N(\alpha)=9$, then $N(\beta)=1$ and hence $\beta\in\mathcal{U}(R)$ by the lemma. Finally, 
			if $N(\alpha)=3$, then $a^2+b^2=3$, which is a contradiction since $a,b\in\Z$. 
		\item $2$ is not irreducible in $R$. In fact, $2=(1+i)(1-i)$ and
			since $N(1+i)=N(1-i)=2$, it follows that $1+i\not\in\mathcal{U}(R)$ 
			and $1-i\not\in\mathcal{U}(R)$. 
	\end{enumerate}	
\end{example}


\chapter{Zorn´s lemma}
\chapter{Algebras}

\index{Group ring}
\index{Group algebra}
We now discuss an important family of examples. 
Fix a field $K$. 
For a finite group $G$ let $K[G]$ be the vector space (over $K$)
with basis $\{g:g\in G\}$. Then $K[G]$ is a ring
with
\[
\left(\sum_{g\in G}\lambda_gg\right)\left(\sum_{h\in G}\mu_hh\right)
=\sum_{g,h\in G}\lambda_g\mu_h(gh).
\] 

Thus $K[G]$ is a ring and also a vector space (over $K$) and these structures
are somewhat compatible. Note that
\[
(\lambda a+\mu g=c=\lambda (ac)+\mu (bc),\quad
a(\lambda b+\mu c)=\lambda (ab)+\mu (ac)
\]
for all $\lambda,\mu\in K$ and $a,b,c\in K[G]$. 

\begin{definition}
\index{Algebra}
Let $A$ be a ring. Then $A$ is an algebra (over the field $K$) if $A$ is a vector space
and the map $K\to Z(A)$, $k\mapsto k1_A$, is an injective ring homomorphism.  
\end{definition}

Thus $K[G]$ is an algebra, as it is ring that contains $K$ in its center (or more precisely,
the map $K\to Z(K[G])$, $k\mapsto k1$, is an injective ring homomorphism.  
Other examples of algebras 
are the polynomial rings $K[X]$ and $K[X,Y]$ and matrix rings $M_n(K)$.  

\begin{example}
	If $A$ is an algebra, then $M_n(A)$ is an algebra.	
\end{example}

The ring $K[G]$ is commutative if and only if $G$ is abelian. Moreove,
$K[G]$ is a vector space of dimension $\dim K[G]=|G|$.

\begin{example}
	Let $G=\langle g:g^3=1\rangle=\{1,g,g^2\}\simeq C_3$ be the cyclic group of order three. 
	If $\alpha=a_11+_2g+a_3g^2$ and $\beta=b_11+b_2g+b_3g^2$, then
	\[
		\alpha\beta=(a_1b_1+a_2b_3+a_3b_2)1+(a_1b_2+a_2b_1+a_3b_3)g+a_1b_3+a_2b_2+a_3b_1)g^2.
	\]
	One can check that $\C[G]\simeq\C[X]/(X^3-1)$. 
\end{example}

In general, one proves that $\C[C_n]\simeq\C[X]/(X^n-1)$ for $n\geq2$.

\begin{example}
	Let $K$ be a field and 
	$G=\{1,g\}\simeq C_2$ be the cyclic group of order two. The product
	of $K[G]$ is 
	\[
	(a1+bg)(c1+gd)=(ac+bd)1+(ad+bc)g.
	\]
	
	If $K=\C$, then the map $K[G]\to K\times K$, $a1+bg\mapsto (a+b,a-b)$, 
	is a linear isomorphism of rings. 
	
	If $K=\Z/2$, then the map $K[G]\to\begin{pmatrix}
		K&K\\
		0&K
	\end{pmatrix}$, $a1+bg\mapsto\begin{pmatrix}
		a+b&b\\
		0&a+b		
	\end{pmatrix}$, is a linear isomorphism of rings. 
\end{example}

\begin{exercise}
	Prove that $\R[C_3]\simeq\R\times\C$. 	
\end{exercise}

The group ring has the following property, which is left as an exercise. 
zzZLet $R$ be a ring and
$G$ be a finite group. If $f\colon G\to\mathcal{U}(R)$ is a group homomorphism, 
then there exists a unique ring homomorphism $\varphi\colon K[G]\to R$ such that
$\varphi|_G=f$. 

\begin{example}
	Let $\D_3=\langle r,s:r^3=s^2=1,\,srs^{-1}=r^{-1}\rangle$ be the dihedral
	group of six elements. We claim that 
	\[
	\C[\D_3]\simeq\C\times\C\times M_2(\C).
	\]
	Let $\omega$ be a primitive root of one. Let 
	\[
	R=\begin{pmatrix}
		\omega&0\\
		0&\omega^2	
	\end{pmatrix},
	\quad
	S=\begin{pmatrix}
		0&1\\
		1&0
	\end{pmatrix}.
 	\]
 	One easily checks that $SRS^{-1}=R^{-1}$ and $R^3=S^2=\begin{pmatrix}
		1&0\\
		0&1	
	\end{pmatrix}$. It follows that there exists a group homomorphism
	$G\to\C\times\C\times M_2(\C)$ such that
	$r\mapsto (1,1,R)$ and $s\mapsto (1,-1,S)$. This group homomorphism
	is a ring isomorphism.  
\end{example}




 


\part{Representations}
%\include{maschke}
%\part{Modules}
%\chapter{Modules, submodules, homomorphisms}

\begin{definition}
    Let $R$ be a ring. A \textbf{module} (over $R$) is an abelian group
    $M$ with a map $R\times M\to M$, $(x,m)\mapsto x\cdot m$, such that
    the following conditions hold:
    \begin{enumerate}
        \item $(r_1+r_2)\cdot m=r_1\cdot m+r_2\cdot m$ for all $r_1,r_2\in R$ y $m\in M$.
		\item $r\cdot (m_1+m_2)=r\cdot m_1+r\cdot m_2$ for all $r\in R$ y $m_1,m_2\in M$.
		\item $r_1\cdot (r_2\cdot m)=(r_1r_2)\cdot m$ for all $r_1,r_2\in R$ y $m\in M$.
		\item $1\cdot m=m$ for all $m\in M$.	
    \end{enumerate}
\end{definition}

Our definition is that of left module. Similarly one defines right modules. We will always
consider left modules, so they will be referred simply as modules.

\begin{example}
A module over a field is a vector space. 
\end{example}

\begin{example}
Every abelian group is a modulo over $\Z$.	
\end{example}

\begin{example}
Let $R$ be a ring. Then $R$ is a module (over $R$) with $x\cdot m=xm$. 
This is the \textbf{(left) regular representation} of $R$ and it usually 
be denoted by $\prescript{}{R}R$. 
\end{example}

\begin{example}
If $R$ is a ring, then $R^n=\{(x_1,\dots,x_n):x_1,\dots,x_n\in R\}$ 
is a module (over $R$) with  
$r\cdot (x_1,\dots,x_n)=(rx_1,\dots,rx_n)$. 
\end{example}

\begin{example}
If $R$ is a ring, then $M_{m,n}(R)$ is a module (over $R$) with usual matrix operations. 
\end{example}

Students usually ask why in the definition of a ring homomorphism one needs
the condition $1\mapsto 1$. The following example provides a good explanation. 

\begin{example}
%%\label{exa:f(1)=1}
If $f\colon R\to S$ is a ring homomorphism and $M$ is a module (over $S$) wih 
$(s,m)\mapsto sm$, then 
$M$ is also a module (over $R$) with $r\cdot m=f(r)m$ for all $r\in R$ and $m\in M$. In fact, 
\begin{align*}
&1\cdot m=f(1)m=1m=m,\\
&r_1\cdot (r_2\cdot m)=f(r_1)(r_2\cdot m)=f(r_1)(f(r_2)m)=(f(r_1)f(r_2))m=f(r_1r_2)m
\end{align*}
for all $r_1,r_2\in R$ and $m\in M$.	  	
\end{example}
%
\begin{example}
Let $R=\R[X]$ and $T\colon\R^n\to\R^n$ be a linear map. Then $M=\R^n$ with 
\[
\left(\sum_{i=0}^na_iX^i\right)\cdot v=\sum_{i=0}^na_iT^i(v)
\]	
is a module (over $R$).   
\end{example}

\begin{example}
If $\{M_i|i\in I\}$ is a family of modules, then  	
\[
\prod_{i\in I}M_i=\{(m_i)_{i\in I}:m_i\in M_i\text{ for all $i\in I$}\}
\]
is a module with 
$x\cdot (m_i)_{i\in I}=(x\cdot m_i)_{i\in I}$, 
where $(m_i)_{i\in I}$ denotes the map $I\to M_i$, $i\mapsto m_i$.
This module is the \textbf{direct product} of the family $\{M_i:i\in I\}$.
\end{example}
%
\begin{example}
If $\{M_i|i\in I\}$ is family of modules, then   	
\[
\bigoplus_{i\in I}M_i=\{(m_i)_{i\in I}:m_i\in M_i\text{ for all $i\in I$ and $m_i=0$ except finitely many $i\in I$}\}
\]
is a module with 
$x\cdot (m_i)_{i\in I}=(x\cdot m_i)_{i\in I}$. 
This module is the \textbf{direct sum} of the family $\{M_i:i\in I\}$. 
\end{example}
%
%\begin{exercise}
If $M$ is a module, then $0\cdot m=0$ and $-m=(-1)\cdot m$ for all $m\in M$ and 
$x\cdot 0=0$ for all $x\in R$. 
%
\begin{example}
Let $M=\Z/6$ as a module (over $\Z$). Note that 
$3\cdot 2=0$ but $3\ne 0$ (in $\Z$) and $2\ne 0$ (in $\Z/6$).  
\end{example}
%
\begin{definition}
	Let $M$ be a module. A subset $N$ of $M$ is a \textbf{submodule} of $M$ if 
	$(N,+)$ is a subgroup of $(M,+)$ and 
	$x\cdot n\in N$ for all $x\in R$ and $n\in N$. 
\end{definition}

Clearly, if $M$ is a module, then $\{0\}$ and $M$ are submodules of $M$. 

\begin{example}
Let $R$ be a field and $M$ be a module over $R$. Then
$N$ is a submodule of $M$ if and only if $N$ is a subspace of $M$. 
\end{example}

\begin{example}
Let $R=\Z$ and $M$ be a module (over $R$). Then
$N$ is a submodule of $M$ if and only if $N$ is a subgroup of $M$
\end{example}

\begin{example}
If $M=\prescript{}{R}R$, then a subset $N\subseteq M$ is a submodule
of $M$ if and only if $N$ is a left ideal of $R$. 
\end{example}

\begin{example}
If $V$ is a vector space and $T\colon V\to V$ is a linear map, then
$V$ is a module (over $\R[X]$) with  
\[
\left(\sum_{i=0}^na_iX^i\right)\cdot v=\sum_{i=0}^na_iT^i(v).
\]
A submodule is a subspace $W$ 
of $V$ such that $T(W)\subseteq W$. 
\end{example}

Clearly, a subset $N$ of $M$ is a submodule if and only 
if $r_1n_1+r_2n_2\in N$ for all
$r_1,r_2\in R$ and $n_1,n_2\in N$. 	

\begin{exercise}
If $N$ and $N_1$ are submodules of $M$, then 
\[
N+N_1=\{n+n_1:n\in N,\,n_1\in N_1\}
\]
is a submodule of $M$.
\end{exercise}

\begin{definition}
Let $M$ and $N$ be modules over $R$. 
A map $f\colon M\to N$ is a \textbf{module homomorphism} if $f(x+y)=f(x)+f(y)$ and 
$f(r\cdot x)=r\cdot f(x)$ for all $x,y\in M$ and $r\in R$. 
\end{definition}

We denote by $\Hom_R(M,N)$ the set of module homomorphisms $M\to N$. 

\begin{exercise}
Let $f\in\Hom_R(M,N)$.  
\begin{enumerate}
\item If $V$ is a submodule of $M$, then $f(V)$ is a submodule of $N$.
\item If $W$ is a submodule of $N$, then $f^{-1}(W)$ is a submodule of $M$.
\end{enumerate}
\end{exercise}

If $f\in\Hom_R(M,N)$, the \textbf{kernel} of $f$ is the submodule  
\[
\ker f=f^{-1}(\{0\})=\{m\in M:f(m)=0\}
\]
of $M$. We say that $f$ is a \textbf{monomorphism} (resp. \textbf{epimorphism}) 
if $f$ is injective (resp. surjective). Moreover, $f$ is an \textbf{isomorphism} 
if $f$ is
bijective. 

\begin{exercise}
Let $f\in\Hom_R(M,N)$. Prove that the following statements are equivalent:
\begin{enumerate}
\item $f$ is a monomorphism.
\item $\ker f=\{0\}$.
\item For every module $V$ and every $g,h\in\Hom_R(V,M)$, $f\circ g=f\circ h\implies g=h$.
\item For every module $V$ and every $g\in\Hom(V,M)$, $f\circ g=0\implies g=0$.
\end{enumerate}
\end{exercise}


\begin{example}
	Let $R=
		\begin{pmatrix}
			\R & 0\\
			0 & \R
		\end{pmatrix}$. 
	We claim that 
	$\begin{pmatrix}
			\R\\
			0
		\end{pmatrix}
		\not\simeq\begin{pmatrix}
			0\\
			\R
		\end{pmatrix}$
	as modules over $R$, where the module structure is given by the usual matrix multiplication. 
	Assume that they are isomorphic. 
	Let $f\colon\begin{pmatrix}
			0\\
			\R
		\end{pmatrix}
		\to\begin{pmatrix}
			\R\\
			0
		\end{pmatrix}$  
	be an isomorphism of modules and let  
	$x_0\in\R\setminus\{0\}$ be such that 
	$f\begin{pmatrix}0\\1\end{pmatrix}=\begin{pmatrix}x_0\\0\end{pmatrix}$. Thus 
	\[
	f\begin{pmatrix}
	0\\
	1\end{pmatrix}
	=f\left(\begin{pmatrix}
	0&0\\
	0&1\end{pmatrix}
	\cdot 
	\begin{pmatrix}
	0\\
	1
	\end{pmatrix}\right)
	=\begin{pmatrix}
	0&0\\
	0&1\end{pmatrix}\cdot f\begin{pmatrix}0\\1\end{pmatrix}
	=\begin{pmatrix}
	0&0\\
	0&1
	\end{pmatrix}
	\cdot 
	\begin{pmatrix}		
	x_0\\
	0
	\end{pmatrix}
	=\begin{pmatrix}
	0\\
	0
	\end{pmatrix},
	\]	
	a contradiction, as $f$ is injective.    
\end{example}

If $N$ and $N_1$ are submodules of $M$, we say that $M$ is the \textbf{direct sum} of $N$ and $N_1$
if $M=N+N_1$ and $N\cap N_1=\{0\}$. In this case, we write $M=N\oplus N_1$. Note that if
$M=N\oplus N_1$, then each $m\in M$ can be written uniquely as $m=n+n_1$ for some
 $n\in N$ and $n_1\in N_1$. 
Such a decomposition exists because $M=S+T$. If $m\in M$ can be written as 
$m=n+n_1=n'+n_1'$ for some $n,n'\in N$ and $n_1,n_1'\in N_1$, then 
$-n'+n=n_1'-n_1\in N\cap N_1=\{0\}$ and hence $n=n'$ and $n_1=n_1'$. If $M=N\oplus N_1$, the submodule
$N$ (resp. $N_1$) is a \textbf{direct summand} of $M$ and the submodule $N_1$ (resp $N$) is a \textbf{complement} of $N$ 
in $M$.   	

\begin{example}
If $M=\R^2$ as a vector space, then every subspace of $M$ is a direct summand of $M$.
\end{example}

Clearly, the submodules $\{0\}$ and $M$ are direct summands of $M$.

\begin{example}
If $M=\Z$ as a module over $\Z$, then $m\Z$ is a direct sum of $M$ if and only if 
$m\in\{0,1\}$, as $n\Z\cap m\Z=\{0\}$ if and only if $nm=0$.
\end{example}

\begin{exercise}
\label{xca:projector}
Let $M$ be a module. 
A module $N$ is isomorphic to a direct summand of $M$ if and only if
there are module homomorphisms $i\colon N\to M$ and $p\colon M\to N$ 
such that $p\circ i=\id_N$. In this case, $M=\ker p\oplus i(N)$.  
\end{exercise}



\part{Modules}

\backmatter
%\include{glossary}
\chapter*{Some hints}
\addcontentsline{toc}{chapter}{Some hints}

%\begin{sol}
%\end{sol}

\begin{sol}{xca:Z[sqrt10]/(2,sqrt10)}
	Use the ring homomorphism $\Z[\sqrt{10}]\to\Z/2$, $a+b\sqrt{10}\mapsto a\bmod 2$. 	
\end{sol}

\begin{sol}{xca:Z[i]/(1+3i)}
	Use the ring homomorphism $\Z\hookrightarrow\Z[i]\xrightarrow{\pi}\Z[i]/(1+3i)$, where
	$\pi$ is the canonical map. 	
\end{sol}

\begin{sol}{xca:extend}
    If $X$ is a linearly independent set, a basis 
    of $V$ that contains $X$ will be found as a maximal linearly independent set 
    containing $X$. 
\end{sol}

\begin{sol}{xca:freshman_dream}
    Use induction on $n$ and the fact that the prime number $p$ divides 
    $\binom{p}{k}$ for all $k\in\{1,2,\dots,p-1\}$. Alternatively, one could use
    that the prime number $p$ divides $\binom{p^n}{k}$ for all 
    $k\in\{1,2,\dots,p^n-1\}$. 
\end{sol}




\chapter{Some solutions}

\section*{Lecture 1}
\section*{Lecture 2}
\section*{Lecture 3}
\section*{Lecture 4}
\section*{Lecture 5}
\section*{Lecture 6}
\section*{Lecture 7}
\section*{Lecture 8}
\section*{Lecture 9}
\section*{Lecture 10}
\section*{Lecture 11}
\section*{Lecture 12}
\section*{Lecture 13}

\begin{sol}{xca:projector}
\end{sol}

\begin{sol}{xca:submodules}
\end{sol}

\begin{sol}{xca:commuting}
\end{sol}

\begin{sol}{xca:Hom}
\end{sol}


\bibliographystyle{abbrv}
\bibliography{refs}

\printindex



\end{document}





