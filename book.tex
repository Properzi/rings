\RequirePackage{amsmath} 

\documentclass[graybox,envcountsect]{svmono}

\usepackage[T1]{fontenc}
\usepackage[utf8]{inputenc}

\usepackage{amsmath}
\usepackage[notref,notcite]{showkeys}
\usepackage{float}
\usepackage{amssymb}
\usepackage{amstext}
\usepackage{mathtools}
\usepackage{xcolor} 
\usepackage{centernot}
\usepackage{listings}
\usepackage{multicol}
\usepackage{mathptmx}
%\usepackage{newtxtext,newtxmath}
%\usepackage{txfonts}
\usepackage{datetime}
\usepackage{tikz-cd}

\usepackage{helvet}
\usepackage{courier}
\usepackage{type1cm}         
\usepackage{makeidx}        
\usepackage{graphicx}        
\usepackage{multicol}        
\usepackage[all]{xy}
\usepackage{hyperref} 
%\usepackage{tikz-cd}
\usepackage{colortbl}


\usepackage[small,bf]{caption}

\usepackage{tikz}
\usetikzlibrary{braids}
	
\usepackage[bottom]{footmisc}

% for QED
\let\proof\relax\let\endproof\relax
\usepackage{amsthm}

\overfullrule=1mm

%%% for Spanish
% \def\abstractname{Resumen}%
% \def\ackname{Agradecimientos}%
% \def\andname{y}%
% \def\bibname{Referencias}%
% \def\lastandname{, y}%
% \def\appendixname{Apéndice}%
% \def\chaptername{Capítulo}%
% \def\claimname{Afirmación}%
% \def\conjecturename{Conjetura}%
% \def\contentsname{Contenidos}%
% \def\corollaryname{Corolario}%
% \def\definitionname{Definici\'on}%
% \def\emailname{e-mail}%
% \def\examplename{Ejemplo}%
% \def\examplesname{Ejemplos}%
% \def\exercisename{Ejercicio}%
\def\figurename{Figure}%
% \def\forewordname{Foreword}%
% \def\keywordname{{\bf Palabras clave:}}%
% \def\indexname{Índice}%
% \def\lemmaname{Lema}%
% \def\listfigurename{Figuras}%
% \def\listtablename{Tablas}%
% \def\notename{Nota}%
% \def\partname{Parte}%
% \def\prefacename{Prefacio}%
\def\problemname{Open problem}%
% \def\proofname{Demostración}%
% \def\propertyname{Propiedad}%
% \def\propositionname{Proposici\'on}%
% \def\questionname{Pregunta}%
% \def\refname{Referencias}%
% \def\remarkname{Observación}%
% \def\seename{see}%
% \def\solutionname{Solución}%
% \def\tablename{Tabla}%
% \def\theoremname{Teorema}
\def\notationname{Notation}
\def\stepsname{Algorithm}
% \def\conventionname{Convención}

% change numbers 
\let\remark\relax
\let\theorem\relax
\let\lemma\relax
\let\definition\relax
\let\proposition\relax
\let\corollary\relax
\let\exercise\relax
\let\example\relax
\let\conjecture\relax
\spnewtheorem{theorem}{\theoremname}[section]{\bfseries}{\itshape}
\renewcommand\thetheorem{\thesection.\arabic{theorem}}
\spnewtheorem{lemma}[theorem]{\lemmaname}{\bfseries}{\itshape}
\spnewtheorem{definition}[theorem]{\definitionname}{\bfseries}{\upshape}
\spnewtheorem{proposition}[theorem]{\propositionname}{\bfseries}{\itshape}
\spnewtheorem{corollary}[theorem]{\corollaryname}{\bfseries}{\itshape}
\spnewtheorem{exercise}[theorem]{\exercisename}{\bfseries}{\upshape}
\spnewtheorem{example}[theorem]{\examplename}{\bfseries}{\upshape}
\spnewtheorem{examples}[theorem]{\examplesname}{\bfseries}{\upshape}
\spnewtheorem{remark}[theorem]{\remarkname}{}{\upshape}
\spnewtheorem{conjecture}[theorem]{\conjecturename}{\bfseries}{\upshape}
\spnewtheorem{notation}[theorem]{\notationname}{\bfseries}{\upshape}
\spnewtheorem{steps}[theorem]{\stepsname}{\bfseries}{\upshape}
\spnewtheorem{convention}[theorem]{\conventionname}{\bfseries}{\upshape}

% para enumerar
\renewcommand{\labelenumi}{\textbf{\arabic{enumi})}}

\setcounter{secnumdepth}{1}

\makeindex             

\renewcommand{\I}{\operatorname{I}}
\newcommand{\II}{\operatorname{II}}

\newcommand{\GAP}{\textsf{GAP}}
\newcommand{\FK}{\mathcal{E}}
\newcommand{\ad}[1]{\operatorname{ad}\,#1}

\newcommand{\N}{\mathbb{N}}
\newcommand{\Q}{\mathbb{Q}}
\newcommand{\Z}{\mathbb{Z}}
\newcommand{\F}{\mathbb{F}}
\newcommand{\R}{\mathbb{R}}
\newcommand{\C}{\mathbb{C}}
\renewcommand{\H}{\mathbb{H}}
\newcommand{\A}{\mathbb{A}}
\newcommand{\K}{\mathbb{K}}
\newcommand{\T}{\mathbb{T}}
\renewcommand{\D}{\mathbb{D}}
\newcommand{\B}{\mathbb{B}}
\newcommand{\Fun}{\operatorname{Fun}}
\newcommand{\mpl}{\operatorname{mpl}}
\newcommand{\cL}{\mathcal{L}}
\newcommand{\cE}{\mathcal{E}}
\newcommand{\cH}{\mathcal{H}}

\newcommand{\GF}{\mathsf{GF}}
\newcommand{\MAX}{\operatorname{MAX}}
\newcommand{\MIN}{\operatorname{MIN}}
\newcommand{\cf}{\operatorname{cf}}
\newcommand{\cl}{\operatorname{cl}}
\newcommand{\cd}{\operatorname{cd}}
\newcommand{\bL}{\mathbf{L}}
\newcommand{\bP}{\mathbf{P}}

\newcommand{\Nil}{\operatorname{Nil}}
\newcommand{\rad}{\operatorname{rad}}
\newcommand{\rank}{\operatorname{rank}}

\newcommand{\Aff}{\mathrm{Aff}}
\newcommand{\Ann}{\operatorname{Ann}}
\newcommand{\Der}{\operatorname{Der}}
\newcommand{\Core}{\operatorname{Core}}
\newcommand{\Soc}{\operatorname{Soc}}
\newcommand{\Fix}{\operatorname{Fix}}
\newcommand{\Rad}{\mathrm{rad}}
\newcommand{\Inn}{\mathrm{Inn}}
\newcommand{\dist}{\mathrm{dist}}
\newcommand{\Out}{\mathrm{Out}}
\newcommand{\Ext}{\mathrm{Ext}}
\newcommand{\Img}{\mathrm{im}}
\newcommand{\Hol}{\operatorname{Hol}}
\newcommand{\Hom}{\operatorname{Hom}}
\newcommand{\Alg}{\operatorname{Alg}}
\newcommand{\Bil}{\operatorname{Bil}}
\newcommand{\op}{\operatorname{op}}
\newcommand{\gr}{\operatorname{gr}}
\newcommand{\Syl}{\mathrm{Syl}}
\newcommand{\id}{\operatorname{id}}
\newcommand{\Aut}{\operatorname{Aut}}
\newcommand{\End}{\operatorname{End}}
\newcommand{\Irr}{\operatorname{Irr}}
\newcommand{\Alt}{\mathbb{A}}
\newcommand{\Sym}{\mathbb{S}}
\newcommand{\lcm}{\mathrm{mcm}}
\newcommand{\diag}{\operatorname{diag}}
\newcommand{\spec}{\operatorname{Spec}}
\newcommand{\supp}{\operatorname{supp}}
\newcommand{\trace}{\operatorname{trace}}
\newcommand{\sgn}{\operatorname{sign}}

\newcommand{\inner}{\operatorname{inn}}
\newcommand{\ext}{\operatorname{ext}}
\newcommand{\im}{\operatorname{im}}
\newcommand{\Ret}{\operatorname{Ret}}

\newcommand{\GL}{\mathbf{GL}}
\newcommand{\SL}{\mathbf{SL}}
\newcommand{\PSL}{\mathbf{PSL}}
\newcommand{\PGL}{\mathbf{PGL}}

\newcommand{\legendre}[2]{\left(\frac{#1}{#2}\right)}

%\newcommand{\char}{\operatorname{char}}

% multiset
\def\multiset#1#2{\ensuremath{\left(\kern-.3em\left(\genfrac{}{}{0pt}{}{#1}{#2}\right)\kern-.3em\right)}}

% column vector
\newcount\colveccount
\newcommand*\colvec[1]{
\global\colveccount#1
\begin{pmatrix}
	\colvecnext
	}
	\def\colvecnext#1{
	#1
	\global\advance\colveccount-1
	\ifnum\colveccount>0
	\\
	\expandafter\colvecnext
	\else
\end{pmatrix}
\fi
}

% numero como secciones
\renewcommand{\thesection}{\arabic{chapter}}
\renewcommand{\thesubsection}{\Alph{section}}

% To remove Springer from the title page
\usepackage{etoolbox}
\makeatletter
\patchcmd{\@maketitle}{{\Large Springer\par}}{}{}{}
\makeatother

\begin{document}
 
\lstset{language=GAP,
  showstringspaces=false,
  xleftmargin=0.6cm,
  xrightmargin=0.6cm,
  basicstyle=\small\ttfamily,
  frame=single,
  framerule=0pt,
}


\author{Leandro Vendramin}
\title{Rings and modules}
\subtitle{Notes}
\maketitle

\frontmatter

%\include{dedic}
\preface

The notes correspond to the bachelor 
course \emph{Ring and Modules} of the 
Vrije Universiteit Brussel, 
Faculty of Sciences, 
Department of Mathematics and Data Sciences. The course
is divided into thirteen two-hours lectures. 

The material is somewhat standard. Basic texts on abstract algebra
are for example \cite{MR1129886}, \cite{MR2286236} and \cite{MR600654}. 
Lang's book \cite{MR783636} is also a standard reference, but 
maybe a little bit more advanced. 
We based the lectures on representation theory of finite
groups on \cite{MR0450380} and 
\cite{MR2867444}. 

We also mention a set of great expository papers by 
Keith Conrad available at 
\url{https://kconrad.math.uconn.edu/blurbs/}. 
The notes are extremely well-written and are useful at  
at every stage of a mathematical career. 
 
Thanks go to Arne van Antwerpen, Luca Descheemaeker, Lucas Simons
and Geoffrey Jassens. 

This version 
was compiled on \today~at~\currenttime.

\begin{flushright}
Leandro Vendramin\\Brussels, Belgium\par
\end{flushright}

%\include{foreword}

\tableofcontents

\mainmatter

\part{Rings}
\chapter{Rings}

\begin{definition}
\index{Ring}
A \textbf{ring} is a set $R$ with two binary operations, the addition
$R\times R\to R$, $(x,y)\mapsto x+y$, and the multiplication
$R\times R\to R$, $(x,y)\mapsto xy$, such that
the following properties hold:
\begin{enumerate}
    \item $(R,+)$ is an abelian group.
    \item $(xy)z=x(yz)$ for all $x,y,z\in R$.
    \item $x(y+z)=xy+xz$ for all $x,y,z\in R$.
    \item $(x+y)z=xz+yz$ for all $x,y,z\in R$.
    \item There exists $e\in R$ such that $xe=ex=x$ for all $x\in R$.
\end{enumerate}
\end{definition}

\begin{definition}
\index{Ring!commutative}
A ring $R$ is said to be \textbf{commutative} if $xy=yx$ for all $x,y\in R$. 
\end{definition}

\begin{example}
$\Z$, $\Q$, $\R$ and $\C$ are commutative rings.
\end{example}

\begin{example}
    If $R$ is a commutative ring, then the set 
    \[
    R[X]=\left\{\sum_{i=0}^na_iX^i:n\in\N_0,\,a_1,\dots,a_n\in R\right\}
    \]
    of polynomials is a commutative ring with the usual operations. 
\end{example}

\begin{example}
    If $A$ is an abelian group, then $\End(A)$ is a ring with
    \[
    (f+g)(x)=f(x)+g(x),\quad
    (fg)(x)=f(g(x)),\quad f,g\in\End(A)\text{ and }x\in A.
    \]
\end{example}

Let $R$ be a ring. 
Some facts:
\begin{enumerate}
    \item $x0=0x=x$ for all $x\in R$.
    \item $x(-y)=-xy$ for all $x,y\in R$.
    \item If $1=0$, then $|R|=1$. 
\end{enumerate}

\begin{example}
    The real vector space $H(\R)=\{a1+bi+cj+dk:a,b,c,d\in\R\}$ with basis $\{1,i,j,k\}$ 
    is a ring with the multiplication induced by
    the formulas 
    \[
    i^2=j^2=k^2=-1,
    \quad ij=k,
    \quad jk=i,
    \quad ki=j.
    \]
    As an example, let us perform a calculation in $H(\R)$: 
    \[
    (1+i+j)(i+k)=i+k-1+ik+ji+jk=i+k-1-j-k+i=-1+2i-j,
    \]
    as $ij=i(ij)=-j$. 
\end{example}

\begin{example}
    Let $n\geq2$. 
    The abelian group $\Z/n=\{0,1,\dots,n\}$ of integers modulo $n$ is a ring 
    with the usual multiplication modulo $n$. 
\end{example}

\begin{example}
    Let $n\geq1$. 
    The set $M_n(\R)$ of real $n\times n$ matrices is a ring with the usual matrix operations. Recall
    that if $a=(a_{ij})$ and $b=(b_{ij})$, the multiplication $ab$ is given by
    \[
    (ab)_{ij}=\sum_{k=1}^n a_{ik}b_{kj}.
    \]
\end{example}

\begin{example}
    Real polynomials in two commuting variables form a ring. This ring will be denoted by $\R[X,Y]$. 
\end{example}

\begin{definition}
\index{Subring}
    Let $R$ be a ring. A \textbf{subring} $S$ of $R$ is a subset $S$ such that
    $(S,+)$ is a subgroup of $(R,+)$ such that $1\in S$ and 
    if $x,y\in S$, then $xy\in S$. 
\end{definition}

\begin{example}
    Clearly $\Z$ is a subring of $\Z$. 
\end{example}

\begin{example}
    $\Z\subseteq\Q\subseteq\R\subseteq\C$ is a chain of subrings. 
\end{example}

\begin{example}
    \index{Gauss integers}
    $\Z[i]=\{a+bi:a,b\in\Z\}$ is a subring of $\C$. This is known as the ring of \textbf{Gauss integers}.  
\end{example}

\begin{example}
    $\Q[\sqrt{2}]=\{a+b\sqrt{2}:a,b\in\Q\}$ is a subring of $\R$. 
\end{example}

\begin{example}
    \index{Center!of ring}
    If $R$ is a ring, then the \textbf{center} $Z(R)=\{x\in R:xy=yx\text{for all $y\in R$}\}$ is a subring of $R$. 
\end{example}

\chapter{Chinese remainder theorem}
\chapter{Noetherian rings}

In this chapter we will work with commutative rings. 

\begin{definition}
	A ring $R$ is said to be \textbf{noetherian} if every (increasing)
	sequence $I_1\subseteq I_2\subseteq\cdots$ of ideals of $R$
	stabilizes, that is $I_n=I_m$ for some $m\in\N$ and all $n\geq m$. 
\end{definition}

The ring $\Z$ of integers is noetherian.

\begin{example}
Let $R=\{f\colon [0,1]\to\R\}$ with 
\[
(f+g)(x)=f(x)+g(x),
\quad
(fg)(x)=f(x)g(x),
\quad
f,g\in R,\,x\in [0,1].
\]
For $n\in\N$ let
$I_n=\{f\in R:f|_{[0,1/n]}=0\}$. Then each $I_n$ is an ideal of $R$ and 
the sequence 
$I_1\subsetneq I_2\subsetneq\cdots$ 
does not stabilizes. Thus $R$ is not noetherian. 
\end{example}

\begin{definition}
	Let $R$ be a ring. An ideal $I$ of $R$ is said to be \textbf{finitely generated} if $I=(X)$ for some
	finite subset $X$ of $R$. 
\end{definition}

The zero ideal is always finitely generated. 

\begin{proposition}
Let $R$ be a ring. Then $R$ is noetherian if and only 
if every ideal of $R$ is finitely generated. 	
\end{proposition}

\begin{proof}
	Assume first that $R$ is noetherian. Let $I$ be an ideal of $R$ that is not finitely generated. 
	Thus $I\ne\{0\}$. Let $x_1\in I\setminus\{0\}$ and let $I_1=(x_1)$. Since $I$ is not finitely
	generated, $I\ne I_1$ and hence   
	$\{0\}\subsetneq I_1\subsetneq I$. Once I have the ideals $I_1,\dots,I_{k-1}$, let 
	$x_k\in I\setminus I_{k-1}$ (such an element exists because $I_{k-1}$ is finitely generated
	and $I$ is not) and $I_k=(I_{k-1},x_k)$. The sequence
	$\{0\}\subsetneq I_1\subsetneq I_2\subsetneq\cdots$ does not stabilize.  
	
	Assume now that every ideal of $R$ is finitely generated and 
	let $I_1\subseteq I_2\subseteq\cdots$ be a sequence of ideals of $R$. Then
	$I=\cup_{i\geq1}I_i$ is an ideal of $R$, so it is finitely generated, sayç
	$I=(x_1,\dots,x_n)$. We may assume that $x_j\in I_{i_j}$ for all $j$. Let 
	$N=\max\{j_1,\dots,j_n\}$ and $n\geq N$. Then 
	$I_N\subseteq I\subseteq I_N$ and therefore the seuqence stabilizes.  
\end{proof}

\begin{exercise}
	Let $R=\C[X_1,X_2,\cdots]$ be the ring of polynomial in an infinite number of 
	commuting variables. Prove that the ideal $I=(X_1,X_2,\dots)$ of polynomials 
	with zero contant term is not finitely generated. 
\end{exercise}


The correspondence theorem and the previous proposition 
allow us to prove easily the following result. 

\begin{proposition}
	Let $I$ be an ideal of $R$. If $R$ is noetherian, then $R/I$ is noetherian.
\end{proposition}

\begin{proof}
	Let $\pi\colon R\to R/I$ be the canonical surjection and let $J$ be an ideal of $R/I$. 
	Then $\pi^{-1}(J)$ is an ideal of $R$ containing $I$. Since 
	$R$ is noetherian, $\pi^{-1}(J)$ is finitely generated, say 
	$\pi^{-1}(J)=(x_1,\dots,x_k)$ for $x_1,\dots,x_k\in R$. Thus 
	\[
	J=\pi(\pi^{-1}(J))=(\pi(x_1),\dots,\pi(x_k))
	\]
	and hence $J$ is finitely generated. 
\end{proof}

Since $\Z$ is noetherian, $\Z/n$ is noetherian for all $n\geq2$. 

\begin{exercise}
	Prove that $\R[X]$ is noetherian. 	
\end{exercise}

% usar qu ees principal
% todo: agregar despues de la prueba de la principalidad para Z que tambien R[X] es principal

\begin{theorem}[Hilbert]
	Let $R$ be a commutative ring. If $R$ is noetherian ring, then $R[X]$ is noetherian.	
\end{theorem}

\begin{proof}
	We need to show that every ideal of $R[X]$ is finitely generated. Assume that
	there is an ideal $I$ of $R[X]$ that is not finitely generated. In particular, $I\ne\{0\}$. 
	Let $f_1(X)\in I\setminus\{0\}$ be of minimal degree. For $i>1$ let 
	$f_i(X)\in I$ be of minimal degree such that $f_i(X)\not\in(f_1(X),\dots,f_{i-1}(X))$ (note
	that such an $f_i(X)$ exists because $I$ is not finitely generated). For each $i$ 
	let $a_i$ be the leading coefficient of $f_i(X)$, that is
	\[
	f_i(X)=a_iX^{n_i}+\cdots,
	\]
	where the dots denote lowest degree terms. Note that 
	$a_i\ne 0$.
	Let $J=(a_1,a_2,\dots)$. Since $R$ is noetherian, the sequence
	\[
	(a_1)\subseteq (a_1,a_2)\subseteq\cdots(a_1,a_2,\dots,a_k)\subseteq\cdots
	\]
	stabilizes, so $J$ is finitely generated, say
	$J=(a_1,\dots,a_m)$ for some $m\in\N$. 
	There exist $u_1,\dots,u_m\in R$ such that 
	\[
	a_{m+1}=\sum_{i=1}^m u_ia_i.
	\]
	Let 
	\[
	g(X)=\sum_{i=1}^mu_if_i(X)X^{n_{m+1}-n_i}\in (f_1(X),\dots,f_m(X)).
	\]
	The leading coefficient of $g(X)$ is $\sum_{i=1}^mu_ia_i=a_{m+1}$ and, moreover, 
	the degree of $g(X)$ is $n_{m+1}$. Thus $\deg(g(X))<n_{m+1}$. 
	Since $f_{m+1}(X)\not\in (f_1(X)\dots,f_n(X))$, 
	\[
	g(X)-f_{m+1}(X)\not\in (f_1(X),\dots,f_n(X)),
	\]
	a contradiction to the minimality of the degree of $f_{m+1}$.  
\end{proof}

Since $R[X_1,\dots,X_n]=(R[X_1,\dots,X_{n-1}])[X_n]$, by induction 
one proves that if $R$ is a commutative noetherian ring, 
then $R[X_1,\dots,X_n]$ is noetherian. 
 
\begin{example}
	Since $\Z$ is noetherian, so is $\Z[X]$ by Hilbert's theorem. Now 
	$\Z[\sqrt{N}]$ is noetherian, as $\Z[\sqrt{N}]\simeq\Z[X]/(X^2-N)$ and quotients
	of noetherian rings are noetherian.  	
\end{example}

\begin{example}
	The ring $\Z[X,X^{-1}]$ is noetherian, as $\Z[X,X^{-1}]\simeq\Z[X,Y]/(XY-1)$. 
\end{example}
 
 
\begin{exercise}
	Prove that $R[[X]]$ is noetherian if $R$ is noetherian. 	
\end{exercise}

\begin{exercise}
	Let $f\colon R\to R$ be surjective ring homomorphism. Prove that $f$ is an isomorphism
	if $R$ is noetherian. 	
\end{exercise}



\chapter{Factorization}

\begin{definition}
\index{Integral domain}
	A commutative ring $R$ is said to be an \textbf{integral domain}
	if $xy=0$ implies $x=0$ or $y=0$.  	
\end{definition}

The rings $\Z$ and $\Z[i]$ are both integral domains. 
More generally, if $N$ is a square-free integer, 
then the ring $\Z[\sqrt{N}]$ is an integral domain.  
The ring $\Z/4$ of 
integers modulo 4 is not an integral domain. 

\begin{definition}
	Let $R$ be an integral domain and $x,y\in R$. Then $x$ \textbf{divides} $y$ 
	if $y=xz$ for some $x\in R$. 
	Notation: $x\mid y$ if and only if $x$ divides $y$. If $x$ does not
	divide $y$ one writes $x\nmid y$.  
\end{definition}

Note that $x\mid y$ if and only if $(y)\subseteq (x)$.
	
\begin{definition}
	Let $R$ be an integral domain and $x,y\in R$. Then $x$ and $y$ are
	\textbf{associate} in $R$ if $x=yu$ for some $u\in\mathcal{U}(R)$. 
\end{definition}

Note that $x$ and $y$ are associate if and only if $(x)=(y)$.  

\begin{example}
	The integers $2$ and $-2$ are associate in $\Z$.	
\end{example}

\begin{example}
	Let $R=\Z[i]$. 
	\begin{enumerate}
		\item Let $d\in\Z$ and $a+ib\in R$. Then $d\mid a+ib$ in $R$ if and only if 
			$d\mid a$ and $d\mid b$ in $\Z$. 
		\item $2$ and $-2i$ are associate in $R$.
	\end{enumerate} 	
\end{example}

\begin{example}
	Let $R=\R[X]$ and $f(X)\in R$. Then $f(X)$ and $\lambda f(X)$ are 
	associate in $R$ for all $\lambda\in\R^{\times}$. 	
\end{example}

\begin{definition}
	Let $R$ be an integral domain and $x\in R\setminus\{0\}$. Then $x$ is \textbf{irreducible} 
	if and only if $x\not\in\mathcal{U}(R)$ 
	and $x=ab$ with $a,b\in R$ implies that $a\in\mathcal{U}(R)$ or $b\in\mathcal{U}(R)$. 
\end{definition}

Note that $x$ is irreducible if and only if $(x)\ne R$ 
and there is no principal ideal $(y)$ such that 
$(x)\subsetneq (y)\subsetneq R$.

\begin{example}
	Let $R=\R[X]$ and $f(X)\in R\setminus\{0\}$. Then $f(X)$ is irreducible if 
	$\lambda\in\R^{\times}$ or $\lambda f(X)$ for $\lambda\in\R^{\times}$ 
	are the only divisors
	of $f(X)$.  
\end{example}

The irreducibles of $\Z$ are the prime numbers. 

\begin{definition}
	Let $R$ be an integral domain and $p\in R\setminus\{0\}$. Then  
	$p$ is \textbf{prime} if $p\not\in\mathcal{U}(R)$ and 
	$yz\in (p)$ implies that $y\in (p)$ or $z\in (p)$. 
\end{definition}

In $\Z$ primes and irreducible coincide. This does not happend in full generality. However,
the following result holds. 

\begin{proposition}
	Let $R$ be an integral domain and $x\in R$. 
	If $x$ is prime, then $x$ is irreducible. 
\end{proposition}

\begin{proof}
	Let $p$ be a prime. Then $p\ne 0$ and $p\not\in\mathcal{U}(R)$. Let $x$ be such that
	$x\mid p$. Then $p=xy$ for some $y\in R$. This means $xy\in (p)$, 
	so $x\in (p)$ or $y\in (p)$ because
	$p$ is prime. If $x\in (p)$, then $x=pz$ for some $z\in R$ and hence
	\[
	p=xy=(pz)y.
	\]
	Since $p-pzy=p(1-zy)$ and $R$ is an integral domain, it follows that 
	$1-zy=0$. Thus $y\in\mathcal{U}(R)$. Similarly, if $y\in (p)$, then 
	$x\in\mathcal{U}(R)$. 
\end{proof}

To show that there rings where some irreducibles are not prime, 
we need the following lemma. 

\begin{lemma}
Let $N\in\Z$ be a square-free integer and $R=\Z[\sqrt{N}]$. The map 
\[
N\colon R\to\N,
\quad a+b\sqrt{N}\mapsto 
|a^2-Nb^2|,
\]
satisfies the following properties:
\begin{enumerate}
	\item $N(\alpha)=0$ if and only if $\alpha=0$. 
	\item $N(\alpha\beta)=N(\alpha)N(\beta)$ for all $\alpha,\beta\in R$. 
	\item $\alpha\in\mathcal{U}(\Z[\sqrt{N}])$ if and only if $N(\alpha)=1$. 
	\item If $N(\alpha)$ is prime in $\Z$, then $\alpha$ is irreducible in $R$. 
\end{enumerate}	
\end{lemma}

\begin{proof}
	The first three items are left as an exercises. Let us prove 4). 
	If $\alpha=\beta\gamma$ for some $\beta,\gamma\in R$, then
	$N(\alpha)=N(\beta)N(\gamma)$. Since $N(\alpha)$ is a prime number, it follows that
	$N(\alpha)=1$ or $N()\beta)=1$. Thus $\beta\in\mathcal{U}(R)$ or $\gamma\in\mathcal{U}(R)$. 	
\end{proof}

\begin{example}
	Let $R=\Z[i]$. 
	\begin{enumerate}
		\item $\mathcal{U}(R)=\{-1,1,i,-i\}$.
		\item $3$ is irreducible in $R$. In fact, if $3=\alpha\beta$, then
			$9=N(\alpha)N(\beta)$. This implies that $N(\alpha)\in\{1,3,9\}$. Write
			$\alpha=a+bi$ for $a,b\in\Z$. If $N(\alpha)=1$, then $\alpha\in\mathcal{U}(R)$ by the lemma. 
			If $N(\alpha)=9$, then $N(\beta)=1$ and hence $\beta\in\mathcal{U}(R)$ by the lemma. Finally, 
			if $N(\alpha)=3$, then $a^2+b^2=3$, which is a contradiction since $a,b\in\Z$. 
		\item $2$ is not irreducible in $R$. In fact, $2=(1+i)(1-i)$ and
			since $N(1+i)=N(1-i)=2$, it follows that $1+i\not\in\mathcal{U}(R)$ 
			and $1-i\not\in\mathcal{U}(R)$. 
	\end{enumerate}	
\end{example}


\chapter{Zorn´s lemma}
\chapter{Algebras}

\index{Group ring}
\index{Group algebra}
We now discuss an important family of examples. 
Fix a field $K$. 
For a finite group $G$ let $K[G]$ be the vector space (over $K$)
with basis $\{g:g\in G\}$. Then $K[G]$ is a ring
with
\[
\left(\sum_{g\in G}\lambda_gg\right)\left(\sum_{h\in G}\mu_hh\right)
=\sum_{g,h\in G}\lambda_g\mu_h(gh).
\] 

Thus $K[G]$ is a ring and also a vector space (over $K$) and these structures
are somewhat compatible. Note that
\[
(\lambda a+\mu g=c=\lambda (ac)+\mu (bc),\quad
a(\lambda b+\mu c)=\lambda (ab)+\mu (ac)
\]
for all $\lambda,\mu\in K$ and $a,b,c\in K[G]$. 

\begin{definition}
\index{Algebra}
Let $A$ be a ring. Then $A$ is an algebra (over the field $K$) if $A$ is a vector space
and the map $K\to Z(A)$, $k\mapsto k1_A$, is an injective ring homomorphism.  
\end{definition}

Thus $K[G]$ is an algebra, as it is ring that contains $K$ in its center (or more precisely,
the map $K\to Z(K[G])$, $k\mapsto k1$, is an injective ring homomorphism.  
Other examples of algebras 
are the polynomial rings $K[X]$ and $K[X,Y]$ and matrix rings $M_n(K)$.  

\begin{example}
	If $A$ is an algebra, then $M_n(A)$ is an algebra.	
\end{example}

The ring $K[G]$ is commutative if and only if $G$ is abelian. Moreove,
$K[G]$ is a vector space of dimension $\dim K[G]=|G|$.

\begin{example}
	Let $G=\langle g:g^3=1\rangle=\{1,g,g^2\}\simeq C_3$ be the cyclic group of order three. 
	If $\alpha=a_11+_2g+a_3g^2$ and $\beta=b_11+b_2g+b_3g^2$, then
	\[
		\alpha\beta=(a_1b_1+a_2b_3+a_3b_2)1+(a_1b_2+a_2b_1+a_3b_3)g+a_1b_3+a_2b_2+a_3b_1)g^2.
	\]
	One can check that $\C[G]\simeq\C[X]/(X^3-1)$. 
\end{example}

In general, one proves that $\C[C_n]\simeq\C[X]/(X^n-1)$ for $n\geq2$.

\begin{example}
	Let $K$ be a field and 
	$G=\{1,g\}\simeq C_2$ be the cyclic group of order two. The product
	of $K[G]$ is 
	\[
	(a1+bg)(c1+gd)=(ac+bd)1+(ad+bc)g.
	\]
	
	If $K=\C$, then the map $K[G]\to K\times K$, $a1+bg\mapsto (a+b,a-b)$, 
	is a linear isomorphism of rings. 
	
	If $K=\Z/2$, then the map $K[G]\to\begin{pmatrix}
		K&K\\
		0&K
	\end{pmatrix}$, $a1+bg\mapsto\begin{pmatrix}
		a+b&b\\
		0&a+b		
	\end{pmatrix}$, is a linear isomorphism of rings. 
\end{example}

\begin{exercise}
	Prove that $\R[C_3]\simeq\R\times\C$. 	
\end{exercise}

The group ring has the following property, which is left as an exercise. 
zzZLet $R$ be a ring and
$G$ be a finite group. If $f\colon G\to\mathcal{U}(R)$ is a group homomorphism, 
then there exists a unique ring homomorphism $\varphi\colon K[G]\to R$ such that
$\varphi|_G=f$. 

\begin{example}
	Let $\D_3=\langle r,s:r^3=s^2=1,\,srs^{-1}=r^{-1}\rangle$ be the dihedral
	group of six elements. We claim that 
	\[
	\C[\D_3]\simeq\C\times\C\times M_2(\C).
	\]
	Let $\omega$ be a primitive root of one. Let 
	\[
	R=\begin{pmatrix}
		\omega&0\\
		0&\omega^2	
	\end{pmatrix},
	\quad
	S=\begin{pmatrix}
		0&1\\
		1&0
	\end{pmatrix}.
 	\]
 	One easily checks that $SRS^{-1}=R^{-1}$ and $R^3=S^2=\begin{pmatrix}
		1&0\\
		0&1	
	\end{pmatrix}$. It follows that there exists a group homomorphism
	$G\to\C\times\C\times M_2(\C)$ such that
	$r\mapsto (1,1,R)$ and $s\mapsto (1,-1,S)$. This group homomorphism
	is a ring isomorphism.  
\end{example}




 


\part{Representations}

\chapter{Group representations}

\begin{definition}
	\index{Representation}
	A \textbf{representation} (over the field $K$) of a group $G$ is a group homomorphism
	$\rho\colon G\to\GL(V)$, $g\mapsto\rho_g$, for some vector space $V$ (over $K$).
\end{definition}

The \textbf{degree} of $V$ will be the dimension of $V$. Note that
if $\dim V=n$, fixing a basis of $V$ we get a \textbf{matrix representation} 
$\rho\colon G\to\GL(V)\simeq\GL_n(V)$ of $G$.  

\begin{example}
	We use group representations to show that 
	$G=\langle x,y:x^2=y^2=1\rangle$ is infinite. Note that
	\[
	\rho\colon G\to\GL_2(\C),
	\quad
	x\mapsto\begin{pmatrix}
		1&0\\
		0&-1	
	\end{pmatrix},\quad
 	y\mapsto\begin{pmatrix}
		1&1\\
		0&-1	
	\end{pmatrix}
 	\]
 	is a group homomorphism, as 
 	$\rho_x^2=\rho_y^2=\begin{pmatrix}
		1&0\\
		0&1	
	\end{pmatrix}$. We claim that the elements of the form $(xy)^n$ are
	different for all $n$. It is enough to show that   
	$(xy)^n=(xy)^m$, then $n=m$. Note that
	\[
	\rho_{xy}=\rho_x\rho_y=\begin{pmatrix}
		1&0\\
		0&-1	
	\end{pmatrix}
	\begin{pmatrix}
		1&0\\
		0&-1	
	\end{pmatrix}
	=\begin{pmatrix}
		1&1\\
		0&1	
	\end{pmatrix}.
	\]
	Thus $\rho_{xy}^n=\begin{pmatrix}
		1&n\\
		0&1	
	\end{pmatrix}$ and 
	$\rho_{xy}^m=\begin{pmatrix}
		1&m\\
		0&1	
	\end{pmatrix}$. From this the claim follows.
\end{example}

The previous example shows the power of group representations, even for infinite groups.   
However, in this course we will work with complex finite-dimensional 
representations of finite
groups.  

\begin{example}
	If $G$ is a group, then $\rho\colon G\to\C^\times$, $g\mapsto 1$, is a representation. This representation is known as the \textbf{trivial representation} of $G$. 	
\end{example}

\begin{example}
	The sign yields a representation of $\Sym_n$. It is the group homomorphism
	$\Sym_n\to\C^\times$, $\sigma\mapsto\sgn(\sigma)$.  	
\end{example}


\begin{example}
	Let $G=\langle g:g^6=1\rangle$ be the cyclic group of order six. Then
	\[
	\rho\colon G\to\GL_2(\C),
	\quad 
	g\mapsto\begin{pmatrix}1&1\\-1&0\end{pmatrix}
	\] 
	is a group representation. 
\end{example}

\begin{proposition}
    Let $G$ be a finite group and $\rho\colon G\to\GL(V)$ be a representation. Then
    each $\rho_g$ is diagonalizable.
\end{proposition}

\begin{proof}
    Let $n=\dim V$. Fix a basis of the finite-dimensional vector space $V$ and consider
    a matrix representation $\rho\colon G\to\GL_n(V)$.
    Since $g$ is finite, $g^m=1$ for some $m\in\N$. This means that $\rho_g$ is a root of $X^m-1\in\C[X]$. Since
    the roots of the polynomial $X^m-1$ are all different and $X^m-1$ factorizes linearly on $\C[X]$, it follows
    that the minimal polynomial of $\rho_g$ also factorizes linearly in $\C[X]$. Hence $\rho_g$ is diagonalizable.
\end{proof}

\begin{definition}
	\index{Invariant map}
    Let $\rho\colon G\to\GL(V)$ and $\psi\colon G\to\GL(W)$ be
    representations of a finite group $G$. A linear map $T\colon V\to W$ is said to be invariant
    if the diagram
    \[\begin{tikzcd}
	V & V \\
	W & W
	\arrow["{\rho_g}", from=1-1, to=1-2]
	\arrow["T"', from=1-1, to=2-1]
	\arrow["{\psi_g}"', from=2-1, to=2-2]
	\arrow["T", from=1-2, to=2-2]
\end{tikzcd}\]
    commutes, i.e. $\psi_gT=T\rho_g$ for all $g\in G$.
\end{definition}

\begin{definition}
	\index{Representations!equivalence}
    The representations $\rho\colon G\to\GL(V)$ and $\psi\colon G\to\GL(W)$ are \textbf{equivalent}
    if there exists a bijective invariant map $T\colon V\to W$.
\end{definition}

\begin{example}
    Let $G=\Z/n$. The representations
    \begin{align*}
    &\rho\colon G\to\GL_2(\C),\quad
    m\mapsto
    \begin{pmatrix}
        \cos(2\pi m/n) & -\sin(2\pi m/n)\\
        \sin(2\pi m/n) & \cos(2\pi m/n)
    \end{pmatrix}
    \shortintertext{and}
    &\psi\colon G\to\GL_2(\C),\quad
    m\mapsto
    \begin{pmatrix}
        e^{2\pi im/n} & 0\\
        0 & e^{-2\pi im/n}
    \end{pmatrix}
    \end{align*}
    are equivalent, as $\rho_mT=T\psi_m$ for all $m\in G$ if $T=\begin{pmatrix}
        i&-i\\
        1&1
    \end{pmatrix}$.
\end{example}

\begin{definition}
	\index{Invariant subspace}
    Let $\rho\colon G\to\GL(V)$
    a representation. A subspace $W$ of $V$ is said to be \textbf{invariant} (with respect to $\rho$)
    if $\rho_g(W)\subseteq W$ for all $g\in G$.
\end{definition}

If $\rho\colon G\to\GL(V)$ is a representation and $W\subseteq V$ is invariant, then
the map $\rho|_W\colon G\to\GL(W)$, $g\mapsto (\rho_g)|_W$, is a representation. The
map $\rho|_W$ is the \textbf{restriction} of $\rho$ to $W$. 

\begin{example}
	If $\rho\colon G\to\GL(V)$ and $\psi\colon G\to\GL(W)$ are representations
	and $T\colon V\to W$ is an invariant map, then the \textbf{kernel} 
	\[
	\ker T=\{v\in V:T(v)=0\}
	\]
	is an invariant of $V$ and the \textbf{image} 
	\[
		T(V)=\{T(v):v\in V\}
	\]
	is an invariant subspace of $W$. 
\end{example}


\begin{definition}
\index{Subrepresentation}
    Let $\rho\colon G\to\GL(V)$
    a representation. A \textbf{subrepresentation} of $\rho$ is a restricted representation 
    of the form $\rho|_W\colon G\to\GL(W)$ for some invariant subspace $W$ of $V$.
\end{definition}

\begin{example}
    Let $G=\langle g:g^3=1\rangle$ be the
    cyclic group of order three
    and
    \[
    \rho\colon G\to\GL_3(\R),
    \quad
    g\mapsto\begin{pmatrix}
        0&1&0\\
        0&0&1\\
        1&0&0
    \end{pmatrix}.
    \]
    The subspace
    \[
    W=\left\{
    \begin{pmatrix}
    x\\
    y\\
    z
    \end{pmatrix}\in\R^3:x+y+z=0\right\}
    \]
    is a invariant subspace of $\R^3$.
\end{example}

\begin{definition}
    \index{Representation!irreducible}
    A representation $\rho\colon G\to\GL(V)$ is \textbf{irreducible} if
    $\{0\}$ and $V$ are the only invariant subspaces of $V$.
\end{definition}

Degree-one representations are irreducible.

\begin{example}
        Let $G=\langle g:g^3=1\rangle$ be the
    cyclic group of order three
    and
    \[
    \rho\colon G\to\GL_3(\R),
    \quad
    g\mapsto\begin{pmatrix}
        0&1&0\\
        0&0&1\\
        1&0&0
    \end{pmatrix}.
    \]
    We claim that the invariant subspace
    \[
    W=\left\{
    \begin{pmatrix}
    x\\
    y\\
    z
    \end{pmatrix}\in\R^3:x+y+z=0\right\}\subseteq\R^3
    \]
    is irreducible. Let $S$ be a non-zero invariant subspace of $W$ and let $s=\begin{pmatrix}x_0\\y_0\\z_0\end{pmatrix}\in S$ be a non-zero element. Then
    \[
    t=\begin{pmatrix}y_0\\z_0\\x_0\end{pmatrix}
    =\begin{pmatrix}
        0&1&0\\
        0&0&1\\
        1&0&0
    \end{pmatrix}
    \begin{pmatrix}x_0\\y_0\\z_0\end{pmatrix}\in S.
    \]
    We claim that $\{s,t\}$ are linearly independent. If not, there exists $\lambda\in\R$ such that
    $\lambda s=t$. Thus $\lambda x_0=y_0$, $\lambda y_0=z_0$ and $\lambda z_0=x_0$. This implies that
    $\lambda^3x_0=x_0$. Since $x_0\ne 0$ (because if $x_0=0$, then $y_0=z_0=0$, a contradiction), it follows that
    $\lambda=1$ and hence $x_0=y_0=z_0$, a contradiction because $x_0+y_0+z_0=0$.
    Therefore $\dim S=2$ and hence $S=W$.
\end{example}

\begin{exercise}
    Let $\rho\colon G\to\GL(V)$ be a degree-two representation. Prove that
    $\rho$ is irreducible if and only if there is no common eigenvector for the $\rho_g$, $g\in G$.
\end{exercise}

The previous exercise can be used to show that the representation
$\Sym_3\to\GL_2(\C)$
of the symmetric group $\Sym_3$
given by
\[
(12)\mapsto\begin{pmatrix}
-1&1\\0&1
\end{pmatrix},
\quad
(123)\mapsto\begin{pmatrix}
1&0\\
1&-1
\end{pmatrix}
\]
is irreducible.

\begin{example}
Let $\rho\colon G\to\GL(V)$ and $\psi\colon G\to\GL(W)$ be representations. One defines
the \textbf{direct sum} $\rho\oplus\psi$ of $\rho$ and $\psi$ as 
\[
\rho\oplus\psi\colon G\to\GL(V\oplus W),\quad
g\mapsto (\rho\oplus\psi)_g,
\]
where $(\rho\oplus\psi)_g\colon V\oplus W\to V\oplus W$ is given by 
$(v,w)\mapsto (\rho_g(v),\psi_g(w))$.
\end{example}

\begin{definition}
    \index{Representation!completely irreducible}
    A representation $\rho\colon G\to\GL(V)$ is \textbf{completely irreducible}
    if $V$ can be decomposed as
    $V=V_1\oplus\cdots\oplus V_n$, where each $V_i$ is a invariant subspace of $V$ and
    each $\rho|_{V_i}$ is irreducible.
\end{definition}

Since we are considering finite-dimensional vector spaces, our vector spaces are
Hilbert spaces, so they have
an inner product $V\times V\to\C$, $(v,w)\mapsto\langle v,w\rangle$.

\begin{definition}
    \index{Representation!unitary}
    A representation $\rho\colon G\to\GL(V)$ is \textbf{unitary} if
    $\langle \rho_gv,\rho_gw\rangle=\langle v,w\rangle$ for all $g\in G$ and $v,w\in V$.
\end{definition}

\begin{definition}
\index{Representation!decomposable}
\index{Representation!indecomposable}
A representation
$\rho\colon G\to\GL(V)$ is \textbf{decomposable} if $V$ can be decomposed as $V=S\otimes T$
where $S$ and $T$ are non-zero invariant subspaces of $V$. 
\end{definition}

A representation is 
\textbf{indecomposable} if it is not decomposable. 

\begin{exercise}
Let $\rho\colon G\to\GL(V)$ be a unitary representation. Prove that $\rho$ is either
irreducible or decomposable.
\end{exercise}

\begin{example}
Let $G$ be a finite group and $V=\C[G]$. The \textbf{left regular representation}
of $G$ is the representation
\[
L\colon G\to\GL(V),
\quad
g\mapsto L_g,
\]
where $L_g(h)=gh$. With the inner product
\[
\left\langle\sum_{g\in G}\lambda_gg,\sum_{g\in G}\mu_gg\right\rangle=\sum_{g\in G}\lambda_g\overline{\mu_g}
\]
the representation $L$ is unitary.
\end{example}

\begin{proposition}[Weyl's trick]
\index{Weyl's trick}
    Every representation of a finite group is equivalent to a unitary representation.
\end{proposition}

\begin{proof}
    Let $\rho\colon G\to\GL(V)$ and $V\times V\to\C$, $(v,w)\mapsto\langle v,w\rangle_0$ be an inner
    product on $V$. A straighforward calculation shows that
    \[
    \langle v,w\rangle=\sum_{g\in G}\langle\rho_gv,\rho_gw\rangle_0
    \]
    is an inner product of $V$. Since
    \begin{align*}
    \langle\rho_gv,\rho_gw\rangle&=\sum_{h\in G}\langle\rho_h\rho_gv,\rho_h\rho_gw\rangle_0\\
    &=\sum_{h\in G}\langle\rho_{hg}v,\rho_{hg}w\rangle_0=\sum_{x\in G}\langle\rho_xv,\rho_xw\rangle_0=\langle v,w\rangle,
    \end{align*}
    the representation $\rho$ is unitary.
\end{proof}

Weyl's trick has several interesting corollaries. Let $\rho\colon G\to\GL(V)$ be a representation
of a finite group $G$. Then 1) every non-zero representation is either
irreducible or decomposable, and 2) every $\rho_g$ is diagonalizable
(as unitary operators are diagonalizable).

\begin{exercise}
    If $G$ is an infinite group it is not longer true that every non-zero representation
    is either irreducible or decomposable. Find an example.
\end{exercise}

Recall that we only consider finite-dimensional representations of finite groups.

\begin{theorem}[Maschke]
\index{Maschke theorem}
    Every representation of a finite group is completely reducible.
\end{theorem}

\begin{proof}
    Let $G$ be a finite group and $\rho\colon G\to\GL(V)$ be a representation of $G$. We proceed
    by induction on $\dim V$.
    If $\dim V=1$, the result is trivial, as degree-one representations are irreducible. Assume that
    the result holds for representations of degree $\leq n$. Let $\rho\colon G\to\GL(V)$ be a representation
    of degree $n+1$. If $\rho$ is irreducible, we are done. If not, write $V=S\otimes T$, where $S$ and $T$
    are non-zero invariant subspaces. Since $\dim S<\dim V$ and $\dim T<\dim V$, it follows from
    the inductive hypothesis that
    both $S$ and $T$ are completely irreducible. Thus $V$ is completely irreducible.
\end{proof}

\begin{example}
    Let $G=\Sym_3$ and $\rho\colon G\to\GL_3(\C)$ be the representation given by
    \[
    (12)\mapsto\begin{pmatrix}
    0&1&0\\
    1&0&0\\
    0&0&1
    \end{pmatrix},\quad
    (123)\mapsto\begin{pmatrix}
    0&0&1\\
    1&0&0\\
    0&1&0
    \end{pmatrix}
    \]
    Then $\rho_g$ is unitary for all $g\in G$ (because $\rho_{(12)}$ and $\rho_{(123)}$ are both
    unitary). Moreover,
    \[
    S=\left\langle \begin{pmatrix}
    1\\1\\1
    \end{pmatrix}
    \right\rangle,
    \quad
    T=S^{\perp}=\left\langle
    \begin{pmatrix}
    -1\\1\\0
    \end{pmatrix},
    \begin{pmatrix}
    0\\-1\\1
    \end{pmatrix}
    \right\rangle,
    \]
    are irreducible invariant subspaces of $V=\C^3$. A direct calculation shows that
    in the orthogonal basis $\left\{\begin{pmatrix}
    1\\1\\1
    \end{pmatrix},
    \begin{pmatrix}
    -1\\1\\0
    \end{pmatrix},
    \begin{pmatrix}
    0\\-1\\1
    \end{pmatrix}
    \right\}$
    the matrices $\rho_{(12)}$ and $\rho_{(123)}$ can be written as
    \[
    \rho_{(12)}=\begin{pmatrix}
        1&0&0\\
        0&-1&1\\
        0&0&1
    \end{pmatrix},
    \quad
    \rho_{(123)}=
    \begin{pmatrix}
        1&0&0\\
        0&0&-1\\
        0&1&-1
    \end{pmatrix}.
    \]
\end{example}

\begin{exercise}
Let $G$ be a finite group.
Prove that there is a bijection between degree-one representations of $G$ and
degree-one representations of $G/[G,G]$.
\end{exercise}



\chapter{Characters}

\begin{definition}
	\index{Character}
	Let $\rho\colon G\to\GL(V)$ be a representation. The \textbf{character} of $\rho$ 
	is the map $\chi_\rho\colon G\to\C$, $g\mapsto\trace\rho_g$. 	
\end{definition}

If a representation $\rho$ is irreducible, its character is said to be an 
\textbf{irreducible character}. The \textbf{degree} of a character is the degree of the affording
representation. 

\begin{proposition}
	Let $\rho\colon G\to\GL(V)$ be a representation, $\chi$ be its character and $g\in G$.
	The following statements hold:
	\begin{enumerate}
		\item $\chi(1)=\dim V$. 
		\item $\chi(g)=\chi(hgh^{-1})$ for all $h\in G$.
		\item $\chi(g)$ is the sum of $\chi(1)$ roots of one of order $|g|$. 
		\item $\chi(g^{-1})=\overline{\chi(g)}$. 
		\item $|\chi(g)|\leq\chi(1)$.  
	\end{enumerate} 
\end{proposition}

\begin{proof}
	The first statement is trivial. 	To prove 2) note that
	\[
	\chi(hgh^{-1})=\trace(\rho_{hgh^{-1}})=\trace(\rho_h\rho_g\rho_h^{-1})=\trace\rho_g=\chi(g).
	\]
	Statement 3) follows from the fact that the trace of $\rho_g$ is the sum
	of the eigenvalues of $\rho_g$ and these numbers are roots of the polynomial
	$X^{|g|}-1\in\C[X]$. To prove 4) write $\chi(g)=\lambda_1+\cdots+\lambda_k$, where 
	the $\lambda_j$ are roots of one. Then
	\[
	\overline{\chi(g)}=\sum^k_{j=1}\overline{\lambda_j}
	=\sum_{j=1}^k\lambda_j^{-1}
	=\trace(\rho_g^{-1})
	=\trace(\rho_{g^{-1}})
	=\chi(g^{-1}).
	\] 
	Finally, we prove 5). Use 3) to write $\chi(g)$ as the sum of
	$\chi(1)$ roots of one, say $\chi(g)=\lambda_1+\cdots+\lambda_k$ for
	$k=\chi(1)$. Then 
	\[
	|\chi(g)|=|\lambda_1+\cdots+\lambda_k|\leq |\lambda_1|+\cdots+|\lambda_k|
	=\underbrace{1+\cdots+1}_{\text{$k$-times}}=k.
	\]
\end{proof}

If two representations are equivalent, their characters are equal.

\begin{definition}
	Let $G$ be a group and 
	$f\colon G\to\C$ be a map. Then $f$ is a \textbf{class function} if
	$f(g)=f(hgh^{-1})$ for all $g,h\in G$. 	
\end{definition}

Characters are class functions. 

\begin{proposition}
    If $\rho\colon G\to\GL(V)$ and
    $\psi\colon G\to\GL(W)$ are representations, then
    $\chi_{\rho\oplus\psi}=\chi_\rho+\chi_\psi$.
\end{proposition}

\begin{proof}
  For $g\in G$, it follows that 
  $(\rho\oplus\psi)_g=
  \begin{pmatrix}
    \rho_g & 0\\ 
    0 & \psi_g
  \end{pmatrix}$. 
  Thus  
  \[
    \chi_{\rho\oplus\psi}(g)=\trace((\rho\oplus\phi)_g)=\trace(\rho_g)+\trace(\psi_g)=\chi_\rho(g)+\chi_\psi(g).\qedhere
  \]
\end{proof}

Let $V$ be a vector space with basis $\{v_1,\dots,v_k\}$ and 
$W$ be a vector space with basis $\{w_1,\dots,w_l\}$. A 
\textbf{tensor product} of $V$ and $W$ is a vector space $X$ with 
together with a bilinear map 
\[
V\times W\to X,
\quad
(v,w)\mapsto v\otimes w,
\]
such that $\{v_i\otimes w_j:1\leq i\leq k,\,1\leq j\leq l\}$ is a basis 
basis of $X$. The tensor product of $V$ and $W$ is unique up to isomorphism 
and it is denoted by $V\otimes W$. Note that
\[
\dim(V\otimes W)=(\dim V)(\dim W).
\]

\begin{definition}
	Let $\rho\colon G\to\GL(V)$ and $\psi\colon G\to\GL(W)$ be representations. The \textbf{tensor product} of $\rho$ and $\psi$ is the representation of $G$ given by 
	\begin{gather*}
	\rho\otimes\psi\colon G\to\GL(V\otimes W),
	\quad 
	g\mapsto (\rho\otimes\psi)_g,
	\shortintertext{where}
	(\rho\otimes\psi)_g(v\otimes w)=\rho_g(v)\otimes \psi_g(w)
	\end{gather*}
	for $v\in V$ and $w\in W$.  	
\end{definition}

A direct calculation shows that the tensor product of representations is indeed a representation. 

\begin{proposition}
  	If $\rho\colon G\to\GL(V)$ and
    $\psi\colon G\to\GL(W)$ are representations, then
    \[
    \chi_{\rho\otimes\psi}=\chi_\rho\chi_\psi.
    \]
    %	\item $\chi_{V^*}=\overline{\chi_V}$.
\end{proposition}

\begin{proof}
	For each $g\in G$ the map $\rho_g$ is diagonalizable. Let $\{v_1,\dots,v_n\}$
	be a basis of eigenvectors of $\phi_g$ and let $\lambda_1,\dots,\lambda_n\in\C$ be such that
	$\rho_g(v_i)=\lambda_iv_i$ for all $i\in\{1,\dots,n\}$. Similarly, 
	let $\{w_1,\dots,w_m\}$ be a basis of eigenvectors of $\psi_g$ and $\mu_1,\dots,\mu_m\in\C$ be such that $\psi_g(w_j)=\mu_jw_j$ for all $j\in\{1,\dots,m\}$. Each 
	$v_i\otimes w_j$ is eigenvector of $\phi\otimes\psi$ with eigenvalue 
	$\lambda_i\mu_j$, as  
	\[
		(\rho\otimes\psi)_g(v_i\otimes w_j)=\rho_gv_i\otimes \psi_gw_j=\lambda_iv_i\otimes \mu_jv_j=(\lambda_i\mu_j)v_i\otimes w_j.
	\]
	Thus  
	$\{v_i\otimes w_j:1\leq i\leq n,1\leq j\leq m\}$ is a basis of eigenvectors and the 
	$\lambda_i\mu_j$ are the eigenvalues of $(\phi\otimes\psi)_g$. It follows that 
	\[
	\chi_{\rho\otimes\psi}(g)
	=\sum_{i,j}\lambda_i\mu_j
	=\left(\sum_i\lambda_i\right)\left(\sum_j\mu_j\right)
	=\chi_\rho(g)\chi_\psi(g).\qedhere 
	\]
\end{proof}

For completeness we mention without proof that
it is also possible to define the dual $\rho^*\colon G\to\GL(V^*)$  
of a representation
$\rho\colon G\to\GL(V)$ by the formula
\[
(\rho^*_gf)(v)=f(\rho^{-1}_gv),\quad
g\in G,\,f\in V^*\text{ and }v\in V.
\]  
We claim that the character of the dual representation is then 
$\overline{\chi_\rho}$. Let $\{v_1,\dots,v_n\}$ be a basis of $V$
and $\lambda_1,\dots,\lambda_n\in\C$ be such that $\rho_gv_i=\lambda_iv_i$ for all $i\in\{1,\dots,n\}$. If $\{f_1,\dots,f_n\}$ is the dual basis of $\{v_1,\dots,v_n\}$, then 
\[
(\rho^*_gf_i)(v_j)=f_i(\rho_g^{-1}v_j)
=\overline{\lambda_j}f_i(v_j)
=\overline{\lambda_j}\delta_{ij}
\]
and the claim follows. 
%\begin{proposition}
%	
%\end{proposition}
%
%\begin{proof}
%	Let $g\in G$ and $\{v_1,\dots,v_n\}$
%	be a basis of eigenvectors of $\rho_g$. Let
%	$\lambda_1,\dots,\lambda_n\in\C$ be such that $\rho_g(v_i)=\lambda_iv_i$ for all
%	$i\in\{1,\dots,n\}$. Let $\{f_1,\dots,f_n\}$ be the dual basis of $\{v_1,\dots,v_n\}$.
%	Since $\rho_g$ is invertible, each eigenvector of $\rho_g$ is non-zero. 
%	Thus $\rho_g(v_i)=\lambda_iv_i$ implies that 
%	$\rho_{g^{-1}}v_i=\lambda_i^{-1}v_i=\overline{\lambda_i}v_i$... 
%%	Now 
%%	\[
%		(\rho_g f_i)(v_j)=f_i(g^{-1}v_j)=\overline{\lambda_j}f_i(v_j)=\overline{\lambda_j}\delta_{ij}.
%	\]
%	
%	  
%	
%	 We claim
%	that $\{f_1,\dots,f_n\}$ is a basis of eigenvectors...
%	with $\overline{\lambda_1},\dots,\overline{\lambda_n}$. En efecto, si $gv_j=\lambda_jv_j$, entonces
%	$g^{-1}v_j=\lambda_j^{-1}v_j=\overline{\lambda_j}v_j$ (observemos que como $\phi_g$ es inversible, los $\lambda_j$ son no nulos). Luego
%	\[
%		(gf_i)(v_j)=f_i(g^{-1}v_j)=\overline{\lambda_j}f_i(v_j)=\overline{\lambda_j}\delta_{ij}.
%	\]
%	En conclusión
%	\[
%		\chi_{V^*}(g)=\sum_{i=1}^n\overline{\lambda_i}=\overline{\chi_V(g)}.\qedhere
%	\]	
%\end{proof}


\chapter{Schur's orthogonality relations}

Let $\rho\colon G\to\GL(V)$ and $\psi\colon G\to\GL(W)$ be representations of a finite group
$G$. Since $V$ and $W$ are vector spaces, the set 
\[
\Hom(V,W)=\{T\colon V\to W:\text{$T$ is linear}\}
\]
is a vector space with 
\begin{align*}
&(\lambda T)(v)=\lambda T(v) && \text{for all $\lambda\in\C$ and all $v\in V$,}\\ 
&(T+T_1)(v)=T(v)+T_1(v) &&\text{for all $v\in V$.}
\end{align*}
We claim that the set $\Hom_G(V,W)$ of invariant maps
is a subspace of $\Hom(V,W)$. Indeed, the zero map is clearly invariant. If $T,T_1\in\Hom_G(V,W)$ 
and $\lambda\in\C$, then
\[
(T+\lambda T_1)(\rho_g v)
=T(\rho_gv)+\lambda T_1(\rho_gv)
=\psi_gT(v)+\lambda \psi_gT_1(v)
=\psi_g((T+\lambda T_1)(v))
\]
for all $v\in V$. 
 
\begin{proposition}
	Let $\rho\colon G\to\GL(V)$ and $\psi\colon G\to\GL(W)$ be representations
	and $T\colon V\to W$ be a linear map. Then
	\[
	T^{\#}=\frac{1}{|G|}\sum_{g\in G}\psi_{g^{-1}}T\rho_g\in\Hom_G(V,W).
	\]
	Moreover, the map $\Hom(V,W)\to\Hom_G(V,W)$, $T\mapsto T^{\#}$, is linear and surjective.  
\end{proposition}

\begin{proof}
  Let $h\in G$ and $v\in V$. Then 
  \begin{align*}
	T^{\#}\phi_h(v)
	&=\frac{1}{|G|}\sum_{g\in G}\psi_{g^{-1}}T\phi_g\phi_h(v)
	=\frac{1}{|G|}\sum_{g\in G}\psi_{g^{-1}}T\phi_{gh}(v)\\
	&=\frac{1}{|G|}\sum_{x\in G}\psi_{hx^{-1}}T\phi_x(v)
	=\frac{1}{|G|}\sum_{x\in G}\psi_h\psi_{x^{-1}}T\phi_x(v)
	=\psi_hT^{\#}(v).
      \end{align*}

	  If $T\in\Hom_{\C[G]}(V,V)$, then 
      \[
	T^{\#}(v)=\frac{1}{|G|}\sum_{g\in G}\psi_{g^{-1}}T\phi_g(v)
	=\frac{1}{|G|}\sum_{g\in G}\psi_{g^{-1}}\psi_gT(v)
	=T(v)
      \]
      for all $v\in V$.
\end{proof}

If $\phi\colon G\to\GL(V)$ and $\psi\colon
	G\to\GL(W)$ are non-equivalent irreducible representations and $T\colon
	V\to W$ is a linear map, then $T^{\#}=0$, as 
	$T^{\#}\in\Hom_{\C[G]}(V,W)=\{0\}$ by the previous proposition and Schur's lemma.

\begin{theorem}[ergodic theorem]
  Let $\rho\colon G\to\GL(V)$ be an irreducible representation. 
  If $T\colon V\to V$ is linear, then 
  $T^{\#}=(\deg\phi)^{-1}\trace(T)\id$.
\end{theorem}

\begin{proof}
  The previous proposition and Schur's lemma imply that
  $T^{\#}=\lambda\id$ for some $\lambda\in\C$.
  We now compute the trace of $T^{\#}$. On the one hand, 
  \[
	\trace(T^{\#})=\trace(\lambda\id)=(\dim V)\lambda.
  \]
  On the other hand.  
  \[
	\trace(T^{\#})
	=\frac{1}{|G|}\sum_{g\in G}\trace(\rho_{g^{-1}}T\rho_g)
	=\frac{1}{|G|}\sum_{g\in G}\trace(T)
	=\trace(T),
  \]
  as $\trace(ABA^{-1})=\trace(B)$ for all $A$ and $B$. 
  Hence 
  \[
  \trace(T^{\#})=(\dim V)^{-1}\trace(T)\id.\qedhere 
  \]
\end{proof}

We now prove Schur's orthogonality relations. We need some preliminary material. First recall that 
the matrix $E_{ij}$ is given by 
\[
(E_{ij})_{kl}=\delta_{ik}\delta_{jl},
\qquad
\delta_{xy}=\begin{cases}
    1 & \text{if $x=y$},\\
    0 & \text{otherwise}.
\end{cases}
\]
If $\rho\colon G\to\GL(V)\simeq\GL_n(\C)$ is a representation, then
$\rho_g$ is the matrix $(\rho_{ij}(g))$ and hence the character of $\rho$ is given by
\[
\chi_\rho(g)=\sum_{i=1}^n\rho_{ii}(g).
\]

\begin{lemma}
Let $\rho\colon G\to\GL(V)$ and $\psi\colon G\to\GL(W)$ be irreducible representations. Then
$(E_{ik}^\#)_{lj}=\langle\rho_{ij},\psi_{kl}\rangle$.
\end{lemma}

\begin{proof}
  We compute
  \begin{align*}
      (E_{ki}^\#)_{lj} &= \frac{1}{|G|}\sum_{g\in G}(\psi_{g^{-1}}E_{ki}\rho_g)_{lj}\\
      &=\frac{1}{|G|}\sum_{g\in G}\sum_{p,q}\psi_{lq}(g^{-1})(E_{ki})_{qp}(\rho_{ij}(g))_{pj}\\
      &=\frac{1}{|G|}\sum_{g\in G}\psi_{lk}(g^{-1})\rho_{ij}(g)\\
      &=\frac{1}{|G|}\sum_{g\in G}\overline{\psi_{kl}(g)}\rho_{ij}(g)=\langle \rho_{ij},\psi_{kl}\rangle.\qedhere
  \end{align*}
\end{proof}

\begin{theorem}[Schur]
    Let $\rho\colon G\to\GL(V)$ and $\psi\colon G\to\GL(W)$ be irreducible representations. 
    Then the following statements hold:
    \begin{enumerate}
        \item $\langle\rho_{ij},\psi_{kl}\rangle=0$ if $\rho$ and $\psi$ are not equivalent.
        \item $\displaystyle{\langle\rho_{ij},\rho_{kl}\rangle=\frac{1}{\dim V}\delta_{ik}\delta_{jl}}$.
    \end{enumerate}
\end{theorem}

\begin{proof}
    Let us prove the first claim. Since 
    $\rho$ and $\psi$ 
    are not equivalent, it follows from Schur's lemma that $\Hom_G(V,W)=\{0\}$.
    Thus $E_{ki}^\#\in\Hom_G(V,W)=\{0\}$ by the Ergodic theorem. 
    
    To prove the second claim, we use the previous lemma:
    \[
    (E_{ki}^\#)_{lj}=\langle\rho_{ij},\psi_{kl}\rangle
    =\frac{1}{\dim V}(\trace E_{ki})\delta_{lj}
    =\frac{1}{\dim V}\delta_{ki}\delta_{lj}.\qedhere
    \]
\end{proof}


\chapter{Some examples}

Let $G$ be a finite group and $\chi_1,\dots,\chi_r$ be the irreducible characters of $G$. Without loss of generality
we may assume that $\chi_1$ is the trivial character, i.e. $\chi_1(g)=1$ for all $g\in G$. 
Recall that $r$ is the number of conjugacy classes of $G$. Each $\chi_j$ is constant on conjugacy classes. 
The \textbf{character table} of 
$G$ is given by 
\begin{center}
\begin{tabular}{|c|cccc|}
\hline 
 & $1$ & $k_{2}$ & $\cdots$ & $k_{r}$\tabularnewline
 & $1$ & $g_{2}$ & $\cdots$ & $g_{r}$\tabularnewline
\hline 
$\chi_{1}$ & $1$ & $1$ & $\cdots$ & $1$\tabularnewline
$\chi_{2}$ & $n_{2}$ & $\chi_{2}(g_{2})$ & $\cdots$ & $\chi_{2}(g_{r})$\tabularnewline
$\vdots$ & $\vdots$ & $\vdots$ & $\ddots$ & $\vdots$\tabularnewline
$\chi_{r}$ & $n_{r}$ & $\chi_{r}(g_{2})$ & $\cdots$ & $\chi_{r}(g_{r})$\tabularnewline
\hline
\end{tabular}
\end{center}
where the $n_j$ are the degrees of the irreducible representations of $G$ and each $k_j$ is 
the size of the conjugacy class of the element $g_j$. By convention, the character table
contains not only the values of the irreducible characters of the group. 

\begin{example}
	Sea $G=\langle g:g^4=1\rangle$ 
	be the cyclic group of order four. The character table of $G$ is given by
	\begin{center}
		\begin{tabular}{|c|cccc|}
			\hline 
			& 1 & 1 & 1 & 1\tabularnewline
			& $1$ & $g$ & $g^2$ & $g^{3}$\tabularnewline
			\hline 
			$\chi_{1}$ & $1$ & $1$ & $1$ & $1$\tabularnewline
			$\chi_{2}$ & $1$ & $\lambda$ & $\lambda^2$ & $\lambda^{3}$\tabularnewline
			$\chi_{3}$ & $1$ & $\lambda^2$ & $\lambda^4$ & $\lambda^{2}$\tabularnewline
			$\chi_{4}$ & $1$ & $\lambda^{3}$ & $\lambda^{2}$ & $\lambda$\tabularnewline
			\hline
		\end{tabular}
	\end{center}
% Let us see how to see this calculation in the computer:
% \begin{lstlisting}
% gap> C4 := CyclicGroup(4);;                       
% gap> T := CharacterTable(C4);;
% gap> Display(T);
% CT1

%      2  2  2  2  2

%       1a 4a 2a 4b

% X.1     1  1  1  1
% X.2     1 -1  1 -1
% X.3     1  A -1 -A
% X.4     1 -A -1  A

% A = E(4)
%   = Sqrt(-1) = i
% \end{lstlisting}
% We need some remarks
% \begin{enumerate}
%     \item The symbol \lstinline{E(4)} denotes a primitive fourth root of 1.
%     \item The function \lstinline{CharacterTable} computes some more information, not only the character table of the group. 
%     esta función calcula algunas otras cosas. Por ejemplo:
% \end{enumerate}
% \begin{lstlisting}
% gap> OrdersClassRepresentatives(T);
% [ 1, 4, 2, 4 ]
% gap> SizesCentralizers(T);
% [ 4, 4, 4, 4 ]
% gap> SizesConjugacyClasses(T);
% [ 1, 1, 1, 1 ]
% \end{lstlisting}
\end{example}

\begin{exercise}
Let $n\in\N$ be such that $n\geq2$. Let 
$C_n=\langle g:g^n=1\rangle$ be the cyclic group of order $n$.
\begin{enumerate}
    \item Prove that the maps  
        $\chi_i\colon C_n\to\C^\times$, $g^k\mapsto e^{2\pi ik/n}$, where $i\in\{0,1,\dots,n-1\}$, 
        are the irreducible representations of $C_n$. 
    \item Let $\lambda$ be a primitive root of 1 of order $n$. Prove that 
        the character table of $C_n$ of order $n$ is given by 
	\begin{center}
		\begin{tabular}{|c|ccccc|}
			\hline 
			& 1 & 1 & 1 & $\cdots$ & 1\tabularnewline
			& $1$ & $g$ & $g^2$ & $\cdots$ & $g^{n-1}$\tabularnewline
			\hline 
			$\chi_{1}$ & $1$ & $1$ & $1$ & $\cdots$ & $1$\tabularnewline
			$\chi_{2}$ & $1$ & $\lambda$ & $\lambda^2$ & $\cdots$ & $\lambda^{n-1}$\tabularnewline
			$\chi_{3}$ & $1$ & $\lambda^2$ & $\lambda^4$ & $\cdots$ & $\lambda^{n-2}$\tabularnewline
			$\vdots$ & $\vdots$ & $\vdots$ & $\vdots$ & $\ddots$ & $\vdots$\tabularnewline
			$\chi_{n}$ & $1$ & $\lambda^{n-1}$ & $\lambda^{n-2}$ & $\cdots$ & $\lambda$\tabularnewline
			\hline
		\end{tabular}
	\end{center}
\end{enumerate}
\end{exercise}

\begin{exercise}
    Let $A$ and $B$ be abelian groups. We write $\Irr(A)=\{\rho_1,\dots,\rho_r\}$ and 
    $\Irr(B)=\{\phi_1,\dots,\phi_s\}$. Prove
    that the maps 
    \[
    \varphi_{ij}\colon A\times B\to\C^\times,\quad
    (a,b)\mapsto\rho_i(a)\phi_j(b),
    \]
    where $i\in\{1,\dots,r\}$ and $j\in\{1,\dots,s\}$, are the irreducible representations of $A\times B$. 
\end{exercise}

Let us show a particular example of the previous exercise. 

\begin{example}
	The character table of the group $C_2\times C_2=\{1,a,b,ab\}$ is 
	\begin{center}
		\begin{tabular}{|c|rrrr|}
			\hline 
			& 1 & 1 & 1 & 1\tabularnewline
			& $1$ & $a$ & $b$ & $ab$\tabularnewline
			\hline 
			$\chi_{1}$ & $1$ & $1$ & $1$ & $1$\tabularnewline
			$\chi_{2}$ & $1$ & $1$ & $-1$ & $-1$\tabularnewline
			$\chi_{3}$ & $1$ & $-1$ & $1$ & $-1$\tabularnewline
			$\chi_{4}$ & $1$ & $-1$ & $-1$ & $1$\tabularnewline
			\hline
		\end{tabular}
	\end{center}
% 	Let us do this by computer:
% \begin{lstlisting}
% gap> Display(CharacterTable(AbelianGroup([2,2])));
% CT2

%      2  2  2  2  2

%       1a 2a 2b 2c

% X.1     1  1  1  1
% X.2     1 -1  1 -1
% X.3     1  1 -1 -1
% X.4     1 -1 -1  1
% \end{lstlisting}
\end{example}

Clearly, the order in which the computer returns the irreducible characters is not necessarily the same we used! 

\begin{example}
	The symmetric group $\Sym_3$ has three conjugacy classes. The representatives are 
	$\id$, $(12)$ and $(123)$. There are three irreducible representations. We already found all the irreducible characters! 
	The character table of $\Sym_3$ is given by 
	\begin{center}
		\begin{tabular}{|c|rrr|}
			\hline
			& $1$ & $3$ & $2$\tabularnewline
			& $1$ & $(12)$ & $(123)$ \tabularnewline
			\hline 
			$\chi_{1}$ & $1$ & $1$ & $1$\tabularnewline
			$\chi_{2}$ & $1$ & $-1$ & $1$ \tabularnewline
			$\chi_{3}$ & $2$ & $0$ & $-1$ \tabularnewline
			\hline
		\end{tabular}
	\end{center}
	Let us recall how this table was computed. Degree-one irreducibles were easy to compute. 
	To compute the third row of the table one possible approach is to use
	the irreducible representation  
	\[
	(12)\mapsto \begin{pmatrix}-1&1\\0&1\end{pmatrix},
	\quad
	(123)\mapsto \begin{pmatrix}0&-1\\1&-1\end{pmatrix}.
	\]
    Then	
    \begin{align*}
		&\chi_3\left( (12) \right)=\trace\begin{pmatrix}-1&1\\0&1\end{pmatrix}=0,\\
		&\chi_3\left( (123) \right)=\chi_3\left( (12)(23)\right)=\trace\begin{pmatrix}0&-1\\1&-1\end{pmatrix}=-1.
	\end{align*}

	We should remark that the irreducible representation mentioned is not really needed to
	compute the third row of the character table. We can, for example, use the regular
	representation $L$. The character of $L$ is given by 
	\[
		\chi_L(g)=\begin{cases}
			6 & \text{si $g=\id$},\\
			0 & \text{si $g\ne\id$}.
		\end{cases}
	\]
	The equality $0=\chi_L\left( (12) \right)=1-1+2\chi_3( (12))$ implies that 
	$\chi_3( (12))=0$ and the equality $0=\chi_L( (123))=1+1+2\chi_3( (123))$
	implies that $\chi_3\left( (123) \right)=-1$. 

	Another approach uses the orthogonality relations. We need to compute $\chi_3( (12) )$ and $\chi_3( (123))$. 
	Let $a=\chi_3( (12) )$ and $b=\chi_3( (123))$. Then 
    we get that $a=0$ and $b=-1$. We just need to solve  
	\begin{align*}
		0&=\langle \chi_3,\chi_1\rangle=\frac16(2+3a+2b),\\
		0&=\langle \chi_3,\chi_2\rangle=\frac16(2-3a+2b).
	\end{align*}
	
% 	Let us use the computer:
% 	\begin{lstlisting}
% gap> S3 := SymmetricGroup(3);;
% gap> T := CharacterTable(S3);;
% gap> Display(T);
% CT3

%      2  1  1  .
%      3  1  .  1

%       1a 2a 3a
%     2P 1a 1a 3a
%     3P 1a 2a 1a

% X.1     1 -1  1
% X.2     2  . -1
% X.3     1  1  1
% \end{lstlisting}
% As we did before, some extra information was computed:
% \begin{lstlisting}
% gap> SizesConjugacyClasses(T);
% [ 1, 3, 2 ]
% gap> SizesCentralizers(T);
% [ 6, 2, 3 ]
% gap> SizesConjugacyClasses(T);
% [ 1, 3, 2 ]
% gap> OrdersClassRepresentatives(T);
% [ 1, 2, 3 ]
% \end{lstlisting}
\end{example}

%A challenging exercise: 
\begin{exercise}
Compute the character table of $\Sym_4$. 
\end{exercise}

\begin{example}
	We now compute the character table of the alternating group $\Alt_4$. This group has $12$ 
	elements and four conjugacy classes.
	\begin{center}
		\begin{tabular}{c|cccc}
			representative & $\id$ & $(123)$ & $(132)$ & $(123)$\tabularnewline
			\hline
			size & $1$ & $4$ & $4$ & $3$ 
		\end{tabular}
	\end{center}
	Since $[\Alt_4,\Alt_4]=\{\id,(12)(34),(13)(24),(14)(23)\}$,
	$\Alt_4/[\Alt_4,\Alt_4]$ has three elements. Thus $\Alt_4$ has three degree-one irreducibles and
	an irreducible character of degree three. Let 
	$\omega=\exp(2\pi i/3)$ be a primitive cubic root of 1. If $\chi$
	is a non-trivial degree-one character, then 
	$\chi\left( (123) \right)=\omega^j$
	for some $j\in\{1,2\}$ and $\chi\left( (132) \right)=\omega^{2j}$. Since 
	$(132)(134)=(12)(34)$ and 
	the permutations $(134)$ and $(123)$ are conjugate,  
	\[
	\chi_i((12)(34))=\chi_i((132)(134))=\chi_i((132))\chi_i((134))=\omega^3=1
	\]
	for all $i\in\{1,2\}$. 
	
	To compute $\chi_4$ we use the regular representation. 
	\begin{align*}
		0&=\chi_L\left( (12)(34) \right)=1+1+1+3\chi_4\left( (12)(34) \right),\\
		0&=\chi_L\left( (123) \right)=1+\omega+\omega^2+3\chi_4\left( (123) \right),\\
		0&=\chi_L\left( (132) \right)=1+\omega+\omega^2+3\chi_4\left( (132) \right).
	\end{align*}
	Then we obtain that $\chi_4\left( (123) \right)=\chi_4\left( (132)
	\right)=0$ and $\chi_4\left( (12)(34) \right)=-1$. Therefore, the character table of $\Alt_4$
	is given by
	\begin{center}
		\begin{tabular}{|c|rrrr|}
			\hline
			& $\id$ & $(123)$ & $(132)$ & $(12)(34)$\tabularnewline
			\hline
			$\chi_1$ & $1$ & $1$ & $1$ & $1$\tabularnewline
			$\chi_2$ & $1$ & $\omega$ & $\omega^2$ & $1$\tabularnewline
			$\chi_3$ & $1$ & $\omega^2$ & $\omega$ & $1$\tabularnewline
			$\chi_4$ & $3$ & $0$ & $0$ & $-1$\tabularnewline
			\hline
		\end{tabular}
	\end{center}
% 	Let us use the computer:
% \begin{lstlisting}
% gap> A4 := AlternatingGroup(4);;
% gap> T := CharacterTable(A4);;
% gap> Display(T);
% CT5

%      2  2  2  .  .
%      3  1  .  1  1

%       1a 2a 3a 3b
%     2P 1a 1a 3b 3a
%     3P 1a 2a 1a 1a

% X.1     1  1  1  1
% X.2     1  1  A /A
% X.3     1  1 /A  A
% X.4     3 -1  .  .

% A = E(3)^2
%   = (-1-Sqrt(-3))/2 = -1-b3
% \end{lstlisting}
% The symbol \lstinline{E(3)} denotes a primitive cubic root of 1, say our $\omega$. 
% To save some space, the compute uses the symbol \lstinline{A} to denote the complex number $\omega^2$ (it is the same as \lstinline{E(3)^2}) and
% the symbol \lstinline{/A} to denote the complex number $\omega$, the multiplicative inverse of $\omega^2$. 
\end{example}

\begin{example}
    Let $Q_8=\{-1,1,i,-i,j,-j\}$ be the quaternion group. Let us compute the character table of $Q_8$.
    The group $Q_8$ is generated by $\{i,j\}$ and the map $\rho\colon Q_8\to\GL_2(\C)$, 
    \[
    i\mapsto\begin{pmatrix}
    i&0\\0&i
    \end{pmatrix},
    \quad
    j\mapsto\begin{pmatrix}
    0&1\\-1&0
    \end{pmatrix},
    \]
    is a representation.
    The conjugacy classes of $Q_8$ are $\{1\}$, $\{-1\}$, $\{-i,i\}$, $\{-j,j\}$ and $\{-k,k\}$. 
    So there are five irreducible representations. 
    We can compute the character of $\rho$:
    	\begin{center}
		\begin{tabular}{|c|c|c|c|c|c|}
		    \hline
			& $1$ & $-1$ & $i$ & $j$ & $k$\tabularnewline
			\hline
			$\chi_\rho$ & 2 & 2 & 0 & 0 & 0\tabularnewline
			\hline
		\end{tabular}
	\end{center}
	Then $\rho$ is irreducible, es $\langle\chi_\rho,\chi_\rho\rangle=1$. 
	
	Since $[Q_8,Q_8]=\{-1,1\}=Z(Q_8)$, the quotient group $Q_8/[Q_8,Q_8]$ has four elements and
	hence there are four irreducible degree-one representations. Since 
	$Q_8$ is non-abelian, $Q_8/Z(Q_8)$ cannot be cyclic. 
	This implies that 
	$Q_8/[Q_8,Q_8]\simeq C_2\times C_2$. This allows us
	to compute almost all the character table of $Q_8$. 
		\begin{center}
		\begin{tabular}{|c|rrrrr|}
			\hline
			& $1$ & $-1$ & $i$ & $j$ & $k$\tabularnewline
			\hline
			$\chi_1$ & $1$ & $1$ & $1$ & $1$ & $1$\tabularnewline
			$\chi_2$ & $1$ & \cellcolor{gray!30}{$1$} & $-1$ & $1$ & $-1$\tabularnewline
			$\chi_3$ & $1$ & \cellcolor{gray!30}{$1$} & $1$ & $-1$ & $-1$\tabularnewline
			$\chi_4$ & $1$ & \cellcolor{gray!30}{$1$} & $-1$ & $-1$ & $1$\tabularnewline
			$\chi_5$ & $2$ & $-2$ & $0$ & $0$ & $0$\tabularnewline
			\hline
		\end{tabular}
	\end{center}
	It remains to compute $\chi_j(-1)$ for $j\in\{2,3,4\}$, these missing values are presented in shaded
    cells. To compute these values that $\langle\chi_i,\chi_j\rangle=0$ whenever $i\ne j$. The calculations
    are left as an exercise. 
% 	To check our character table we can use the computer. 
% \begin{lstlisting}
% gap> Q8 := QuaternionGroup(8);;
% gap> Display(CharacterTable(Q8));
% CT6

%      2  3  2  2  3  2

%       1a 4a 4b 2a 4c
%     2P 1a 2a 2a 1a 2a
%     3P 1a 4a 4b 2a 4c

% X.1     1  1  1  1  1
% X.2     1 -1 -1  1  1
% X.3     1 -1  1  1 -1
% X.4     1  1 -1  1 -1
% X.5     2  .  . -2  .
% \end{lstlisting}
\end{example}

\begin{exercise}
    Compute the character table of the dihedral group of eight elements. 
\end{exercise}





\part{Modules}

\chapter{Modules, submodules, homomorphisms}

\begin{definition}
    Let $R$ be a ring. A \textbf{module} (over $R$) is an abelian group
    $M$ with a map $R\times M\to M$, $(x,m)\mapsto x\cdot m$, such that
    the following conditions hold:
    \begin{enumerate}
        \item $(r_1+r_2)\cdot m=r_1\cdot m+r_2\cdot m$ for all $r_1,r_2\in R$ y $m\in M$.
		\item $r\cdot (m_1+m_2)=r\cdot m_1+r\cdot m_2$ for all $r\in R$ y $m_1,m_2\in M$.
		\item $r_1\cdot (r_2\cdot m)=(r_1r_2)\cdot m$ for all $r_1,r_2\in R$ y $m\in M$.
		\item $1\cdot m=m$ for all $m\in M$.	
    \end{enumerate}
\end{definition}

Our definition is that of left module. Similarly one defines right modules. We will always
consider left modules, so they will be referred simply as modules.

\begin{example}
A module over a field is a vector space. 
\end{example}

\begin{example}
Every abelian group is a modulo over $\Z$.	
\end{example}

\begin{example}
Let $R$ be a ring. Then $R$ is a module (over $R$) with $x\cdot m=xm$. 
This is the \textbf{(left) regular representation} of $R$ and it usually 
be denoted by $\prescript{}{R}R$. 
\end{example}

\begin{example}
If $R$ is a ring, then $R^n=\{(x_1,\dots,x_n):x_1,\dots,x_n\in R\}$ 
is a module (over $R$) with  
$r\cdot (x_1,\dots,x_n)=(rx_1,\dots,rx_n)$. 
\end{example}

\begin{example}
If $R$ is a ring, then $M_{m,n}(R)$ is a module (over $R$) with usual matrix operations. 
\end{example}

Students usually ask why in the definition of a ring homomorphism one needs
the condition $1\mapsto 1$. The following example provides a good explanation. 

\begin{example}
%%\label{exa:f(1)=1}
If $f\colon R\to S$ is a ring homomorphism and $M$ is a module (over $S$) wih 
$(s,m)\mapsto sm$, then 
$M$ is also a module (over $R$) with $r\cdot m=f(r)m$ for all $r\in R$ and $m\in M$. In fact, 
\begin{align*}
&1\cdot m=f(1)m=1m=m,\\
&r_1\cdot (r_2\cdot m)=f(r_1)(r_2\cdot m)=f(r_1)(f(r_2)m)=(f(r_1)f(r_2))m=f(r_1r_2)m
\end{align*}
for all $r_1,r_2\in R$ and $m\in M$.	  	
\end{example}
%
\begin{example}
Let $R=\R[X]$ and $T\colon\R^n\to\R^n$ be a linear map. Then $M=\R^n$ with 
\[
\left(\sum_{i=0}^na_iX^i\right)\cdot v=\sum_{i=0}^na_iT^i(v)
\]	
is a module (over $R$).   
\end{example}

\begin{example}
If $\{M_i|i\in I\}$ is a family of modules, then  	
\[
\prod_{i\in I}M_i=\{(m_i)_{i\in I}:m_i\in M_i\text{ for all $i\in I$}\}
\]
is a module with 
$x\cdot (m_i)_{i\in I}=(x\cdot m_i)_{i\in I}$, 
where $(m_i)_{i\in I}$ denotes the map $I\to M_i$, $i\mapsto m_i$.
This module is the \textbf{direct product} of the family $\{M_i:i\in I\}$.
\end{example}
%
\begin{example}
If $\{M_i|i\in I\}$ is family of modules, then   	
\[
\bigoplus_{i\in I}M_i=\{(m_i)_{i\in I}:m_i\in M_i\text{ for all $i\in I$ and $m_i=0$ except finitely many $i\in I$}\}
\]
is a module with 
$x\cdot (m_i)_{i\in I}=(x\cdot m_i)_{i\in I}$. 
This module is the \textbf{direct sum} of the family $\{M_i:i\in I\}$. 
\end{example}
%
%\begin{exercise}
If $M$ is a module, then $0\cdot m=0$ and $-m=(-1)\cdot m$ for all $m\in M$ and 
$x\cdot 0=0$ for all $x\in R$. 
%
\begin{example}
Let $M=\Z/6$ as a module (over $\Z$). Note that 
$3\cdot 2=0$ but $3\ne 0$ (in $\Z$) and $2\ne 0$ (in $\Z/6$).  
\end{example}
%
\begin{definition}
	Let $M$ be a module. A subset $N$ of $M$ is a \textbf{submodule} of $M$ if 
	$(N,+)$ is a subgroup of $(M,+)$ and 
	$x\cdot n\in N$ for all $x\in R$ and $n\in N$. 
\end{definition}

Clearly, if $M$ is a module, then $\{0\}$ and $M$ are submodules of $M$. 

\begin{example}
Let $R$ be a field and $M$ be a module over $R$. Then
$N$ is a submodule of $M$ if and only if $N$ is a subspace of $M$. 
\end{example}

\begin{example}
Let $R=\Z$ and $M$ be a module (over $R$). Then
$N$ is a submodule of $M$ if and only if $N$ is a subgroup of $M$
\end{example}

\begin{example}
If $M=\prescript{}{R}R$, then a subset $N\subseteq M$ is a submodule
of $M$ if and only if $N$ is a left ideal of $R$. 
\end{example}

\begin{example}
If $V$ is a vector space and $T\colon V\to V$ is a linear map, then
$V$ is a module (over $\R[X]$) with  
\[
\left(\sum_{i=0}^na_iX^i\right)\cdot v=\sum_{i=0}^na_iT^i(v).
\]
A submodule is a subspace $W$ 
of $V$ such that $T(W)\subseteq W$. 
\end{example}

Clearly, a subset $N$ of $M$ is a submodule if and only 
if $r_1n_1+r_2n_2\in N$ for all
$r_1,r_2\in R$ and $n_1,n_2\in N$. 	

\begin{exercise}
If $N$ and $N_1$ are submodules of $M$, then 
\[
N+N_1=\{n+n_1:n\in N,\,n_1\in N_1\}
\]
is a submodule of $M$.
\end{exercise}

\begin{definition}
Let $M$ and $N$ be modules over $R$. 
A map $f\colon M\to N$ is a \textbf{module homomorphism} if $f(x+y)=f(x)+f(y)$ and 
$f(r\cdot x)=r\cdot f(x)$ for all $x,y\in M$ and $r\in R$. 
\end{definition}

We denote by $\Hom_R(M,N)$ the set of module homomorphisms $M\to N$. 

\begin{exercise}
Let $f\in\Hom_R(M,N)$.  
\begin{enumerate}
\item If $V$ is a submodule of $M$, then $f(V)$ is a submodule of $N$.
\item If $W$ is a submodule of $N$, then $f^{-1}(W)$ is a submodule of $M$.
\end{enumerate}
\end{exercise}

If $f\in\Hom_R(M,N)$, the \textbf{kernel} of $f$ is the submodule  
\[
\ker f=f^{-1}(\{0\})=\{m\in M:f(m)=0\}
\]
of $M$. We say that $f$ is a \textbf{monomorphism} (resp. \textbf{epimorphism}) 
if $f$ is injective (resp. surjective). Moreover, $f$ is an \textbf{isomorphism} 
if $f$ is
bijective. 

\begin{exercise}
Let $f\in\Hom_R(M,N)$. Prove that the following statements are equivalent:
\begin{enumerate}
\item $f$ is a monomorphism.
\item $\ker f=\{0\}$.
\item For every module $V$ and every $g,h\in\Hom_R(V,M)$, $f\circ g=f\circ h\implies g=h$.
\item For every module $V$ and every $g\in\Hom(V,M)$, $f\circ g=0\implies g=0$.
\end{enumerate}
\end{exercise}


\begin{example}
	Let $R=
		\begin{pmatrix}
			\R & 0\\
			0 & \R
		\end{pmatrix}$. 
	We claim that 
	$\begin{pmatrix}
			\R\\
			0
		\end{pmatrix}
		\not\simeq\begin{pmatrix}
			0\\
			\R
		\end{pmatrix}$
	as modules over $R$, where the module structure is given by the usual matrix multiplication. 
	Assume that they are isomorphic. 
	Let $f\colon\begin{pmatrix}
			0\\
			\R
		\end{pmatrix}
		\to\begin{pmatrix}
			\R\\
			0
		\end{pmatrix}$  
	be an isomorphism of modules and let  
	$x_0\in\R\setminus\{0\}$ be such that 
	$f\begin{pmatrix}0\\1\end{pmatrix}=\begin{pmatrix}x_0\\0\end{pmatrix}$. Thus 
	\[
	f\begin{pmatrix}
	0\\
	1\end{pmatrix}
	=f\left(\begin{pmatrix}
	0&0\\
	0&1\end{pmatrix}
	\cdot 
	\begin{pmatrix}
	0\\
	1
	\end{pmatrix}\right)
	=\begin{pmatrix}
	0&0\\
	0&1\end{pmatrix}\cdot f\begin{pmatrix}0\\1\end{pmatrix}
	=\begin{pmatrix}
	0&0\\
	0&1
	\end{pmatrix}
	\cdot 
	\begin{pmatrix}		
	x_0\\
	0
	\end{pmatrix}
	=\begin{pmatrix}
	0\\
	0
	\end{pmatrix},
	\]	
	a contradiction, as $f$ is injective.    
\end{example}

If $N$ and $N_1$ are submodules of $M$, we say that $M$ is the \textbf{direct sum} of $N$ and $N_1$
if $M=N+N_1$ and $N\cap N_1=\{0\}$. In this case, we write $M=N\oplus N_1$. Note that if
$M=N\oplus N_1$, then each $m\in M$ can be written uniquely as $m=n+n_1$ for some
 $n\in N$ and $n_1\in N_1$. 
Such a decomposition exists because $M=S+T$. If $m\in M$ can be written as 
$m=n+n_1=n'+n_1'$ for some $n,n'\in N$ and $n_1,n_1'\in N_1$, then 
$-n'+n=n_1'-n_1\in N\cap N_1=\{0\}$ and hence $n=n'$ and $n_1=n_1'$. If $M=N\oplus N_1$, the submodule
$N$ (resp. $N_1$) is a \textbf{direct summand} of $M$ and the submodule $N_1$ (resp $N$) is a \textbf{complement} of $N$ 
in $M$.   	

\begin{example}
If $M=\R^2$ as a vector space, then every subspace of $M$ is a direct summand of $M$.
\end{example}

Clearly, the submodules $\{0\}$ and $M$ are direct summands of $M$.

\begin{example}
If $M=\Z$ as a module over $\Z$, then $m\Z$ is a direct sum of $M$ if and only if 
$m\in\{0,1\}$, as $n\Z\cap m\Z=\{0\}$ if and only if $nm=0$.
\end{example}

\begin{exercise}
\label{xca:projector}
Let $M$ be a module. 
A module $N$ is isomorphic to a direct summand of $M$ if and only if
there are module homomorphisms $i\colon N\to M$ and $p\colon M\to N$ 
such that $p\circ i=\id_N$. In this case, $M=\ker p\oplus i(N)$.  
\end{exercise}


\chapter{Finitely generated modules}
\chapter{Free modules}
\chapter{Structure theorems}

\backmatter
%\include{glossary}
\chapter*{Some hints}
\addcontentsline{toc}{chapter}{Some hints}

%\begin{sol}
%\end{sol}

\begin{sol}{xca:Z[sqrt10]/(2,sqrt10)}
	Use the ring homomorphism $\Z[\sqrt{10}]\to\Z/2$, $a+b\sqrt{10}\mapsto a\bmod 2$. 	
\end{sol}

\begin{sol}{xca:Z[i]/(1+3i)}
	Use the ring homomorphism $\Z\hookrightarrow\Z[i]\xrightarrow{\pi}\Z[i]/(1+3i)$, where
	$\pi$ is the canonical map. 	
\end{sol}

\begin{sol}{xca:extend}
    If $X$ is a linearly independent set, a basis 
    of $V$ that contains $X$ will be found as a maximal linearly independent set 
    containing $X$. 
\end{sol}

\begin{sol}{xca:freshman_dream}
    Use induction on $n$ and the fact that the prime number $p$ divides 
    $\binom{p}{k}$ for all $k\in\{1,2,\dots,p-1\}$. Alternatively, one could use
    that the prime number $p$ divides $\binom{p^n}{k}$ for all 
    $k\in\{1,2,\dots,p^n-1\}$. 
\end{sol}




\chapter{Some solutions}

\section*{Lecture 1}
\section*{Lecture 2}
\section*{Lecture 3}
\section*{Lecture 4}
\section*{Lecture 5}
\section*{Lecture 6}
\section*{Lecture 7}
\section*{Lecture 8}
\section*{Lecture 9}
\section*{Lecture 10}
\section*{Lecture 11}
\section*{Lecture 12}
\section*{Lecture 13}

\begin{sol}{xca:projector}
\end{sol}

\begin{sol}{xca:submodules}
\end{sol}

\begin{sol}{xca:commuting}
\end{sol}

\begin{sol}{xca:Hom}
\end{sol}


\bibliographystyle{abbrv}
\bibliography{refs}

\printindex



\end{document}





